\chapter{Constitutive Laws}

\section{Introduction}
Continuum Mechanics deals with the movement of materials under the action of
applied forces where the materials are continuous and deformable (we will only
consider solid mechanics here).  The result of a set of forces acting on a
deformable material is a (possibly time-varying) displacement field: each
material point moves a certain amount dependent both on its position relative
to the applied forces and on the mechanical properties of the material. A
displacement ``field'' implies a continuous variation of displacement with
position. The concepts of strain, a measure of length change or displacement
gradient, and stress, the force per unit area acting on an infinitesimally
small plane surface within the material, are of fundamental importance.

We derive the equations governing the motion of deformable
materials in the following four steps:
\begin{enumerate}
\item \textbf {Kinematic Relations}, which define the components of strain in
  terms of displacement gradients and, in the case of incompressible
  materials, define the incompressibility constraint.
\item \textbf {Stress Equilibrium}, or equations of motion, derived from the laws
  of conservation of linear momentum and conservation of angular momentum.
\item \textbf {Constitutive Relations}, which express the relationship between
  stress and strain and must be established from experimental measurement,
  subject to certain theoretical restrictions.
\item \textbf {Boundary Conditions}, which specify the external loads or
  displacement constraints acting on the deforming body.
\end{enumerate}
The first two steps are concerned with relationships which hold for all
materials and will be considered in detail in \secref{sec:Kinrel} and 7.4; the
third step is concerned with relations determined experimentally for a
particular material and is treated more fully in 7.5.
\remark {fix up section references no 7.5}
\section{Kinematic Relations}
\label{sec:Kinrel} 

\index{Kinematic relations}The key to analyzing strain in a material
undergoing large displacements and deformation is to establish two types of
coordinate system and the relationship between them. The first are fixed
reference coordinates and the second (material coordinates) are embedded in
the deforming body. From the base vectors of the material coordinates we
define metric tensors - measures of the physical lengths of coordinate
increments - and thence strain tensors.
 
\subsection{Coordinate systems and metric tensors}
The coordinates $\pbrac{X_{1},X_{2},X_{3}}$ define the position of a material point
in the undeformed tissue (see Fig.2.1) \remark{fix up figure reference} with 
respect to a fixed reference
rectangular cartesian coordinate system. The same material point in the
deformed tissue has rectangular cartesian coordinates $\pbrac{x_{1},x_{2},x_{3}}$.
Another system of material coordinates $\pbrac{x_{1},x_{2},x_{3}}$ is defined in
which the coordinate lines can be thought of as being attached to material
particles and so move with the deforming body. During the deformation the
initially straight and orthogonal $X_{M}$-coordinate axes remain ``attached''
to material particles and therefore become distorted and non-orthogonal in the
deformed configuration. $\bbrac{X_{M}}$ are called material coordinates or
convected coordinates. On the other hand the $x_{i}$-coordinate axes are fixed
throughout the deformation and $\bbrac{x_{i}}$ are called spatial coordinates (see
Figure 3.1).
\remark{fix reference}
\remark{missing figure}
\begin{figure} \centering
  \caption{Coordinate systems used in the kinematic
    analysis of large deformation. $\pbrac{X_{1},X_{2},X_{3}}$ are the
    rectangular cartesian coordinates of the point $P$
    (position vector $R$) in the undeformed body. $\pbrac{x_{1},x_{2},x_{3}}$
    are the rectangular cartesian coordinates of the same
    material point $p$ (position vector $r$) in the deformed
    body after a displacement $u$. $\pbrac{x_{1},x_{2},x_{3}}$ are material
    coordinates.}
  \label{fig:coord}
\end{figure} 

The base vectors for the $x_{i}$-coordinates in the undeformed reference state
are found by differentiating the position vector $\vect{R}=X_{k}\vect{i}_{k}$
\begin{equation}
  \vect{G}_{i} = \dfrac{\delta X_{k}}{\delta x_{i}} \vect{i}_{k}
  \label{eq:pv}
\end{equation}

The covariant components $G_{ij}$ of the undeformed metric tensor
are obtained from the inner products of the base vectors
\begin{equation}
  G_{ij}= \vect{G}_{i}\vect{G}_{j}=\dfrac{\delta X_{k}}{\delta x_{i}}
  \dfrac{\delta X_{k}}{\delta x_{j}}
  \label{eq:bv}
\end{equation}
 
Similarly, derivatives of the position vector $\vect{r}$ of a material
point in the deformed body (see \figref{fig:coord}) with respect to the
same material coordinates yield base vectors
\begin{equation}
 \vect{g}_{i} = \dfrac{\delta x_{k}}{\delta x_{i}} \vect{i}_{k}
 \label{eq:bv2}
\end{equation}
and metric tensors
\begin{equation}
 g_{ij}= \vect{g}_{i} \vect{g}_{j} = \dfrac{\delta x_{k}}{\delta x_{i}}
 \dfrac{\delta x_{k}}{\delta x_{j}}
 \label{eq:mt}
\end{equation}

\subsection{Strain measures}
\index{Strain measures}An initially undeformed line segment vector $d\vect{X}$
with components $dX^{1}, dX^{2}, dX^{3}$ (expressed more succinctly as
$d\vect{X} =\bbrac{dX^{M}}$) deforms into the line segment $d\vect{x}=
\bbrac{dx^{i}}$.  The deformation field is defined by expressing $\vect{x}$ as
a function of $\vect{X}$ and the components of deformation gradient,
$\dfrac{\delta x_{i}}{\delta X_{M}}$, collectively give the deformation
gradient tensor $\matr{F} = \bbrac{\dfrac{\delta x_{i}}{\delta X_{M}}}$.
Thus, the line segment $d\vect{X}$ is carried by the deformation into
$d\vect{x} = \matr{F}d\vect{X}$, or, in component form, $dx^{i} =
F^{i}_{M}dX^{M}$, where summation is implied ($M$ = 1,2,3) over the repeated
index $M$ and
\begin{equation}
  F^{i}_{M} = \dfrac{\delta x_{i}}{\delta X_{M}}
  \label{eq:ri}
\end{equation}
 
\remark {missing figure}
\begin{figure}\centering
  \caption{Coordinate systems used in the kinematic
    analysis of large deformation. $\pbrac{X_{1},X_{2},X_{3}}$ are the
    rectangular cartesian coordinates of the point $P$
    (position vector $\vect{R}$) in the undeformed body. $\pbrac{x_{1},x_{2},x_{3}}$
    are the rectangular cartesian coordinates of the same
    material point $p$ (position vector $\vect{r}$) in the deformed
    body after a displacement $u$. $\pbrac{x_{1},x_{2},x_{3}}$ are material
    coordinates.}
  \label{fig:cs}
\end{figure} 

A very important concept is that of ``polar decomposition'' 
\index{Polar decomposition}. The deformation gradient tensor $\matr{F}$ 
can be considered
as the product $\matr{F}=\matr{R}\matr{U}$ of an orthogonal rotation tensor
$\matr{R}$ and a symmetric positive stretch tensor $\matr{U}$. Equivalently,
the line segment components $dX^{M}$ are stretched into components $dy^{L} =
U^{L}_{M }dX^{M}_{ }$ before being rotated into the components $dx^{i} =
R^{i}_{L} dy^{L}$. An equally valid interpretation of the deformation would be
to consider the rigid body rotation before the stretching but for the present
purpose it is more convenient to interprete ``stretch'' in terms of material
coordinates and then use the rotation tensor $\matr{R}$ to relate the
stretched material lines to the spatial coordinates (for further details and a
full justification of the polar decomposition theorem see Atkin and Fox (1980)
or Spencer (1980)). The main point here is that the stretch tensor $\matr{U}$
contains a complete description of the material strain, independent of rigid
body motion (see also \Figref{fig:advection}).
 
The length of the line segments $d\vect{X}$ and $d\vect{x}$ are denoted by $dS$
and $ds$, respectively, where, from Pythagoras, $dS^{2} =dX^{M} dX^{M}$ and
$ds^{2}= dx^{i}dx^{i}$

Strain is a measure of relative length change and the so-called 
``Right Cauchy-Green'' strain tensor
\begin{displaymath}
  \matr{C} = \transpose{\matr{F}}\matr{F} = \frac{\delta x_{k}}{\delta
    X_{M}}\frac{\delta x_{k}}{\delta X_{N}}
\end{displaymath}
with components $C_{MN}$ , indicates how each component of undeformed line
segment $d\vect{X}$ contributes to the squared length of the deformed line
segment $d\vect{x}$ 
\begin{displaymath} 
  ds^{2} = dx^{k} dx^{k} = \dfrac{\delta x_{k}}{\delta X_{M}} dX^{M}
  \dfrac{\delta x_{k}}{\delta X_{N}} dX^{N} = \matr{C}_{MN} dX^{M} 
  dX^{N} = \transpose{dX} \matr{C} dX
\end{displaymath}

Using polar decomposition $\matr{F} = \matr{R}\matr{U}$, gives
\begin{displaymath}
  \matr{C} = \transpose{\matr{F}}\matr{F} =
  \transpose{\pbrac{\matr{R}\matr{U}}} 
  \matr{R}\matr{U} = \transpose{\matr{U}}\transpose{\matr{R}}
  \matr{R}\matr{U} = \transpose{\matr{U}} \matr{U} = \matr{U}^{2}
\end{displaymath}
since $\matr{R}$ is orthogonal ($\transpose{\matr{R}} = \matr{R}^{-1}$ ) and
$\matr{U}$ is symmetric ($\transpose{\matr{U}} = \matr{U}$).

Thus the stretch tensor $\matr{U}$ can be obtained by taking the positive
square-root of the strain tensor (see below).  Both $\matr{U}$ and $\matr{C}$ are
expressed in terms of material coordinates.

%%% Local Variables: 
%%% mode: latex
%%% TeX-master: t
%%% End: 
