
\chapter{Data fitting with finite elements}
\label{cha:datafitting}


\section{Introduction}

We show here how finite elements can be used for fitting one, two or
three-dimensional fields. The data could, for examples, be a series of
temperature measurememts (as in figure ***) and there is a need to find the
finite element nodal parameteres which, together with the chosen basis
functions, 'fit' the data in a least-sqquares sense. This linear least-squares
probelm is considerd in \secref{cha:linearfieldfitting} Another common
requirement is to fit nodal geometric parameters describing a finite element
surface in 3D space to a set of non-uniformly spced coordinates defining the
original surface. Geometric fitting problems are nonlinear when the $\xi_i$
-coordinates found at the orthogonal projection of the data onto the finite
element mesh are changed by the fitting process. A jucicious choice of
coordinate system can sometimes avoid the problem and keep the geometric
fitting linear, (this situation considers in **\secref{}. Full nonlinear
geometric fitting is considered in \secref)




\section{Linear field fitting}
\label{sec:linearfieldfitting}

\subsection{The linear field fitting problem}

Consider the two-dimensional bilinear element shown in
\figref{fig:linfieldfit}. The element surrounds a set of data points (shown by
the x's in \figref{fig:linfieldfit}) which consist of measured values $u_{d}$,
$d=1..D$, of some field (such as temperature) at specified locations $(x_{d},
y_{d})$. The field fitting problem is to find the values of the finite element
nodal parameters $u_{n}$, $n=1..4$, which minimise the sum of squared
differences $(u(\xi_{d})-u_{d})^{2}$ between $u_{d}$ and the finite element
field $u$ evaluated at the $(\xi_{1},\xi_{2})$ coordinates of data point $d$ 
$(u(\xi_{d}))$.

\begin{figure}[htpb] \centering
  \input{figs/datafitting/linfieldfit.pstex}
  \caption{Least-squares fitting of finite element nodal parameters $u_{n}$,
    $n=1..4$, to measured data values $u_{d}$, $d=1..D$, at point locations
    $(x_{d}, y_{d})$.}
  \label{fig:linfieldfit}
\end{figure}

\subsection{Calculation of data point projections}

The first requirement, therefore, is to find the $\xi_{i}$ coordinates of each
data point $(x_{d}, y_{d}$). For a bilinear element (with the basis functions
given by \eqref{eqn:2,3DE}) this is a straightforward inversion of the
following relations for each data point:

\begin{equation}
  \begin{array}{rcl}
    x_{d}&=&(1-\xi_{1}^{d})(1-\xi_{2}^{d})x_{1}+\xi_{1}^{d}(1-\xi_{2}^{d})x_{2}
    +(1-\xi_{1}^{d})\xi_{2}^{d}x_{3}+\xi_{1}^{d}\xi_{2}^{d}x_{4} \\
    y_{d}&=&(1-\xi_{1}^{d})(1-\xi_{2}^{d})y_{1}+\xi_{1}^{d}(1-\xi_{2}^{d})y_{2}
    +(1-\xi_{1}^{d})\xi_{2}^{d}y_{3}+\xi_{1}^{d}\xi_{2}^{d}y_{4}
  \end{array}
  \label{eqn:datapointposfield}
\end{equation}

where $(x_{n}, y_{n}), n=1..4$, are the specified node positions and $(x_{d},
y_{d}), d=1..D$, are the specified data point positions. To solve
\bref{eqn:datapointposfield} for $\xi_{1}^{d}$ and $\xi_{2}^{d}$ we rearrange
\bref{eqn:datapointposfield} as

\begin{equation}
  \begin{array}{rcl}
    a\xi_{1}^{d}+b\xi_{2}^{d}+c\xi_{1}^{d}\xi_{2}^{d}&=&d \\
    A\xi_{1}^{d}+B\xi_{2}^{d}+C\xi_{1}^{d}\xi_{2}^{d}&=&D 
  \end{array}
  \label{eqn:dataxipos1}
\end{equation}
where
\begin{displaymath}
  a=x_{2}-x_{1}, \quad b=x_{3}-x_{1}, \quad c=x_{1}-x_{2}-x_{3}+x_{4}, \quad
  \mbox{and} \quad d=x_{d}-x_{1}
\end{displaymath}
and
\begin{displaymath}
  A=y_{2}-y_{1}, \quad B=y_{3}-y_{1}, \quad C=y_{1}-y_{2}-y_{3}+y_{4}, \quad
  \mbox{and} \quad D=y_{d}-y_{1}
\end{displaymath}
then
\begin{equation}
  \xi_{2}^{d}=\dfrac{d-a\xi_{1}^{d}}{b+c\xi_{1}^{d}}=\dfrac{D-A\xi_{1}^{d}}
  {B+C\xi_{1}^{d}}
  \label{eqn:dataxipos2}
\end{equation}
or
\begin{equation}
  \alpha(\xi_{1}^{d})^{2}+\beta(\xi_{1}^{d})+\gamma=0
  \label{eqn:dataxipos3}
\end{equation}
where
\begin{displaymath}
  \alpha=Ac-aC, \quad \beta=dC-Dc+Ab-aB, \quad \mbox{and} \quad \gamma=Bd-bD
\end{displaymath}
Solving \bref{eqn:dataxipos3} gives
\begin{equation}
  \xi_{1}^{d}= \left\{ \begin{array}{ll} 
      \dfrac{-\beta\pm\sqrt{\beta^{2}-4\alpha\gamma}}{2\alpha} & 
      \mbox{when } \alpha\neq 0 \\
      -\dfrac{\gamma}{\beta} & \mbox{when } \alpha = 0
    \end{array} \right.
  \label{eqn:dataxipos4}
\end{equation}
and then $\xi_{2}^{d}$ is recovered from \bref{eqn:dataxipos2}.

\subsection{Least squares field fitting}

Once the $\xi_{i}$ coordinate positions of each data point
$\xi_{d}=(\xi_{1}^{d},\xi_{2}^{d})$ is known, and interpolation of the
(unknown) nodal values of $u_{n}$ ($n=1..4$) gives
\begin{displaymath}
  u(\xi_{d})=\Psi_{n}u_{n}
\end{displaymath}
where there is an implied sum over $n=1..4$. Now the weighted sum of squared
differences between this value and the measured value $u_{d}$ for $d=1..D$, is

\begin{equation}
%  {\cal F(\vect{u})}=\dsum_{d=1}^{D}w_{d}\left(\Psi_n(\xi_{d})u_{n}-
%    u_{d}\right)^{2}
  eqn here
  \label{eqn:datasumsqs}
\end{equation}
where $w_{d}$ is the weight for data point $d$. For measured data a good
choice for $w_{d}$ is one over the variance of the error for data point $d$.

Minimising \bref{eqn:datasumsqs} with respect to the nodal parameters $u_{m}$
gives
\begin{equation}
  \delby{{\cal F}}{u_{m}}=2\dsum_{d=1}^{D}w_{d}\left(\Psi_n(\xi_{d})u_{n}-
    u_{d}\right)\Psi_m(\xi_{d}) = 0
\end{equation}
or
\begin{equation}
  \left[\dsum_{d=1}^{D}w_{d}\Psi_m(\xi_{d})\Psi_n(\xi_{d})\right]u_{n}=
  \dsum_{d=1}^{D}w_{d}\Psi_m(\xi_{d})u_{d} \quad \quad m=1..4
  \label{eqn:linfieldfiteqn}
\end{equation}
Note that there is an implied sum over $n=1..4$ on the left hand side of
\eqref{eqn:linfieldfiteqn}. For the single 4-node element shown in
\figref{fig:linfieldfit}. 
\todo{check capital?}
\eqref{eqn:linfieldfiteqn} gives four unknowns
$u_{1}, u_{2}, u_{3}$ and $u_{4}$. If more than one element is used in the
fitting, the matrix and vector on the left and right hand sides, respectively,
of \bref{eqn:linfieldfiteqn} are
\begin{equation}
  E_{mn}=\dsum_{d=1}^{D}w_{d}\Psi_m(\xi_{d})\Psi_n(\xi_{d}) \quad \mbox{and} 
  \quad f_{m}=\dsum_{d=1}^{D}w_{d}\Psi_m(\xi_{d})u_{d}
\end{equation}
which can then be assembled into a global system of equations in exactly the
same fashion as occurs in the finite element solution of a boundary value
problem (see \secref{sec:OdSSHC-2.1}).

\subsection{Gauss point fitting}

An extension to linear field fitting is Gauss point fitting. Consider the
problem of finding the nodal stress field from a finite element analysis. The
finite element formulation for a stress analysis typically solves for
displacements, not stresses (see \chapref{cha:linearelasticity}). Stresses are
a derived quantity and are calculated from the constitutive law from the
strains which are calculated (in terms of nodal displacements) at the Gauss
points. The result of the analysis is that it is only possible to obtain a
Gauss point based description of the stresses. To obtain a nodal field based
description of the stress field we can use a fitting approach very similar to
the one described above for field fitting. The only difference in this case is
that our 'data points' are located at the Gauss points and hence the $\xi_d$
location is just the Gauss point location $\xi_{g}$. The values of the field
$u_{d}$ are just the values of stress calculated at the Gauss points and the
weights for each data point $w_{d}$ are 1.

\section{Linear geometric fitting}
\label{sec:lineargeometricfitting}

\subsection{Introduction}

Another common requirement is to fit nodal geometric parameters describing a
finite element geometry to a set of non-uniformly spaced coordinates defining
the original geometry. Geometric fitting differs from field fitting in that
when fitting with more than one geometric variable we have a non-linear
problem since the mesh data projection is no longer orthogonal.

Geometric fitting problems can be turned into a linear problem in two
ways. The first is to only fit geometries were there is one geometric
variable. The second is to keep the data projections constant throughout the
fit. Both options will now be considered.

\subsection{Fitting with one geometric variable}

A Geometric fitting problem with one variable is linear hence it is desirable
to try and arrange a geometric fitting problem so that only one variable is
being fitted. This can often be achieved by adopting a non-rectangular
cartesian coordinate system. For example consider a fitting problem in polar
coordinates as shown in \figref{fig:polargeomfit}

\todo{missing figure}

\begin{figure}[htpb] \centering
%  \input{figs/datafitting/polargeomfit.pstex}
%  \caption{Fitting the radial coordinate in a polar coordinate mesh. There are
%    four elements and four nodes. The dashed lines show the projections of the
%    data points (x) onto the starting mesh.}
  \label{fig:polargeomfit}
\end{figure}

In this case the orthogonal data point projections $\xi_{d}$ are independent
of the radius and can be easily found from the theta coordinate of the data
point $\theta_{d}$. This independence from radius means we can formulate the
geometric fitting problem with only one geometric variable, $r$ being
fitted. As with the linear field fitting case we can formulate an error
function as the weighted sum of squares of the individual errors, that is

\begin{equation}
  {\cal F(\vect{r})}=\dsum_{d=1}^{D}w_{d}\left(\Psi_{n}(\xi_{d})r_{n}-r_{d}
  \right)^{2}
\end{equation}

Minimising this error function with respect to the nodal radii $r_{n}$ we
obtain
\begin{equation}
  \delby{{\cal F}}{r_{m}}=2\dsum_{d=1}^{D}w_{d}\left(\Psi_n(\xi_{d})r_{n}-
    r_{d}\right)\Psi_m(\xi_{d}) = 0
\end{equation}
or in terms of element stiffness matrices
\begin{equation}
  E_{mn}r_{n}=f_{m}
\end{equation}
where
\begin{equation}
  E_{mn}=\dsum_{d=1}^{D}w_{d}\Psi_m(\xi_{d})\Psi_n(\xi_{d}) \quad \mbox{and} 
  \quad f_{m}=\dsum_{d=1}^{D}w_{d}\Psi_m(\xi_{d})u_{d}
\end{equation}

Another example of a change of coordinate systems is a prolate spheroidal
coordinates for fitting a heart geometry.

\subsection{Linear geometric fitting with fixed $\xi_{d}$ locations}

In some geometric fitting problems there is no appropriate change of 
coordinates and a rectangular cartesian system must be used. This means that,
in general, we require both x and y (and z) in the fit at the same time.
This results in a non-linear problem as the data point projection $\xi_{d}$
will now, in general, change as we change any geometric variable during the
fit. Hence in order to obtain a linear problem the data point projections
must remain fixed during the fit. To see how this works consider the geometric
fitting problem as shown in \figref{fig:fixedxifitbefore}.

\begin{figure}[htpb] \centering
  %\epsfig{file=epsfiles/beforefit.eps,width=15cm}
  \caption{Geometric fitting problem. The three nodal y-coordinates 
    $y_{1}, y_{2}$ and $y_{3}$ of a two linear element basis function mesh are
    fitted to five data points. The initial locations of the nodes are
    $(x_{1},y_{1})=(0,0), (x_{2},y_{2})=(1,0)$ and $(x_{3},y_{3})=(2,0)$. The
    location of the data points are
    $(x^{1},y^{1})=(0.2,1.2),(x^{2},y^{2})=(0.8,1.5),(x^{3},y^{3})=(1.3,1.5),
    (x^{4},y^{4})=(1.5,1.1)$ and $(x^{5},y^{5})=(1.7,0.8)$. The initial $\xi$
    projections are $\xi^{1}=0.2$ and $\xi^{2}=0.8$ in element 1 and
    $\xi^{3}=0.3,\xi^{4}=0.5$ and $\xi^{5}=0.7$ in element 2.}
  \label{fig:fixedxifitbefore}
\end{figure}

Now computing the error functions as a weighted sum of squares we obtain
\begin{equation}
  {\cal F(\vect{r})}=\dsum_{d=1}^{D}w_{d}\left(\Psi_{n}(\xi_{d})y_{n}-y_{d}
  \right)^{2}
\end{equation}

Minimising this error function with respect to the nodal parameters $y_{n}$ we
obtain the element matrix equation
\begin{equation}
  E_{mn}y_{n}=f_{m}
\end{equation}
where, as before,
\begin{equation}
  E_{mn}=\dsum_{d=1}^{D}w_{d}\Psi_m(\xi_{d})\Psi_n(\xi_{d}) \quad \mbox{and} 
  \quad f_{m}=\dsum_{d=1}^{D}w_{d}\Psi_m(\xi_{d})u_{d}
\end{equation}

Now for element 1
\begin{equation}
  \begin{array}{rcl}
    E_{11}&=&(1-\xi^{1})(1-\xi^{1})+(1-\xi^{2})(1-\xi^{2}) \\
    &=&(1-0.2)(1-0.2)+(1-0.8)(1-0.8)=0.68 \\
    E_{12}&=&E_{21}=(1-\xi^{1})\xi^{1}+(1-\xi^{2})\xi^{2} \\
    &=&(1-0.2)(0.2)+(1-0.8)(0.8)=0.32 \\
    E_{22}&=&\xi^{1}\xi^{1}+\xi^{2}\xi^{2} \\
    &=& (0.2)(0.2)+(0.8)(0.8)=0.68 \\
    f_{1}&=&(1-\xi^{1})y^{1}+(1-\xi^{2})y^{2} \\
    &=&(1-0.2)(1.2)+(1-0.8)(1.5)=1.26 \\
    f_{2}&=&\xi^{1}y^{1}+\xi^{2}y^{2} \\
    &=&(0.2)(1.2)+(0.8)(1.5)=1.44
  \end{array}
\end{equation}
and for element 2
\begin{equation}
  \begin{array}{rcl}
    E_{11}&=&(1-\xi^{3})(1-\xi^{3})+(1-\xi^{4})(1-\xi^{4})+(1-\xi^{5})
    (1-\xi^{5})\\
    &=&(1-0.3)(1-0.3)+(1-0.5)(1-0.5)+(1-0.7)(1-0.7)=0.83 \\
    E_{12}&=&E_{21}=(1-\xi^{3})\xi^{3}+(1-\xi^{4})\xi^{4}+(1-\xi^{5})\xi^{5}\\
    &=&(1-0.3)(0.3)+(1-0.5)(0.5)+(1-0.7)(0.7)=0.67 \\
    E_{22}&=&\xi^{3}\xi^{3}+\xi^{4}\xi^{4}+\xi^{5}\xi^{5}\\
    &=&(0.3)(0.3)+(0.5)(0.5)+(0.7)(0.7)=0.83 \\
    f_{1}&=&(1-\xi^{3})y^{3}+(1-\xi^{4})y^{4}+(1-\xi^{5})y^{5}\\
    &=&(1-0.3)(1.5)+(1-0.5)(1.1)+(1-0.7)(0.8)=1.84\\
    f_{2}&=&\xi^{3}y^{3}+\xi^{4}y^{4}+\xi^{5}y^{5}\\
    &=&(0.3)(1.5)+(0.5)(1.1)+(0.7)(0.8)=1.56
  \end{array}
\end{equation}

Assembling these element matrices and vectors into a global system of
equations we get
\begin{equation}
  \begin{bmatrix}
    0.68 & 0.32 & 0.00 \\
    0.32 & 1.51 & 0.67 \\
    0.00 & 0.67 & 0.83 
  \end{bmatrix}
  \begin{bmatrix}
    y_{1} \\
    y_{2} \\
    y_{3}
  \end{bmatrix} =
  \begin{bmatrix}
    1.26 \\
    3.28 \\
    1.56
  \end{bmatrix}
\end{equation}

Solving these gives
\begin{equation}
  \begin{bmatrix}
    y_{1} \\
    y_{2} \\
    y_{3}
  \end{bmatrix} =
  \begin{bmatrix}
    1.03210 \\
    1.74420 \\
    0.47152
  \end{bmatrix}
\end{equation}

The fitted solution is shown in \figref{fig:fixedxifitafter}.
\begin{figure}[htpb] \centering
  %\epsfig{file=epsfiles/afterfit.eps,width=15cm}
  \caption{Fitted mesh. The $\xi_{d}$ locations on the fitted mesh are
    $\xi^{1}=0.212$ and $\xi^{2}=0.752$ in element 1 and $\xi^{3}=0.233,
    \xi^{4}=0.504$ and $\xi^{5}=0.726$ in element 2.}
  \label{fig:fixedxifitafter}
\end{figure}
It should be noted that after the fit the new data point projections have
changed after the fit. Hence it general there maybe some benefit from
reapplying the fitting procedure to the new data point projection i.e. 
iterating on the fit.

\section{Geometric fitting with Hermite elements}

\subsection{Review of cubic Hermite interpolation}

\subsubsection{Cubic Hermite basis functions}

One of the most commonly used basis functions in finite elements are Lagrange
basis functions which preserve continuity of the geometric coordinates across
element boundaries by interpolating nodal coordinates which are shared by
adjacent elements i.e.  $C^{0}$ continuity.  The interpolation formula for
linear Lagrange interpolation is given in \eqref{eqn:linLagrangeinterp}.
\begin{equation}
  \fnof{\vect{x}}{\xi}=\varphi_{1}(\xi)\vect{x}_{1}+\varphi_{2}(\xi)\vect{x}_{2}
  \label{eqn:linLagrangeinterp}
\end{equation}
where $\vect{x}_{n}$ is the geometric position of local node $n$ and the two
one-dimensional linear Lagrange basis functions are given in
\eqref{eqn:linLagrangeBfuns}.
\begin{eqnarray}
  \varphi_{1}(\xi)=1-\xi & \varphi_{2}(\xi)=\xi
  \label{eqn:linLagrangeBfuns}
\end{eqnarray}

Cubic Hermite functions, on the other hand, also preserve continuity of the
derivative of these coordinates with respect to $\xi$ across element
boundaries by defining additional nodal parameters \todo{here}
%\dbyat{\vect{x}}{\xi}{n}
i.e. $C^{1}$ continuity. The interpolation formula within an element is given
by
\begin{equation}
%  \fnof{\vect{x}}{\xi}=\Psi_{1}^{0}(\xi)\vect{x}_{1}+\Psi_{1}^{1}(\xi)\dbyat{\vect{x}}
%  {\xi}{1}+\Psi_{2}^{0}(\xi)\vect{x}_{2}+\Psi_{2}^{1}(\xi)\dbyat{\vect{x}}
%  {\xi}{2}
  \todo{eqn}
  \label{eqn:cubHermxiinterp}
\end{equation}
where the four one-dimensional cubic Hermite basis functions are given in 
\eqref{eqn:cubHermBfuns}.
\begin{eqnarray}
  \Psi_{1}^{0}(\xi)=1-3\xi^{2}+2\xi^{3} & \Psi_{1}^{1}(\xi)=\xi(\xi-1)^{2} 
  \nonumber \\
  \Psi_{2}^{0}(\xi)=\xi^{2}(3-2\xi) & \Psi_{2}^{1}(\xi)=\xi^{2}(\xi-1)
  \label{eqn:cubHermBfuns}
\end{eqnarray}

One further step is required to make cubic Hermite basis functions useful in
practice.  Consider now the two cubic Hermite elements as shown in 
\figref{fig:cubHermelem}.

\begin{figure}[htbp] \centering
  \input{figs/datafitting/cubichermiteelem.pstex}
  \caption{Two cubic Hermite elements (denoted by {\bf 1} and {\bf 2}) formed
    from three nodes (shown as a $\bullet$ and denoted by 1, 2 and 3) and 
    having arc-lengths $s_{1}$ and $s_{2}$.}
  \label{fig:cubHermelem}
\end{figure}

The derivative \todo{here}
%\dbyat{\vect{x}}{\xi}{n} defined at node $n$ is dependent upon
the local element $\xi$-coordinate and is therefore, in general, different in
the two adjacent elements. Therefore we carry a physical derivative
%$\dbyat{\vect{x}}{s}{n}$ at nodes and use
\begin{equation}
%  \dbyat{\vect{x}}{\xi}{n}=\brdby{\vect{x}}{s}_{\Delta(n,e)}.\brdby{s}{\xi}_{e}
  \label{eqn:xitoarclength}
\end{equation}
%to determine \dbyat{\vect{x}}{\xi}{n}. Here \dby{\vect{x}}{s} is a physical arc-
length derivative, $\Delta(n,e)$ is the global node number of local node $n$
%in element $e$, $\brdby{s}{\xi}_{e}$ is an element 'scale factor', denoted by
$S_{e}$, which scales the arc-length derivative to the $\xi$-coordinate
derivative.  Thus $\dby{\vect{x}}{s}$ is constrained to be continuous across
element boundaries rather than $\dby{\vect{x}}{\xi}$.

There is one condition that must be placed on the $\xi$ to arc-length
transformation to ensure that we have arc-length derivatives. This condition
is that the arc-length derivative vector at a node must have unit magnitude,
that is
\begin{equation}
%  \norm{\brdby{\vect{x}}{s}_{n}}=1
  \label{eqn:cubHermnormconst}
\end{equation}
This ensures that we have continuity with respect to a physical parameter
rather than with respect to a mathematical parameter $\xi$. The set of mesh
parameters, \vect{u}, for cubic Hermite interpolation hence contains the set
of nodal values (or positions), the set of nodal arc-length derivatives and
the set of scale factors.

\subsubsection{Bicubic Hermite basis functions}

Bicubic Hermite basis functions are the two-dimensional extension of the
one-dimensional cubic Hermite basis functions. They are formed from the tensor
(or outer) product of two one-dimensional basis functions as defined in
\eqref{eqn:cubHermBfuns}. The interpolation formula for a point $(\xi_{1},
\xi_{2})$ within an element is obtained from the bicubic Hermite interpolation
formula,
\begin{eqnarray}
%\vect{x}(\xi_{1},\xi_{2}) &=& \Psi_{1}^{0}(\xi_{1})\Psi_{1}^{0}(\xi_{2})
%    \vect{x}_{1}+\Psi_{2}^{0}(\xi_{1})\Psi_{1}^{0}(\xi_{2})\vect{x}_{2} + 
%    \nonumber \\
%    & & \Psi_{1}^{0}(\xi_{1})\Psi_{2}^{0}(\xi_{2})\vect{x}_{3} +
%    \Psi_{2}^{0}(\xi_{1})\Psi_{2}^{0}(\xi_{2})\vect{x}_{4} + \nonumber \\
%    & & \Psi_{1}^{1}(\xi_{1})\Psi_{1}^{0}(\xi_{2})\delbyat{\vect{x}}
%    {\xi_{1}}{1}+\Psi_{2}^{1}(\xi_{1})\Psi_{1}^{0}(\xi_{2})\delbyat{\vect{x}}
%    {\xi_{1}}{2} + \nonumber \\
%    & & \Psi_{1}^{1}(\xi_{1})\Psi_{2}^{0}(\xi_{2})\delbyat{\vect{x}}
%    {\xi_{1}}{3}+\Psi_{2}^{1}(\xi_{1})\Psi_{2}^{0}(\xi_{2})\delbyat{\vect{x}}
%    {\xi_{1}}{4} + \nonumber \\
%    & & \Psi_{1}^{0}(\xi_{1})\Psi_{1}^{1}(\xi_{2})\delbyat{\vect{x}}
%    {\xi_{2}}{1}+\Psi_{2}^{0}(\xi_{1})\Psi_{1}^{1}(\xi_{2})\delbyat{\vect{x}}
%    {\xi_{2}}{2} + \nonumber \\
%    & & \Psi_{1}^{0}(\xi_{1})\Psi_{2}^{1}(\xi_{2})\delbyat{\vect{x}}
%    {\xi_{2}}{3}+\Psi_{2}^{0}(\xi_{1})\Psi_{2}^{1}(\xi_{2})\delbyat{\vect{x}}
%    {\xi_{2}}{4} + \nonumber \\
%    & & \Psi_{1}^{1}(\xi_{1})\Psi_{1}^{1}(\xi_{2})\deltwobyat{\vect{x}}
%    {\xi_{1}}{\xi_{2}}{1} +
%    \Psi_{2}^{1}(\xi_{1})\Psi_{1}^{1}(\xi_{2})\deltwobyat{\vect{x}}
%    {\xi_{1}}{\xi_{2}}{2} + \nonumber \\
%    & & \Psi_{1}^{1}(\xi_{1})\Psi_{2}^{1}(\xi_{2})\deltwobyat{\vect{x}}
%    {\xi_{1}}{\xi_{2}}{3} +
%    \Psi_{2}^{1}(\xi_{1})\Psi_{2}^{1}(\xi_{2})\deltwobyat{\vect{x}}
%    {\xi_{1}}{\xi_{2}}{4}
    \label{eqn:bicubHerminterp}
\end{eqnarray}

As with the one-dimensional cubic Hermite elements, the derivatives with
respect to $\xi$ in the two-dimensional interpolation formula above are
expressed as the product of a nodal arc-length derivative and a nodal scale
factor. This is, however, complicated by the fact that there are now scale
factors for each $\xi$ direction at each node. For node $n$ we
have
\begin{equation}
%  \delbyat{\vect{x}}{\xi_{i}}{n}=\brdelby{\vect{x}}{s_{i}}_{\Delta(n,e)}.
%  \left(S_{i}\right)_{e}
  \label{eqn:xitoarclength2}
\end{equation}
and for the cross-derivative
\begin{equation}
%  \deltwobyat{\vect{x}}{\xi_{1}}{\xi_{2}}{n}=\brdeltwoby{\vect{x}}{s_{1}}{s_{2}}_
%  {\Delta(n,e)}.\left(S_{1}\right)_{e}.\left(S_{2}\right)_{e}
  \label{eqn:xitoarclength3}
\end{equation}

As with the one-dimensional cubic Hermite case conditions must be placed on
this transformation in order to maintain $C^{1}$ continuity. A sufficient
condition is that the scale factor at a node in one element must be the same
as the scale factor at the same node in an adjacent element. That is, the
elemental scale factor should be nodally based so that the same scale factor
is used at a given node regardless of the current element. With this condition
satisfied any choice of scale factor will give $C^{1}$ continuity across
element boundaries. The choice of the scale factor will, however, affect the
spacing of $\xi$ with arc-length. It is often computationally desirable to
have a uniform spacing of $\xi$ with respect to arc-length (for example, not
biasing the Gaussian quadrature scheme to one end of the element). To achieve
this uniform spacing a good choice of the nodal scale factor (denoted from now
on as ${\cal S}_{n}$ for node $n$) is the average of the two arc-lengths on
either side of the node. This condition must also be applied together with
arc-length derivatives for each $\xi$ direction as defined by
\eqref{eqn:cubHermnormconst}.

If $n_{\ominus}$ is the node or line immediately before node $n$ (in the sense
of increasing $\xi$) and $n_{\oplus}$ is the node or line immediately after
node $n$, the nodal scale factor is given by
\begin{equation}
  {\cal S}_{n}=\frac{s_{n_{\ominus}}+s_{n_{\oplus}}}{2}
  \label{eqn:avearclenscale}
\end{equation}
This is termed average arc-length scaling.  For example consider node 2 in
\figref{fig:cubHermelem} then
\[{\cal S}_{2}=\frac{s_{1}+s_{2}}{2}\]
Thus, for an element $e$, the one-dimensional cubic Hermite interpolation
formula in \bref{eqn:cubHermxiinterp} becomes
\begin{equation}
%  \fnof{\vect{x}}{\xi}=\sum_{n=1,2}\left[\Psi_{n}^{0}(\xi)\vect{x}_{n}+\Psi_{n}^{1}(\xi)
%  \brdby{\vect{x}}{s}_{n}.{\cal S}_{n}\right]
  \label{eqn:cubHerminterp}
\end{equation}
To calculate the arc-length for a particular element an iterative process is
needed. The arc-length for an one-dimensional element is defined as
\begin{equation}
%  \mbox{arc-length}=\int_{0}^{1}\norm{\dby{\vect{x}(\xi)}{\xi}}d\xi=
%  \int_{0}^{1}\sqrt{\brdby{x(\xi)}{\xi}^{2}+\brdby{y(\xi)}{\xi}^{2}}d\xi
  \label{eqn:arclendef}
\end{equation}
However, since the interpolation of \fnof{\vect{x}}{\xi}, as defined in
\bref{eqn:cubHerminterp}, uses the arc-length in the calculation of the
scaling factor, an iterative root finding technique is needed to obtain the
arc-length.

With nodal based scale factors and arc-length derivatives we achieve a good
arc-length to $\xi$ spacing, continuity is maintained with respect to a
physically meaningful parameter and the numerical results when using the mesh
are more accurate.

\subsection{Least squares fitting with cubic Hermite interpolation}

\subsubsection{Problem formulation}

Consider a set of rectangular cartesian data with geometric positions
$\vect{z}_{d},d=1..D$. For each data point we can find the position on the mesh
which has the smallest distance to that data point. This point is the
orthogonal projection of the data point onto the mesh and has geometric
position $\vect{z}$. The point $\vect{z}$ is also given by the local element
co-ordinate $\vect{\xi}_{d}$ as is shown in \figref{fig:dataproj}.

To calculate $\vect{\xi}_{d}$ a non-linear iterative procedure is required.
Given a $\vect{\xi}$ position for the data point projection within an element
the geometric position of this projection is given by the standard
interpolation formula (\eqref{eqn:cubHermxiinterp} or
\eqref{eqn:bicubHerminterp}). An error function can then be set up as the
Euclidean distance between this position and the actual position of the data
point. The $\vect{\xi}$ position which minimises this function can the be found
by using the Newton-Rhapson root finding method on the derivative of this
function. This $\vect{\xi}$ position is the orthogonal projection of the data
point.

\begin{figure} \centering
  \input{figs/datafitting/datapos.pstex}
  \caption{Definition of a data point projection into an element. The data
    point at geometric location $\vect{z}_{d}$ is projected into an element at a
    geometric position $\vect{z}$ and element co-ordinate $\xi_{d}$.}
  \label{fig:dataproj}
\end{figure}

For simplicity only two-dimensional fitting (i.e. cubic Hermite elements) will
be covered in this section. Hence, given $\xi_{d}$, $\vect{z}$ can be found from
interpolation i.e.
\begin{equation}
%  \vf{z}{\xi_{d}}=\sum_{n=1,2}\left[\onevf{\Psi_{n}^{0}}{\xi_{d}}\vect{x}_{n}+
%  \onevf{\Psi_{n}^{1}}{\xi_{d}}\brdby{\vect{x}}{s}_{n}.{\cal S}_{n}\right]
  \label{eqn:datapointpos}
\end{equation}

The measure of error for each data point is defined as the Euclidean distance
between the data point and its closest projection onto the current mesh:
\begin{equation}
%  \onevf{f_{d}}{\xi_{d}}=\norm{\vf{z}{\xi_{d}}-\vect{z}_{d}}
\end{equation} 

For a given projection of the data points onto the mesh (i.e. $\xi_{d}$ is
held constant) the objective function to be minimised in the fit is then formed
as the sum-of-squares of the individual errors.
\begin{equation}
%  \vvf{{\cal F}}{u}=\sum_{d=1,D}\gamma_{d}\onevf{f_{d}^{2}}{\xi_{d}}=
%  \sum_{d=1,D}\gamma_{d}\norm{\onevf{\vect{z}}{\xi_{d}}-\vect{z}_{d}}^{2}
  \label{eqn:linfitobjfun}
\end{equation}
where $\gamma_{d}$ is a weight for each data point and \vect{u} is a vector of
mesh parameters.

The fitting problem is to find the set of mesh parameters that minimises 
this objective function. Substituting \bref{eqn:datapointpos} into
\bref{eqn:linfitobjfun} and differentiating we obtain
\begin{equation}
%  \delby{\vvf{{\cal F}}{u}}{\left(x_{j}\right)_{m}}=2\sum_{d=1,D}\gamma_{d}
%  \left(\sum_{n=1,2}\left[\onevf{\Psi_{n}^{0}}{\xi_{d}}\left(x_{j}\right)_{n}+
%  \onevf{\Psi_{n}^{1}}{\xi_{d}}\brdelby{x_{j}}{s}_{n}.{\cal S}_{n}\right]-
%  {z_{d}}_{j}\right)\onevf{\Psi_{m}^{0}}{\xi_{d}} 
  \label{eqn:lindatajacnode}
\end{equation}
\begin{equation}
%  \delby{\vvf{{\cal F}}{u}}{\brdelby{x_{j}}{s}_{m}}=2\sum_{d=1,D}\gamma_{d}
%  \left(\sum_{n=1,2}\left[\onevf{\Psi_{n}^{0}}{\xi_{d}}\left(x_{j}\right)_{n}+
%  \onevf{\Psi_{n}^{1}}{\xi_{d}}\brdelby{x_{j}}{s}_{n}.{\cal S}_{n}\right]-
%  {z_{d}}_{j}\right)\onevf{\Psi_{m}^{1}}{\xi_{d}}{\cal S}_{m} 
  \label{eqn:lindatajacderiv}
\end{equation}

A minimum can thus be found to the objective function by setting both 
\eqnrefs{eqn:lindatajacnode}{eqn:lindatajacderiv} to zero. This will result in
a linear system only if the scale factors are kept constant during the fit. 
That is the vector $\vect{u}$ will contain the nodal positions and the nodal
arc-length derivatives. With this restriction we can obtain
\begin{eqnarray*}
%  \sum_{d=1,D}\gamma_{d}\left(\sum_{n=1,2}\left[\onevf{\Psi_{n}^{0}}{\xi_{d}}
%  \left(x_{j}\right)_{n}+\onevf{\Psi_{n}^{1}}{\xi_{d}}{\cal S}_{n}
%  \brdelby{x_{j}}{s}_{n}\right]\onevf{\Psi_{m}^{0}}{\xi_{d}}\right)
%  &=&\sum_{d=1,D}\gamma_{d}\onevf{\Psi_{m}^{0}}{\xi_{d}}{z_{d}}_{j} \\ 
%  \sum_{d=1,D}\gamma_{d}\left(\sum_{n=1,2}\left[\onevf{\Psi_{n}^{0}}{\xi_{d}}
%  \left(x_{j}\right)_{n}+\onevf{\Psi_{n}^{1}}{\xi_{d}}{\cal S}_{n}
%  \brdelby{x_{j}}{s}_{n}\right]\onevf{\Psi_{m}^{1}}{\xi_{d}}{\cal S}_{m}\right)
%  &=&\sum_{d=1,D}\gamma_{d}\onevf{\Psi_{m}^{1}}{\xi_{d}}{\cal S}_{m}{z_{d}}_{j}
\end{eqnarray*}
This is a linear system of equations for the element of the form
\begin{equation}
  K_{mn}u_{n}=f_{m}
  \label{eqn:linsystem}
\end{equation}

A linear system of equations governing the entire mesh can then be found by
assembling a global stiffness matrix from all the individual element matrices.
This can then be solved to yield the nodal positions and derivatives which
minimises the error in the mesh. An example of fitting is shown in
\figref{fig:cubHermfit}.

\begin{figure} \centering
  \input{figs/datafitting/fitting.pstex}
  \caption{Geometric data fitting with cubic Hermite elements. (a) The data
     points (+) are shown projected onto the two element mesh at some 
     intermediate stage in the fitting procedure. (b) The final fitted mesh.}
  \label{fig:cubHermfit}
\end{figure}

\subsubsection{Sobelov Smoothing}

With an insufficient number of data points, fitting 'noisy' data or fitting
data that has an uneven spread, a smoothness constraint \cite{young:1989}
can be introduced by adding a second term to the objective function:
%\[\vvf{F}{u}=\sum_{d=1,D}\gamma_{d}\norm{\onevf{\vect{z}}{\vect{\xi}_{d}}-
%\vect{z}_{d}}^{2}+\int_{\Omega} \onevf{g}{\vvvf{u}{\xi}}d\vect{\xi}\]

The first term measures the error in the surface from the data and the second
term measures deformation of the surface. To measure the deformation of the
surface a $p^{th}$ order weighted Sobelov norm 
\cite{terzopoulos:1986,tikhonov:1977} is used, defined by
\begin{equation}
%  \onevf{g_{p,w}}{\vvvf{u}{\xi}}=\sum_{q=0}^{p}\sum_{i+j=q}w_{ij}
%  \norm{\frac{\del^{q}\vect{u}}{\del^{i}\xi_{1}\del^{j}\xi_{2}}}^{2}
  \label{eqn:Sobnorm}
\end{equation}
where $w_{ij}$ are the weights for the norm. The addition to the objective
function, called the Sobelov value, is defined as
\begin{equation}
%  \onevf{G}{\vvvf{u}{\xi}}=\int_{\Omega}{\onevf{g}{\vvvf{u}{\xi}}d
%    \vect{\xi}}
  \label{eqn:Sobvalue}
\end{equation}
where $\Omega$ is the mesh domain.

For the case of two-dimensional fitting (i.e. cubic Hermite elements) $j=0$
(as there is no $\xi_{2}$ direction) and for the case of three-dimensional
fitting (i.e. bicubic Hermite elements) $j=0..2$. Consider the case for $p=2$,
and for 2D: $w_{0}=0, w_{1}=\alpha, w_{2}=\beta$, and for 3D: $w_{00}=0,
w_{01}=w_{10}=\alpha, w_{20}=w_{02}= \beta, w_{11}=2\beta$. The Sobelov value
now becomes
\begin{equation}
  \begin{array}{lllll}
%    &\onevf{G}{\vvvf{u}{\xi}}&=&\displaystyle\int_{\Omega}{\left\{
%    \alpha\norm{\dby{\vect{u}}{\xi}}^{2}+\beta\norm{\dtwosqby{\vect{u}}{\xi}}^{2}
%    \right\}d\xi} & \mbox{for 2D} \nonumber \\ 
%    \mbox{or}&\onevf{G}{\vvvf{u}{\xi}}&=&\displaystyle\int_{\Omega}
%    \left\{\alpha\left(\norm{\delby{\vect{u}}{\xi_{1}}}^{2}+\norm{\delby{\vect{u}}
%    {\xi_{2}}}^{2}\right)\right.+ & \nonumber \\ 
%    &&&\left.\beta\left(\norm{\deltwosqby{\vect{u}}{\xi_{1}}}^{2}+
%    2\norm{\deltwoby{\vect{u}}{\xi_{1}}{\xi_{2}}}^{2}+
%    \norm{\deltwosqby{\vect{u}}{\xi_{2}}}^{2}\right)\right\}d\vect{\xi}
%    & \mbox{for 3D} \nonumber
  \end{array}
  \label{eqn:ptwoSobnorm}
\end{equation}

The parameter $\alpha$ controls the tension of the surface and the parameter
$\beta$ controls the degree of surface curvature
\cite{terzopoulos:1986}.

\subsection{Non-linear geometric fitting with Hermite elements}

\subsubsection{Problem formulation}

One problem that arises when using linear fitting with cubic Hermite elements
is that arc-length derivatives and average arc-length scaling are not
maintained during the fit. In linear fitting there is no information supplied
in the linear fitting model that enforces arc-length derivatives and the scale
factors are held constant during the fit. Here we consider how to fit the data
whilst maintaining arc-length derivatives and an even spacing of arc-length
with $\xi$. Because both the value of the arc-length for the element and the
relationship between the derivatives in the various spatial directions depend
upon the mesh parameters in a non-linear fashion, the only way to ensure
arc-length derivatives are maintained during fitting is to use a non-linear
fitting procedure.

Consider the following: Let \vect{u} be the vector of mesh parameters, 
$\vect{z}_{d}$ the vector of the location of the data points in space and 
%\twovf{\vect{z}}{\vect{u}}{\vect{\xi}_{d}} 
the vector of the location of the closest projection (at the points given by
%the vector $\vect{\xi}_{d}$) of data point $d$ onto the mesh. Now the error in
the $d^{th}$ data point can be expressed by an error vector
\begin{equation}
%\twovf{\vect{e}_{d}}{\vect{u}}{\vect{\xi}_{d}}=\twovf{\vect{z}}{\vect{u}}{\vect{\xi}_{d}}
%-\vect{z}_{d}
\label{eqn:errorvec}
\end{equation}
and by a residual,
\begin{equation}
%\twovf{f_{d}}{\vect{u}}{\vect{\xi}_{d}}=\norm{\twovf{\vect{e}_{d}}{\vect{u}}
%{\vect{\xi}_{d}}}^{2}
\label{eqn:dataresidvec}
\end{equation}

%An objective function, \twovf{{\cal F}}{\vect{u}}{\vect{\xi}_{d}}, is formed 
as the sum of squares of the individual residuals for the mesh \vect{u} and
projections $\vect{\xi}_{d}$
\begin{equation}
%\twovf{{\cal F}}{\vect{u}}{\vect{\xi}_{d}}= \twovft{\vect{f}}{\vect{u}}{\vect{\xi}_{d}}
%\twovf{\vect{f}}{\vect{u}}{\vect{\xi}_{d}}
\label{eqn:dataresfun}
\end{equation}
%The fitting problem then becomes, for constant $\vect{\xi}_{d}$,
\begin{eqnarray*}
%\min_{\vect{u} \in \Re} & \vvf{{\cal F}}{u} = \twovft{\vect{f}}{\vect{u}}
%{\vect{\xi}_{d}}\twovf{\vect{f}}{\vect{u}}{\vect{\xi}_{d}}
\end{eqnarray*}

In order to maintain arc-length derivatives a non-linear constraint is needed.
This constraint comes from the geometric properties of arc-length derivatives.
For a node $n$ and $\xi$-direction $i$ the magnitude of the vector of
arc-length derivatives in the various spatial directions must be 1 as detailed
in \eqref{eqn:cubHermnormconst}.  Hence the constraint is
\begin{equation}
%  c=\norm{\brdelby{\vect{x}}{s_{i}}_{n}}=1
  \label{eqn:normconst}
\end{equation}
This also implies simple bounds on the derivative variables:
%\[-1 \leq \brdelby{x_{j}}{s_{i}}_{n} \leq 1\]

Thus the fitting problem can be written in terms of a non-linearly constrained,
non-linear optimisation problem
\begin{eqnarray}
%  \min_{\vect{u} \in \Re,~\vect{\xi}_{d} \mbox{const}} & \vvf{{\cal F}}{u} = 
%  \twovft{\vect{f}}{\vect{u}}{\vect{\xi}_{d}}\twovf{\vect{f}}{\vect{u}}{\vect{\xi}_{d}} 
  \label{eqn:nloproblem} \\
%  \mbox{subject to} & \vect{a} \leq \left(\begin{array}{c}
%  \vect{u} \nonumber \\ \vvvf{c}{u} \end{array} \right) \leq \vect{b} \nonumber
\end{eqnarray}
where \vect{a} is a vector of lower bounds, \vect{b} a vector of upper bounds 
%and \vvvf{c}{u} a vector of non-linear constraints. This type of optimisation 
problem can be solved with readily available non-linear optimisation packages 
such as NPSOL \cite{gill:1986}. 

In order to ensure that there is an approximately uniform spacing of $\xi$
with arc-length two approaches can be used. The first approach is to use
an iterative technique for the scale factors and is detailed here. The
second approach is to include the scale factors in the optimisation problem
and is detailed in the next section.

The constraint on the nodal scale factors (being the average of the line
lengths either side of the node) is placed upon the problem to ensure that the
arc-length to $\xi$ spacing is approximately uniform. As we are only getting
approximately uniform arc-length to $\xi$ spacing we can relax the constraint
on nodal scale factors. If the nodal scale factors are held fixed during the
fit we will not have average arc-length scaling throughout the fitting process
but we will have a reasonable approximation. With the nodal scale factors
fixed the variables ${\cal S}_{n}$ can be removed from the vector of mesh
%parameters \vect{u}.

The approach is to hold the scale factors constant, fit the mesh with
these scale factors, and then update the scale factors (based on the new mesh)
to be average arc-length. This process can be repeated iteratively until the
desired fit has be achieved. It should also be noted that
\eqref{eqn:nloproblem} is defined only for a constant data point projection.
With this iterative approach the data point projections are also updated
at the same time as the scale factors.

The convergence of the fitting problem can be measured in two ways. The first
is the convergence of the RMS error in the data and second is the convergence
in the magnitude of the nodal scale factors.

The algorithm is therefore:
\begin{enumerate}
  \item Define initial mesh (and calculate initial nodal scale factors)
  \item Calculate the initial data point projections
  \item Repeat until converged or the maximum number of iterations is exceeded
  \begin{enumerate}
    \item Fit the mesh to the data by solving \bref{eqn:nloproblem}
    \item Update the scale factors to be average arc-lengths based on the new
      mesh
    \item Recalculate the data point projections on the new mesh
  \end{enumerate}
\end{enumerate}

\subsubsection{Alternative formulation}
\label{sec:alternativeformulation}

An alternative approach to ensuring average arc-length scale factors can be
formulated by including the scale factors in the optimisation problem as
optimisation variables.  The vector of mesh parameters, \vect{u}, is hence
extended to include the nodal scale factors. With this a new constraint
can be introduced to \eqref{eqn:nloproblem} to ensure that there is an
approximately uniform spacing of $\xi$ with arc-length. This can be achieved
if the nodal scale factor is the average on the arc-lengths on either side of
that node. If $\left({\cal S}_{i}\right)_{n}$ is the average arc-length for
node $n$ in the $\xi_{i}$ direction as given by \bref{eqn:avearclenscale} then
the constraint that the nodal scale factor equals the average arc-length is
given by
\begin{equation}
%  c=\frac{\left(s_{i}\right)_{n\ominus}+\left(s_{i}\right)_{n\oplus}}{2}-
%  \left({\cal S}_{i}\right)_{n}=0
\end{equation}
or
\begin{equation} 
%  c=\frac{1}{2}\left(\int_{0}^{1}\norm{\delby{\vvvf{x}{\xi}}{\xi_{i}}}_
%  {n_{\ominus}}d\xi_{i}+\int_{0}^{1}\norm{\delby{\vvvf{x}{\xi}}{\xi_{i}}}_
%  {n_{\oplus}}d\xi_{i}
%  \right)-\left({\cal S}_{i}\right)_{n}=0
  \label{eqn:arclenconst}
\end{equation}
Note that no summation over $\xi_{i}$ is implied.

This also generates a simple bound on the scale-factors:
%\[\left({\cal S}_{i}\right)_{n} > 0\]

This formulation of the non-linear fitting problem does have one practical
limitation. With average arc-length scaling the interpolation within an
element depends on the arc-length on the neighbouring elements, and the
arc-length of the neighbouring elements depends on their neighbouring elements
and so on.  This results in a 'global mesh' in that every part of the mesh
is dependent on every other part of the mesh. The implication of this is that
the entire mesh must be included in the fit otherwise average arc-length
scaling cannot be achieved. This is not a desirable feature for very large
problems or for problems where only a small part of the mesh is in error and
needs to be fitted. The formulation still requires iteration for the data
point projections but does have the advantage that the number of iterations 
required is reduced as the scale factors are found during the fit. This is 
at the expense of having to solve a much larger non-linear optimisation
problem with more variables and, more importantly, more non-linear constraints.

The fit can be considered converged when either the data point projections or
the RMS error in the fit does not change significantly between iterations.
The algorithm for the non-linear data fitting procedure is as follows:

\begin{enumerate}
  \item Define the initial mesh (and calculate the initial nodal scale factors)
  \item Calculate the initial data point projections
  \item Repeat until converged or the maximum number of iterations is exceeded
  \begin{enumerate}
    \item Fit the mesh to the data by solving \bref{eqn:nloproblem} (with
      the new constraints)
    \item Recalculate the data point projections on the new mesh
  \end{enumerate}
\end{enumerate}

\subsubsection{Residual and constraint Jacobians}
\label{sec:resandcontjacs}

Solution of the non-linear problem given by \bref{eqn:nloproblem} will
generally require the evaluation of objective gradient (or residual vector
Jacobian) and the constraint Jacobian with respect to the optimisation (mesh)
parameters. For simplicity in this section we will be concerned with
two-dimensions only (i.e.  one-dimensional cubic Hermite elements). The
residual and constraint Jacobians will be given for the alternative
formulation as this covers all cases. If the scale factors are found by
iteration (i.e. the problem as described in \secref{sec:probform}) the
Jacobians with respect to the nodal scale factors can be ignored, as can the
second constraint (\eqref{eqn:arclenconst}) which ensures average arc-length
scale factors.  In this case there are three basic types of variable within an
element: the nodal variables $\left(x_{j} \right)_{n}$, the derivative
%variables $\brdelby{x_{j}}{s}_{n}$ and the nodal scale factors ${\cal S}_{n}$.
The residual and constraint Jacobians are given below for each of these three
variable types.

%Consider the error vector for the data point $d$, $\vect{e}_{d}$, defined in
\bref{eqn:errorvec}, and its corresponding residual $f_{d}$, defined in
\bref{eqn:dataresidvec}. The position of the projection of data point $d$
within the element is given by \bref{eqn:datapointpos}.  Substituting
\bref{eqn:datapointpos} into \eqnrefs{eqn:errorvec}{eqn:dataresidvec} and
differentiating with respect to the various optimisation variables we can
obtain the Jacobian of the residual vector:
\begin{eqnarray}
%  \delby{f_{d}}{\left(x_{j}\right)_{n}}&=&2\Psi^{0}_{n}(
%  \xi_{d})~{e_{d}}_{j} \label{eqn:dataresjacnode} \\
%  \delby{f_{d}}{\brdelby{x_{j}}{s}_{n}}&=&2\Psi^{1}_{n}(\xi_{d})~{\cal S}_{n}~
%  {e_{d}}_{j} \label{dataresjacderiv} \\
%  \delby{f_{d}}{{\cal S}_{n}}&=&2\Psi^{1}_{n}(\xi_{d})~\brdelby{\vect{x}}
%  {s}_{n}\cdot\vect{e}_{d} \label{eqn:dataresjacline}
\end{eqnarray}

Note that if the data residual was
%\[\twovf{f_{d}}{\vect{u}}{\vect{\xi}_{d}}=\norm{\twovf{\vect{e}_{d}}{\vect{u}}
%{\vect{\xi}_{d}}}\]
\eqref{eqn:dataresjacnode} would become
%\[\delby{f_{d}}{\left(x_{j}\right)_{n}}=\frac{\Psi^{0}_{n}(\xi_{d})~{e_{d}}_{j}}
%{f_{d}}\] which is singular at the optimal solution $f_{d}=0$. To avoid
numerical problems the residual is therefore defined by
\bref{eqn:dataresidvec}.

Now consider the constraint Jacobians. Differentiating constraint
\bref{eqn:normconst} with respect to the mesh parameters gives constraint
gradients:
\begin{eqnarray}
%  \delby{c}{\left(x_{j}\right)_{n}}&=&0 \label{eqn:normjacnode} \\
%  \delby{c}{\brdelby{x_{j}}{s}_{n}}&=&\frac{\brdelby{x_{j}}{s}_{n}}{\norm{
%  \brdelby{\vect{x}}{s}_{n}}}\label{eqn:normjacderiv} \\
%  \delby{c}{{\cal S}_{n}}&=&0 \label{eqn:normjacline}
\end{eqnarray}

To calculate the gradients for constraint \bref{eqn:arclenconst} we first
compute the rate of change of \vect{x} with respect to $\xi$ within an element
by differentiating \bref{eqn:cubHerminterp}:
\begin{equation}
%  \delby{\fnof{\vect{x}}{\xi}}{\xi}=\sum_{n=1,2}\left[\delby{\onevf{\Psi_{n}^{0}}{\xi}}
%  {\xi}\vect{x}_{n}+\delby{\onevf{\Psi_{n}^{1}}{\xi}}{\xi}\brdelby{\vect{x}}
%  {s}_{n}.{\cal S}_{n}\right]
  \label{eqn:delxdelxiinterp}
\end{equation}

Substituting \bref{eqn:delxdelxiinterp} into
\bref{eqn:arclenconst} and differentiating with respect to the optimisation
variables gives the constraint gradients:
\begin{eqnarray}
%  \delby{c}{\left(x_{j}\right)_{n}}&=&\half\left({\int_{0}^{1}\frac{\left(
%  \delby{\onevf{\Psi_{2}^{0}}{\xi}}{\xi}\delby{\onevf{x_{j}}{\xi}}{\xi}
%  \right)_{n_{\ominus}}}{\norm{\delby{\fnof{\vect{x}}{\xi}}{\xi}}_{n_{\ominus}}}d\xi}+
%  \int_{0}^{1}{\frac{\left(\delby{\onevf{\Psi_{1}^{0}}{\xi}}{\xi}\delby{
%  \onevf{x_{j}}{\xi}}{\xi}\right)_{n_{\oplus}}}{\norm{\delby{\fnof{\vect{x}}{\xi}}
%  {\xi}}_{n_{\oplus}}}d\xi}\right) \label{eqn:arclenjacnode} \\
%  \delby{c}{\brdelby{x_{j}}{s}_{n}}&=&\half\left({\int_{0}^{1}\frac{\left(
%  \delby{\onevf{\Psi_{2}^{1}}{\xi}}{\xi}{\cal S}_{n}\delby{\onevf{x_{j}}
%  {\xi}}{\xi}\right)_{n_{\ominus}}}{\norm{\delby{\fnof{\vect{x}}{\xi}}{\xi}}_{n_{
%  \ominus}}}d\xi}+\int_{0}^{1}{\frac{\left(\delby{\onevf{\Psi_{1}^{1}}{\xi}}
%  {\xi}{\cal S}_{n}\delby{\onevf{x_{j}}{\xi}}{\xi}\right)_{n_{\oplus}}}
%  {\norm{\delby{\fnof{\vect{x}}{\xi}}{\xi}}_{n_{\oplus}}}d\xi}\right) 
%  \label{eqn:arclenjacderiv} \\
%  \delby{c}{{\cal S}_{n}}&=&\half\left(\int_{0}^{1}{\frac{\left(\delby{
%  \onevf{\Psi_{2}^{1}}{\xi}}{\xi}\brdelby{\vect{x}}{s}_{n}\cdot\delby{
%  \fnof{\vect{x}}{\xi}}{\xi}\right)_{n_{\ominus}}}{\norm{\delby{\fnof{\vect{x}}{\xi}}
%  {\xi}}_{n_{\ominus}}}d\xi}\right. + \nonumber \\
%  &&\hspace{3cm}\left.\int_{0}^{1}{\frac{\left(\delby{\onevf{\Psi_{1}^{1}}
%  {\xi}}{\xi}\brdelby{\vect{x}}{s}_{n}\cdot\delby{\fnof{\vect{x}}{\xi}}{\xi}\right)_{
%  n_{\oplus}}}{\norm{\delby{\fnof{\vect{x}}{\xi}}{\xi}}_{n_{\oplus}}}d\xi}\right) 
%  - 1 
   \label{eqn:arclenjacline}
\end{eqnarray}

Note that when the integrand in these formula contains $n_{\ominus}$
($n_{\oplus}$) the $\xi$ variable is assumed to be taken over the previous
(next) element.

Similar results can be found for three-dimensions (i.e. bicubic Hermite
elements).

\subsubsection{Sobelov smoothing}

In order to implement Sobelov smoothing in the form of the non-linear
optimisation problem given in \bref{eqn:nloproblem} we need to modify the
residual vector by defining an additional residual as the Sobelov value for
the mesh being fitted, that is
\begin{equation}
%  \twovf{f_{D+1}}{\vect{u}}{\vect{\xi}}=\onevf{G}{\vvvf{u}{\xi}}
  \label{eqn:addSobresidual}
\end{equation}
%where \onevf{G}{\vvvf{u}{\xi}} is defined in \bref{eqn:ptwoSobnorm}.
The objective function to minimise now becomes
%\[\vvf{F}{u}=\vvf{{\cal F}}{u}+\vvf{G^{2}}{u}=\vvf{{\cal F}}{u}+
%\vvf{{\cal G}}{u}\]
%where \vvf{{\cal F}}{u} is the data objective defined in \bref{eqn:dataresfun}
%and $\vvf{{\cal G}}{u}=\vvf{G^{2}}{u}$ is the Sobelov objective.

The gradients of this additional residual can then be found by differentiating
\bref{eqn:ptwoSobnorm} with respect to the mesh parameters. Considering the
case of two-dimensional fitting and breaking the domain up into elements we
obtain
%\[\vvf{G}{u}=\sum_{e}\int_{0}^{1}\left\{\alpha\normdelby{\fnof{\vect{x}}{\xi}}
%{\xi}^{2}+\beta\norm{\frac{\del^{2}\fnof{\vect{x}}{\xi}}{\del \xi^{2}}}^{2}
%\right\}d\xi\]
Interpolating within the element using cubic Hermite elements and 
differentiating with respect to the three different mesh parameters within
an element we obtain the additional residual gradients given in equations 
(\ref{eqn:sobjacnode}--\ref{eqn:sobjacline}).
\begin{eqnarray}
%  \delby{f_{D+1}}{\left(x_{j}\right)_{n}}&=&2\int_{0}^{1}\left\{\alpha
%  \delby{\onevf{\Psi_{n}^{0}}{\xi}}{\xi}\delby{\onevf{x_{j}}{\xi}}{\xi}+
%  \beta\frac{\del^{2}\onevf{\Psi_{n}^{0}}{\xi}}{\del\xi^{2}}\frac{\del^{2}
%  \onevf{x_{j}}{\xi}}{\del\xi^{2}}\right\}d\xi \label{eqn:sobjacnode} \\
%  \delby{f_{D+1}}{\brdelby{x_{j}}{s}_{n}}&=&2\int_{0}^{1}\left\{\alpha
%  \delby{\onevf{\Psi_{n}^{1}}{\xi}}{\xi}{\cal S}_{n}\delby{\onevf{x_{j}}
%  {\xi}}{\xi}+\beta\frac{\del^{2}\onevf{\Psi_{n}^{1}}{\xi}}{\del\xi^{2}}
%  {\cal S}_{n}\frac{\del^{2}\onevf{x_{j}}{\xi}}{\del\xi^{2}}\right\}d\xi 
%  \label{eqn:sobjacderiv} \\
%  \delby{f_{D+1}}{{\cal S}_{n}}&=&2\int_{0}^{1}\left\{\alpha\delby{
%  \onevf{\Psi_{n}^{1}}{\xi}}{\xi}\brdelby{\vect{x}}{s}_{n}\cdot\delby{\fnof{\vect{x}}
%  {\xi}}{\xi}+\beta
%  \frac{\del^{2}\onevf{\Psi_{n}^{1}}{\xi}}{\del\xi^{2}}\brdelby{\vect{x}}{s}_{n}
%  \cdot\frac{\del^{2}\fnof{\vect{x}}{\xi}}{\del\xi^{2}}\right\}d\xi \label{eqn:sobjacline}
\end{eqnarray}

Note that all the above integrals are with respect to one element. All the
contributions from each element need to be included to obtain the complete
residual gradients for the optimisation parameters.

%%% Local Variables: 
%%% mode: latex
%%% TeX-master: "/product/cmiss/documents/notes/fembemnotes/fembemnotes"
%%% compile-command: "cd ..; latex '\\nonstopmode\\documentclass[12pt,twoside,a4paper]{book}

%\includeonly{titlepage}
%\includeonly{fem_basis_fns/fem_basis_fns}
%\includeonly{heat_conduction/heat_conduction}
%\includeonly{bem/bem}
%\includeonly{transient_heat_condn/transient_heat_condn}
%\includeonly{lin_elasticity/lin_elasticity}
%\includeonly{modal_analysis/modal_analysis}
%\includeonly{con_mechanics/con_mechanics}

%\includeonly{domints_in_bem/domints_in_bem}
%\includeonly{timedep_bem/timedep_bem}
%\includeonly{datafitting/datafitting}


%%% the first three chapters
%\includeonly{titlepage,fem_basis_fns/fem_basis_fns,heat_conduction/heat_conduction,bem/bem}
%%% chapters 4-6
%\includeonly{titlepage,transient_heat_condn/transient_heat_condn,lin_elasticity/lin_elasticity,modal_analysis/modal_analysis}
%%% chapters 7-9
%\includeonly{titlepage,con_mechanics/con_mechanics,domints_in_bem/domints_in_bem,timedep_bem/timedep_bem}

\input{../latex/macros} %define new commands etc.
\usepackage{epic}  %package for epic 
\usepackage{eepic} %package for extended epic
\def\tenrm{}
\usepackage{rotating}
\usepackage{subfigure}
\usepackage{ifthen}
\usepackage{epsfig}
\usepackage{graphicx}
\usepackage{harvard}
\usepackage{color}
\usepackage{multirow}
\usepackage{makeidx}        % makes index

\gdef\SetFigFont#1#2#3{#3}
\gdef\SetFigFontNFSS#1#2#3{#3}

% line spacing definitions
\usepackage{setspace}       % Line spacing package
\newcommand{\singlespc}{\setstretch{1.00}}     % Single spacing
\newcommand{\oneandhalfspc}{\setstretch{1.24}} % One and a half spacing
\newcommand{\doublespc}{\setstretch{1.66}}     % Double spacing
\oneandhalfspc

% Page style definitions
%\setlength{\parskip}{\baselineskip} % these two lines for
%\setlength{\parindent}{0mm}         % no paragraph indents
\usepackage{vmargin}
\setpapersize{A4}
\setmargrb{25mm}{35mm}{25mm}{35mm}
\addtolength{\headheight}{6pt}

% Local Variables: 
% mode: latex
% TeX-master: t
% End: 
 %define packages etc.
\input{../latex/abbreviations} %define packages etc.

\usepackage{../latex/mybook} %define book style
%\usepackage{../latex/lofexamples} %list of examples style

\usepackage{times} %fonts

%\usepackage{mathptm}

\renewcommand{\baselinestretch}{1.24}

% Note: single spacing used for student notes
\singlespc
\raggedbottom

% NOTE: many of the \todo and \remarks have been commented out
% so the current version can tbe printed out for teaching these
% should still be looked at NPS 22/2/99 

\normalsize
\makeindex

\title{The FEM-BEM-notes}
\author{}              % Author's name
\date{\today}          % Revision Date


\begin{document}

\thispagestyle{empty}

\begin{center}
   \huge FEM/BEM NOTES
   \vspace{10mm}   

   \large http://www.cmiss.org/

   \vspace{40mm}   
%   \vspace{30mm}   
%   \large   
%      Professor Peter Hunter\\   
%   \normalsize
%      p.hunter@auckland.ac.nz\\
%   \vspace{10mm}
%   \large
%      Associate Professor Andrew Pullan\\
%   \normalsize
%      a.pullan@auckland.ac.nz\\

   \large
   \vspace{20mm}
      \includegraphics[height=45mm]{epsfiles/crest.eps}\\
   \vspace{15mm}      
      Bioengineering Institute \\ 
      The University of Auckland \\ 
      New Zealand \\

   \vspace{5mm}
   \today\\   % today's date
   \vspace{20mm}
   \small
   \copyright \thickspace Copyright 1997-\\
   Bioengineering Institute\\
   The University of Auckland
\end{center}





%%% Local Variables: 
%%% mode: latex
%%% TeX-master: t
%%% End: 

\clearemptydoublepage 

\pagenumbering{roman}
\tableofcontents

%\clearemptydoublepage
%\listofexamples
\clearemptydoublepage 
\pagenumbering{arabic}

\chapter{Finite Element Basis Functions}

\section{Representing a One-Dimensional Field}

Consider the problem of finding a mathematical expression $\fnof{u}{x}$ to
represent a one-dimensional field \eg measurements of temperature $u$
against distance $x$ along a bar, as shown in \figref{fig:1Dfield}a.

\begin{figure}[htbp] \centering
  \input{figs/fem_basis_fns/1Dfield.pstex}
  \caption{(a) Temperature distribution $\fnof{u}{x}$ along a bar. 
    The points are the measured temperatures. (b) A least-squares polynomial
    fit to the data, showing the unacceptable oscillation between data points.}
  \label{fig:1Dfield}
\end{figure} 

One approach would be to use a polynomial expression $\fnof{u}{x} = a + bx +
cx^{2} + dx^{3} + \ldots$ and to estimate the values of the parameters
$a$, $b$, $c$ and $d$ from a least-squares fit to the data.  As the degree of the
polynomial is increased the data points are fitted with increasing accuracy
and polynomials provide a very convenient form of expression because they can
be differentiated and integrated readily. For low degree polynomials this is a
satisfactory approach, but if the polynomial order is increased further to
improve the accuracy of fit a problem arises: the polynomial can be made to
fit the data accurately, but it oscillates unacceptably between the data
points, as shown in \figref{fig:1Dfield}b.  

To circumvent this, while retaining the advantages of low degree polynomials,
we divide the bar into three subregions and use low order polynomials over
each subregion - called \emph{elements}. For later generality we also
introduce a parameter $s$ which is a measure of distance along the bar. $u$ is
plotted as a function of this arclength in \figref{fig:Tempmeas}a.
\figref{fig:Tempmeas}b shows three linear polynomials in $s$ fitted by
least-squares separately to the data in each element. 
%\todo{ref data fitting chapter}
%(we return to least-squares data fitting in \Chapref{cha:datafitting}).

\begin{figure}[htbp] \centering
  \input{figs/fem_basis_fns/Tempmeas.pstex}
  \caption{(a) Temperature measurements replotted against arclength 
    parameter $s$. (b) The $s$ domain is divided into three subdomains,
    \emph{elements}, and linear polynomials are independently fitted to the data
    in each subdomain.}
  \label{fig:Tempmeas}
\end{figure}

\section{Linear Basis Functions}

\index{Basis functions|(}\index{Basis functions!Lagrange!linear|(}A new
problem has now arisen in \figref{fig:Tempmeas}b: the piecewise linear
polynomials are not continuous in $u$ across the boundaries between elements.
One solution would be to constrain the parameters $a$, $b$, $c$ \etc to 
ensure continuity of $u$ across the element boundaries, but a better solution 
is to replace the parameters $a$ and $b$ in the first element with parameters
$u_{1}$ and $u_{2}$, which are the values of $u$ at the two ends of that
element. We then define a linear variation between these two values by
\begin{equation*}
  \fnof{u}{\xi} = \pbrac{1-\xi}u_{1} + \xi u_{2}
\end{equation*}
where $\xi (0 \leq \xi \leq 1)$ is a normalized measure of distance along the
curve.

We define 
\begin{align*}
    \lbfn{1}{\xi} &= 1-\xi  \\ 
    \lbfn{2}{\xi} &= \xi
\end{align*}
such that 
\begin{equation*}
  \fnof{u}{\xi} = \lbfn{1}{\xi} u_{1} + \lbfn{2}{\xi} u_{2}            
\end{equation*}
and refer to these expressions as the \emph{basis} functions associated with
the \emph{nodal} parameters $u_{1}$ and $u_{2}$. The basis functions
$\lbfn{1}{\xi}$ and $\lbfn{2}{\xi}$ are straight lines varying between $0$ and $1$
as shown in \figref{fig:Linearbf}.

\begin{figure}[htbp] \centering
  \input{figs/fem_basis_fns/Linearbf.pstex}
  \caption{Linear basis functions $\lbfn{1}{\xi}=1-\xi$ and 
    $\lbfn{2}{\xi}=\xi$.}
  \label{fig:Linearbf}
\end{figure}
   
It is convenient always to associate the nodal quantity $u_{n}$ with
\emph{element node} $n$ and to map the temperature $U_{\Delta}$ defined at
\emph{global node} $\Delta$ onto local node $n$ of element $e$ by using a
connectivity matrix $\fnof{\Delta}{n,e}$ \ie
\begin{equation*}
  u_{n} = U_{\fnof{\Delta}{n,e}}
\end{equation*}
where $\fnof{\Delta}{n,e}$ = global node number of local node $n$ of element $e$.
This has the advantage that the interpolation 
\begin{equation*}
  \fnof{u}{\xi} = \lbfn{1}{\xi} u_{1} + \lbfn{2}{\xi} u_{2}
\end{equation*}
holds for any element provided that $u_{1}$ and $u_{2}$ are correctly identified
with their global counterparts, as shown in \figref{fig:Relationship}.
\begin{figure}[htbp] \centering
  \input{figs/fem_basis_fns/Relationship.pstex}
  \caption{The relationship between global nodes and element nodes.}
  \label{fig:Relationship}
\end{figure}
Thus, in the first element
\begin{equation}
  \fnof{u}{\xi} = \lbfn{1}{\xi}u_{1} + \lbfn{2}{\xi}u_{2}
  \label{eqn:1stelement}
\end{equation}
with $u_{1}=U_{1}$ and $u_{2}=U_{2}$.

In the second element $u$ is interpolated by
\begin{equation}
  \fnof{u}{\xi}  =  \lbfn{1}{\xi}u_{1} + \lbfn{2}{\xi}u_{2}
  \label{eqn:2ndelement}
\end{equation}
with $u_{1}=U_{2}$ and $u_{2}=U_{3}$, since the parameter $U_{2}$ is shared
between the first and second elements the temperature field $u$ is implicitly
continuous. Similarly, in the third element $u$ is interpolated by
\begin{equation}
  \fnof{u}{\xi} = \lbfn{1}{\xi}u_{1} + \lbfn{2}{\xi}u_{2}
  \label{eqn:3rdelement}
\end{equation}
with $u_{1}=U_{3}$ and $u_{2}=U_{4}$, with the parameter $U_{3}$ being shared
between the second and third elements. \Figref{fig:Weighting} shows the
temperature field defined by the three interpolations
\bref{eqn:1stelement}--\bref{eqn:3rdelement}. \index{Basis functions!Lagrange!linear|)}

\begin{figure}[htbp] \centering
  \input{figs/fem_basis_fns/Tmfit.pstex}
  \caption{Temperature measurements fitted with nodal parameters and linear 
    basis functions.  The fitted temperature field is now continuous across
    element boundaries.}
  \label{fig:Tmfit}
\end{figure}

\section{Basis Functions as Weighting Functions}

It is useful to think of the basis functions as weighting functions on the
nodal parameters.  Thus, in element 1
\begin{equation*}
  \text{at } \xi=0 \qquad \fnof{u}{0} = \pbrac{1-0}u_{1} + 0u_{2} = u_{1}
\end{equation*}
which is the value of $u$ at the left hand end of the element and has no
dependence on $u_{2}$
\begin{equation*}
  \text{at } \xi=\frac14 \qquad   \fnof{u}{\frac14} = \pbrac{1-\frac14} u_{1} +
  \frac14 u_{2} = \frac34 u_{1}+\frac14 u_{2}
\end{equation*}
which depends on $u_{1}$ and $u_{2}$, but is weighted more towards $u_{1}$
than $u_{2}$
\begin{equation*}
  \text{at } \xi=\frac12 \qquad \fnof{u}{\frac12} = \pbrac{1-\frac12} u_{1}+
  \frac12 u_{2} = \frac12 u_{1}+\frac12 u_{2}
\end{equation*}
which depends equally on $u_{1}$ and $ u_{2}$
\begin{equation*}
  \text{at } \xi=\frac34   \qquad \fnof{u}{\frac34} = \pbrac{1-\frac34} u_{1}
  + \frac34 u_{2} = \frac14 u_{1}+\frac34 u_{2}
\end{equation*}
which depends on $u_{1}$ and $u_{2}$ but is weighted more towards $u_{2}$
than $u_{1}$
\begin{equation*}
    \text{at } \xi=1   \qquad \fnof{u}{1} = \pbrac{1- 1}u_{1} +{1}u_{2} = u_{2}
\end{equation*}
which is the value of $u$ at the right hand end of the region and has no
dependence on $u_{1}$.

Moreover, these weighting functions can be considered as \emph{global} 
functions, as shown in \figref{fig:Weighting}, where the weighting function
$w_{n}$ associated with global node $n$ is constructed from the basis functions 
in the elements adjacent to that node.

\begin{figure}[htbp] \centering
  \input{figs/fem_basis_fns/Weighting.pstex}
  \caption{(a) $\ldots$ (d) The weighting functions $w_{n}$ associated with 
    the global nodes $n=1\ldots4$, respectively. Notice the linear fall off 
    in the elements adjacent to a node.  Outside the immediately adjacent 
    elements, the weighting functions are defined to be zero.}
  \label{fig:Weighting}
\end{figure}

For example, $w_{2}$ weights the global parameter $U_{2}$ and the influence of
$U_{2}$ falls off linearly in the elements on either side of node 2.

We now have a continuous piecewise parametric description of the temperature
field $\fnof{u}{\xi}$ but in order to define $\fnof{u}{x}$ we need to define
the relationship between $x$ and $\xi$ for each element. A convenient way to
do this is to define $x$ as an interpolation of the nodal values of $x$.

For example, in element 1
\begin{equation}
  \fnof{x}{\xi} = \lbfn{1}{\xi}x_{1} + \lbfn{2}{\xi}x_{2}
\end{equation}
and similarly for the other two elements. The dependence of temperature on $x$,
$\fnof{u}{x}$, is therefore defined by the parametric expressions
\begin{align*}
  \fnof{u}{\xi} &= \dsuml{n}{} \lbfn{n}{\xi} u_{n} \\
  x(\xi) &= \dsuml{n}{} \lbfn{n}{\xi} x_{n}
\end{align*}
where summation is taken over all element nodes (in this case only $2$) and
the parameter $\xi$ (the ``element coordinate'') links temperature $u$ to
physical position $x$. $\fnof{x}{\xi}$ provides the mapping between the
mathematical space $0 \leq \xi \leq 1$ and the physical space $x_{1} \leq x \leq
x_{2}$, as illustrated in \figref{fig:xandurel}.

\begin{figure}[htbp] \centering
  \input{figs/fem_basis_fns/xandurel.pstex}
  \caption{Illustrating how $x$ and $u$ are related through the normalized 
    element coordinate $\xi$. The values of $\fnof{x}{\xi}$ and $\fnof{u}{\xi}$ are 
    obtained from a linear interpolation of the nodal variables and then 
    plotted as $\fnof{u}{x}$. The points at $\xi=0.2$ are emphasized.}
  \label{fig:xandurel}
\end{figure}

\section{Quadratic Basis Functions}

\index{Basis functions!Lagrange!quadratic|(}The essential property of the 
basis functions defined above is that the basis function associated with a 
particular node takes the value of $1$ when evaluated at that node and is zero at
every other node in the element (only one other in the case of linear basis 
functions). This ensures the linear independence of the basis functions. 
It is also the key to establishing the form of the basis functions for higher order
interpolation.  For example, a quadratic variation of $u$ over an element
requires three nodal parameters $u_{1}$, $u_{2}$ and $u_{3}$
\begin{equation}
  \fnof{u}{\xi} = \lbfn{1}{\xi}u_{1} + \lbfn{2}{\xi}u_{2} + \lbfn{3}{\xi}u_{3}  
  \label{eqn:Qbf}
\end{equation}
The quadratic basis functions are shown, with their mathematical expressions,
in \figref{fig:1DQbf}. Notice that since $\lbfn{1}{\xi}$ must be zero at
$\xi=0.5$ (node $2$), $\lbfn{1}{\xi}$ must have a factor $\pbrac{\xi-0.5}$
and since it is also zero at $\xi=1$ (node $3$), another factor is
$\pbrac{\xi-1}$.  Finally, since $\lbfn{1}{\xi}$ is $1$ at $\xi=0$ (node $1$)
we have $\lbfn{1}{\xi} = 2\pbrac{\xi-1}\pbrac{\xi-0.5}$.  Similarly for the
other two basis functions.

\begin{figure}[htbp] \centering
  \input{figs/fem_basis_fns/1DQbf.pstex}
  \caption{One-dimensional quadratic basis functions.}
  \label{fig:1DQbf}
\end{figure}
\index{Basis functions!Lagrange!quadratic|)}

\section{Two- and Three-Dimensional Elements}

\index{Basis functions!Lagrange!bilinear|(}Two-dimensional bilinear 
basis functions are constructed from the products of the above 
one-dimensional linear functions as follows

Let   
\begin{equation*}
  \fnof{u}{\xione,\xitwo} = \lbfn{1}{\xione,\xitwo}u_{1} +
  \lbfn{2}{\xione,\xitwo}u_{2} +
  \lbfn{3}{\xione,\xitwo}u_{3} + \lbfn{4}{\xione,\xitwo}u_{4}
\end{equation*}
where  
\begin{equation}
\begin{split}
    \lbfn{1}{\xione,\xitwo} &= \pbrac{1- \xione}\pbrac{1- \xitwo} \\
    \lbfn{2}{\xione,\xitwo} &= \xione\pbrac{1- \xitwo} \\ 
    \lbfn{3}{\xione,\xitwo} &= \pbrac{1- \xione}\xitwo \\
    \lbfn{4}{\xione,\xitwo} &= \xione\xitwo
  \end{split}
  \label{eqn:2,3DE}
\end{equation}

Note that
\lbfn{1}{\xione,\xitwo} = \lbfn{1}{\xione}\lbfn{1}{\xitwo} 
where \lbfn{1}{\xione} and \lbfn{1}{\xitwo} are the one-dimensional linear basis
functions. Similarly, \lbfn{2}{\xione,\xitwo} =
\lbfn{2}{\xione}\lbfn{1}{\xitwo} \ldots \etc

These four bilinear basis functions are illustrated in \figref{fig:2Dbbf}.
\begin{figure}[htbp] \centering
  \input{figs/fem_basis_fns/2Dbbf.pstex}
  \caption{Two-dimensional bilinear basis functions.}
  \label{fig:2Dbbf}
\end{figure}

Notice that $\lbfn{n}{\xione,\xitwo}$ is $1$ at node $n$ and zero at the
other three nodes.  This ensures that the temperature
$\fnof{u}{\xione,\xitwo}$ receives a contribution from each nodal parameter
$u_{n}$ weighted by $\lbfn{n}{\xione,\xitwo}$ and that when
$\fnof{u}{\xione,\xitwo}$ is evaluated at node $n$ it takes on the value
$u_{n}$.

As before the geometry of the element is defined in terms of the node
positions $\pbrac{x_{n},y_{n}}$, $n=1,\ldots,4$ by
\begin{align*}
  x &= \dsuml{n}{} \lbfn{n}{\xione,\xitwo} x_{n} \\ 
  y &= \dsuml{n}{} \lbfn{n}{\xione,\xitwo} y_{n}
\end{align*}
which provide the mapping between the mathematical space $\pbrac{\xione, \xitwo}$
(where $0 \leq \xione,  \xitwo \leq 1$) and the physical space $\pbrac{x,y}$.

%\begin{example}{Defining a 2D bilinear finite element mesh}
%  {Defining a 2D bilinear finite element mesh}

%  To define a 2D bilinear finite element mesh run the CMISS example number
%  $111$.  The nodes should be position as shown in \figref{fig:example111_node}.
%  After defining elements the mesh should appear like the one shown in
%  \figref{fig:example111_elem}

%  \begin{figure}[htbp] \centering
%    \input{figs/fem_basis_fns/egdiag1.pstex}
%    \caption{Node positions for example $111$.}
%    \label{fig:example111_node}
%  \end{figure}
%   \begin{figure}[hbtp] \centering
%    \input{figs/fem_basis_fns/egdiag2.pstex}
%    \caption{2D bilinear finite element mesh for example $111$.}
%    \label{fig:example111_elem}
%  \end{figure} 
%  \label{xmp:Def2D}
%\end{example}

%\begin{example}{Refining a mesh}
%  {Refining a mesh} 
%  To refine a mesh run the CMISS example $113$. After the first refine the
%  mesh should appear like the one shown in \figref{fig:example113}.  
%\begin{figure}[htbp] \centering
%    \input{figs/fem_basis_fns/egdiag3.pstex}
%    \caption{Refined mesh.}
%    \label{fig:example113}
%  \end{figure}
%  \label{xmp:Refining}
%\end{example}

\index{Basis functions!Lagrange!bilinear} 

Higher order 2D basis functions can be similarly constructed from products
of the appropriate 1D basis functions. For example, a six-noded
(see \figref{fig:6node}) quadratic-linear element (quadratic in $\xione$ and
linear in $\xitwo$) would have 
\begin{equation*} u=\dsuml{n=1}{6}
  \lbfn{n}{\xione,\xitwo}u_{n}
\end{equation*}
where
  \begin{alignat}{2}
    \lbfn{1}{\xione,\xitwo} &=
    2\pbrac{\xione-1}\pbrac{\xione-0.5}\pbrac{1-\xitwo} &\qquad
    \lbfn{2}{\xione,\xitwo} &= 4\xione\pbrac{1-\xione}\pbrac{1-\xitwo} \\
    \lbfn{3}{\xione,\xitwo} &=
    2\xione\pbrac{\xione-0.5}\pbrac{1-\xitwo} &\qquad
    \lbfn{4}{\xione,\xitwo} &=
    2\pbrac{\xione-1}\pbrac{\xione-0.5}\xitwo \\ 
    \lbfn{5}{\xione,\xitwo} &= 4\xione\pbrac{1-\xione}\xitwo &\qquad
    \lbfn{6}{\xione,\xitwo} &= 2\xione\pbrac{\xione-0.5}\xitwo
  \label{eqn:QLE}
  \end{alignat}



\begin{figure}[htbp] \centering
  \input{figs/fem_basis_fns/6node.pstex}
  \caption{A $6$-node quadratic-linear element (node numbers circled).}
  \label{fig:6node}
\end{figure} 

Three-dimensional basis functions are formed similarly, \eg a trilinear
element basis has eight nodes (see \figref{fig:8nte}) with basis functions
\begin{alignat}{2}
        \lbfn{1}{\xione,\xitwo,\xithree} &=
        \pbrac{1-\xione}\pbrac{1-\xitwo}\pbrac{1-\xithree} 
        &\qquad \lbfn{2}{\xione,\xitwo,\xithree}
        &=\xione\pbrac{1-\xitwo}\pbrac{1-\xithree} \\
        \lbfn{3}{\xione,\xitwo,\xithree} &=\pbrac{1-\xione}\xitwo\pbrac{1-\xithree} 
        &\qquad \lbfn{4}{\xione,\xitwo,\xithree} &=\xione\xitwo\pbrac{1-\xithree} \\
        \lbfn{5}{\xione,\xitwo,\xithree} &=\pbrac{1-\xione}\pbrac{1-\xitwo}\xithree 
        &\qquad \lbfn{6}{\xione,\xitwo,\xithree} &=\xione\pbrac{1-\xitwo}\xithree \\
        \lbfn{7}{\xione,\xitwo,\xithree} &=\pbrac{1-\xione}\xitwo\xithree          
        &\lbfn{8}{\xione,\xitwo,\xithree} &=\xione\xitwo\xithree
        \label{eqn:trilinear}
\end{alignat}


\begin{figure}[htbp] \centering
  \input{figs/fem_basis_fns/8nte.pstex}
  \caption{An $8$-node trilinear element.}
  \label{fig:8nte}
\end{figure}

%\begin{example}{Defining a quadratic-linear element mesh}
%  {Defining a quadratic-linear element mesh.} 
%  To define a quadratic-linear element run the cmiss example $115$.
%  \label{xmp:Defining_quadlin}
%\end{example}

%\begin{example}{Defining a 3D trilinear element mesh}
%  {Defining a 3D trilinear element mesh} 
%  To define a 3D trilinear element run CMISS example $121$.
%  \label{xmp:Def3D}
%\end{example}

\section{Higher Order Continuity}
\index{Basis functions!Hermite|(}\index{Basis functions!Hermite!cubic|(}
All the basis functions mentioned so far are \emph{Lagrange}\footnote{Joseph-Louis
Lagrange (1736-1813).} basis functions\index{Basis functions!Lagrange} and provide continuity of $u$ across element boundaries 
but not higher order continuity.  Sometimes it is desirable to use basis
functions which also preserve continuity of the derivative of $u$ with respect
to $\xi$ across element boundaries. A convenient way to achieve this is by
defining two additional nodal parameters $\pbrac{\dby{u}{\xi}}_{n}$.  The basis
functions are chosen to ensure that 
\begin{equation*}
 \evalat{\dby{u}{\xi}}{\xi=0} =\pbrac{\dby{u}{\xi}}_{1} = u'_{1} \text{ and }
 \evalat{\dby{u}{\xi}}{\xi=1} =\pbrac{\dby{u}{\xi}}_{2} = u'_{2}
\end{equation*}
and since $u_n$ is shared between adjacent elements derivative continuity is 
ensured.  Since the number of element parameters is 4 the basis functions must
be cubic in $\xi$.  To derive these
cubic \emph{Hermite}\footnote{Charles Hermite (1822-1901).} basis functions
let
\begin{align*}
  \fnof{u}{\xi} &= a + b\xi + c\xi^2 + d\xi^3 , \\
  \dby{u}{\xi} &= b + 2c\xi + 3d\xi^2 ,
\end{align*}
and impose the constraints    
\begin{alignat*}{2}
  \fnof{u}{0} &= a \quad &&= u_{1} \\ 
  \fnof{u}{1} &= a + b + c + d \quad &&= u_{2} \\
  \fnof{\dby{u}{\xi}}{0} &= b \quad &&= u_{1}^{'}\\
  \fnof{\dby{u}{\xi}}{1} &= b + 2c + 3d \quad &&= u_{2}^{'}
\end{alignat*}
These four equations in the four unknowns $a$, $b$, $c$ and $d$ are solved to
give
\begin{align*}
    a &= u_{1} \\   
    b &= u'_{1} \\
    c &= 3u_{2} - 3u_{1} - 2 u'_{1} -  u'_{2}   \\
    d &= u'_{1} + u'_{2}  + 2u_{1}  - 2u_{2}
\end{align*}
Substituting $a$, $b$, $c$ and $d$ back into the original cubic then gives 
\begin{equation*}
  \fnof{u}{\xi} =u_{1} + u'_{1}\xi + \pbrac{3u_{2} - 3u_{1} - 2u'_{1} -
    u'_{2}}\xi^2 + \pbrac{u'_{1} + u'_{2}  + 2u_{1} - 2u_{2}}\xi^3 
\end{equation*}
or, rearranging,
\begin{equation}
  \fnof{u}{\xi} =\chbfn{1}{0}{\xi}u_{1} + \chbfn{1}{1}{\xi}u'_{1} +
  \chbfn{2}{0}{\xi}u_{2} + \chbfn{2}{1}{\xi}u'_{2} 
  \label{eqn:Hoc}
\end{equation}
where the four cubic Hermite basis functions are drawn in \figref{fig:cubic}.

\begin{figure}[htbp] \centering
  \input{figs/fem_basis_fns/cubic.pstex}
  \caption{Cubic Hermite basis functions.}
  \label{fig:cubic}
\end{figure}

One further step is required to make cubic Hermite basis functions useful in
practice. The derivative $\pbrac{\dby{u}{\xi}}_{n}$ defined at node $n$
is dependent upon the element $\xi$-coordinate in the two adjacent elements.
It is much more useful to define a global node derivative
 $\pbrac{\dby{u}{s}}_{n}$ where $s$ is arclength and then use
\begin{equation}
  \pbrac{\dby{u}{\xi}}_{n}=\pbrac{\dby{u}{s}}_{\fnof{\Delta}{n,e}} \cdot
  \pbrac{\dby{s}{\xi}}_{n} 
  \label{eqn:cubichbf}
\end{equation}
where $\pbrac{\dby{s}{\xi}}_{n}$ is an element \emph{scale factor} which scales
the arclength derivative of global node $\Delta$ to the $\xi$-coordinate
derivative of element node $n$. Thus $\dby{u}{s}$ is constrained to be
continuous across element boundaries rather than $\dby{u}{\xi}$.\index{Basis
  functions!Hermite!cubic|)} \index{Basis functions!Hermite!bicubic|)}A two-
dimensional bicubic Hermite basis requires four derivatives per node
\begin{equation*}
  u, \delby{u}{\xione}, \delby{u}{\xitwo} \text{ and }
  \deltwoby{u}{\xione}{\xitwo}
\end{equation*}
The need for the second-order cross-derivative term can be explained as
follows; If $u$ is cubic in $\xione$ and cubic in $\xitwo$, then
$\delby{u}{\xione}$ is quadratic in $\xione$ and cubic in $\xitwo$ , and
$\delby{u}{\xitwo}$ is cubic in $\xione$ and quadratic in $\xitwo$ . Now
consider the side 1--3 in \figref{fig:interpol}. The cubic variation of $u$
with $\xitwo$ is specified by the four nodal parameters $u_{1}$,
$\pbrac{\delby{u}{\xitwo}}_{1}$, $u_{3}$ and $\pbrac{\delby{u}{\xitwo}}_{3}$.
But since $\delby{u}{\xione}$ (the normal derivative) is also cubic in
$\xitwo$ along that side and is entirely independent of these four parameters,
four additional parameters are required to specify this cubic. Two of these
are specified by $\pbrac{\delby{u}{\xione}}_{1}$ and
$\pbrac{\delby{u}{\xione}}_{3}$, and the remaining two by
$\pbrac{\deltwoby{u}{\xione}{\xitwo}}_{1}$ and
$\pbrac{\deltwoby{u}{\xione}{\xitwo}}_{3}$.

\begin{figure}[htbp] \centering
  \input{figs/fem_basis_fns/interpol.pstex}
  \caption{Interpolation of nodal derivative $\delby{u}{\xione}$ along side 
    1--3.}
  \label{fig:interpol}
\end{figure}
The bicubic interpolation of these nodal parameters is given by
\begin{gather}
  \begin{aligned}
    \fnof{u}{\xione,\xitwo} &= \chbfn{1}{0}{\xione}\chbfn{1}{0}{\xitwo}u_{1} +
    \chbfn{2}{0}{\xione}\chbfn{1}{0}{\xitwo}u_{2} \\ 
    &+ \chbfn{1}{0}{\xione}\chbfn{2}{0}{\xitwo}u_{3} +
    \chbfn{2}{0}{\xione}\chbfn{2}{0}{\xitwo}u_{4} \\ &+
    \chbfn{1}{1}{\xione}\chbfn{1}{0}{\xitwo}\pbrac{\delby{u}{\xione}}_{1} +
    \chbfn{2}{1}{\xione}\chbfn{1}{0}{\xitwo}\pbrac{\delby{u}{\xione}}_{2} \\ &+
    \chbfn{1}{1}{\xione}\chbfn{2}{0}{\xitwo}\pbrac{\delby{u}{\xione}}_{3} +
    \chbfn{2}{1}{\xione}\chbfn{2}{0}{\xitwo}\pbrac{\delby{u}{\xione}}_{4} \\ &+
    \chbfn{1}{0}{\xione}\chbfn{1}{1}{\xitwo}\pbrac{\delby{u}{\xitwo}}_{1} +
    \chbfn{2}{0}{\xione}\chbfn{1}{1}{\xitwo}\pbrac{\delby{u}{\xitwo}}_{2} \\ &+
    \chbfn{1}{0}{\xione}\chbfn{2}{1}{\xitwo}\pbrac{\delby{u}{\xitwo}}_{3} +
    \chbfn{2}{0}{\xione}\chbfn{2}{1}{\xitwo}\pbrac{\delby{u}{\xitwo}}_{4} \\ &+
    \chbfn{1}{1}{\xione}\chbfn{1}{1}{\xitwo}
    \pbrac{\deltwoby{u}{\xione}{\xitwo}}_{1} +
    \chbfn{2}{1}{\xione}\chbfn{1}{1}{\xitwo}
    \pbrac{\deltwoby{u}{\xione}{\xitwo}}_{2} \\ &+ 
    \chbfn{1}{1}{\xione}\chbfn{2}{1}{\xitwo}
    \pbrac{\deltwoby{u}{\xione}{\xitwo}}_{3} +
    \chbfn{2}{1}{\xione}\chbfn{2}{1}{\xitwo}
    \pbrac{\deltwoby{u}{\xione}{\xitwo}}_{4}  
  \end{aligned}
  \label{eqn:bicubic}
\end{gather}
where
\begin{gather}
  \begin{aligned}
    \chbfn{1}{0}{\xi} \quad &= \quad 1-3\xi^2+2\xi^3 \\ 
    \chbfn{1}{1}{\xi} \quad &= \quad \xi\pbrac{\xi -1}^2 \\
    \chbfn{2}{0}{\xi} \quad &= \quad \xi^2\pbrac{3-2\xi} \\
    \chbfn{2}{1}{\xi} \quad &= \quad \xi^2\pbrac{\xi -1}
  \end{aligned}
\end{gather}
are the one-dimensional cubic Hermite basis functions (see \figref{fig:cubic}).

As in the one-dimensional case above, to preserve derivative continuity in
physical x-coordinate space as well as in $\xi$-coordinate space the global
node derivatives need to be specified with respect to physical arclength.
There are now two arclengths to consider: $s_{1}$, measuring arclength along
the $\xione$-coordinate, and $s_{2}$, measuring arclength along the 
$\xitwo$-coordinate. Thus
\begin{gather}
  \begin{aligned}
    \pbrac{\delby{u}{\xione}}_{n} &=
    \pbrac{\delby{u}{s_{1}}}_{\fnof{\Delta}{n,e}} \cdot
    \pbrac{\delby{s_{1}}{\xione}}_{n} \\ \pbrac{\delby{u}{\xitwo}}_{n} &=
    \pbrac{\delby{u}{s_{2}}}_{\fnof{\Delta}{n,e}} \cdot
    \pbrac{\delby{s_{2}}{\xitwo}}_{n} \\ 
    \pbrac{\deltwoby{u}{\xione}{\xitwo}}_{n} &=
    \pbrac{\deltwoby{u}{s_{1}}{s_{2}}}_ {\fnof{\Delta}{n,e}} \cdot
    \pbrac{\dby{s_{1}}{\xione}}_{n} \cdot \pbrac{\dby{s_{2}}{\xitwo}}_{n}
  \end{aligned}
  \label{eqn:Hbf}
\end{gather}
where $\pbrac{\dby{s_{1}}{\xione}}_{n}$ and $\pbrac{\dby{s_{2}}{\xitwo}}_{n}$
are element \emph{scale factors} which scale the arclength derivatives of
global node $\Delta$ to the $\xi$-coordinate derivatives of element node $n$.

The bicubic Hermite basis is a powerful shape descriptor for curvilinear
surfaces.  \figref{fig:surface} shows a four element bicubic Hermite
surface in 3D space where each node has the following twelve parameters
\begin{equation*}
   x,\/ \delby{x}{s_{1}},\/ \delby{x}{s_{2}},\/
   \deltwoby{x}{s_{1}}{s_{2}},\/y,\/ \delby{y}{s_{1}},\/\delby{y}{s_{2}},\/
   \deltwoby{y}{s_{1}}{s_{2}},\/  z,\/
   \delby{z}{s_{1}},\/ \delby{z}{s_{2}} \text{ and } \deltwoby{z}{s_{1}}{s_{2}}
\end{equation*}

\begin{figure}[htbp] \centering
  \input{figs/fem_basis_fns/surface.pstex}
  \caption{A surface formed by four bicubic Hermite elements.}
  \label{fig:surface}
\end{figure}


%\begin{example}{Defining a 2D cubic Hermite-linear finite element mesh}
%  {Defining a 2D cubic Hermite-linear finite element mesh}  
%  To define a 2D cubic Hermite-linear finite element mesh run example 114
%  \label{sec:Def2Db}
%\end{example}

\index{Basis functions!Hermite!bicubic|)}\index{Basis functions!Hermite|)}

\section{Triangular Elements} \index{Triangular elements|(}
Triangular elements cannot use the $\xione$ and $\xitwo$ coordinates defined 
above for \emph{tensor product} elements (\ie
two- and three- dimensional elements whose basis functions are formed as the
product of one-dimensional basis functions). The natural coordinates for
triangles are based on area ratios and are called \emph{Area Coordinates}
\index{Area Coordinates}.  
Consider the ratio of the area formed from the points $2$, $3$ and 
 $\fnof{P}{x,y}$ in \figref{fig:areacoord} to the total area of the triangle
\begin{figure}[htbp] \centering
  \input{figs/fem_basis_fns/areacoord.pstex}
  \caption{Area coordinates for a triangular element.}
  \label{fig:areacoord}
\end{figure}

\begin{equation*}
  L_{1}=\dfrac{\text{Area $<P23>$}}{\text{Area $<123>$}}=\frac12
    \begin{vmatrix}
            1 & x & y\\
            1 & x_{2} & y_{2}\\
            1 & x_{3} & y_{3}
    \end{vmatrix}
  / \Delta = \pbrac{a_{1} + b_{1}x + c_{1}y}/ \pbrac{2 \Delta}
\end{equation*}
where $ \Delta = \frac12             
    \begin{vmatrix}
            1 & x_{1} & y_{1}\\
            1 & x_{2} & y_{2}\\
            1 & x_{3} & y_{3}
    \end{vmatrix}$ 
    is the area of the triangle with vertices $123$, and $a_{1} = x_{2} y_{3}
    - x_{3} y_{2}, b_{1} = y_{2} - y_{3}, c_{1} = x_{3} - x_{2}$.

  Notice that $L_{1}$ is linear in $x$ and $y$. Similarly, area coordinates for
  the other two triangles containing $P$ and two of the element vertices are
\begin{equation*}
  L_{2}=\dfrac{\text{Area $<P13>$}}{\text{Area $<123>$}}=\frac12
    \begin{vmatrix}
            1 & x & y\\
            1 & x_{3} & y_{3}\\
            1 & x_{1} & y_{1}
    \end{vmatrix}
  / \Delta = \pbrac{a_{2} + b_{2}x + c_{2}y}/ \pbrac{2 \Delta}
\end{equation*}
\begin{equation*}
  L_{3}=\dfrac{\text{Area $<P12>$}}{\text{Area $<123>$}}=\frac12
    \begin{vmatrix}
            1 & x & y\\
            1 & x_{1} & y_{1}\\
            1 & x_{2} & y_{2}
    \end{vmatrix}
  / \Delta = \pbrac{a_{3} + b_{3}x + c_{3}y}/ \pbrac{2 \Delta}
\end{equation*}
where $a_{2} = x_{3} y_{1} - x_{1} y_{3}, b_{2} = y_{3} - y_{1}, c_{2} = x_{1}
- x_{3}$ and $a_{3} = x_{1} y_{2} - x_{2} y_{1}, b_{3} = y_{1} - y_{2}, c_{3}
= x_{2} - x_{1}$.

Notice that $L_{1} + L_{2} + L_{3} =1$.

Area coordinate $L_{1}$ varies linearly from $L_{1}=0$ when $P$ lies at node $2$
or $3$ to $L_{1}=1$ when $P$ lies at node $1$ and can therefore be used directly as
the basis function for node $1$ for a three node triangle.  Thus, interpolation
over the triangle is given by
\begin{equation*}
  \fnof{u}{x,y} = \lbfn{1}{x,y}u_{1} + \lbfn{2}{x,y}u_{2} + \lbfn{3}{x,y}u_{3}
\end{equation*} %reference needed???
where $\lbfnsymb{1} = L_{1}$, $\lbfnsymb{2} = L_{2}$ and $\lbfnsymb{3} = L_{3}
= 1 - L_{1} - L_{2}$.

Six node quadratic triangular elements are constructed as shown in 
\figref{fig:6nodeqt}.
\begin{figure}[htbp] \centering
  \input{figs/fem_basis_fns/6nodeqt.pstex}
  \caption{Basis functions for a six node quadratic triangular element.}
  \label{fig:6nodeqt}
\end{figure}

%\begin{example}{Defining a triangular element mesh}
%  {Defining a triangular element mesh}
%  \begin{figure}[htbp] \centering
%    \input{figs/fem_basis_fns/egdiag5.pstex}
%    \label{fig:egdiag5}
%  \end{figure} 
%  \label{sec:Deftem} 
%\end{example}
\index{Triangular elements|)}\index{Basis functions|)}

\section{Curvilinear Coordinate Systems}
\label{sec:ccs-1.8}
\index{Curvilinear coordinate systems|(}
It is sometimes convenient to model the geometry of the region (over which a 
finite element solution is sought) using an orthogonal curvilinear coordinate 
system. A 2D circular annulus, for example, can be modelled geometrically 
using one element with cylindrical polar $\pbrac{r,\theta}$-coordinates, \eg 
the annular plate in \figref{fig:dca}a has two global nodes, the first with 
$r=r_{1}$ and the second with $r=r_{2}$.

\begin{figure}[htbp] \centering
 \input{figs/fem_basis_fns/defca.pstex}
  \caption{Defining a circular annulus with one cylindrical polar element. 
    Notice that element vertices $1$ and $2$ in $\pbrac{r,\theta}$-space or
    $\pbrac{\xione,\xitwo}$-space, as shown in (b) and (c), respectively, map onto
    the single global node $1$ in $\pbrac{x ,y}$-space in (a).  Similarly, element
    vertices $3$ and $4$ map onto global node $2$.}
  \label{fig:dca}
\end{figure}
 
Global nodes $1$ and $2$, shown in $\pbrac{x,y}$-space in \figref{fig:dca}a, each 
map to two element vertices in $\pbrac{r,\theta}$-space, as shown in
\figref{fig:dca}b, and in $\pbrac{\xione,\xitwo}$-space, as shown in
\figref{fig:dca}c. The $\pbrac{r,\theta}$ coordinates at any
$\pbrac{\xione,\xitwo}$ point are given by a bilinear interpolation of the nodal
coordinates $r_{n}$ and $\theta_{n}$ as
\begin{align*}
  r &= \dotprod{\lbfn{n}{\xione,\xitwo}}{r_{n}}\\
  \theta &= \dotprod{\lbfn{n}{\xione,\xitwo}}{\theta_{n}} 
\end{align*}
where the basis functions $\lbfn{n}{\xione,\xitwo}$ are given by \bref{eqn:2,3DE}.

%\begin{example}{Defining a bilinear mesh in cylindrical polar coordinates}
%  {Defining a bilinear mesh in cylindrical polar coordinates}
%  \label{sec:Defbm}
%\end{example} 

Three orthogonal curvilinear coordinate systems are defined here for use in
later sections.

\begin{description}
\item \textbf{Cylindrical polar\index{Curvilinear coordinate
      systems!Cylindrical polar} $\pbrac{r,\theta, z}$ :} 
  \begin{equation}
    \begin{split}
      x & = r \cos\theta \\
      y & = r \sin\theta \\
      z & = z 
    \end{split}
  \end{equation}

\item \textbf{Spherical polar\index{Curvilinear coordinate
      systems!Spherical polar} $\pbrac{r,\theta, \phi}$ :}
  \begin{equation}
    \begin{split}
      x & = r \cos\theta \cos\phi \\
      y & = r \sin\theta \cos\phi \\
      z & = r \sin\phi 
    \end{split}
  \end{equation}
  
\item \textbf{Prolate spheroidal\index{Curvilinear coordinate
      systems!Prolate spheroidal} $\pbrac{\lambda, \mu, \theta}$ :}
  \begin{equation}
    \begin{split}
      x & = d \cosh\lambda \cos\mu \\ 
      y & = d \sinh\lambda \sin\mu \cos\theta \\ 
      z & = d \sinh\lambda \sin\mu \sin\theta 
    \end{split}
  \end{equation}
\end{description}


\begin{figure}[htbp] \centering
  \input{figs/fem_basis_fns/Psc.pstex}
  \caption{Prolate spheroidal coordinates.}
  \label{fig:psc}
\end{figure}


The prolate spheroidal coordinates rae illustrated in \figref{fig:psc} and a single prolate spheroidal element is shown in \figref{fig:spse}. The
coordinates $\pbrac{\lambda, \mu, \theta}$ are all trilinear in
$\pbrac{\xione,\xitwo,\xithree}$.  Only four global nodes are required provided
the four global nodes map to eight element nodes as shown in \figref{fig:spse}.

\begin{figure}[H] \centering
  \input{figs/fem_basis_fns/spse.pstex}
  \caption{A single prolate spheroidal element, shown (a) in 
    $\pbrac{x,y,z}$-coordinates, (c) in $\pbrac{\lambda, \mu,
      \theta}$-coordinates and (d) in
    $\pbrac{\xione,\xitwo,\xithree}$-coordinates, (b) shows the orientation
    of the $\xi_{i}$-coordinates on the prolate spheroid.}
  \label{fig:spse}
\end{figure}
\index{Curvilinear coordinate systems|)}

\clearpage

\section{CMISS Examples}

\begin{enumerate}
\item  To define a 2D bilinear finite element mesh run the CMISS example number
  $111$.  The nodes should be positioned as shown in \figref{fig:example111_node}.
  After defining elements the mesh should appear like the one shown in
  \figref{fig:example111_elem}.

  \begin{figure}[htbp] \centering
    \input{figs/fem_basis_fns/egdiag1.pstex}
    \caption{Node positions for example $111$.}
    \label{fig:example111_node}
  \end{figure}

   \begin{figure}[hbtp] \centering
    \input{figs/fem_basis_fns/egdiag2.pstex}
    \caption{2D bilinear finite element mesh for example $111$.}
    \label{fig:example111_elem}
  \end{figure} 
  \label{xmp:Def2D}

\item  To refine a mesh run the CMISS example $113$. After the first refine the
  mesh should appear like the one shown in \figref{fig:example113}.  
  \begin{figure}[htbp] \centering
    \input{figs/fem_basis_fns/egdiag3.pstex}
    \caption{First refined mesh for example $113$}
    \label{fig:example113}
  \end{figure}
  \begin{figure}[htbp] \centering
    \input{figs/fem_basis_fns/egdiag4.pstex}
    \caption{Second refined mesh for example $113$}
    \label{fig:example113b}
  \end{figure}
  \label{xmp:Refining}

\item  To define a quadratic-linear element run the cmiss example $115$.
\item  To define a 3D trilinear element run CMISS example $121$.
\item  To define a 2D cubic Hermite-linear finite element mesh run 
    example $114$.

\item  To define a triangular element mesh run CMISS example $116$ (see \figref{fig:example116}).
  \begin{figure}[htbp] \centering
    \input{figs/fem_basis_fns/egdiag5.pstex}
    \label{fig:egdiag5}
    \caption{Defining a triangular mesh for example $116$}
    \label{fig:example116}
  \end{figure} 
  \label{sec:Deftem} 

\item  To define a bilinear mesh in cylindrical polar coordinates run CMISS
  example $122$.
  \label{sec:Defbm}
\end{enumerate}

%%% Local Variables: 
%%% mode: latex
%%% TeX-master: "/product/cmiss/documents/notes/fembemnotes/fembemnotes"
%%% End: 

\clearemptydoublepage
\chapter{Steady-State Heat Conduction}
\label{cha:steadystate}

\section{One-Dimensional Steady-State Heat Conduction}
\label{sec:OdSSHC-2.1}

Our first example of solving a partial differential equation by finite
elements is the one-dimensional steady-state heat equation. The equation
arises from a simple heat balance over a region of conducting material:

\begin{sloppypar}
  \begin{center}
    Rate of change of heat flux = heat source per unit volume
  \end{center}
  or
  \begin{equation*}
    \dby{ }{x}\text{ (heat flux) + heat sink per unit volume = 0}
  \end{equation*}
  or
  \begin{equation*}
    \dby{ }{x}\pbrac{-k \dby{u}{x}} + \fnof{q}{u,x} = 0
  \end{equation*}
  where $u$ is temperature, $\fnof{q}{u,x}$ the heat sink and $k$ the thermal 
  conductivity (\units{Watts/m/\degree C}).
\end{sloppypar}

Consider the case where $q=u$ 
\begin{equation}
  -\dby{ }{x}\pbrac{k\dby{u}{x}} + u = 0 \quad 0<x<1
  \label{eqn:heat_sink}
\end{equation}
subject to boundary conditions: $\fnof{u}{0}=0$ and $\fnof{u}{1}=1$. 

This equation (with $k=1$) has an exact solution
\begin{equation}
  \fnof{u}{x} = \dfrac{e}{e^{2}-1} \pbrac{e^{x} -e^{-x}}
  \label{eqn:h.s.solutn}
\end{equation}
with which we can compare the approximate finite element solutions.

To solve \eqnref{eqn:heat_sink} by the finite element method requires the 
following steps:
\begin{enumerate}
\item Write down the integral equation form of the heat equation.
\item Integrate by parts (in 1D) or use Green's Theorem (in 2D or 3D) to
  reduce the order of derivatives.
\item Introduce the finite element approximation for the temperature field 
  with nodal parameters and element basis functions.
\item Integrate over the elements to calculate the element stiffness
  matrices and RHS vectors.
\item Assemble the global equations.
\item Apply the boundary conditions.
\item Solve the global equations.
\item Evaluate the fluxes.
\end{enumerate}

\subsection{Integral equation}

Rather than solving \eqnref{eqn:heat_sink} directly, we form the 
weighted residual \index{Weighted residual} 
\begin{equation}
  \dint R\omega.dx = 0
  \label{eqn:integral_eqn}
\end{equation}
where $R$ is the residual 
\begin{equation}
   R = -\dby{ }{x}\pbrac{k\dby{u}{x}} + u 
   \label{eqn:int_equation_2}
\end{equation}
for an approximate solution $u$ and $\omega$ is a weighting function
\index{Weighting function} to be chosen below. If $u$ were an exact solution
over the whole domain, the residual $R$ would be zero everywhere. But, given
that in real engineering problems this will not be the case, we try to obtain
an approximate solution $u$ for which the residual or error (\ie the amount
by which the differential equation is not satisfied exactly at a point) is
distributed evenly over the domain.  Substituting \eqnref{eqn:int_equation_2}
into \eqnref{eqn:integral_eqn} gives
\begin{equation}
  \gint{0}{1}{\bbrac{- \dby{ }{x}\pbrac{k\dby{u}{x}}\omega + u\omega}}{x} = 0
  \label{eqn:substitution}
\end{equation}
This formulation of the governing equation can be thought of as forcing the
residual or error to be zero in a spatially averaged sense.  More precisely,
$\omega$ is chosen such that the residual is kept orthogonal to the space of
functions used in the approximation of $u$ (see step 3 below).


\subsection{Integration by parts}

A major advantage of the integral equation is that the order of the
derivatives inside the integral can be reduced from two to one by integrating
by parts (or, equivalently for 2D problems, by applying Green's theorem - see
later).  Thus, substituting $f=\omega$ and $g=-k\dby{u}{x}$ into the
\emph{integration by parts}\index{integration by parts} formula
\begin{equation*}
  \gint{0}{1}{f\dby{g}{x}}{x} = \inteval{f.g}{0}{1}- \gint{0}{1}{g\dby{f}{x}}{x}
\end{equation*}
gives
\begin{equation*}
  \gint{0}{1}{\omega\dby{ }{x}\pbrac{-k\dby{u}{x}}}{x} 
  = \inteval{\omega\pbrac{-k\dby{u}{x}}}{0}{1}-
  \gint{0}{1}{\pbrac{-k\dby{u}{x}\dby{\omega}{x}}}{x}
\end{equation*}
and \eqnref{eqn:substitution} becomes 
\begin{equation}
  \gint{0}{1}{\pbrac{k \dby{u}{x}\dby{\omega}{x}+u\omega}}{x} = \inteval{k\dby{u}{x}\omega}{0}{1}
  \label{eqn:integration_by_parts}
\end{equation}

\subsection{Finite element approximation}

We divide the domain  $0<x<1$ into 3 equal length elements and replace the 
continuous field variable $\fnof{u}{x}$ within each element by the parametric
finite element approximation
\begin{align*}
  \fnof{u}{\xi}&=\lbfn{1}{\xi}u_{1}+\lbfn{2}{\xi}u_{2}=\lbfn{n}{\xi}u_{n}\\
  \fnof{x}{\xi}&=\lbfn{1}{\xi}x_{1}+\lbfn{2}{\xi}x_{2}=\lbfn{n}{\xi}x_{n}
\end{align*}
(summation implied by repeated index)
where $\lbfn{1}{\xi}=1-\xi$ and $\lbfn{2}{\xi}=\xi$ are the linear basis 
functions for both $u$ and $x$. 

We also choose $\omega=\lbfnsymb{m}$ (called the
\emph{Galerkin}\footnote{Boris G. Galerkin (1871-1945). 
    Galerkin was a Russian engineer who published his
    first technical paper on the buckling of bars while imprisoned in 1906 by
    the Tzar in pre-revolutionary Russia. In many Russian
    texts the Galerkin finite element method is known as the
    Bubnov-Galerkin method.
    He published a paper using this idea in 1915. The method was also attributed
    to I.G. Bubnov in 1913.}
assumption). This forces the
residual $R$ to be orthogonal to the space of functions used to represent the
dependent variable $u$, thereby ensuring that the residual, or error, is
monotonically reduced as the finite element mesh is refined (see later for a
more complete justification of this very important step)
\index{Galerkin formulation}.

The domain integral in \eqnref{eqn:integration_by_parts} can now be 
replaced by the sum of integrals taken separately over the three elements
\begin{equation*}
  \gint{0}{1}{\cdot}{x}=\gint{0}{\frac13}{\cdot}{x} + \gint{\frac13}{\frac23}{\cdot}{x} +
  \gint{\frac23}{1}{\cdot}{x}
\end{equation*}
and each element integral is then taken over $\xi$-space
\begin{equation*}
  \gint{x_{1}}{x_{2}}{\cdot}{x} = \gint{0}{1}{\cdot J}{\xi}
\end{equation*}
where $J =\abs{\dby{x}{\xi}}$ is the Jacobian of the transformation 
from $x$ coordinates to $\xi$ coordinates.

\subsection{Element integrals}

The element integrals arising from the LHS of \eqnref{eqn:integration_by_parts}
have the form
\begin{equation}
  \gint{0}{1}{\pbrac{k\dby{u}{x}\dby{\omega}{x} + u\omega}J}{\xi}
  \label{eqn:element_integrals_1}
\end{equation}
where $u = \lbfnsymb{n}u_{n}$ and $\omega = \lbfnsymb{m}$. Since $\lbfnsymb{n}$ and
$\lbfnsymb{m}$ are both functions of $\xi$ the derivatives with respect to $x$
need to be converted to derivatives with respect to $\xi$.  Thus
\eqnref{eqn:element_integrals_1} becomes
\begin{equation}
  u_{n}\gint{0}{1}{\pbrac{k\dby{\lbfnsymb{n}}{\xi}\dby{\xi}{x}
    \dby{\lbfnsymb{m}}{\xi}\dby{\xi}{x}+\lbfnsymb{n}\lbfnsymb{m}}J}{\xi}
  \label{eqn:element_integrals_2}
\end{equation}
Notice that $u_{n}$ has been taken outside the integral because it is not a
function of $\xi$. The term $\dby{\xi}{x}$ is evaluated by substituting the
finite element approximation $\fnof{x}{\xi}=\lbfnsymb{n}.x_{n}$. In this case
 $x=\dfrac13\xi$ or $\dby{\xi}{x}= 3$ and the Jacobian is
$J=\dby{x}{\xi}=\frac13$. The term multiplying the nodal parameters $u_{n}$ is
called the element stiffness matrix\index{element stiffness matrix}, $E_{mn}$
\begin{equation*}
  E_{mn}=\gint{0}{1}{\pbrac{k\dby{\lbfnsymb{m}}{\xi}\dby{\xi}{x}
    \dby{\lbfnsymb{n}}{\xi}\dby{\xi}{x}+\lbfnsymb{m}\lbfnsymb{n}}J}{\xi}
    =\gint{0}{1}{\pbrac{k\dby{\lbfnsymb{m}}{\xi}3\dby{\lbfnsymb{n}}{\xi}3
    +\lbfnsymb{m}\lbfnsymb{n}}\frac13}{\xi}
\end{equation*}
where the indices $m$ and $n$ are $1$ or $2$. To evaluate $E_{mn}$ we substitute 
the basis functions
\begin{alignat*}{2}
  \lbfn{1}{\xi}&= 1-\xi &\quad\text{ or } \dby{\lbfnsymb{1}}{\xi}&= -1 \\
  \lbfn{2}{\xi}&= \xi &\quad\text{ or } \dby{\lbfnsymb{2}}{\xi}&= 1
\end{alignat*}
Thus,
\begin{equation*}
  E_{11}=\frac13\gint{0}{1}{\pbrac{9k\pbrac{\dby{\lbfnsymb{1}}{\xi}}^{2}
    +\pbrac{\lbfnsymb{1}}^{2}}}{\xi}=\frac13\gint{0}{1}{\pbrac{9k\pbrac{-1}^{2}+
    \pbrac{1-\xi}^{2}}}{\xi}=\frac13 \pbrac{9k+\frac13}
\end{equation*}
and, similarly, 
\begin{align*}
  E_{12} &= E_{21} =\frac13 \pbrac{-9k + \frac16} \\
  E_{22} &= \frac13 \pbrac{9k+\frac13} \\ 
  E_{mn} &=
  \begin{bmatrix}
    \frac13 \pbrac{9k+\frac13} & \frac13 \pbrac{-9k+\frac16} \\
    \frac13 \pbrac{-9k+\frac16} & \frac13 \pbrac{9k+\frac13}
  \end{bmatrix}
\end{align*}
Notice that the element stiffness matrix is symmetric. Notice also that 
the stiffness matrix, in this particular case, is the same for all elements. 
For simplicity we put $k=1$ in the following steps.

\subsection{Assembly}

The three element stiffness matrices (with $k=1$) are assembled into one
global stiffness matrix\index{Global stiffness matrix}. This process is
illustrated in \figref{fig:asfig} where rows $1, .. , 4$ of the global
stiffness matrix (shown here multiplied by the vector of global unknowns) are
generalised from the weight function associated with nodes $1, .. , 4$.
\begin{figure} \centering
 \input{figs/heat_conduction/assembly.pstex}
 \caption{The rows of the global stiffness matrix are generated from the
   global weight functions. The bar is shown at the top divided into three
   elements.}
 \label{fig:asfig}
\end{figure}
 
Note how each element stiffness matrix (the smaller square brackets in
\figref{fig:asfig}) overlaps with its neighbour because they share a common
global node. The assembly process gives
\begin{equation*}
  \begin{bmatrix}
    \frac{28}{9} & -\frac{53}{18} & 0 & 0 \\
    -\frac{53}{18} & \frac{28}{9}+\frac{28}{9} & -\frac{53}{18} & 0 \\
    0 & -\frac{53}{18} & \frac{28}{9}+\frac{28}{9} & -\frac{53}{18} \\
    0 & 0 &-\frac{53}{18} & \frac{28}{9}
  \end{bmatrix}
  \begin{bmatrix}
    U_{1} \\
    U_{2} \\
    U_{3} \\
    U_{4}
  \end{bmatrix}
\end{equation*}
Notice that the first row (generating heat flux at node $1$) has zeros
multiplying $U_{3}$ and $ U_{4}$ since nodes $3$ and $4$ have no direct connection
through the basis functions to node $1$. Finite element matrices are always
\emph{sparse} matrices - containing many zeros - since the basis functions are
local to elements.

The RHS of \eqnref{eqn:integration_by_parts} is 
\begin{equation}
  \inteval{k\dby{u}{x}\omega}{x=0}{x=1}=\evalat{\pbrac{k\dby{u}{x}\omega}}{x=1}
    -\evalat{\pbrac{k\dby{u}{x}\omega}}{x=0}
  \label{eqn:a1}
\end{equation}
To evaluate these expressions consider the weighting function $\omega$
corresponding to each global node (see Fig.1.6). For node $1$ $\omega_{1}$ is
obtained from the basis function $\lbfnsymb{1}$ associated with the first node of
element $1$ and therefore $\evalat{\omega_{1}}{x=0}=1$. Also, since $\omega_{1}$ is
identically zero outside element $1$, $\evalat{\omega_{1}}{x=1}=0$. Thus
\eqnref{eqn:a1} for node $1$ reduces to
\begin{equation*}
  \inteval{k\dby{u}{x}\omega_{1}}{x=0}{x=1}=-\evalat{\pbrac{k\dby{u}{x}}}{x=0}
    \text{= flux entering node $1$.}
\end{equation*}
Similarly,
\begin{equation*}
  \inteval{k\dby{u}{x}\omega_{n}}{x=0}{x=1}=0\quad\quad\text{(nodes $2$ and $3$)}
\end{equation*}
and
\begin{equation*}
  \inteval{k\dby{u}{x}\omega_{4}}{x=0}{x=1}=\evalat{\pbrac{k\dby{u}{x}}}{x=1}
    \mbox{= flux entering node $4$.}
\end{equation*}
Note: $k$ has been left in these expressions to emphasise that they are heat 
fluxes.

Putting these global equations together we get
\begin{equation}
  \begin{bmatrix}    
    \frac{28}{9} & -\frac{53}{18} & 0 & 0 \\
    -\frac{53}{18} & \frac{28}{9}+\frac{28}{9} & -\frac{53}{18} & 0 \\
    0 & -\frac{53}{18} & \frac{28}{9}+\frac{28}{9} & -\frac{53}{18} \\
    0 & 0 &-\frac{53}{18} & \frac{28}{9}
  \end{bmatrix}
  \begin{bmatrix}
    U_{1} \\
    U_{2} \\
    U_{3} \\
    U_{4}
  \end{bmatrix} =
  \begin{bmatrix}
    -\evalat{\pbrac{k\dby{u}{x}}}{x=0} \\
    0 \\
    0 \\
    \evalat{\pbrac{k\dby{u}{x}}}{x=1}
  \end{bmatrix}
  \label{eqn:globaleqn}
\end{equation}    
or
\begin{equation*}
  \matr{K}\vect{u}=\vect{f}
\end{equation*}
where $\matr{K}$ is the global ``stiffness'' matrix, $\vect{u}$ the vector of
unknowns and $\vect{f}$ the global ``load'' vector.

Note that if the governing differential equation had included a distributed 
source term that was independent of $u$, this term would appear - via its 
weighted integral - on the RHS of \eqnref{eqn:globaleqn} rather than on the LHS
as here. Moreover, if the source term was a function of $x$, the contribution 
from each element would be different - as shown in the next section.

\subsection{Boundary conditions}

\index{Boundary conditions!application of}The boundary conditions
$\fnof{u}{0}=0$ and $\fnof{u}{1}=1$ are applied directly to the first and last
nodal values: \ie $U_{1}=0$ and $U_{4}=1$.  These so-called \emph{essential}
boundary conditions then replace the first and last rows in the global
\eqnref{eqn:globaleqn}, where the flux terms on the RHS are at present
unknown
\begin{equation*}
  \begin{array}{lrrrrl}
    \mbox{$1^{\text{st}}$ equation} & U_{1} & & & & = 0 \\ 
    \mbox{$2^{\text{nd}}$ equation} & -\frac{53}{18}U_{1} & +\frac{56}{9}U_{2}
    & -\frac{53}{18}U_{3} & & = 0 \\
    \mbox{$3^{\text{rd}}$ equation}& & -\frac{53}{18}U_{2} & +\frac{56}{9}U_{3}
    & -\frac{53}{18}U_{4} &= 0 \\
    \mbox{$4^{\text{th}}$ equation}& & & & U_{4} & =1
  \end{array}
\end{equation*}

Note that, if a flux boundary condition had been applied, rather than an 
essential boundary condition, the known value of flux would enter the 
appropriate RHS term and the value of $U$ at that node would remain an unknown 
in the system of equations. An applied boundary flux of zero, corresponding to 
an insulated boundary, is termed a \emph{natural} boundary condition, since 
effectively no additional constraint is applied to the global equation. At
least one essential boundary condition must be applied.

\subsection{Solution}
 
Solving these equations gives: $U_{2} = 0.2885$ and $U_{3} = 0.6098$.  From
\eqnref{eqn:h.s.solutn} the exact solutions at these points are $0.2889$ and
 $0.6102$, respectively. The finite element solution is shown in
\figref{fig:f.e.soln}.
\begin{figure} \centering
  \input{figs/heat_conduction/fesoln.pstex}
  \caption{Finite element solution of one-dimensional heat equation.}
  \label{fig:f.e.soln}
\end{figure}

\subsection{Fluxes}

The fluxes at nodes $1$ and $4$ are evaluated by substituting the nodal solutions
$U_{1}=0$, $U_{2}=0.2885$, $U_{3}=0.6098$ and $U_{4}=1$ into \eqnref{eqn:globaleqn}
\begin{alignat*}{2}
  \text{flux entering node $1$} &= -\evalat{\pbrac{k\dby{u}{x}}}{x=0} = -0.8496
        && \quad \text{ ($k=1$; exact solution $0.8509$)}\\      
  \text{flux entering node $4$} &= \evalat{\pbrac{k\dby{u}{x}}}{x=1} = 1.3157
        && \quad \text{ ($k=1$; exact solution $1.3131$)}
\end{alignat*}       
These fluxes are shown in \figref{fig:f.e.soln} as heat entering node $4$ 
and leaving node $1$, consistent with heat flow down the temperature gradient.

\section{An $x$-Dependent Source Term}

Consider the addition of a source term dependent on $x$ in 
\eqnref{eqn:heat_sink}:
\begin{equation*}
    -\dby{ }{x}\pbrac{k\dby{u}{x}}+u-x=0 \quad 0<x<1
\end{equation*}    
\eqnref{eqn:integration_by_parts} now becomes
\begin{equation}
  \gint{0}{1}{\pbrac{k\dby{u}{x}\dby{\omega}{x} + u\omega}}{x} 
    = \inteval{k\dby{u}{x}\omega}{0}{1} +\gint{0}{1}{x\omega}{x}        
  \label{eqn:source_term_dep}
\end{equation}
where the $x$-dependent source term appears on the RHS because it is not 
dependent on $u$. Replacing the domain integral for this source term by the
sum of three element integrals
\begin{equation*}
  \gint{0}{1}{x\omega}{x}=\gint{0}{\frac13}{x\omega}{x} 
   + \gint{\frac13}{\frac23}{x\omega}{x} + \gint{\frac23}{1}{x\omega}{x}
\end{equation*}
and putting $x$ in terms of $\xi$ gives (with $\dby{x}{\xi}=\dfrac13 $ 
for all three elements)
\begin{equation}
  \gint{0}{1}{x\omega}{x}=\dfrac13\gint{0}{1}{\dfrac{\xi}{3}\omega}{\xi}
    + \dfrac13\gint{0}{1}{\dfrac{\pbrac{1+\xi}}{3}\omega}{\xi} +
    \dfrac13\gint{0}{1}{\dfrac{\pbrac{2+\xi}}{3}\omega}{\xi}
  \label{eqn:x-d.s.t.}
\end{equation}
where $\omega$ is chosen to be the appropriate basis function within each element. 
For example, the first term on the RHS of \bref{eqn:x-d.s.t.} corresponding to 
element $1$ is $\dfrac{1}{9}\gint{0}{1}{\xi\lbfnsymb{m}}{\xi}$,
where $\lbfnsymb{1}=1 -\xi$ and $\lbfnsymb{2}=\xi$ . Evaluating these expressions, 
\begin{equation*}
  \gint{0}{1}{\dfrac{1}{9}\xi\pbrac{1-\xi}}{\xi} = \dfrac{1}{54}
\end{equation*}
and
\begin{equation*}
  \gint{0}{1}{\dfrac{1}{9} \xi^{2}}{\xi} = \dfrac{1}{27}
\end{equation*}
Thus, the contribution to the element $1$ RHS vector from the source term is
$\begin{bmatrix}
  \frac{1}{54} \\
  \frac{1}{27}
\end{bmatrix}$.

Similarly, for element $2$,  
\begin{equation*}
  \gint{0}{1}{\dfrac{1}{9}\pbrac{1+\xi}\pbrac{1-\xi}}{\xi}=\dfrac{2}{27}
  \mbox{ and }\gint{0}{1}{\dfrac{1}{9}\pbrac{1+\xi}\xi}{\xi}=\dfrac{5}{54}
  \mbox{ gives }
  \begin{bmatrix}  
    \frac{2}{27} \\
    \frac{5}{54}
  \end{bmatrix}
\end{equation*}
and for element $3$,
\begin{equation*}
  \gint{0}{1}{\dfrac{1}{9}\pbrac{2+\xi}\pbrac{1-\xi}}{\xi}=\dfrac{7}{54}
  \mbox{ and }\gint{0}{1}{\dfrac{1}{9}\pbrac{2+\xi}\xi}{\xi}=\dfrac{5}{54}
  \mbox{ gives }
  \begin{bmatrix}
    \frac{7}{54} \\
    \frac{5}{54}
  \end{bmatrix}
\end{equation*}
Assembling these into the global RHS vector, \eqnref{eqn:globaleqn} becomes
\begin{equation*}
  \begin{bmatrix}
    \frac{28}{9} & -\frac{53}{18} & 0 & 0 \\
    -\frac{53}{18} & \frac{56}{9} & -\frac{53}{18} & 0 \\
    0 & -\frac{53}{18} & \frac{56}{9} & -\frac{53}{18} \\
    0 & 0 &-\frac{53}{18} & \frac{28}{9}
  \end{bmatrix}
  \begin{bmatrix}
    U_{1} \\
    U_{2} \\
    U_{3} \\
    U_{4}
  \end{bmatrix} =
  \begin{bmatrix}
    -\evalat{\pbrac{k\dby{u}{x}}}{x=0} \\
    0 \\
    0 \\
    \evalat{\pbrac{k\dby{u}{x}}}{x=1}
  \end{bmatrix} + 
  \begin{bmatrix}
    \frac{1}{54} \\
    \frac{1}{27} + \frac{2}{27} \\
    \frac{5}{54}+\frac{7}{54} \\
    \frac{5}{54}
  \end{bmatrix}
\end{equation*}

\section{The Galerkin Weight Function Revisited}

\index{Galerkin formulation}A key idea in the Galerkin finite element method
is the choice of weighting functions which are orthogonal to the equation
residual (thought of here as the error or amount by which the equation fails
to be exactly zero).  This idea is illustrated in \figref{fig:galerkin}.
\begin{figure} \centering
  \input{figs/heat_conduction/galerkin.pstex}
  \caption{Showing how the Galerkin method maintains orthogonality between the 
    residual vector $\vect{R}$ and the set of basis vectors
    $\vect{\lbfnsymb{i}}$ as
    $i$ is increased from (a) $1$ to (b) $2$ to (c) $3$.}
  \label{fig:galerkin}
\end{figure}

In \figref{fig:galerkin}a an exact vector $\vect{u}_{e}$ (lying in 3D space)
is approximated by a vector $\vect{u}=\vect{u_{1}\lbfnsymb{1}}$ where
$\vect{\lbfnsymb{1}}$ is a basis vector along the first coordinate axis
(representing one degree of freedom in the system). The difference between the
exact vector $\vect{u_{e}}$ and the approximate vector $\vect{u}$ is the error
or residual $\vect{R}=\vect{u_{e}}-\vect{u}$ (shown by the broken line in
\figref{fig:galerkin}a).  The Galerkin technique minimises this residual by
making it orthogonal to $\lbfnsymb{1}$ and hence to the approximating vector
$\vect{u}$.  If a second degree of freedom (in the form of another coordinate
axis in \figref{fig:galerkin}b) is added, the approximating vector is
$\vect{u}=u_{1}\vect{\lbfnsymb{1}}+u_{2}\vect{\lbfnsymb{2}}$ and the residual
is now \emph{also} made orthogonal to $\lbfnsymb{2}$ and hence to $\vect{u}$.
Finally, in \figref{fig:galerkin}c, a third degree of freedom (a third axis in
\figref{fig:galerkin}c) is permitted in the approximation
$\vect{u}=u_{1}\lbfnsymb{1}+u_{2}\lbfnsymb{2}+u_{3}\lbfnsymb{3}$ with the
result that the residual (now also orthogonal to $\lbfnsymb{3}$) is reduced to
zero and $\vect{u}=\vect{u_{e}}$. For a 3D vector space we only need three
axes or basis vectors to represent the true vector $\vect{u}$, but in the
infinite dimensional vector space associated with a spatially continuous field
$\fnof{u}{x}$ we need to impose the equivalent orthogonality condition
$\pbrac{\dint R\lbfnsymb{} dx = 0}$ for every basis function $\lbfnsymb{}$ used in
the approximate representation of $\fnof{u}{x}$. The key point is that in this
analogy the residual is made orthogonal to the current set of basis vectors -
or, equivalently, in finite element analysis, to the set of basis functions
used to represent the dependent variable. This ensures that the error or
residual is minimal (in a least-squares sense) for the current number of
degrees of freedom and that as the number of degrees of freedom is increased
(or the mesh refined) the error decreases monotonically.

\section{Two and Three-Dimensional Steady-State Heat Conduction}
\label{sec:2and3-DSSHC}
Extending \eqnref{eqn:heat_sink} to two or three spatial dimensions 
introduces some additional complexity which we examine here. Consider the
three-dimensional steady-state heat equation with no source terms:
\begin{equation*}
  -\delby{ }{x}\pbrac{k_{x}\delby{u}{x}} -\delby{ }{y}\pbrac{k_{y}\delby{u}{y}}
  -\delby{ }{z}\pbrac{k_{z}\delby{u}{z}}=0
\end{equation*}       
where $k_{x},k_{y}$ and $k_{z}$ are the thermal diffusivities along the $x$, $y$
and $z$ axes respectively. If the material is assumed to be isotropic, $k_{x}
= k_{y} = k_{z} = k$, and the above equation can be written
as
\begin{equation}
  -\diverg{\pbrac{k \grad u}} = 0
  \label{eqn:S-SH}
\end{equation}
and, if $k$ is spatially constant (in the case of a homogeneous material), this reduces to Laplace's equation 
$k\laplacian{u}=0$. Here we consider the solution of \eqnref{eqn:S-SH} over the 
region $\Omega$, subject to boundary conditions on $\Gamma$ (see \figref{fig:regionbdy}).
\pstexfigure{figs/heat_conduction/regionbdy.pstex}{}{The region $\Omega$ and
  the boundary $\Gamma$.}{fig:regionbdy}

The weighted integral equation, corresponding to \eqnref{eqn:S-SH}, is
\begin{equation}
  \goneint{-\diverg{\pbrac{k\grad u}}\omega}{\Omega} = 0
  \label{eqn:weighted_integral}
\end{equation}

The multi-dimensional equivalent of integration by parts is the Green-Gauss
theorem:
\begin{equation}
  \goneint{\pbrac{f\diverg{\grad g}+\dotprod{\grad f}{\grad g}}}{\Omega} =
  \goneint{f \delby{g}{n}}{\Gamma}
  \label{eqn:nabla_eq}
\end{equation}
(see p553 in Advanced Engineering Mathematics'' by E. Kreysig, 7th edition,
Wiley, 1993).

This is used (with $f=\omega$, $g=-ku$ and assuming that $k$ is constant) 
to reduce the derivative order from two to one as follows:
\begin{equation}
  \goneint{-\diverg{\pbrac{k\grad u}}\omega}{\Omega} 
  = \goneint{k \dotprod{\grad u}{\grad  \omega}}{\Omega} 
  - \goneint{k\delby{u}{n} \omega}{\Gamma}      
  \label{eqn:Green-Gauss}
\end{equation}
\cf Integration by parts is $\goneint{-\dby{ }{x}\pbrac{k\dby{u}{x}}\omega}{x} =
\goneint{k \dby{u}{x} \dby{\omega}{x}}{x} - \inteval{k\dby{u}{x}\omega}{x_{1}}{x_{2}}$.

Using \eqnref{eqn:Green-Gauss} in \eqnref{eqn:weighted_integral} gives the 
two-dimensional equivalent of \eqnref{eqn:integration_by_parts} 
(but with no source term):
\begin{equation}
  \goneint{k \dotprod{\grad u}{\grad \omega}}{\Omega}
  = \goneint{k \delby{u}{n}\omega}{\Gamma}
  \label{eqn:2-D_equiv}
\end{equation}
subject to $u$ being given on one part of the boundary and $\delby{u}{n}$
being given on another part of the boundary.

The integrand on the LHS of \bref{eqn:2-D_equiv} is evaluated using
\begin{equation}
  \dotprod{\grad u}{\grad \omega} = \dotprod{\delby{u}{x_{k}}}{\delby{\omega}{x_{k}}} = 
  \dotprod{\delby{u}{\xi_{i}} \delby{\xi_{i}}{x_{k}}}{\delby{\omega}{\xi_{j}} 
  \delby{\xi_{j}}{x_{k}}}
  \label{eqn:integrand}
\end{equation}
        
where $u=\lbfnsymb{n}u_{n}$ and $\omega=\lbfnsymb{m}$, as before, and the geometric terms
$\delby{\xi_{i}}{x_{k}}$ are found from the inverse matrix
\begin{equation*}
  \sqbrac{\delby{\xi_{i}}{x_{k}}}=\sqbrac{\delby{x_{k}}{\xi_{i}}}^{-1} 
\end{equation*} 
or, for a two-dimensional element,
\begin{equation*}
  \begin{bmatrix}
    \delby{\xi_{1}}{x} & \delby{\xi_{1}}{y} \\
    \delby{\xi_{2}}{x} & \delby{\xi_{2}}{y}
  \end{bmatrix} =
  \begin{bmatrix}
    \delby{x}{\xi_{1}} & \delby{x}{\xi_{2}} \\
    \delby{y}{\xi_{1}} & \delby{y}{\xi_{2}}
  \end{bmatrix}^{-1} =
  \dfrac{1}{\delby{x}{\xi_1}\delby{y}{\xi_2}- \delby{x}{\xi_2}\delby{y}
    {\xi_1}}
  \begin{bmatrix}
    \delby{y}{\xi_{2}} & -\delby{x}{\xi_{2}}\\
    -\delby{y}{\xi_{1}} & \delby{x}{\xi_{1}}
  \end{bmatrix}
\end{equation*}

%\begin{example}{2D steady-state heat conduction}
%  {2D steady-state heat conduction}

%  \todo{example31}

%  \label{xmp:2DSSH}
%\end{example}

%\remark{Inserted new text here}
%inserted file :chapter2/test.txt Jan 20 1997

\section{Basis Functions - Element Discretisation}
%Example,

Let $\Omega = \displaystyle{\bigcup^{I}_{i=1}} \medspace \Omega_{i}$, \ie the
solution region is the union of the individual elements. In each $\Omega_{i}$ let $u =
\lbfnsymb{n}u_{n} = \lbfnsymb{1}u_{1} + \lbfnsymb{2}u_{2} + \ldots + 
\lbfnsymb{N}u_{N}$ and map each $\Omega_{i}$ to the $\xi_{1}, \xi_{2}$
plane. \figref{fig:mapping} shows an example of this mapping. 
\pstexfigure{figs/heat_conduction/mapping.pstex}{}{Mapping each $\Omega$ to the
  $\xi_{1},\xi_{2}$ plane in a $2 \times 2$ element plane.}{fig:mapping}

For each element, the basis functions and their derivatives are:
\begin{alignat}{2}
  \lbfnsymb{1} &= (1-\xi_{1})(1-\xi_{2})  
  &\qquad \delby{\lbfnsymb{1}}{\xi_{1}}&=-(1-\xi_{2})\\ 
  &&\delby{\lbfnsymb{1}}{\xi_{2}}
  &=-(1-\xi_{1})\\\\
%%line2
  \lbfnsymb{2}&=\xi_{1}(1-\xi_{2}) &\quad \delby{\lbfnsymb{2}}{\xi_{1}}&=1-\xi_{2}\\ 
  &&\delby{\lbfnsymb{1}}{\xi_{2}}&=-\xi_{1} \\\\ 
%%line3
  \lbfnsymb{3}&=(1-\xi_{1})\xi_{2} &\quad \delby{\lbfnsymb{3}}{\xi_{1}}&=-\xi_{2}\\ 
  &&\delby{\lbfnsymb{3}}{\xi_{2}}&=1-\xi_{1} \\\\  
%%line4
  \lbfnsymb{4}&=\xi_{1}\xi_{2} &\quad \delby{\lbfnsymb{4}}{\xi_{1}}&=\xi_{2}\\ 
  &&\delby{\lbfnsymb{4}}{\xi_{2}}&=\xi_{1} 
%%
\end{alignat}

\section{Integration}

The equation is 
\begin{equation}
  \goneint{k\dotprod{\grad u}{\grad \omega}}{\Omega} 
     = \goneint{k\delby{u}{n}\omega}{\Gamma}
\end{equation}
\ie
\begin{equation}
  \goneint{k\pbrac{\delby{u}{x}\delby{\omega}{x} + 
    \delby{u}{y}\delby{\omega}{y}}}{\Omega} = \goneint{k\delby{u}{n}\omega}{\Gamma}
\end{equation}

u has already been approximated by $\lbfnsymb{n}u_{n}$ and $\omega$ is a weight 
function but what should this be chosen to be?
For a \emph{Galerkin} formulation choose $\omega = \lbfnsymb{m}$
\ie weight function is one of the basis functions used to approximate the
dependent variable.

This gives
\begin{equation}
  \sum_{i}u_{n}
\goneint{k\pbrac{ \delby{\lbfnsymb{n}}{x}\delby{\lbfnsymb{m}}{x}
      +\delby{\lbfnsymb{n}}{y}\delby{\lbfnsymb{m}}{y} }}{\Omega} 
= \goneint{k\delby{u}{n}\lbfnsymb{m}}{\Gamma}
\end{equation}

where the stiffness matrix is $E_{mn}$ where $m=1,\dots,4$ and $n=1,\ldots,4$
and $F_{m}$ is the (element) load vector. 

The names originated from earlier finite element applications and extension of spring
systems, \ie $F=kx$ where $k$ is the stiffness of spring and $F$ is the force/load.

This yields the system of equations $E_{mn}u_{n} = F_{m}$. \eg heat flow in a
unit square (see \figref{fig:heatflow}).
\pstexfigure{figs/heat_conduction/heatflow.pstex}{}{Considering heat flow in a unit square.}
   {fig:heatflow}

The first component $E_{11}$ is calculated as
\begin{align*}
E_{11} &= k \giint{0}{1}{0}{1}{ (1-y)^{2}+(1-x)^{2} }{x}{y}\\ 
       &= \dfrac{2}{3}k
\end{align*}
and similarly for the other components of the matrix.

Note that if the element was not the unit square we would need to transform
from $(x,y)$ to $(\xi_{1},\xi_{2})$ coordinates. In this case we would have to
include the Jacobian of the transformation and also use the chain rule to
calculate $\delby{\lbfnsymb{i}}{x_{j}}$. \eg 
$\delby{\lbfnsymb{n}}{x} = 
   \delby{\lbfnsymb{n}}{\xi_{1}}\delby{\xi_{1}}{x} + 
   \delby{\lbfnsymb{n}}{\xi_{2}}\delby{\xi_{2}}{x}
   =\delby{\lbfnsymb{n}}{\xi_{i}}\delby{\xi_{i}}{x}$.

The system of $E_{mn}u_{n}=F_{m}$ becomes

\begin{equation}
  k
  \begin{bmatrix}
    \frac{2}{3} & -\frac{1}{6} & -\frac{1}{6} & -\frac{1}{3} \\
    -\frac{1}{6} & \frac{2}{3} & -\frac{1}{3} & -\frac{1}{6} \\
    -\frac{1}{6} & -\frac{1}{3} & \frac{2}{3} & -\frac{1}{6} \\
    -\frac{1}{3} & -\frac{1}{6} & -\frac{1}{6} & \frac{2}{3}
  \end{bmatrix}
  \begin{bmatrix}
    u_{1} \\
    u_{2} \\
    u_{3} \\
    u_{4} 
  \end{bmatrix}
  = RHS \qquad \text{(Right Hand Side)}
\end{equation}

Note that the Galerkin formulation generates a symmetric stiffness matrix (this is
true for self adjoint operators which are the most common).

Given that boundary conditions can be applied and it is possible to solve for 
unknown nodal temperatures or fluxes. 
However, typically there is more than one element and so the next step is required.

\section{Assemble Global Equations}

Each element stiffness matrix must be assembled into a global stiffness
matrix.  For example, consider $4$ elements (each of unit size) and nine nodes. Each
element has the same element stiffness matrix as that given above. This is
because each element is the same size, shape and interpolation.


\pstexfigure{figs/heat_conduction/assemble.pstex}{}{Assembling $4$ unit sized elements into a
  global stiffness matrix.}
   {fig:assemble}
\begin{equation}
  \begin{bmatrix}
    \frac{2}{3} &-\frac{1}{6} &\frac{}{} &-\frac{1}{6} 
          &-\frac{1}{3} 
          &\frac{}{} &\frac{}{} &\frac{}{} &\frac{}{} \\
%line2
    -\frac{1}{6} &\frac{2}{3}+\frac{2}{3} &-\frac{1}{6} &-\frac{1}{3} 
          &-\frac{1}{6}-\frac{1}{6} 
          &-\frac{1}{3} &\frac{}{} &\frac{}{} &\frac{}{} \\
%line3
    \frac{}{} &-\frac{1}{6} &\frac{2}{3} &\frac{}{} 
          &-\frac{1}{3} 
          &-\frac{1}{6} &\frac{}{} &\frac{}{} &\frac{}{} \\
%line4
    -\frac{1}{6} &-\frac{1}{3} &\frac{}{} &\frac{2}{3}+\frac{2}{3} 
          &-\frac{1}{6}-\frac{1}{6} 
          &\frac{}{} &-\frac{1}{6} &-\frac{1}{3} &\frac{}{} \\
%line5
    -\frac{1}{3} &-\frac{1}{6}-\frac{1}{6} &-\frac{2}{3} &-\frac{1}{6}-\frac{1}{6} 
          & \frac{2}{3}+\frac{2}{3}+\frac{2}{3}+\frac{2}{3} 
          & -\frac{1}{6}-\frac{1}{6}
          &-\frac{1}{3} &-\frac{1}{6}-\frac{1}{6} &-\frac{1}{3} \\
%line6
    \frac{}{} &-\frac{1}{3} &-\frac{1}{6} &\frac{}{} 
          &-\frac{1}{6}-\frac{1}{6}
          &\frac{2}{3}+\frac{2}{3} &\frac{}{} &-\frac{1}{3} &-\frac{1}{6} \\
%line7
    \frac{}{} &\frac{}{} &\frac{}{} &-\frac{1}{6} 
          &-\frac{1}{3} 
          &\frac{}{} &\frac{2}{3} &-\frac{1}{6} &\frac{}{} \\
%line8
    \frac{}{} &\frac{}{} &\frac{}{} &-\frac{1}{3} 
          &-\frac{1}{6}-\frac{1}{6} 
          &-\frac{1}{3} &-\frac{1}{6} &\frac{2}{3}+\frac{2}{3} &-\frac{1}{6} \\
%line9
    \frac{}{} &\frac{}{} &\frac{}{} &\frac{}{} 
          &-\frac{1}{3} 
          &-\frac{1}{6} &\frac{}{} &-\frac{1}{6} &\frac{2}{3}
  \end{bmatrix}
  \begin{bmatrix}
    U_{1} \\
    U_{2} \\
    U_{3} \\
    U_{4} \\
    U_{5} \\
    U_{6} \\
    U_{7} \\
    U_{8} \\
    U_{9}
  \end{bmatrix}
  = RHS
\end{equation}
This yields the system of equations 
\begin{equation*}
  \begin{bmatrix}
    \frac{2}{3} &-\frac{1}{6} &\frac{}{} &-\frac{1}{6} &-\frac{1}{3} 
          &\frac{}{} &\frac{}{} &\frac{}{} &\frac{}{} \\
%line2
    -\frac{1}{6} &\frac{4}{3} &-\frac{1}{6} &-\frac{1}{3} &-\frac{1}{3} 
          &-\frac{1}{3} &\frac{}{} &\frac{}{} &\frac{}{} \\
%line3
    \frac{}{} &-\frac{1}{6} &\frac{2}{3} &\frac{}{} &-\frac{1}{3} 
          &-\frac{1}{6} &\frac{}{} &\frac{}{} &\frac{}{} \\
%line4
    -\frac{1}{6} &-\frac{1}{3} &\frac{}{} &\frac{4}{3} &-\frac{1}{3} 
          &\frac{}{} &-\frac{1}{6} &-\frac{1}{3} &\frac{}{} \\
%line5
    -\frac{1}{3} &-\frac{1}{3} &-\frac{1}{3} &-\frac{1}{3} &\frac{8}{3} 
          &-\frac{1}{3} &-\frac{1}{3} &-\frac{1}{3} &-\frac{1}{3} \\
%line6
    \frac{}{} &-\frac{1}{3} &-\frac{1}{6} &\frac{}{} &-\frac{1}{3} 
          &\frac{4}{3} &\frac{}{} &-\frac{1}{3} &-\frac{1}{6} \\
%line7
    \frac{}{} &\frac{}{} &\frac{}{} &-\frac{1}{6} &-\frac{1}{3} 
          &\frac{}{} &\frac{2}{3} &-\frac{1}{6} &\frac{}{} \\
%line8
    \frac{}{} &\frac{}{} &\frac{}{} &-\frac{1}{3} &-\frac{1}{3} 
          &-\frac{1}{3} &-\frac{1}{6} &\frac{4}{3} &-\frac{1}{6} \\
%line9
    \frac{}{} &\frac{}{} &\frac{}{} &\frac{}{} &-\frac{1}{3} 
          &-\frac{1}{6} &\frac{}{} &-\frac{1}{6} &\frac{2}{3}
   \end{bmatrix}
   \begin{bmatrix}
     U_{1} \\
     U_{2} \\
     U_{3} \\
     U_{4} \\
     U_{5} \\
     U_{6} \\
     U_{6} \\
     U_{7} \\
     U_{8} \\
     U_{9} \\
   \end{bmatrix}
   = RHS
\end{equation*}
Note that the matrix is symmetric. It should also be clear that the matrix
will be sparse (i.e. contains many zeros) if there is a larger number of elements.

From this system of equations, boundary conditions can be applied and the
equations solved. To solve, firstly boundary conditions are applied to reduce
the size of the system.

If at global node $i$, $U_{i}$ is known, we can remove the \nth{i} equation
and replace it with the known value of $U_{i}$. This is because the RHS at
node $i$ is unknown, but the RHS equation is uncoupled from other equations so these
equation can be removed.
Therefore the size of the system is reduced. The final system to solve is only
as big as the number of unknown values of $U$. 

As an example to illustrate this consider fixing the temperature ($U$) at the
left and right sides of the plate in \figref{fig:assemble} and insulating the top (node
 $8$) and the bottom (node $2$). 
This means that there are only $3$ unknown values of $U$ at nodes (2,5 and 8), 
therefore there is a $3 \times 3$ matrix to solve. The RHS is known at these
three nodes (see below). We can then solve the $3 \times 3$ matrix and then multiply out
the original matrix to find the unknown RHS values.

The RHS is $0$ at nodes $2$ and $8$ because it is insulated.
To find out what the RHS is at node $5$ we need to examine the RHS expression
 $\goneint{\delby{u}{n}\omega}{\Gamma} = 0$ at node $5$. This is zero as flux is always $0$ at
internal nodes. This can be explained in two ways.
\pstexfigure{figs/heat_conduction/zeroflux.pstex}{}{``Cancelling'' of flux in internal nodes.}
   {fig:zeroflux} 
\begin{description}
   \item [Correct way:] $\Gamma$ does not pass through node $5$ and each
     basis function that is
     not zero at $5$ is zero on $\Gamma$
   \item [Other way:] $\delby{u}{n}$ is opposite in neighbouring elements so
     it cancels (see \figref{fig:zeroflux}).
\end{description}

%Located on the University Macintosh system is the program \emph{Phebe} - The
%Finite Element/Boundary Element Tutorial Package. This program can calculate
%the elemental stiffness matrix, assemble the global matrix and apply boundary
%conditions to solve for steady state diffusion. It calculates these step by
%step to show the user how each step occurs.


\section{Gaussian Quadrature}
\label{sec:Gquad}

\index{Gaussian quadrature|(} The element integrals arising from two- or
three-dimensional problems can seldom be evaluated analytically. Numerical
integration or \emph{quadrature} is therefore required and the most efficient
scheme for integrating the expressions that arise in the finite element method
is Gauss-Legendre quadrature.

Consider first the problem of integrating $\fnof{f}{\xi}$ between the limits
$0$ and $1$ by the sum of weighted samples of $\fnof{f}{\xi}$ taken at points
$\xi_{1},\xi_{2},\ldots,\xi_{I}$ (see \figref{eqn:integral_eqn}):
\begin{equation*}
  \gint{0}{1}{\fnof{f}{\xi}}{\xi} = \dsuml{i=1}{I}W_{i}\fnof{f}{\xi_{i}} + E
\end{equation*}
Here $W_{i}$ are the weights associated with sample points $\xi_{i}$ -
called \emph{Gauss points} - and $E$ is the error in the approximation of the
integral. We now choose the Gauss points and weights to exactly integrate a
polynomial of degree $2I-1$ (since a general polynomial of degree $2I-1$ has
$2I$ arbitrary coefficients and there are $2I$ unknown Gauss points and
weights).
\begin{figure} \centering
  \input{figs/heat_conduction/Gaussq.pstex}
  \caption{Gaussian quadrature. $\fnof{f}{\xi}$ is sampled at $I$ Gauss points 
  $\xi_{1},\xi_{2} \ldots \xi_{I}.$}
  \label{fig:Gaussian_quadrature}
\end{figure}

For example, with $I=2$ we can exactly integrate a polynomial of degree 3:

\begin{equation*}
  \text{Let}\quad\gint{0}{1}{\fnof{f}{\xi}}{\xi}=W_{1}\fnof{f}{\xione}+
  W_{2}\fnof{f}{\xitwo} 
\end{equation*}
and choose $\fnof{f}{\xi} = a + b\xi + c\xi^{2} + d\xi^{3}$. Then 
\begin{equation}
  \gint{0}{1}{\fnof{f}{\xi}}{\xi} = a\gint{0}{1}{}{\xi} + 
  b\gint{0}{1}{\xi}{\xi} + c\gint{0}{1}{\xi^{2}}{\xi} + 
  d\gint{0}{1}{\xi^{3}}{\xi} 
  \label{eqn:Gq1}
\end{equation}
Since $a$, $b$, $c$ and $d$ are arbitrary coefficients, each integral on the RHS of 
\ref{eqn:Gq1} must be integrated exactly. Thus,
\begin{align}
\gint{0}{1}{}{\xi} &= 1 = W_{1}.1 + W_{2}.1 \label{eqn:Gq2} \\
\gint{0}{1}{\xi}{\xi} &= \dfrac12 = W_{1}.\xi_{1} + W_{2}.\xi_{2} \label{eqn:Gq3}\\
\gint{0}{1}{\xi^{2}}{\xi} &= \dfrac13 = W_{1}.\xi_{1}^{2} + W_{2}.\xi_{2}^{2} \label{eqn:Gq4}\\
\gint{0}{1}{\xi^{3}}{\xi} &= \dfrac14 = W_{1}.\xi_{1}^{3} + W_{2}.\xi_{2}^{3} \label{eqn:Gq5}
\end{align}

These four equations yield the solution for the two Gauss points and weights
as follows: 

From symmetry and \eqnref{eqn:Gq2}, 
\begin{equation*}
  W_{1} = W_{2} =  \dfrac{1}{2}. 
\end{equation*}
Then, from \bref{eqn:Gq3}, 
\begin{equation*}
  \xi_{2}=1-\xi_{1}
\end{equation*}
and, substituting in \bref{eqn:Gq4},
\begin{equation*}
  \xi_{1}^{2} + \pbrac{1- \xi_{1}}^{2} = \dfrac{2}{3}
\end{equation*}
\begin{equation*}
  2\xi_{1}^{2} - 2\xi_{1} + \dfrac{1}{3} = 0,
\end{equation*}  
giving  
\begin{equation*}
  \xi_{1} = \dfrac{1}{2} \pm \dfrac{1}{2\sqrt{3}} .
\end{equation*}
\Eqnref{eqn:Gq5} is satisfied identically. Thus, the two Gauss points are 
given by 
\begin{equation}
\begin{split}
    \xi_{1} &= \dfrac12 - \dfrac{1}{2\sqrt{3}}, \\
    \xi_{2} &= \dfrac12 + \dfrac{1}{2\sqrt{3}}, \\
    W_{1} &= W_{2} = \dfrac12
  \label{eqn:2Gp}
\end{split}
\end{equation}
A similar calculation for a \nth{5} degree polynomial using three Gauss points gives
\begin{alignat}{2}
  \xi_1 &= \dfrac12 - \dfrac12\sqrt{\dfrac35},&\qquad W_{1} &= \dfrac{5}{18}
  \notag \\ \xi_{2} &= \dfrac12, \;\; &\qquad\quad \qquad W_{2} &=
  \dfrac{4}{9} \\ \xi_{3} &= \dfrac12 + \dfrac12\sqrt{\dfrac{3}{5}},&\qquad
  W_{3} &= \dfrac{5}{18} \notag
  \label{eqn:3Gp}
\end{alignat}
For two- or three-dimensional Gaussian quadrature the Gauss point positions
are simply the values given above along each $\xi_{i}$-coordinate with the
weights scaled to sum to $1$ \eg for $2$x$2$ Gauss quadrature the $4$ weights
are all $\dfrac{1}{4}$ . The number of Gauss points chosen for each
$\xi_{i}$-direction is governed by the complexity of the integrand in the
element integral \bref{eqn:element_integrals_2}. In general two- and
three-dimensional problems the integral is not polynomial (owing to the
$\delby{\xi_{i}}{x_{j}}$ terms which come from the inverse of the matrix
$\sqbrac{\delby{x_{i}}{\xi_{j}}}$) and no attempt is made to achieve exact
integration. The quadrature error must be balanced against the discretization
error. 
% (see \secref{sec:Aoe}) 
For example, if the two-dimensional basis is
cubic in the $\xione$-direction and linear in the $\xitwo$-direction, three
Gauss points would be used in the $\xione$-direction and two in the
$\xitwo$-direction. \index{Gaussian quadrature|)}

%\section{Analysis of Error and Convergence Rate}
%\label{sec:Aoe}

\section{CMISS Examples}

\begin{enumerate}
\item  To solve for the steady state temperature distribution inside  a plate run CMISS example $311$ 
\item   To solve for the steady state temperature distribution inside  
an annulus run CMISS example $312$
 \item   To investigate the convergence of the steady state temperature
   distribution with mesh refinement  run CMISS examples $3141$, $3142$, $3143$
   and $3144$.
\end{enumerate}



%%% Local Variables: 
%%% mode: latex
%%% TeX-master: "/product/cmiss/documents/notes/fembemnotes/fembemnotes"
%%% End: 

\clearemptydoublepage
\chapter{The Boundary Element Method} 

\section{Introduction}

Having developed the basic ideas behind the finite element method, we now
develop the basic ideas of the boundary element method.  There are several key
differences between these two methods, one of which involves the choice of
weighting function (recall the Galerkin finite element method used as a
weighting function one of the basis functions used to approximate the solution
variable).  Before launching into the boundary element method we must briefly
develop some ideas that are central to the weighting function used in the
boundary element method.

\section{The Dirac-Delta Function and Fundamental Solutions}

\index{Dirac-Delta function|(}Before one applies the boundary element method
to a particular problem one must obtain a \emph{fundamental
  solution}\index{Fundamental solution} (which is similar to the idea of a
particular solution in ordinary differential equations and is the weighting
function). Fundamental solutions are tied to the Dirac\footnote{Paul A.M.
  Dirac (1902-1994) was awarded the Nobel Prize (with Erwin Schrodinger) in
  1933 for his work in quantum mechanics.  Dirac introduced the idea of the
  ``Dirac Delta'' intuitively, as we will do here, around 1926-27.  It was
  rigorously defined as a so-called generalised function by Schwartz in
  1950-51, and strictly speaking we should talk about the ``Dirac Delta
  Distribution''.} Delta function and we deal with both here.

\subsection{Dirac-Delta function} 

What we do here is very non-rigorous.  To gain an intuitive feel for this 
unusual function, consider the following sequence of force distributions 
applied to a large plate as shown in \Figref{fig:unitf}
\begin{equation*}
  \fnof{w_{n}}{x} = \left\{ \begin{matrix}
      \frac{n}{2} & \abs{x} < \frac{1}{n} \\ 0 & \abs{x} > \frac{1}{n}
    \end{matrix} \right.
\end{equation*}
Each has the property that
\begin{equation*}
    \gint{-\infty}{\infty}{\fnof{w_{n}}{x}}{x} = 1 \qquad 
    \text{ (\ie the total force applied is unity)}
\end{equation*}
but as $n$ increases the area of force application decreases and the 
force/unit area increases.

\begin{figure}[htbp] \centering
  \input{figs/bem/unitf.pstex}
  \caption{Illustrations of unit force distributions $w_{n}$.}
  \label{fig:unitf}
\end{figure}

As $n$ gets larger we can easily see that the area of application of the force
becomes smaller and smaller, the magnitude of the force increases but the
total force applied remains unity.  If we imagine letting $n \rightarrow
\infty$ we obtain an idealised ``point'' force of unit strength, given the
symbol $\fnof{\delta}{x}$, acting at $x$ = 0.  Thus, in a nonrigorous sense we have
\begin{equation*}
    \fnof{\delta}{x} = \displaystyle{\lim_{n \rightarrow \infty}}
    \fnof{w_{n}}{x} \quad \text{the Dirac Delta``function''.}
\end{equation*}

This is not a function that we are used to dealing with because we have
$\fnof{\delta}{x} = 0$ if $x \neq 0$ and ``$\fnof{\delta}{0} = \infty $'' \ie
the ``function'' is zero everywhere except at the origin, where it is
infinite.  However, we have $\gint{-\infty}{\infty}{\fnof{\delta}{x}}{x} = 1$
since each $\gint{-\infty}{\infty}{\fnof{w_{n}}{x}}{x} = 1$.

The Dirac delta ``function'' is not a function in the usual sense, and it is
more correctly referred to as the Dirac delta distribution.  It also has the
property that for any continuous function $\fnof{h}{x}$
\begin{equation}
  \gint{-\infty}{\infty}{\fnof{\delta}{x}\fnof{h}{x}}{x} = \fnof{h}{0}
  \label{eqn:contfn}
\end{equation}
A rough proof of this is as follows
  \begin{alignat*}{2}
    \gint{-\infty}{\infty}{\fnof{\delta}{x}\fnof{h}{x}}{x} &=
    \displaystyle{\lim_{n \rightarrow \infty}}
    \gint{-\infty}{\infty}{\fnof{w_{n}}{x}\fnof{h}{x}}{x} \quad && \text{by
      definition of $\fnof{\delta}{x}$} \\ 
    &= \displaystyle{\lim_{n\rightarrow \infty}} \dfrac{n}{2}
    \gint{-\frac{1}{n}}{\frac{1}{n}}{\fnof{h}{x}}{x} && \text{by definition
      of $\fnof{w_{n}}{x}$} \\ &= \displaystyle{\lim_{n \rightarrow
        \infty}} \dfrac{n}{2} \fnof{h}{\xi}\dfrac{2}{n} && \text{by the Mean
      Value Theorem, where $\xi \in\pbrac{-\dfrac{1}{n},\dfrac{1}{n}}$} \\ &=
    \fnof{h}{0} && \text{since $\xi \in\pbrac{-\dfrac{1}{n},\dfrac{1}{n}}$ and
      as $n\rightarrow \infty,\medspace \xi \rightarrow 0$}
  \end{alignat*}

The above result (\Eqnref{eqn:contfn}) is often used as the defining property of
the Dirac delta in more rigorous derivations.  One does not usually talk about
the values of the Dirac delta at a particular point, but rather its integral
behaviour. Some properties of the Dirac delta are listed below
\begin{equation}
  \gint{-\infty}{\infty}{\fnof{\delta}{\xi - x}\fnof{h}{x}}{x} = \fnof{h}{\xi}
  \label{eqn:Ddelta}
\end{equation} 
(Note: $\fnof{\delta}{\xi - x}$ is the Dirac delta distribution centred at $x = \xi$
instead of $x = 0$) 
\begin{equation}
  \fnof{\delta}{\xi - x} = \fnof{H'}{\xi - t} 
  \label{eqn:Ddelta2}
\end{equation}

where \fnof{H}{\xi - t} = 
   $\begin{cases}$
     $0 & \text{if }  \xi<t \\$
     $1 & \text{if }  \xi>t$
   $\end{cases}$
(\ie the Dirac Delta function is the slope of the Heaviside\footnote{Oliver 
  Heaviside (1850-1925) was a British physicist, who pioneered the
  mathematical study of electrical circuits and helped develop vector
  analysis.} step function.)
\begin{equation}
  \fnof{\delta}{\xi - x,\eta - y} = \fnof{\delta}{\xi - x}\fnof{\delta}{\eta - y}
  \label{eqn:Ddelta3}
\end{equation}
(\ie the two dimensional Dirac delta is just a product of two one-dimensional
Dirac deltas.)\index{Dirac-Delta function|)}

\subsection{Fundamental solutions}

\index{Fundamental solution|(}We develop here the fundamental solution (also
called the freespace Green's\footnote{George Green (1793-1841) was a
  self-educated miller's son.  Most widely known for his integral theorem (the
  Green-Gauss theorem).} function) for Laplace's Equation in two variables.
The fundamental solution of a particular equation is the weighting function
that is used in the boundary element formulation of that equation.  It is
therefore important to be able to find the fundamental solution for a
particular equation.  Most of the common equations have well-known fundamental
solutions (see \Appendref{app:fundamentalsolutions}). We briefly illustrate
here how to find a simple fundamental solution.

Consider solving the Laplace Equation $\deltwosqby{u}{x} + \deltwosqby{u}{y} 
= 0$ in some domain $\Omega \in \Re^{2}$. %Check this 
\index{Fundamental solution!Laplace}

The fundamental solution for this equation (analogous to a particular solution
in ODE work) is a solution of
\begin{equation}
  \deltwosqby{\omega}{x} + \deltwosqby{\omega}{y} + \fnof{\delta}{\xi - x,\eta - y} = 0
  \label{eqn:Fsoln}
\end{equation}
in $\Re^{2}$ (\ie we solve the above without reference to the original domain
$\Omega$ or original boundary conditions). The method is to try and find
solution to $\laplacian{\omega}=0$ in $\Re^{2}$ which contains a singularity at the
point $\pbrac{\xi,\eta}$. This is not as difficult as it sounds.  We expect
the solution to be symmetric about the point $\pbrac{\xi,\eta}$ since
$\fnof{\delta}{\xi - x,\eta - y}$ is symmetric about this point.  So we adopt a
local polar coordinate system about the \emph{singular point} $\pbrac{\xi,\eta}$.

Let  
\begin{equation*}
  r = \sqrt{\pbrac{\xi - x}^{2} + \pbrac{\eta - y}^{2}}
\end{equation*}      
Then, from \Secref{sec:ccs-1.8} we have
\begin{equation}
  \laplacian{\omega} = \dfrac{1}{r}\delby{ }{r} \pbrac{r\delby{\omega}{r}}
  + \dfrac{1}{r^{2}} \deltwosqby{\omega}{\theta}    
  \label{eqn:Fsoln2}
\end{equation}
For $r > 0, \fnof{\delta}{\xi - x,\eta - y} = 0$ and owing to symmetry,
$\deltwosqby{\omega}{\theta}$ is zero.  Thus \eqnref{eqn:Fsoln2} becomes
\begin{equation*}
  \dfrac{1}{r}\delby{ }{r}\pbrac{r\delby{\omega}{r}} = 0 
\end{equation*}
This can be solved by straight (one-dimensional) integration.  The solution is 
\begin{equation}
  \omega = A \log r + B
  \label{eqn:Fsoln3}
\end{equation}
Note that this function is singular at $r = 0$ as required.

To find $A$ and $B$ we make use of the integral property of the Delta
function. From \eqnref{eqn:Fsoln} we must have
\begin{equation}
  \goneint{\laplacian{\omega}}{D} = - \goneint{\delta}{D} = -1
  \label{eqn:Fsoln4}
\end{equation}
where $D$ is any domain containing $r = 0$.

\begin{figure}[htbp] \centering
  \input{figs/bem/domused.pstex}
  \caption{Domain used to evaluate fundamental solution coefficients.}
  \label{fig:domused}
\end{figure}

We choose a simple domain to allow us to evaluate the above integrals. If $D$
is a small disk of radius $\varepsilon > 0$ centred at $r = 0$
(\Figref{fig:domused}) then from the Green-Gauss theorem
\begin{alignat*}{2}
    \goneint{\laplacian{\omega}}{D} &= \gint{\del D}{}{\delby{\omega}{n}}{S}
    \qquad && \text{$\del D$ is the surface of the disk $D$} \\
    &= \gint{\del D}{}{\delby{\omega}{r}}{S} && 
    \text{since $D$ is a disk centred at $r=0$ so $n$ and $r$ are in the 
      same direction} \\ 
    &= \dfrac{A}{\varepsilon} 2\pi \varepsilon && 
   \text{from \eqnref{eqn:Fsoln3}, and the fact that $D$ is a disc of 
      radius $\varepsilon$} \\
    &=2 \pi A &&
\end{alignat*}

Therefore, from \eqnref{eqn:Fsoln4}
\begin{equation*}
  A = -\dfrac{1}{2 \pi}.
\end{equation*}
So we have  
\begin{equation*}
  \omega  =  - \dfrac{1}{2\pi}\log r  + B
\end{equation*} 
$B$ remains arbitrary but usually put equal to zero, so that the fundamental
solution for the two-dimensional Laplace Equation is
\begin{equation}
  \omega  =  - \dfrac{1}{2\pi}\log r \quad \pbrac{= \dfrac{1}{2\pi}\log\dfrac{1}{r}}
  \label{eqn:2dLe}
\end{equation}
where $r = \sqrt{\pbrac{\xi - x}^{2} + \pbrac{\eta - y}^{2}}$ (singular at the
point $\pbrac{\xi,\eta}$).

The fundamental solution for the three-dimensional Laplace Equation can be 
found by a similar technique.  The result is
\begin{equation*}
  \omega = \dfrac{1}{4\pi r}
\end{equation*}
where $r$ is now a distance measured in three-dimensions.
\index{Fundamental solution|)}

\section{The Two-Dimensional Boundary Element Method}
\label{sec:The2-Dbem}

We are now at a point where we can develop the boundary element method for the
solution of $\laplacian{u} = 0$ in a two-dimensional domain $\Omega$.
The basic steps are in fact quite similar to those used for the finite element
method (refer \Secref{sec:OdSSHC-2.1}).  We firstly must form an integral
equation from the Laplace Equation by using a weighted integral equation and
then use the Green-Gauss theorem.  From \secref{sec:2and3-DSSHC} we have
seen that
\begin{equation}
  0 = \goneint{\laplacian{u}.\omega}{\Omega} 
    = \gint{\del \Omega}{}{\delby{u}{n} \omega}{\Gamma}
    - \goneint{\grad u. \grad \omega}{\Omega}
  \label{eqn:2Dbe}
\end{equation}

This was the starting point for the finite element method. To derive the 
starting equation for the boundary element method we use the Green-Gauss 
theorem again on the second integral. This gives
\begin{equation}
  \begin{split}
    0 &= \gint{\del\Omega}{}{\delby{u}{n} \omega}{\Gamma}
    - \goneint{\grad u. \grad \omega}{\Omega} \\
    &= \gint{\del\Omega}{}{\delby{u}{n} \omega}{\Gamma}
    - \gint{\del\Omega}{}{u \delby{\omega}{n}}{\Gamma} 
    + \goneint{u \laplacian{\omega}}{\Omega} 
  \label{eqn:startingeq}
  \end{split}
\end{equation}
For the Galerkin FEM we chose $\omega$, the weighting function, to be $\lbfnsymb{m}$,
one of the basis functions used to approximate $u$.  For the boundary element
method we choose $\omega$ to be the fundamental solution of Laplace's Equation
derived in the previous section \ie
\begin{equation*}
  \omega = -\dfrac{1}{2\pi} \log r
\end{equation*}
where $r = \sqrt{\pbrac{\xi - x}^{2} + \pbrac{\eta - y}^{2}}$ (singular at the point
$\pbrac{\xi,\eta}\in \Omega$).

Then from \eqnref{eqn:startingeq}, using the property of the Dirac delta
\begin{equation}
  \goneint{u \laplacian{\omega}}{\Omega} = - \goneint{u \fnof{\delta}{\xi -
      x,\eta - y}}{\Omega} = -\fnof{u}{\xi,\eta} \quad \pbrac{\xi,\eta} \in
  \Omega
  \label{eqn:propdd}
\end{equation}
\ie the domain integral has been replaced by a point value.

Thus \eqnref{eqn:startingeq} becomes
\begin{equation}
  \fnof{u}{\xi,\eta} + \gint{\del\Omega}{}{u \delby{\omega}{n}}{\Gamma} =
  \gint{\del \Omega}{}{\delby{u}{n} \omega}{\Gamma} \quad \pbrac{\xi,\eta} \in
  \Omega
  \label{eqn:seq2}
\end{equation}
This equation contains only boundary integrals (and no domain integrals as in
Finite Elements) and is referred to as a boundary integral equation.  It
relates the value of $u$ at some point inside the solution domain to integral
expressions involving $u$ and $\delby{u}{n}$ over the boundary of the solution
domain.  Rather than having an expression relating the value of $u$ at some
point inside the domain to boundary integrals, a more useful expression would
be one relating the value of $u$ at some point \emph{on the boundary} to
boundary integrals.  We derive such an expression below.

The previous equation (\Eqnref{eqn:seq2}) holds if $\pbrac{\xi,\eta} \in
\Omega$ (\ie the singularity of Dirac Delta function is inside the domain). If
$\pbrac{\xi,\eta}$ is outside $\Omega$ then
\begin{equation*}
  \goneint{u \laplacian{\omega}}{\Omega}  = 
  - \goneint{u\fnof{\delta}{\xi - x,\eta - y}}{\Omega} = 0
\end{equation*}
since the integrand of the second integral is zero at every point except
$\pbrac{\xi,\eta}$ and this point is outside the region of integration.  The
case which needs special consideration is when the singular point
$\pbrac{\xi,\eta}$ is on the boundary of the domain $\Omega$.  This case also
happens to be the most important for numerical work as we shall see.  The
integral expression we will ultimately obtain is simply \eqnref{eqn:seq2} with
$\fnof{u}{\xi,\eta}$ replaced by $\dfrac{1}{2} \fnof{u}{\xi,\eta}$.  We can
see this in a non-rigorous way as follows.  When $\pbrac{\xi,\eta}$ was inside
the domain, we integrated around the entire singularity of the Dirac Delta to
get $\fnof{u}{\xi,\eta}$ in \eqnref{eqn:seq2}.  When $\pbrac{\xi,\eta}$ is on
the boundary we only have half of the singularity contained inside the domain,
so we integrate around one-half of the singularity to get $\dfrac{1}{2}
\fnof{u}{\xi,\eta}$.  Rigorous details of where this coefficient
$\dfrac{1}{2}$ comes from are given below.

Let $P$ denote the point $\pbrac{\xi,\eta} \in \Omega$.  In order to be able to
evaluate $\goneint{u \laplacian{\omega}}{\Omega}$ in this case we
enlarge $\Omega$ to include a disk of radius $\varepsilon$ about $P$
(\Figref{fig:illus}).  We call this enlarged region $\Omega^{\prime}$ and
let $\Gamma^{\prime} = \Gamma_{-\varepsilon} \cup \Gamma_{\varepsilon}$.

\begin{figure}[htbp] \centering
  \input{figs/bem/illus.pstex}
  \caption{Illustration of enlarged domain when singular point is on the 
    boundary.}
  \label{fig:illus}
\end{figure}

Now, since $P$ is inside the enlarged region $\Omega^{\prime}$, \eqnref{eqn:seq2}
holds for this enlarged domain \ie
\begin{equation}
  \fnof{u}{P}+ \gint{\Gamma_{-\varepsilon} \cup \Gamma_{\varepsilon}}{}{u  
  \delby{\omega}{n}}{\Gamma} = \gint{\Gamma_{-\varepsilon} \cup 
    \Gamma_{\varepsilon}}{}{\delby{u}{n}\omega}{\Gamma} 
  \label{eqn:endom}
\end{equation}
We must now investigate this equation as $\limita{\varepsilon}{0}$.  There
are $4$ integrals to consider, and we look at each of these in turn.

Firstly consider 
  \begin{alignat*}{2}
    \gint{\Gamma_{\varepsilon}}{}{u \delby{\omega}{n}}{\Gamma} 
    &= \gint{\Gamma_{\varepsilon}}{}{u \delby{ }{n} 
    \pbrac{-\dfrac{1}{2 \pi} \log r}}{\Gamma} \qquad && 
    \text{by definition of $\omega$} \\
    &= \gint{\Gamma_{\varepsilon}}{}{u \delby{ }{r} 
    \pbrac{-\dfrac{1}{2 \pi} \log r}}{\Gamma} &&
    \text{since $\delby{ }{n} \equiv \delby{ }{ r}$ on 
      $\Gamma_{\varepsilon}$} \\
    &= -\dfrac{1}{2 \pi} \gint{\Gamma_{\varepsilon}}{}{\dfrac{u}{r}}{\Gamma} && \\
    &= -\dfrac{1}{2 \pi} \dfrac{1}{\varepsilon} 
    \gint{\Gamma_{\varepsilon}}{}{u}{\Gamma} && 
    \text{since $r =\varepsilon$ on $\Gamma_{\varepsilon}$} \\ 
    &\rightarrow -\dfrac{1}{2 \pi} \dfrac{1}{\varepsilon} \fnof{u}{P} \pi 
    \varepsilon &&
  \end{alignat*}
by the mean value theorem for a surface with a unique tangent at $P$. 

Thus
\begin{equation}
  \limita{\varepsilon}{0}\gint{\Gamma_{\varepsilon}}{}{u  
  \delby{\omega}{n}}{\Gamma} =  \limita{\varepsilon}{0} 
  \pbrac{-\dfrac{1}{2 \pi} \dfrac{\fnof{u}{P}}{\varepsilon} \pi{\varepsilon}} 
  = - \dfrac{\fnof{u}{P}}{2} 
  \label{eqn:integ1}
\end{equation}
By a similar process we obtain
\begin{equation}
  \limita{\varepsilon}{0}\gint{\Gamma_{\varepsilon}}{}{\omega  
  \delby{u}{n}}{\Gamma} = \limita{\varepsilon}{0} 
  \pbrac{-\dfrac{1}{2 \pi} \fnof{\delby{u}{n}}{P} \pi \varepsilon \log
    \varepsilon} = 0 
  \label{eqn:integ2}
\end{equation}
since $\limita{\varepsilon \log \varepsilon}{0}$ as $\limita{\varepsilon}{0}$.

It only remains to consider the integrand over $\Gamma_{-\varepsilon}$.  For
``nice'' integrals (which includes the integrals we are dealing with here) we
have
\begin{equation*}
  \limita{\varepsilon}{0}\pbrac{\gint{\;\; \Gamma_{-\varepsilon}}{}
    {\text{(nice integrand)}}{\Gamma}} = \gint{\Gamma}{}{\text{(nice
    integrand)}}{\Gamma} 
\end{equation*}
since $\Gamma_{-\varepsilon} \rightarrow \Gamma$ as $\limita{\varepsilon}{0}$.

\textbf{Note}: If the integrand is too badly behaved we cannot always
replace $\Gamma_{-\varepsilon}$ by $\Gamma$ in the limit and one must deal
with Cauchy Principal Values.  (refer \Secref{sec:BIE,sec4.10})

Thus we have
\begin{align}
  \lim_{\varepsilon \rightarrow 0}\pbrac{\;\;\gint{\Gamma_{-\varepsilon}}{}
    {\delby{u}{n} \omega}{\Gamma}} &= \gint{\Gamma}{}{\delby{u}{n} \omega}{\Gamma}  
  \label{eqn:integ3} \\
  \lim_{\varepsilon \rightarrow 0}\pbrac{\;\;\gint{\Gamma_{-\varepsilon}}{}
    {\delby{\omega}{n} u}{\Gamma}} &= \gint{\Gamma}{}{\delby{\omega}{n} u}{\Gamma}  
  \label{eqn:integ4}
\end{align}

Combining \eqnthrurefs{eqn:endom}{eqn:integ4} we get
\begin{equation*}
  \fnof{u}{P} + \goneint{u \delby{\omega}{n}}{\Gamma} 
  = \dfrac{1}{2} \fnof{u}{P} + \goneint{\delby{u}{n} \omega}{\Gamma}
\end{equation*} 
or
\begin{equation*}
  \dfrac{1}{2} \fnof{u}{P} + \goneint{u \delby{\omega}{n}}{\Gamma}  = 
  \goneint{\delby{u}{n} \omega}{\Gamma} 
\end{equation*}
where $P = \pbrac{\xi,\eta} \in \del \Omega$ (\ie singular point is on the boundary
of the region).

\textbf{Note}: The above is true if the point $P$ is at a smooth point (\ie
a point with a unique tangent) on the boundary of $\Omega$.  If $P$ happens to
lie at some nonsmooth point e.g. a corner, then the coefficient $\dfrac{1}{2}$
is replaced by $\dfrac{\alpha}{2 \pi}$ where $\alpha$ is the internal angle at
$P$ (\Figref{fig:intang}).

\begin{figure}[htbp] \centering
  \input{figs/bem/intang.pstex}
  \caption{Illustration of internal angle $\alpha$.}
  \label{fig:intang}
\end{figure}

Thus we get the boundary integral equation.
\begin{equation}
  \fnof{c}{P} \fnof{u}{P} + \goneint{u \delby{\omega}{n}}{\Gamma}  = 
  \goneint{\delby{u}{n} \omega}{\Gamma}
  \label{eqn:bie}
\end{equation}
where 
\begin{align*}
  \omega &= - \dfrac{1}{2 \pi} \log r \\
  r &= \sqrt{\pbrac{\xi - x}^{2} + \pbrac{\eta - y}^{2}}\\
  \fnof{c}{P} &= \left\{ \begin{array}{cll}
      1 & & \text{if } P \in \Omega  \\
      \frac{1}{2} & &\text{if } P \in \Gamma \text{ and $\Gamma$ smooth at $P$}\\
      \dfrac{\text{internal angle}}{2 \pi} & & \text{if } P \in \Gamma 
      \text{ and $\Gamma$ not smooth at $P$}
    \end{array} \right.
\end{align*}

For three-dimensional problems, the boundary integral equation expression
above is the same, with
\begin{align*}
  \omega &= \dfrac{1}{4 \pi r} \\
  r &= \sqrt{\pbrac{\xi - x}^{2} + \pbrac{\eta - y}^{2} +\pbrac{\gamma - z}^{2}} \\
  \fnof{c}{P} &= \left\{ \begin{array}{cll}
      1 & & \text{if } P \in \Omega  \\
      \frac{1}{2} & & \text{if } P \in \Gamma \text{ and $\Gamma$ smooth at $P$}\\
      \dfrac{\text{inner solid angle}}{4 \pi} & & \text{if } P \in \Gamma  
      \text{ and $\Gamma$ not smooth at $P$}
    \end{array} \right.
\end{align*}

\Eqnref{eqn:bie} involves only the surface distributions of $u$ and
$\delby{u}{n}$ and the value of $u$ at a point $P$.  Once the surface
distributions of $u$ and $\delby{u}{n}$ are known, the value of $u$ at
any point $P$ inside $\Omega$ can be found since all surface integrals in
\eqnref{eqn:bie} are then known. The procedure is thus to use \eqnref{eqn:bie} to
find the surface distributions of $u$ and $ \delby{u}{n}$ and then
(if required) use \eqnref{eqn:bie} to find the solution at any point $P \in
\Omega$ .  Thus we solve for the boundary data first, and find the volume data
as a separate step.

Since \eqnref{eqn:bie} only involves surface integrals, as opposed to volume
integrals in a finite element formulation, the overall size of the problem has
been reduced by one dimension (from volumes to surfaces).  This can result in
huge savings for problems with large volume to surface ratios (\ie problems
with large domains).  Also the effort required to produce a volume mesh of a
complex three-dimensional object is far greater than that required to produce
a mesh of the surface.  Thus the boundary element method offers some distinct
advantages over the finite element method in certain situations.  It also has
some disadvantages when compared to the finite element method and these will
be discussed in \Secref{sec:The3-DBE}.  We now turn our attention to
solving the boundary integral equation given in \eqnref{eqn:bie}.

\section{Numerical Solution Procedures for the Boundary Integral Equation} 
\label{sec:Numsol}

The first step is to discretise the surface $\Gamma$ into some set of elements
(hence the name boundary elements).
\begin{equation}
  \Gamma = \displaystyle\bigcup_{j=1}^{N} \Gamma_{j}
  \label{eqn:ns1}
\end{equation}

\begin{figure}[htbp] \centering
  \input{figs/bem/schemi.pstex}
  \caption{Schematic illustration of a boundary element mesh (a) and a finite 
    element mesh (b).}
  \label{fig:schemi}
\end{figure}

Then \eqnref{eqn:bie} becomes
\begin{equation}
  \fnof{c}{P} \fnof{u}{P} + \dsuml{j=1}{N} \gint{\Gamma_{j}}{}{u \delby{\omega}{n}}  
  {\Gamma}  = \dsuml{j=1}{N} \gint{\Gamma_{j}}{}{\delby{u}{n} \omega}{\Gamma}
  \label{eqn:ns2}
\end{equation}
Over each element $\Gamma_{j}$ we introduce standard (finite element) 
basis functions\index{Basis functions}
\begin{equation}
  u_{j} = \dsuml{\alpha}{} \lbfnsymb{\alpha} u_{j\alpha} \quad \text{and} \quad
  q_{j} \equiv \delby{u_{j}}{ n} = \dsuml{\alpha}{} \lbfnsymb{\alpha} q_{j\alpha}
  \label{eqn:febf}
\end{equation}
where $u_{j}, q_{j}$ are values of $u$ and $q$ on element $\Gamma_{j}$ and
$u_{j\alpha}, q_{j\alpha}$ are values of $u$ and $q$ at node $\alpha$ on
element $\Gamma_{j}$.

These basis functions for $u$ and $q$ can be any of the standard
one-dimensional finite element basis functions (although we are dealing with a
two-dimensional problem, we only have to interpolate the functions over a
one-dimensional element).  In general the basis functions used for $u$ and $q$
do not have to be the same (typically they are) and these basis functions can
even be different to the basis functions used for the geometry, but are
generally taken to be the same (this is termed an 
isoparametric formulation\index{Isoparametric formulation}).

This gives
\begin{equation}
  \fnof{c}{P} \fnof{u}{P}+ \dsuml{j=1}{N} \dsuml{\alpha}{} u_{j\alpha} 
  \gint{\Gamma_{j}}{}{\lbfnsymb{\alpha} \delby{\omega}{n}}{\Gamma} = 
  \dsuml{j=1}{N}\dsuml{\alpha}{}
  q_{j\alpha}\gint{\Gamma_{j}}{}{\lbfnsymb{\alpha} \omega}{\Gamma}
  \label{eqn:isopar}
\end{equation}
This equation holds for any point $P$ on the surface $\Gamma$.  We now
generate one equation per node by putting the point $P$ to be at each node in
turn.  If $P$ is at node $i$, say, then we have
\begin{equation}
  c_{i}u_{i}+\dsuml{j=1}{N} \dsuml{\alpha}{} u_{j\alpha} 
  \gint{\Gamma_{j}}{}{\lbfnsymb{\alpha} \delby{\omega_i}{n}}{\Gamma}  = 
  \dsuml{j=1}{N}\dsuml{\alpha}{} q_{j\alpha}\gint{\Gamma_{j}}{}{\lbfnsymb{\alpha} 
  \omega_{i}}{\Gamma}
  \label{eqn:isopar2}
\end{equation}
where $\omega_{i}$ is the fundamental solution with the singularity at node $i$
(recall $\omega$ is $-\dfrac{1}{2 \pi} \log r$ , where $r$ is the distance from the
singularity point). We can write \eqnref{eqn:isopar2} in a more abbreviated form
as
\begin{equation}
  c_{i} u_{i} + \dsuml{j=1}{N} \dsuml{\alpha}{} u_{j\alpha}a_{ij}^{\alpha} =
  \dsuml{j=1}{N} \dsuml{\alpha}{} q_{j\alpha}b_{ij}^{\alpha}
  \label{eqn:isopar3}
\end{equation}
where
\begin{equation}
  a_{ij}^{\alpha} = \gint{\Gamma_{j}}{}{\lbfnsymb{\alpha}
    \delby{\omega_i}{n}}{\Gamma} \quad \text{and} \quad  b_{ij}^{\alpha} = 
  \gint{\Gamma_{j}}{}{\lbfnsymb{\alpha} \omega_{i}}{\Gamma}
  \label{eqn:isopar4}
\end{equation}
\Eqnref{eqn:isopar3} is for node $i$ and if we have $L$ nodes, then we can
generate $L$ equations.  

We can assemble these equations into the matrix system
\begin{equation}
  \matr{A}\vect{u} = \matr{B}\vect{q}
  \label{eqn:ms}
\end{equation}
(compare to the global finite element equations $\matr{K}\vect{u} = \vect{f}$)
where the vectors $\vect{u}$ and $\vect{q}$ are the vectors of nodal values of
$u$ and $q$.  Note that the \nth{ij} component of the $\matr{A}$ matrix in general
is \emph{not} $a_{ij}^{\alpha}$ and similarly for $\matr{B}$.

\index{Boundary conditions!application of}At each node, we must specify either
a value of $u$ or $q$ (or some combination of these) to have a well-defined
problem.  We therefore have $L$ equations (the number of nodes) and have $L$
unknowns to find.  We need to rearrange the above system of equations to get
\begin{equation}
  \matr{C}\vect{x} = \vect{f}
  \label{eqn:ms2}
\end{equation}
where $\vect{x}$ is the vector of unknowns.  This can be solved using standard
linear equation solvers, although specialist solvers are required if the
problem is large (refer \todo{todo : Section ???}).

The matrices $\matr{A}$ and $\matr{B}$ (and hence $\matr{C}$) are fully populated
and not symmetric (compare to the finite element formulation where the global
stiffness matrix $\matr{K}$ is sparse and symmetric).  The size of the $\matr{A}$
and $\matr{B}$ matrices are dependent on the number of surface nodes, while the
matrix $\matr{K}$ is dependent on the number of finite element nodes (which
include nodes in the domain). As mentioned earlier, it depends on the surface
to volume ratio as to which method will generate the smallest and quickest
solution.

The use of the fundamental solution as a weight function ensures that the
$\matr{A}$ and $\matr{B}$ matrices are generally well conditioned (see
\Secref{sec:Numevalcoeff} for more on this).  In fact the $\matr{A}$ matrix is
diagonally dominant (at least for Laplace's equation). The matrix $\matr{C}$ is
therefore also well conditioned and \eqnref{eqn:ms2} can be solved reasonably
easily.

The vector $\vect{x}$ contains the unknown values of $\vect{u}$ and $\vect{q}$ on
the boundary.  Once this has been found, all boundary values of $\vect{u}$ and
$\vect{q}$ are known.  If a solution is then required at a point inside the
domain, then we can use \eqnref{eqn:isopar3} with the singular point $P$ located
at the required solution point \ie
\begin{equation}
  \fnof{u}{P} = \dsuml{j=1}{N} \dsuml{\alpha}{} q_{j\alpha}b_{Pj}^{\alpha}  - 
  \dsuml{j=1}{N} \dsuml{\alpha}{} u_{j\alpha}a_{Pj}^{\alpha}
  \label{eqn:rsp}
\end{equation}
The right hand side of \eqnref{eqn:rsp} contains no unknowns and only involves
evaluating the surface integrals using the fundamental solution with the
singular point located at $P$.

\section{Numerical Evaluation of Coefficient Integrals}
\label{sec:Numevalcoeff}

We consider in detail here how one evaluates the $a_{ij}^{\alpha}$ and
$b_{ij}^{\alpha}$ integrals for two-dimensional problems.  These integrals
typically must be evaluated numerically, and require far more work and effort
than the analogous finite element integrals.

Recall that 
\begin{equation*}
  a_{ij}^{\alpha} = \gint{\Gamma_{j}}{}{\lbfnsymb{\alpha}
    \delby{\omega_i}{n}}{\Gamma} \quad \text{and} \quad b_{ij}^{\alpha} = 
  \gint{\Gamma_{j}}{}{\lbfnsymb{\alpha} \omega_{i}}{\Gamma}
\end{equation*}
where 
\begin{align*}
  \omega_{i} &= - \dfrac{1}{2 \pi} \log r_{i}\\
  r_{i} &= \text{distance measured from node $i$}
\end{align*} 

In terms of a local $\xi$ coordinate we have
\begin{align}
  b_{ij}^{\alpha} &= \gint{0}{1}{\lbfn{\alpha}{\xi} \fnof{\omega_{i}}{\xi}
    \abs{\fnof{J}{\xi}}}{\xi} \\
  a_{ij}^{\alpha} &= \gint{\Gamma_{j}}{}{\lbfn{\alpha}{\xi} 
  \delby{\fnof{\omega_{i}}{\xi}}{n} \abs{\fnof{J}{\xi}}}{\xi} = \gint{0}{1} 
  {\lbfn{\alpha}{\xi} \fnof{\delby{\omega_{i}}{r_{i}}}{\xi}  \dby{r_{i}}{n}
    \abs{\fnof{J}{\xi}}}{\xi}
\end{align}
  
The Jacobian $\fnof{J}{\xi}$ can be found by
\begin{equation}
  \fnof{J}{\xi} = \dby{\Gamma}{\xi} = \dby{s}{\xi} = 
  \sqrt{\pbrac{\dby{x}{\xi}}^{2} + \pbrac{\dby{y}{\xi}}^{2}}
  \label{eqn:Jacobian}
\end{equation}
where $s$ represents the arclength and $\dby{x}{\xi}$ and $\dby{y}{\xi}$ can be found by
straight differentiation of the interpolation expression for $\fnof{x}{\xi}$ and
$\fnof{y}{\xi}$.

The fundamental solution is
\begin{align*}
  \omega_{i} &= -\dfrac{1}{2 \pi} \log \pbrac{\fnof{r_{i}}{\xi}} \\
  \fnof{r_{i}}{\xi} &= \sqrt{\pbrac{\fnof{x}{\xi}-x_{i}}^{2}+
    \pbrac{\fnof{y}{\xi}-y_{i}}^{2}}    
\end{align*}
where $\pbrac{x_{i},y_{i}}$ are the coordinates of node $i$.

To find $\dby{r_{i}}{n}$  we note that  
\begin{equation}
  \dby{r_{i}}{n} = \dotprod{\grad r_{i}}{\hat{\vect{n}}}
  \label{eqn:dridn}
\end{equation}
where $\hat{\vect{n}}$ is a unit outward normal vector.  To find a unit normal
vector, we simply rotate the tangent vector (given by
$\pbrac{\fnof{x'}{\xi},\fnof{y'}{\xi}}$ ) by $\dfrac{\pi}{2}$ in the
appropriate direction and then normalise.

Thus every expression in the integrands of the $a_{ij}^{\alpha}$ and
$b_{ij}^{\alpha}$ integrals can be found at any value of $\xi$, and the
integrals can therefore be evaluated numerically using some suitable
quadrature schemes.

If node $i$ is well removed from element $\Gamma_{j}$ then standard 
\Index{Gaussian quadrature} can be used to evaluate these integrals.  However,
if node $i$ is in $\Gamma_{j}$ (or close to it) we see that $r_{i}$ approaches
0 and the fundamental solution $\omega_{i}$ tends to $\infty$.  The integral
still exists, but the integrand becomes singular.  In such cases special care
must be taken - either by using special quadrature schemes, large numbers of
Gauss points or other special treatment.

\begin{figure}[htbp] \centering
  \input{figs/bem/decri.pstex}
  \caption{Illustration of the decrease in $r_{i}$ as node $i$ approaches 
    element $\Gamma_{j}$.}
  \label{fig:decri}
\end{figure}

The integrals for which node $i$ lies in element $\Gamma_{j}$ are in general
the largest in magnitude and lead to the diagonally dominant matrix equation.
It is therefore important to ensure that these integrals are calculated as
accurately as possible since these terms will have most influence on the
solution.  This is one of the disadvantages of the BEM - the fact that
singular integrands must be accurately integrated.

A relatively straightforward way to evaluate all the integrals is simply to use
Gaussian quadrature with varying number of quadrature points, depending on how
close or far the singular point is from the current element.  This is not very
elegant or efficient, but has the benefit that it is relatively easy to
implement.  For the case when node $i$ is contained in the current element one
can use special quadrature schemes which are designed to integrate log-type
functions.  These are to be preferred when one is dealing with Laplace's
equation.  However, these special log-type schemes cannot be so readily used
on other types of fundamental solution so for a general purpose
implementation, Gaussian quadrature is still the norm. It is possible to
incorporate adaptive integration schemes that keep adding more quadrature
points until some error estimate is small enough, or also to subdivide the
current element into two or more smaller elements and evaluate the integral
over each subelement.  It is also possible to evaluate the ``worst'' integrals
by using simple solutions to the governing equation, and this technique is the
norm for elasticity problems (\Secref{sec:BIE,sec4.10}). Details on each
of these methods is given in \Secref{sec:MNI}.  It should be noted that
research still continues in an attempt to find more efficient ways of
evaluating the boundary element integrals.

%\begin{example}{2D steady-state heat conduction inside an annulus}
%  {2D steady-state heat conduction inside an annulus}
  
%  To determine the steady-state heat conduction inside an annulus run the
%  CMISS example ``example34''.

%  \label{xmp:2Dsshc} 
%\end{example}

\section{The Three-Dimensional Boundary Element Method}
\label{sec:The3-DBE}

The three-dimensional boundary element method is very similar to the
two-dimensional boundary element method discussed above.  As noted above, the
three-dimensional boundary integral equation is the same as the
two-dimensional equation \bref{eqn:bie}, with $\omega$ and $\fnof{c}{P}$ being
defined as in \Secref{sec:The2-Dbem}.  The numerical solution procedure also
parallels that given in \Secref{sec:Numsol}, and the expressions given for
$a_{ij}^{\alpha}$ and $b_{ij}^{\alpha}$ apply equally well to the
three-dimensional case.  The only real difference between the two procedures
is how to numerically evaluate the terms in each integrand of these
coefficient integrals.

As in \Secref{sec:Numevalcoeff} we illustrate how to evaluate each of the
terms in the integrand of $a_{ij}^{\alpha}$ and $b_{ij}^{\alpha}$.  The
relevant expressions are
\begin{align}
  a_{ij}^{\alpha} &= \gint{\Gamma_{j}}{}{\lbfnsymb{\alpha}
    \delby{\omega_{i}}{n}}{\Gamma} \nonumber \\ &=
  \gint{0}{1}{\gint{0}{1}{\fnof{\lbfnsymb{\alpha}}{\xione,\xitwo}
      \fnof{\delby{\omega_{i}}{r_{i}}}{\xione,\xitwo} \dby{r_{i}}{n}
      \abs{\fnof{J}{\xione,\xitwo}}}{\xione}}{\xitwo} \\ 
%%
  b_{ij}^{\alpha}
  &= \gint{\Gamma_{j}}{}{\lbfnsymb{\alpha} \omega_{i}}{\Gamma} \nonumber \\ 
  &= \gint{0}{1}{\gint{0}{1}{\fnof{\lbfnsymb{\alpha}} {\xione,\xitwo}
      \fnof{\omega_{i}}{\xione,\xitwo} \abs{\fnof{J}{\xione,\xitwo}}}
    {\xione}}{\xitwo}
\end{align}

The fundamental solution is 
\begin{alignat*}{2}
  && \fnof{\omega_{i}}{\xione,\xitwo} &= \dfrac{1}{4 \pi
    \fnof{r_{i}}{\xione,\xitwo}} \\ \text{where} && \fnof{r_{i}}{\xione,\xitwo}
  &= \sqrt{\pbrac{\fnof{x}{\xione,\xitwo}-x_{i}}^{2} +
    \pbrac{\fnof{y}{\xione,\xitwo}-y_{i}}^{2} +
    \pbrac{\fnof{z}{\xione,\xitwo}-z_{i}}^{2}}
\end{alignat*}
where $\pbrac{x_{i},y_{i},z_{i}}$ are the coordinates of node $i$.  As before
we use $\dby{r_{i}}{n} = \dotprod{\grad r_{i}}{\hat{\vect{n}}}$ to find
$\dby{r_{i}}{n}$.  The unit outward normal $\hat{\vect{n}}$ is found by
normalising the cross product of the two tangent vectors $\vect{t}_{1} =
\pbrac{\delby{x}{\xi_{1}}, \delby{y}{\xi_{1}},\delby{z}{\xi_{1}}}$ and
$\vect{t}_{2} = \pbrac{\delby{x}{\xi_{2}},\delby{y}{\xi_{2}},\delby{z}{\xi_{2}}
}$ (it relies on the user of any BEM code to ensure that the elements
have been defined with a consistent set of element coordinates $\xione$ and
$\xitwo$).

The Jacobian $\fnof{J}{\xione,\xitwo}$ is given by \lnorm{}{\vect{t}_{1} \times
\vect{t}_{2}} (where $\vect{t}_{1}$ and $\vect{t}_{2}$ are the two tangent
vectors).  Note that this is different for the determinant in a
two-dimensional finite element code - in that case we are dealing with a
two-dimensional surface in two-dimensional space, whereas here we have a
(possibly curved) two-dimensional surface in three-dimensional space.

The integrals are evaluated numerically using some suitable quadrature schemes
(see \Secref{sec:MNI}) (typically a Gauss-type scheme in both the $\xione$
and $\xitwo$ directions).

%\begin{example}{3D steady-state heat conduction}
%  {3D steady-state heat conduction}

%  \todo{Yet to complete}

%\end{example}

\section{A Comparison of the FE and BE Methods}

We comment here on some of the major differences between the two methods.
Depending on the application some of these differences can either be
considered as advantageous or disadvantageous to a particular scheme.

\begin{enumerate}
\item \textbf{FEM}: An entire domain mesh is required. \\ 
  \textbf{BEM}: A mesh of the boundary only is required. \\ 
  \textbf{Comment}: Because of the reduction in size of the mesh, one 
  often hears of people saying that the problem size has been reduced by one
  dimension.  This is one of the major pluses of the BEM - construction of
  meshes for complicated objects, particularly in 3D, is a very time consuming
  exercise. 
\item \textbf{FEM}: Entire domain solution is calculated as part of the 
  solution. \\
  \textbf{BEM}: Solution on the boundary is calculated first, and then the 
  solution at domain points (if required) are found as a separate step. \\ 
  \textbf{Comment}: There are many problems where the details of interest occur 
  on the boundary, or are localised to a particular part of the domain, and 
  hence an entire domain solution is not required.
\item \textbf{FEM}: Reactions on the boundary typically less accurate than the 
  dependent variables. \\
  \textbf{BEM}: Both $\vect{u}$ and $\vect{q}$ of the same accuracy.
\item \textbf{FEM}: Differential Equation is being approximated. \\
  \textbf{BEM}: Only boundary conditions are being approximated. \\ 
  \textbf{Comment}: The use of the Green-Gauss theorem and a fundamental 
  solution in the formulation means that the BEM involves no approximations 
  of the differential Equation in the domain - only in its approximations 
  of the boundary conditions.
\item \textbf{FEM}: Sparse symmetric matrix generated. \\
  \textbf{BEM}: Fully populated nonsymmetric matrices generated. \\
  \textbf{Comment}: The matrices are generally of different sizes due to the 
  differences in size of the domain mesh compared to the surface mesh.  There 
  are problems where either method can give rise to the smaller system and 
  quickest solution - it depends partly on the volume to surface ratio.  
  For problems involving infinite or semi-infinite domains, BEM is to be 
  favoured.
\item \textbf{FEM}: Element integrals easy to evaluate. \\
  \textbf{BEM}: Integrals are more difficult to evaluate, and some contain 
  integrands that become singular. \\
  \textbf{Comment}: BEM integrals are far harder to evaluate.  Also the 
  integrals that are the most difficult (those containing singular integrands)
  have a significant effect on the accuracy of the solution, so these integrals
  need to be evaluated accurately.
\item \textbf{FEM}: Widely applicable.  Handles nonlinear problems well. \\
  \textbf{BEM}: Cannot even handle all linear problems. \\
  \textbf{Comment}: A fundamental solution must be found (or at least an 
  approximate one) before the BEM can be applied.  There are many linear 
  problems (\eg virtually any nonhomogeneous equation) for which fundamental
  solutions are not known.  There are certain areas in which the BEM is 
  clearly superior, but it can be rather restrictive in its applicability.
\item \textbf{FEM}: Relatively easy to implement. \\
  \textbf{BEM}: Much more difficult to implement. \\
  \textbf{Comment}: The need to evaluate integrals involving singular integrands
  makes the BEM at least an order of magnitude more difficult to implement 
  than a corresponding finite element procedure.
\end{enumerate}

%\begin{example}{CMISS comparison of 2d FEM and BEM calculations}
%  {CMISS comparison of 2d FEM and BEM calculations}

%  \todo{Yet to complete}

%  \label{xmp:CMISScomp}
%\end{example}

%\begin{example}{CMISS comparison of 3d FEM and BEM calculations}
%  {CMISS comparison of 3d FEM and BEM calculations}
  
%  \todo{Yet to complete (SS temp in a cube)}

%  \label{xmp:CMISScomp2}
%\end{example}

\section{More on Numerical Integration}
\label{sec:MNI}

The BEM involves integrals whose integrands in generally become singular when
the source point is contained in the element of integration.  If one uses
constant or linear interpolation for the geometry and dependent variable, then
it is possible to obtain analytic expressions to most (if not all) of the
integrals that will appear in the BEM (at least for two-dimensional problems).
The expressions can become quite lengthy to write down and evaluate, but
benefit from the fact that they will be exact.  However, when one begins to
use general curved elements and/or solve three-dimensional problems then the
integrals will not be available as analytic expressions.  The basic tool for
evaluation of these integrals is quadrature.  As discussed in
\Secref{sec:Gquad} a one-dimensional integral is approximated by a sum in
which the integrand is evaluated at certain discrete points or abscissa
\begin{equation*}
  \gint{0}{1}{\fnof{f}{\xi}}{\xi} \approx \dsuml{i=1}{N} \fnof{f}{\xi_{i}}w_{i}
\end{equation*}
where $w_{i}$ are the weights and $\xi_{i}$ are the abscissa.

The weights and abscissa for the Gaussian quadrature scheme of order $N$ are
chosen so that the above expression will exactly integrate any polynomial of
degree $2N-1$ or less.  For the numerical evaluation of two or
three-dimensional integrals, a Gaussian scheme can be used of each variable of
integration if the region of integration is rectangular.  This is generally
not the optimal choice for the weights and abscissae but it allows easy
extension to higher order integration.

\subsection{Logarithmic quadrature and other special schemes}

Low order Gaussian schemes are generally sufficient for all FEM integrals, but
that is not the case for BEM.  For a two-dimensional BEM solution of Laplace's
equation, integrals of the form $\gint{0}{1}{\log \pbrac{\xi}\fnof{f}{\xi}}{\xi}$
will be required.  It is relatively common to use logarithmic schemes for
this.  These are obtained by approximating the integral as
\begin{equation*} 
  \gint{0}{1}{\log \pbrac{\xi}\fnof{f}{\xi}}{\xi} \approx
  \dsuml{i=1}{N} \fnof{f}{\xi_{i}}w_{i}
\end{equation*}   
\ie the log function has been factored out.  

In the same way as Gaussian quadrature schemes were developed in
\Secref{sec:Gquad}, log quadrature schemes can be developed which will exactly
integrate polynomial functions $\fnof{f}{\xi}$. Tables of these are given below
\begin{table}[htbp] \centering
  \begin{tabular}{ccccccccc}
    \multicolumn{4}{c}{Abscissas = $r_{i}$} & \multicolumn{1}{c}{ } &
    \multicolumn{4}{c}{Weight Factors = $w_{i}$} \\
    $n$ & $\xi_{i}$ & $-w_{i}$ & $n$ &  $\xi_{i}$ & $-w_{i}$ & $n$ & 
    $\xi_{i}$ & $-w_{i}$ \\
    2 & 0.112009 & 0.718539 & 3 & 0.063891 & 0.513405 & 4 & 0.041448 &
    0.383464 \\
    & 0.602277 & 0.281461 & & 0.368997 & 0.391980 & & 0.245275 & 0.386875 \\
    & & & & 0.766880 & 0.094615 & & 0.556165 & 0.190435 \\
    & & & & & & & 0.848982 & 0.039225
  \end{tabular}
  \caption{Abscissas and weight factors for Gaussian integration for
    integrands with a logarithmic singularity.}
  \label{table:abscissas}
\end{table}
  
It is possible to develop similar quadrature schemes for use in the BEM 
solution of other PDEs, which use different fundamental solutions to the log 
function.  The problem with this approach is the lack of generality - each new equation to be used requires its own special quadrature scheme.     

\subsection{Special solutions}

Another approach, particularly useful if Cauchy principal values are to be
found (see \Secref{sec:BIE,sec4.10}) is to use special solutions of the
governing equation to find one or more of the more difficult integrals.

For example $u=x$ is a solution to Laplaces' equation (assuming the boundary
conditions are set correctly). Thus if one sets both $u$ and $q$ in \eqnref{eqn:ms}
at every node according to the solution $u=x$, one can then use this to solve
for some entry in either the $\matr{A}$ or $\matr{B}$ matrix (typically the
diagonal entry since this is the most important and difficult to
find). Further solutions to Laplaces equation (\eg $u=x^2-y^2$) can be used to
find the other matrix entries (or just used to check the accuracy of the matrices).


\section{The Boundary Element Method Applied to other Elliptic PDEs}

Helmholtz, modified Helmholtz (CMISS example) Poisson Equation (domain
integral and MRM, DRM, Monte-carlo integration.


\section{Solution of Matrix Equations}

The standard BEM approach results in a system of equations of the form
 $\matr{C}\vect{x} = \vect{f}$ (refer \eqref{eqn:ms2}).  As mentioned above the
matrix $\matr{C}$ is generally well conditioned, fully populated and
nonsymmetric.  For small problems, direct solution methods, based on LU
factorisations, can be used.  As the problem size increases, the time taken for
the matrix solution begins to dominate the matrix assembly stage.  This
usually occurs when there is between $500$ and $1000$ degrees of freedom, although
it is very dependent on the implementation of the BE method.  The current
technique of favour in the BE community for solution of large BEM matrix
equations is a preconditioned Conjugate Gradient solver. Preconditioners are
generally problem dependent - what works well for one problem may not be so
good for another problem.  The conjugate gradient technique is generally
regarded as a solution technique for (sparse) symmetric matrix equations. 

%\section{Sample BEM code}
%\remark{to be added to}

\section{Coupling the FE and BE techniques}

There are undoubtably situations which favour FEM over BEM and vice versa.
Often one problem can give rise to a model favouring one method in one region
and the other method in another region, \eg in a detailed analysis of stresses
around a foundation one needs FEM close to the foundation to handle
nonlinearities, but to handle the semi-infinite domain (well removed from the
foundation), BEM is better.  There has been a lot of research on coupling FE
and BE procedures - we will only talk about the basic ideas and use Laplace's
Equation to illustrate this.  There are at least two possible methods.  
\begin{enumerate}
\item Treat the BEM region as a finite element and combine with FEM
\item Treat the FEM region as an equivalent boundary element and combine with
  BEM
\end{enumerate}

Note that these are essentially equivalent - the use of one or the other
depends on the problem, in the sense of which part is more dominant FEM or
BEM)

Consider the region shown in \Figref{fig:region}, where
\begin{align*}
  \Omega_{f} &= \text{FEM region}\\ \Omega_{B} &= \text{BEM region}\\ 
  \Gamma_{f} &= \text{FEM boundary}\\ \Gamma_{B} &= \text{BEM
    boundary}\\ \Gamma_{I} &= \text{interface boundary}
\end{align*}
\begin{figure} \centering
  \input{figs/bem/region.pstex}
  \caption{Coupled finite element/boundary element solution domain.}
  \label{fig:region}
\end{figure}
The BEM matrices for $\Omega_{B}$ can be written as
\begin{equation}
  \matr{A}\vect{u} = \matr{B}\vect{q}
  \label{eqn:BEM matrices}
\end{equation}
where $\vect{u}$ is a vector of the nodal values of $u$ and $\vect{q}$ is a vector
of the nodal values of $\delby{u}{n}$

The FEM matrices for $\Omega_{F}$ can be written as
\begin{equation}
  \matr{K}\vect{u} = \vect{f}
  \label{eqn:FEM matrices}
\end{equation}
where $\matr{K}$ is the stiffness matrix and $\vect{f}$ is the load vector.  

To apply method $1$ (\ie treating BEM as an equivalent FEM region) we get (from
\eqnref{eqn:BEM matrices})
\begin{equation}
  \matr{B}^{-1}\matr{A}\vect{u}=\vect{q}
  \label{eqn:appliedmethod}
\end{equation}

If we recall what the elements of $\vect{f}$ in \eqnref{eqn:FEM matrices}
contained, then we can convert $\vect{q}$ in \eqnref{eqn:appliedmethod} to an
equivalent load vector by weighting the nodal values of $\vect{q}$ by the
appropriate basis functions, producing a matrix $\matr{M}$ \ie 
$\vect{f}_{B}=\matr{M}\vect{q}$

Therefore \eqnref{eqn:appliedmethod} becomes 
\begin{equation*}
  \matr{M}\pbrac{\matr{B}^{-1}\matr{A}}\vect{u}=  \matr{M}\vect{q} = \vect{f}_{B}
\end{equation*}
\ie 
\begin{alignat*}{2}
    && \matr{K}_{B}\vect{u} &= \vect{f}_{B} \\
    \text{where} && \matr{K}_{B} &= \matr{M}\matr{B}^{-1}\matr{A}
\end{alignat*}
an equivalent stiffness matrix obtain from BEM.

Therefore we can assemble this together with original FEM matrix to produce an
FEM-type system for the entire region $\Omega_{B}$.

\textbf{Notes}:
\begin{enumerate}
\item $\matr{K}_{B}$ is in general not symmetric and not sparse.  This means
  that different matrix equation solvers must be used for solving the new
  combined FEM-type system (most solvers in FEM codes assume sparse and
  symmetric). Attempts have been made to ``symmetricise'' the $\matr{K}_{B}$
  matrix - of doubtful quality. (\eg replace $\matr{K}_{B}$ by 
  $\dfrac{1}{2}\pbrac{ \matr{K}_{B}-\matr{K}_{B}^{T}}$ - 
  often yields inaccurate results).
\item On $\Gamma_{I}$ nodal values of $\vect{u}$ and $\vect{q}$ are unknown.  One
  must make use of the following
  \begin{alignat*}{2}
    \vect{u}_{B}^{I} &= \vect{u}_{F}^{I} && \text{($\vect{u}$ is continuous)}
    \\ \delby{\vect{u}_{B}^{I}}{\vect{n}_{B}} &= -\delby{\vect{u}_{F}^{I}}
    {\vect{n}_{F}} && \text{($\vect{q}$ is continuous, but $\Gamma_{B} =
      -\Gamma_{F}$)}
  \end{alignat*}
\end{enumerate}

To apply method $2$ (\ie to treat the FEM region as an equivalent BEM region)
we firstly note that, as before, $\vect{f}=\matr{M}\vect{q}$. Applying this to
\bref{eqn:FEM matrices} yields $\matr{K}\vect{u}=\matr{M}\vect{q}$ an
equivalent BEM system.  This can be assembled into the existing BEM system
(using compatability conditions) and use existing BEM matrix solvers.

\textbf{Notes}:
\begin{enumerate}
\item This approach does not require any matrix inversion and is hence easier
  (cheaper) to implement
\item Existing BEM solvers will not assume symmetric or sparse matrices
  therefore no new matrix solvers to be implemented
\end{enumerate}

\section{Other BEM techniques}

What we have mentioned to date is the so-called singular (direct) BEM.  Given
a BIE there are other ways of solving the Equation although these are not so
widely used.

\subsection{Trefftz method}

\index{Trefftz method}Trefftz was the first person to perform a BEM
calculation (in 1917 - calculated the value (numerical) of the contraction
coefficient of a round jet issuing from an infinite tank - a nonlinear free
surface problem).  This method basically uses a ``complete'' set of solutions
instead of a Fundamental Solution.  \eg Consider Laplaces Equation in a
(bounded) domain $\Omega$
\begin{equation*}
    \text{weighted residuals} \Rightarrow \gint{\del\Omega}{}{\omega 
    \delby{u}{n}}{\Gamma} = \gint{\del\Omega}{}{u \delby{\omega}{n}}{\Gamma} 
    \quad \text{if $\laplacian{\omega}=0$}
\end{equation*}

The procedure is to express $u$ as a series of (complete) functions satisfying
Laplace's equation with coefficients which need to be numerically determined
through utilisation of the boundary conditions.

\textbf{Notes}:
\begin{enumerate}
\item Doesn't introduce singular functions so integrals are easy to evaluate
\item Must find a (complete) set of functions (If you just use usual
  approximations for $u$ matrix system is not diagonally dominant so not so
  good)
\item Method is not so popular - Green's functions more widely available that
  complete systems
\end{enumerate}

\subsection{Regular BEM}

\index{Regular BEM}Consider the BIE for Laplace's equation
\begin{align*}
  \fnof{c}{P} \fnof{u}{P} + \gint{\del\Omega}{}{u \delby{\omega}{n}}{\Gamma} &=
  \gint{\del\Omega}{}{\delby{u}{n}\omega}{\Gamma} \\ \text{with } \quad \omega &=
  -\dfrac{1}{2\pi} \log r
\end{align*}
The usual procedure is to put point $P$ at each solution variable node - creating
an equation for each node. This leads to singular integrands.

Another possibility is to put point $P$ outside of the domain $\Omega$ - this
yields
\begin{equation*}
  \gint{\del\Omega}{}{u \delby{\omega_{p}}{n}}{\Gamma} = 
  \gint{\del\Omega}{}{\delby{u}{n}_{p}}{\Gamma}
\end{equation*}
Following discretisation as before gives
\begin{equation*}
  \dsuml{j=1}{N} \dsuml{\alpha}{} u_{j\alpha} \gint{\Gamma_{j}}{}
  {\lbfnsymb{\alpha} \delby{\omega_{p}}{n}}{\Gamma} =
  \dsuml{j=1}{N} \dsuml{\alpha}{} q_{j\alpha} \gint{\Gamma_{j}}{}
  {\lbfnsymb{\alpha}\omega_{p}}{\Gamma}
\end{equation*}   
- an equation involving $u$ and $q$ at each surface node.

By placing the point $P$ (the singular point) at other distinct points outside
$\Omega$ one can generate as many equations as there are unknowns (or more if
required).

\textbf{Notes}:
\begin{enumerate}
\item This method does not involve singular integrands, so that
integrals are inexpensive to calculate.
\item There is considerable choice for the location of the point $P$.  Often
  the set of Equations generated are ill-conditioned unless $P$ chosen
  carefully.  In practise $P$ is chosen along the unit outward normal of the
  surface at each solution variable node. The distance along each node is
  often found by experimentation - various research papers suggesting
  ``ideal'' distances (Patterson \& Shiekh).
\item This method is not very popular.
\item The idea of placing the singularity point $P$ away from the solution
  variable node is often of use in other situations \eg \emph{Exterior
    Acoustic Problems}. For an acoustic problem (governed by Helmholtz Equation
  $\laplacian{u} + k^{2}u=0$) in an unbounded region the system of Equations
  produced by the usual (singular) BEM approach is singular for certain
  ``fictitious'' frequencies (\ie certain values of $k$).  To overcome this
  further equations are generated (by placing the singular point $P$ at
  various locations outside $\Omega$). The system of equations are then
  overdetermined and are solved in a least squares sense.
\end{enumerate}

\section{Symmetry}

\begin{figure} \centering
  \input{figs/bem/symcirc.pstex}
  \caption{A problem exhibiting symmetry.} 
  \label{fig:symcirc}
\end{figure}

Consider the problem given in \Figref{fig:symcirc} (the domain is
\emph{outside} the circle).  Both the boundary conditions and the governing
Equation exhibit symmetry about the vertical axis. \ie putting $x$ to $-x$
makes no difference to the problem formulation.  Thus the solution
$\fnof{H}{x,z}$ has the property that $\fnof{H}{x,z} = \fnof{H}{-x,z} \forall
x$. This behaviour can be found in many problems and we can make use of this
as follows. The Boundary Element Equation is (with $N=2M$ (\ie $N$ is even)
\emph{constant} elements)

\begin{equation}
  \dfrac{1}{2} u_{i} + \dsuml{j=1}{N} u_{j} \gint{\Gamma_{j}}{}
  {\delby{\omega_{i}}{n}}{\Gamma} = \dsuml{j=1}{N} q_{j}
  \gint{\Gamma_{j}}{}{\omega_{i}}{\Gamma} \qquad i=1,\ldots,N
  \label{eqn:bemeq}
\end{equation}
We have $N$ Equations and $N$ unknowns (allowing for the boundary conditions).
From symmetry we know that (refer to \Figref{fig:symcirc2}).
\begin{equation}
  u_{i} = u_{n+1-i} \qquad i=1,\ldots,M
  \label{eqn:symmetry}
\end{equation}
\begin{figure} \centering
  \input{figs/bem/symcirc2.pstex}
  \caption{Illustration of a symmetric mesh.}
  \label{fig:symcirc2}
\end{figure}
So we can write
\begin{equation}
  \dfrac{1}{2} u_{i} + \dsuml{j=1}{M} u_{j} \bbrac{\gint{\Gamma_{j}}{}
    {\delby{\omega_{i}}{n}}{\Gamma} + \gint{\Gamma_{N+1-j}}{}
    {\negthickspace \delby{\omega_{i}}{n}}{\Gamma}} = \dsuml{j=1}{M} q_{j} 
  \bbrac{\gint{\Gamma_{j}}{}{\omega_{i}}{\Gamma} 
    + \gint{\Gamma_{N+1-j}}{}{\negthickspace \negthickspace\omega_{i}}{\Gamma}}
  \label{eqn:nodes}
\end{equation}
for nodes $i=1,\ldots,M$. (The Equations for nodes $i=M+1,\ldots,N$ are the same as
the Equations for nodes $i=1,\ldots,M$). The above $M$ Equations have only $M$
unknowns.

If we define
\begin{align}
    a_{ij} &= \gint{\Gamma_{j}}{}{\delby{\omega_{i}}{n}}{\Gamma} +
    \gint{\Gamma_{N+1-j}}{}{\delby{\omega_{i}}{n}}{\Gamma} \\ 
    b_{ij} &= \gint{\Gamma_{j}}{}{\omega_{i}}{\Gamma} + 
    \gint{\Gamma_{N+1-j}}{}{\negthickspace\negthickspace\negthickspace\omega_{i}^{*}}{\Gamma}
\end{align}
then we can write \eqnref{eqn:nodes} as
\begin{equation}
  \dfrac{1}{2} u_{i} + \dsuml{j=1}{M} a_{ij}u_{j} = \dsuml{j=1}{M} b_{ij}q_{j} \quad
  i=1,\ldots,M
  \label{eqn:nodes2}
\end{equation}
and solve as before. (This procedure has halved the number of unknowns.)
\newline Note: Since $i=1,\ldots,M$ this means that the integrals over the
elements $\Gamma_{M+1}$ to $\Gamma_{N}$ will never contain a singularity
arising from the fundamental solution, except possibly on the axis of symmetry
if linear or higher order elements are used.

An alternative approach to the method above arises from the implied no flux
across the $z$ axis. This approach ignores the negative $x$ axis and considers
the half plane problem shown.

However now the surface to be discretised extends to infinity in the positive
and negative $z$ directions and the resulting systems of equations produced is
much larger.  

Further examples of how symmetry can be used (\eg radial symmetry) are given
in the next section.

\section{Axisymmetric Problems}

If a three-dimensional problem exhibits radial or axial symmetry (\ie
$\fnof{u}{r,\theta_{1},z}=\fnof{u}{r,\theta_{2},z}$) it is possible to reduce the
two-dimensional integrals appearing in the standard boundary Equation to
one-dimensional line integrals and thus substantially reduce the amount of
computer time that would otherwise be required to solve the fully
three-dimensional problem. The first step in such a procedure is to write the
standard boundary integral equation in terms of cylindrical polars
$\pbrac{r,\theta,z}$ \ie
\begin{equation}
  \fnof{c}{P} \fnof{u}{P} +
  \gint{\overline{\Gamma}}{}{u\pbrac{\gint{0}{2\pi}
      {\delby{\omega_{p}}{n}}{\theta_{q}}}r_{q}}{\overline{\Gamma}} =
    \gint{\overline{\Gamma}}{}{q \pbrac{\gint{0}{2 \pi}
      {\omega_{p}}{\theta_{q}}} r_{q}}{\overline{\Gamma}}
  \label{eqn:BoInteq}
\end{equation}
where $\pbrac{r_{p},\theta_{p},z_{p}}$ and $\pbrac{r_{q}, \theta_{q}, z_{q}}$ are the
polar coordinates of $P$ and $Q$ respectively, and $\overline{\Gamma}$ is the
intersection of $\Gamma$ and $\theta=0$ semi-plane (Refer
\Figref{fig:cylinder}). (\nb $Q$ is a point on the surface being integrated
over.)
\begin{figure} \centering
 \input{figs/bem/cylinder.pstex}
 \caption{Illustration of surface $\overline{\Gamma}$ for an 
  axisymmetric problem.}
\label{fig:cylinder}
\end{figure}

For three-dimensional problems governed by Laplace's equation
\begin{equation*}
  \omega_{p} = \dfrac{1}{4 \pi r}
\end{equation*}
where $r_{p}$ is the distance from $P$ to $Q$. From \Figref{fig:rdiag}
\begin{figure} \centering
  \input{figs/bem/rdiag.pstex}
  \caption{The distance from the source point ($P$) to the point of 
   interest ($Q$) in terms of cylindrical polar coordinates.}
 \label{fig:rdiag}
\end{figure}
\begin{align}
  r_{1}^{2} &= r_{p}^{2} + r_{q}^{2} -2r_{p}r_{q}\cos\pbrac{\theta_{p}-
  \theta_{q}} \nonumber \\ 
  r^{2} &= \sqrt{r_{1}^{2} + r_{2}^{2}} \nonumber \\ 
  r &= \sqrt{r_{p}^{2} + r_{q}^{2}-2r_{p}r_{q}\cos\pbrac{\theta_{p}-\theta_{q}}+
    \pbrac{z_{p}-z_{q}}^2} \nonumber \\
  &= \sqrt{a-b \cos\pbrac{\theta_{p}-\theta_{q}}+\pbrac{z_{p}-z_{q}}^2}
  \label{eqn:diageq}
\end{align}

We define
\begin{equation}
  \overline{\omega}_{p} = \dfrac{1}{4 \pi} \gint{0}{2 \pi} 
  {\omega_{p}}{\theta_{q}} \equiv \dfrac{\fnof{K}{m}}{\pi \sqrt{a+b}} \quad 
  \text{where } m=\dfrac{2b}{a+b}
  \label{eqn:defined}
\end{equation}
and $\fnof{K}{m}$ is the complete elliptic integral of the first kind.

$\overline{\omega}_{p}$ is called the axisymmetric fundamental solution and is
the Green's function for a ring source as opposed to a point source. \ie
$\overline{\omega}_{p}$ is a solution of
\begin{equation}
  \laplacian{\omega}+\fnof{\delta}{r-r_{p}} =0
  \label{eqn:solution}
\end{equation}
instead of
\begin{equation}
  \laplacian{\omega}+\delta_{p} =0
  \label{eqn:notsol}
\end{equation}
where $\delta_{p}$ is the dirac delta centered at the point $P$ and
$\fnof{\delta}{r-r_{P}}$ is the dirac delta centered on the ring $r=r_{p}$.

Unlike the two- and three-dimensional cases, the axisymmetric fundamental
solution cannot be written as simply a function of the distance between two
points $P$ and $Q$, but it also depends upon the distance of these points to
the axis of revolution.

We also define
\begin{equation}  
  \overline{q}_{p}^{*} = \dfrac{1}{4 \pi}\gint{0}{2 \pi} 
  {\delby{\omega_{p}}{n}}{\theta_{q}} \equiv \delby{\overline{\omega}_{p}}{n}
 \label{eqn:qeq}
\end{equation}
For Laplace's equation \eqnref{eqn:qeq} becomes
\begin{equation} 
  \overline{q}_{p}^{*} = \dfrac{1}{\pi \sqrt{a+b}} \sqbrac{\dfrac{1}{2r_{q}}
    \bbrac{\dfrac{r_{p}^{2}-r_{q}^{2} +\pbrac{z_{p}-z_{q}}}{a-b} \fnof{E}{m} -
      \fnof{K}{m}} \fnof{n_{r}}{Q} + \dfrac{z_{p}-z_{q}}{a-b}
    \fnof{E}{m}\fnof{n_{z}}{Q}} 
  \label{eqn:number98}
\end{equation}
where $\fnof{E}{m}$ is the complete elliptic integral of the second kind.

Using \eqnref{eqn:defined} and \eqnref{eqn:qeq} we can write \eqnref{eqn:BoInteq} as
\begin{equation}
  \fnof{c}{P} \fnof{u}{P} + \goneint{u \delby{\overline{\omega}_{p}}{n}}
  {\overline{\Gamma}} = \goneint{q \overline{\omega}_{p}}{\overline{\Gamma}}
  \label{eqn:BoInteq2}
\end{equation}
and the solution procedure for this Equation follows the same lines as the
solution procedure given previously for the two-dimensional version of
boundary element method.

\section{Infinite Regions}

The boundary integral equations we have been using have been derived assuming
the domain $\Omega$ is bounded (although this was never stated). However all
concepts presented thus far are also valid for infinite regular (\ie nice)
regions provided the solution and its normal derivative behave appropriately
as $\Gamma \rightarrow \infty$.

Consider the problem of solving $\laplacian{u}=0$ outside some surface
$\Gamma$.

\begin{figure} \centering
  \input{figs/bem/symm.pstex}
  \caption{Derivation of infinite domain boundary integral equations.}
  \label{fig:symm}
\end{figure}

$\overline{\Gamma}$ is the centre of a circle (or sphere in three dimensions) 
of radius centred at some point $x_{0}$ on $\Gamma$ and surrounding $\Gamma$
(see \Figref{fig:symm}).  The boundary integral equations for the bounded
domain $\Omega_{R}$ can be written as
\begin{equation}
  \fnof{c}{P} \fnof{u}{P} + \goneint{u \delby{\omega_{P}}{n}}{\Gamma} +
  \gint{\overline{\Gamma}}{}{u \delby{\omega_{P}}{n}}{\Gamma} =
  \goneint{q \omega_{P}}{\Gamma} + \gint{\overline{\Gamma}}{} 
  {q \omega_{P}}{\Gamma}
  \label{eqn:boundinteg}
\end{equation}

If we let the radius $R \rightarrow \infty$ \eqnref{eqn:boundinteg}
will only be valid for the points on $\Gamma$ if
\begin{equation}
  \lim_{R \rightarrow\infty}\gint{\overline{\Gamma}}{}{\pbrac{u 
    \delby{\omega_{P}}{n} - q \omega_{P}}}{\Gamma} = 0
  \label{eqn:limit}
\end{equation}
If this is satisfied, the boundary integral Equation for $\Omega$ will be as
expected \ie
\begin{equation}
  \fnof{c}{P} \fnof{u}{P} + \goneint{u \delby{\omega_{P}}{n}}{\Gamma}
  = \goneint{q \omega_{P}}{\Gamma}
  \label{eqn:omega}
\end{equation}

For three-dimensional problems with $\omega^{*}=\dfrac{1}{4 \pi r}$
\begin{alignat*}{2}
  d\Gamma &= \abs{J} d\theta d\phi && \qquad \text{where } 
          \abs{J} = \orderof{R^{2}} \\ 
  \omega^{*} &= \orderof{R^{-1}} & & \\ 
  \delby{\omega^{*}}{n} &= \orderof{R^{-2}} &&
  \label{eqn:3-Dprob}
\end{alignat*}
where $\abs{J}$ is the Jacobian and $\orderof{}$ represents the asymptotic behaviour
of the function as $R \rightarrow\infty$. In this case \eqnref{eqn:boundinteg}
will be satisfied if $u$ behaves at most as $\orderof{R^{-1}}$ so that
$q=\orderof{R^{-2}}$.  These are the regularity conditions at infinity and these
ensure that each term in the integral \eqnref{eqn:boundinteg} behaves
at most as $\orderof{R^{-1}}$ (\ie each term will $\rightarrow 0$ as $R
\rightarrow\infty$)

For two-dimensional problems with $\omega^{*}=\orderof{\log \pbrac{R}}$ we
require $u$ to behave as $\log \pbrac{R}$ so that $q=\orderof{R^{-1}}$. For
almost all well posed infinite domain problems the solution behaves
appropriately at infinity.
\clearpage
\section{Appendix: Common Fundamental Solutions}
\label{app:fundamentalsolutions}
\subsection{Two-Dimensional equations}
Here $r=\sqrt{\left(x_{1}^{2} + x_{2}^{2} \right)}$.\index{Fundamental solution}

\begin{tabular}{llp{9cm}}
  Laplace & Equation & $\deltwosqby{u^{*}}{x_1}+\deltwosqby{u^{*}}{x_2}+
  \delta_{0}=0$ \\ & Solution &
  $u^{*}=\dfrac{1}{2\pi}\log\pbrac{\dfrac{1}{r}}$ \\ \\ Helmholtz &
  Equation\index{Fundamental solution!Helmholtz} &
  $\deltwosqby{u^{*}}{x_1}+\deltwosqby{u^{*}}{x_2}+
  \lambda^{2}u^{*}+\delta_{0}=0$  \\ 
  & Solution & $u^{*}=\dfrac{1}{4i}H_{0}^{(2)}(\lambda r)$ \\
  &&  where $H$ is the Hankel funtion.
  \\ \\ 
%
  Wave & Equation\index{Fundamental solution!wave equation} &
  $c^2\pbrac{\deltwosqby{u^{*}}{x_1}+
    \deltwosqby{u^{*}}{x_2}}-\deltwosqby{u^{*}}{t}+ \fnof{\delta_{0}}{t} = 0$\\
  && where $c$ is the wave speed. \\ 
  & Solution & $u^{*}=
  -\dfrac{\fnof{H}{ct-r}}{2\pi c \pbrac{c^{2}t^{2}-r^{2}}}$ \\ \\ 
%
  Diffusion & Equation\index{Fundamental solution!diffusion equation} &
  $\deltwosqby{u^{*}}{x_1}+\deltwosqby{u^{*}}{x_2}
  -\dfrac{1}{k}\delby{u^{*}}{t}=0$ \\
  && where $k$ is the diffusivity.\\ 
  & Solution & $u^{*}=-\dfrac{1}{\pbrac{4 \pi kt}^{\frac{3}{2}}}exp
  \pbrac{-\dfrac{r^{2}}{4kt}}$ \\ \\ 
%
  Navier's & Equation\index{Fundamental solution!Navier} 
        & $\delby{\sigma_{jk}^{*}}{x_{j}} + \delta_{l}=0$ for a point load 
        in direction $l$. \\ 
  & Solution & $p_{i}^{*}=p_{ji}^{*}e_{j}$ \\ 
  && $p_{ji}^{*}=-\dfrac{1}{8 \pi \pbrac{1-\nu^{2}}r^{2}}$\\
  && $\pbrac{ \delby{r}{n} 
        \sqbrac{\pbrac{1-2\nu}\delta_{ij}+3r_{,i}r_{,j}}+
        \pbrac{1-2\nu} \pbrac{n_{j}r_{,i}-n_{i}r_{,j}} }
        e_{j}$\\ 
  && for a traction in direction $k$ where $\nu$ is Poisson's ratio.
\end{tabular}

  \subsection{Three-Dimensional equations}

Here $r=\sqrt{(x_{1}^{2} +x_{2}^{2} + x_{3}^{2})}$.

\begin{tabular}{llp{9cm}}
  Laplace & Equation & $\deltwosqby{u^{*}}{x_1}+\deltwosqby{u^{*}}{x_2}+
  \deltwosqby{u^{*}}{x_3}+\delta_{0}=0$ \\
  & Solution & $u^{*}=\dfrac{1}{4 \pi r}$ \\ \\
%
  Helmholtz & Equation & $\deltwosqby{u^{*}}{x_1}+\deltwosqby{u^{*}}{x_2}+
  \deltwosqby{u^{*}}{x_3}+\lambda^{2}u^{*}+\delta_{0}=0$ \\
  & Solution & $u^{*}=\dfrac{1}{4\pi r}\exp\pbrac{-i\lambda r}$ \\ \\
%
  Wave & Equation & $c^{2}\pbrac{\deltwosqby{u^{*}}{x_1}+
    \deltwosqby{u^{*}}{x_2}+\deltwosqby{u^{*}}{x_3}}-
  \deltwosqby{u^{*}}{t}+\delta_{t}=0$ \\
  &&where $c$ is the wave speed. \\
  & Solution & $u^{*}= \dfrac{\delta \pbrac{t-\dfrac{r}{c}}}{4 \pi r}$ \\ \\
%
  Navier's & Equation & $\delby{\sigma_{jk}^{*}}{x_{j}} + \delta_{l} = 0$ 
  for a isotropic homogenenous Kelvin solution for a point load in direction
  $l$. \\
  & Solution & $u_{k}^{*} = u_{lk}^{*} e_{l}$ \\
  && $u_{lk}^{*}= \dfrac{1}{16 \pi G\pbrac{1-\nu}} \pbrac{\dfrac{3-4\nu}{r}
    \delta_{lk} + \delby{r}{x_{1}}\delby{r}{x_{2}}}$ \\
  && for a displacement in direction $k$ where $\nu$ is Poisson's ratio
  and $G$ is the shear modulus.
\end{tabular}

\subsection{Axisymmetric problems}

\begin{tabular}{llp{10cm}}
  Laplace && For $u^{*}$ see \eqnref{eqn:defined} and for $q^{*}$ 
  see \eqnref{eqn:number98}
\end{tabular}

\section{CMISS Examples}

\begin{enumerate}
  \item  2D steady-state heat conduction inside an annulus
    To determine the steady-state heat conduction inside an annulus run the
    CMISS example $324$.
    \label{xmp:2Dsshc} 

  \item  3D steady-state heat conduction inside a sphere. To determine the
    steady-state heat conduction inside a sphere run the
    CMISS example $328$.

  \item  CMISS comparison of 2-D FEM and BEM calculations
    To determine the CMISS comparison of 2-D FEM and BEM calculations run
    examples $324$ and $312$.
    \label{xmp:CMISScomp}

  \item  CMISS biopotential problems C4 and C5.
    \label{xmp:CMISScomp2}

\end{enumerate}

%%% Local Variables: 
%%% mode: latex
%%% TeX-master: "/product/cmiss/documents/notes/fembemnots/fembemnotes"
%%% End: 

\clearemptydoublepage
\chapter{Transient Heat Conduction}

\section{Introduction}

In the previous discussion of steady state boundary value problems the
principal advantage of the finite element method over the finite difference
method has been the greater ease with which complex boundary shapes can be
modelled. In time-dependent problems the solution proceeds from an initial
solution at $t = 0$ and it is almost always convenient to calculate each new
solution at a constant time ($t>0$) throughout the entire spatial domain
$\Omega$.  There is, therefore, no need to use the greater flexibility (and
cost) of finite elements to subdivide the time domain: finite difference
approximations of the time derivatives are usually preferred.  Finite
difference techniques are introduced in \Secref{sec:finite} to solve the
transient one dimensional heat equation. A combination of finite elements for
the spatial domain and finite differences for the time domain is used in
\Secref{sec:transient} to solve the transient advection-diffusion
equation - a slight generalization of the heat equation.

%%
%% Finite Differences
%%
\section{Finite Differences}
\label{sec:finite} 

%%
%% Explicit Transient Finite Differences
%%
\subsection{Explicit Transient Finite Differences}

Consider the transient one-dimensional heat equation 
\begin{equation}
  \delby{u}{t} = D\deltwosqby{u}{x}, \qquad \pbrac{0<x<L, t>0}
  \label{eqn:1DHT}
\end{equation}
where $D$ is the conductivity and $u=\fnof{u}{x,t}$ is the temperature,
subject to the boundary conditions $\fnof{u}{0,t} = u_{0}$ and $\fnof{u}{L,t}
= u_{1}$ and the initial conditions $\fnof{u}{x,0} = 0$. A finite difference
approximation of this equation is obtained by defining a grid with spacing
$\Delta x$ in the x-domain and $\Delta t$ in the time domain, as shown in
\Figref{fig:findiff}.

Grid points are labelled by the indices $i=0,1,\ldots,I$ (for the
$x$-direction) and $n=0,1,\ldots,N$ (for the $t$-direction). The temperature at the
grid point $\pbrac{i,n}$ is therefore labelled as
\begin{equation}
  \fnof{u}{x,t}=\fnof{u}{i\Delta x, n\Delta t}=u_{i}^{n}.
  \label{eqn:temp}
\end{equation}
Finite difference equations are derived by writing Taylor Series expansions
for $u_{i+1}^{n},u_{i-1}^{n}u_{i}^{n+1}$ about the grid point $\pbrac{i,n}$
\begin{align}
  u_{i+1}^{n}=u_{i}^{n} &+ \Delta x.\pbrac{\delby{u}{x}}_{i}^{n}
  +\dfrac{1}{2} \Delta x^{2}.\pbrac{\deltwosqby{u}{x}}_{i}^{n}
  +\frac{1}{6}\Delta x^{3}.\pbrac{\frac{\del^{3}u}{\del x^{3}}
  }_{i}^{n}+\orderof{\Delta x^{4}} \label{eqn:Taylor1} \\ 
  u_{i-1}^{n}=u_{i}^{n} &- \Delta x.\pbrac{\delby{u}{x}}_{i}^{n}
  +\dfrac{1}{2} \Delta x^{2}.\pbrac{\deltwosqby{u}{x}}_{i}^{n}
  -\frac{1}{6}\Delta x^{3}.\pbrac{\frac{\del^{3}u}{\del x^{3}}
  }_{i}^{n}+\orderof{\Delta x^{4}} \label{eqn:Taylor2} \\
  u_{i}^{n+1}=u_{i}^{n} &+ \Delta t.\pbrac{\delby{u}{t}}_{i}^{n}
  +\orderof{\Delta t^{2}} \label{eqn:Taylor3}
\end{align}
where $\orderof{\Delta x^{4}}$ and $\orderof{\Delta t^{2}}$ represent all the
remaining terms in the Taylor Series expansions.

Adding \eqnrefs{eqn:Taylor1}{eqn:Taylor2} gives
\begin{equation*}
  u_{i+1}^{n}+u_{i-1}^{n} = 2u_{i}^{n} + \Delta x^{2}.\pbrac{
    \deltwosqby{u}{x}}_{i}^{n}+\orderof{\Delta x^{4}}
\end{equation*}  
or
\begin{equation}
  \pbrac{\deltwosqby{u}{x}}_{i}^{n} = \dfrac{u_{i+1}^{n}
    -2u_{i}^{n}+u_{i-1}^{n}}{\Delta x^{2}} + \orderof{\Delta x^{2}},
  \label{eqn:expns}
\end{equation} 
which is a ``central difference'' approximation of the second order spatial
derivative.

Rearranging \eqnref{eqn:Taylor3} gives a ``difference'' approximation of the
first order time derivative
\begin{equation}
  \pbrac{\delby{u}{t}}_{i}^{n} = \dfrac{u_{i}^{n+1} -u_{i}^{n}}{\Delta t}
  + \orderof{\Delta t}.
  \label{eqn:fordiff}
\end{equation} 
 
Substituting \eqnref{eqn:expns} and \eqnref{eqn:fordiff} into the transient heat
equation \eqnref{eqn:1DHT} gives the finite difference approximation
\begin{equation*}
  \dfrac{u_{i}^{n+1}-u_{i}^{n}}{\Delta t} + \orderof{\Delta t}
   = D\dfrac{u_{i+1}^{n}-2u_{i}^{n}+u_{i-1}^{n}}{\Delta x^{2}} 
  + \orderof{\Delta x^{2}}
\end{equation*}
which is rearranged to give an expression for $u_{i}^{n+1}$ in terms of the
values of $u$ at the $\nth{n}$ time step
\begin{equation}
  u_{i}^{n+1}  =  u_{i}^{n} + D \dfrac{\Delta t}{\Delta x^{2}}\pbrac{
    u_{i+1}^{n} -2u_{i}^{n} +u_{i-1}^{n}} 
  + \orderof{\Delta t^{2}, \Delta x^{2}}.
  \label{eqn:rearr}
\end{equation}
 
\begin{figure} \centering
 \input{figs/transient_heat_condn/findiff.pstex}
 \caption{A finite difference grid for the solution of the transient 1D 
   heat equation. The equation is centred at grid point $\pbrac{i,n}$ shown by the
   $\mathbf{O}$. The lightly shaded region shows where the solution is known at
   time step $n$. With central differences in $x$ and a forward difference in
   $t$ an explicit finite difference formula gives the solution at time step
   $n+1$ explicitly in terms of the solution at the three points below it at
   step $n$, as indicated by the dark shading.}
 \label{fig:findiff}
\end{figure}

Given the initial values of $u_{i}^{n}$ at $n=0$ (\ie $t=0$), the values of
$u_{i}^{n+1}$ for the next time step are found from \eqnref{eqn:rearr} with
$i=1,2,\ldots,I$. Applying \eqnref{eqn:rearr} iteratively for time steps
$n=1,2,\ldots$ \etc yields the time dependent temperatures at the grid points
(see \Figref{fig:findiff}). This is an \emph{ explicit} finite difference
formula because the value of $u_{i}^{n}$ depends only on the values of
$u_{i}^{n} \pbrac{i=1,2,\ldots,I}$ at the previous time step and not on the
neighbouring terms $u_{i+1}^{n+1}$ and $u_{i-1}^{n+1}$ at the latest time
step. The accuracy of the solution depends on the chosen values of $\Delta x$
and $\Delta t$ and in fact the stability of the scheme depends on these
satisfying the \emph{Courant} condition:
\begin{equation}
  D\dfrac{\Delta t}{\Delta x^{2}} \leq \frac{1}{2}.
  \label{eqn:Courant}
\end{equation}

%%
%% Von Neumann Stability Analysis
%%
\subsection{Von Neumann Stability Analysis}

The concept behind the Von Neumann analysis is that all Fourier
components decay as time advances or as they are processed by an iterative
solver. Considering \eqnref{eqn:rearr}, we can rearrange this to be of the
form,
\begin{equation}
  u_i^{n+1} = \Upsilon u_{i+1}^n  + (1-2\Upsilon)u_i^n + \Upsilon u_{i-1}^n
\end{equation}
where $\Upsilon=D\dfrac{\Delta t}{\Delta x^2}$. By subsituting the general Fourier
component $u_j^n = A_k^n e^{i \pbrac{\frac{\pi kj\Delta x}{L}}}$, we obtain,
\begin{equation}
  A_k^{n+1}e^{i \pbrac{\frac{\pi kj \Delta x}{L}}} = 
  A_k^n \sqbrac{\Upsilon e^{i \pbrac{\frac{\pi k(j+1)\Delta x}{L}}}
   + \pbrac{1-2\Upsilon} e^{i \pbrac{\frac{\pi kj\Delta x}{L}}}
    + \Upsilon e^{i \pbrac{\frac{\pi k(j-1) \Delta x}{L}}} }
  \label{eqn:genfourier}
\end{equation}
%%
If divide \eqnref{eqn:genfourier} by, 
$A_k^n e^{i \pbrac{\frac{\pi kj \Delta x}{L}}}$ we obtain (no sum on $k$),
%%
\begin{equation}
\begin{split}
 \dfrac{A_k^{n+1}}{A_k^n} 
   &= \pbrac{1-2\Upsilon} + \Upsilon e^{i \pbrac{\frac{\pi k \Delta x}{L}}} 
      + \Upsilon e^{-i \pbrac{\frac{\pi k \Delta x}{L}}} \\
   &= 1-2\Upsilon + 2\Upsilon cos\pbrac{\frac{\pi k \Delta x}{L}} \qquad
   \text{using} \qquad cos(x)=\frac{e^{ix}+e^{-ix}}{2} \\
   &= 1- 4\Upsilon sin^2\pbrac{\frac{\pi k \Delta x}{2L}} \qquad
   \text{using} \qquad cos (2x) = 1 - 2 sin^{2} (x)
\end{split}
\label{eqn:stability}
\end{equation}

\Eqnref{eqn:stability} predicts the growth of any component (specified by $k$)
admitted by the system. If all components are to decay,
\begin{equation}
  \abs{\dfrac{A_k^{n+1}}{A_k^n}} \leq 1 \qquad \text{for stability (no sum on $k$)}
\end{equation}
Since the $sin^2$ term in \eqnref{eqn:stability} is always between $0$ and $1$,
we effectively have the stablity criteria that
\begin{equation}
  1 - 4 \Upsilon \leq 1 \qquad \text{and} \qquad 1 - 4 \Upsilon \geq -1
\end{equation}
The first inequality is trivially satisfied, since $\Upsilon \geq 0$ for
positive values of $\Delta t$ and $D$, and the second condition will always
hold if
\begin{equation}
   \Upsilon = D\dfrac{\Delta t}{\Delta x^2} \leq \dfrac{1}{2}
\end{equation}
Thus, to ensure stability, the time step should be chosen such that
\begin{equation}
   \Delta t \leq \dfrac{\Delta x^2}{2D} \qquad \text{The \emph{Courant}
   condition}
\end{equation}


%%
%% Higher Order Approximations
%%
\subsection{Higher Order Approximations}

An improvement in accuracy and stability can be obtained by using a higher
order approximation for the time derivative. For example, if a central
difference approximation is used for $\delby{u}{t}$ by centering the equation
at $(i \Delta x,\pbrac{n+\frac{1}{2}} \Delta t)$ rather than $\pbrac{i\Delta
  x,n\Delta t}$ we get
\begin{equation}
  \pbrac{\delby{u}{t}}_{i}^{n+\frac{1}{2}} = \dfrac{u_{i}^{n+1}
    -u_{i}^{n}}{\Delta t} + \orderof{\Delta t^{2}}
  \label{eqn:centering}
\end{equation}
in place of \eqnref{eqn:fordiff} and \eqnref{eqn:1DHT} is approximated with the
``Crank-Nicolson''formula
\begin{equation}
  \dfrac{u_{i}^{n+1}-u_{i}^{n}}{\Delta t}=D\bbrac{\dfrac{1}{2}\pbrac{
      \deltwosqby{u}{x}}_{i}^{n+1} + \dfrac{1}{2}\pbrac{\deltwosqby{u}{x}}_{i}^{n}}
  \label{eqn:C-N}
\end{equation} 
in which the spatial second derivative term is weighted by $\frac{1}{2}$ at the old
time step $n$ and by $\frac{1}{2}$ at the new time step $n+1$. Notice that the
finite difference time derivative has not changed - only the time position at
which it is centred. The price paid for the better accuracy (for a given
$\Delta t$) and unconditional stability (\ie stable for \textbf{any}
$\Delta t$) is that \eqnref{eqn:C-N} is an \emph{implicit} scheme - the
equations for the new time step are now coupled in that $u_{i}^{n+1}$ depends
on the neighbouring terms $u_{i+1}^{n+1}$ and $u_{i-1}^{n+1}$. Thus each new
time step requires the solution of a system of coupled equations.
\begin{figure} \centering
 \input{figs/transient_heat_condn/impfd.pstex}
 \caption{An implicit finite difference scheme based
   on central differences in $t$, as well as $x$, which tie together the 6
   points shown by $\mathbf{x}$. The equation is centred at the point
   ($i,n+\dfrac{1}{2}$) shown by the $\mathbf{O}$. The lightly shaded region shows
   where the solution is known at time step $n$. The dark shading shows the
   region of the coupled equations.}
 \label{fig:impfd}
\end{figure}

%
%
% NOTE:
%
%\remark{In CMISS this is implemented as $\pbrac{1-\theta}
%    \pbrac{\deltwosqby{u}{x}_{i}}^{n+1} + 
%    \theta\pbrac{\deltwosqby{u}{x}_{i}}^{n}$}\\
%
%

A generalization of \bref{eqn:C-N} is
\begin{equation}
  \frac{u_{i}^{n+1}-u_{i}^{n}}{\Delta t} =D\bbrac{\theta
    \pbrac{\deltwosqby{u}{x}_{i}}^{n+1} + \pbrac{1-\theta}
    \pbrac{\deltwosqby{u}{x}_{i}}^{n}}
  \label{eqn:C-N2}
\end{equation}
in which the spatial second derivative of \eqnref{eqn:1DHT} has been weighted
by $\theta$ at the new time step and by $\pbrac{1-\theta}$ at the old time step. The
original explicit forward difference scheme \eqnref{eqn:rearr} is recovered
when $\theta =0$ and the implicit central difference (Crank-Nicolson) scheme
\bref{eqn:C-N2} when $\theta=\frac{1}{2}$. An implicit backward difference
scheme is obtained when $\theta =1$.

In the following section the transient heat equation is approximated for
numerical analysis by using finite differences in time and finite elements in
space. We also generalize the partial differential equation to include an
advection term and a source term.


\section{The Transient Advection-Diffusion Equation}
\label{sec:transient}

Consider a linear parabolic equation
\begin{equation}
  \delby{u}{t} +\dotprod{\vect{v}}{\grad u} = D\laplacian{u} + f
  \label{eqn:lpe}
\end{equation}
where $u$ is a scalar variable (\eg the advection-diffusion
equation\index{Advection-diffusion equation}, where $u$ is concentration or
temperature; $\dotprod{\vect{v}}{\grad u}$ then represents advective transport
by a velocity field $\vect{v}, D$ is the diffusivity and $f$ is source term.
The ratio of advective to diffusive transport is characterised by the
\emph{Peclet number} $VL/D$ where $V = \norm{\vect{v}}$ and $L$ is a
characteristic length).

Applying the Galerkin weighted residual method to \eqnref{eqn:lpe} with
weight \vect{\omega} gives
\begin{displaymath}
  \goneint{\pbrac{\delby{u}{t} + \dotprod{\vect{v}}{\grad u} 
    - D\laplacian{u}-f}\omega}{\Omega} =0
  \label{eqn:Gal}
\end{displaymath}
or
\begin{equation}
  \goneint{\sqbrac{\pbrac{\delby{u}{t} + \dotprod{\vect{v}}{\grad u}} 
    \omega + D\dotprod{\grad u}{\grad \omega}}}{\Omega}= \goneint{f\omega}{\Omega}
  + D \gint{\del \Omega}{}{\delby{u}{n}\omega}{\Gamma}
  \label{eqn:Gal2}
\end{equation}
where $\delby{ }{n}$is the normal derivative to the boundary $\del \Omega$.

Putting $u=\lbfnsymb{n}u_{n}$ and  $\omega=\lbfnsymb{m}$ and summing
the element contributions to the global equations, \eqnref{eqn:Gal2} can be 
represented by a system of first order ordinary differential equations,
\begin{equation}
  \matr{M}\dby{\vect{u}}{t} + \matr{K}\vect{u} = \matr{K}\vect{u}_{\infty}
  \label{eqn:orddiff}
\end{equation}
where $\matr{M}$ is the global mass matrix, $\matr{K}$ the global stiffness matrix
and $\vect{u}$ a vector of global nodal unknowns with steady state values ($t
\rightarrow\infty$) $\vect{u}_{\infty}$ . The element contributions to $\matr{M}$
and $\matr{K}$ are given by
\begin{equation}
  M_{{mn}_{e}} = \gint{0}{1}{\lbfnsymb{m} \lbfnsymb{n} J}{\xi}
 \label{eqn:elemcont}
\end{equation}
and
\begin{equation}
  K_{{mn}_{e}} = \gint{0}{1}{D\delby{\lbfnsymb{m}}{\xi_{i}}
  \delby{\lbfnsymb{n}}{\xi_{j}}\cdot \delby{\xi_{i}}{x_{k}} 
  \delby{\xi_{j}}{x_{k}}J}{\xi} + \gint{0}{1}{v_{j}\lbfnsymb{m}
  \delby{\lbfnsymb{n}}{\xi_{i}}\delby{\xi_{i}}{x_{j}}J}{\xi}
  \label{eqn:elemcont2}
\end{equation}

If the time domain is now discretized $\pbrac{t = n \Delta t, n = 0,1,2,\ldots}$
\eqnref{eqn:orddiff} can be replaced by
\begin{equation}
  \matr{M}\dfrac{\vect{u}^{n+1}-\vect{u}^{n}}{\Delta t} +
        \matr{K}\sqbrac{\theta \vect{u}^{n+1} + \pbrac{1-\theta}\vect{u}^{n}} =
        \matr{K}\vect{u}_{\infty} \qquad 0 \leq \theta \leq 1
  \label{eqn:elemcont3}
\end{equation}
where $\theta$ is a weighting factor discussed in \Secref{sec:finite}.
Note that for $\theta = \dfrac{1}{2}$ the method is known as the
\emph{Crank-Nicolson-Galerkin} method and errors arising from the time domain
discretization are $\orderof{\Delta t^{2}}$. Rearranging \eqnref{eqn:elemcont3}
as
\begin{eqnarray}
  \sqbrac{\matr{M} + \theta \Delta t \matr{K}}\vect{u}^{n+1}
  =\sqbrac{\matr{M}-\pbrac{1-\theta}\Delta t\matr{K}}\vect{u}^{n} + \Delta t 
  \matr{K}\vect{u}_{\infty}
  \label{eqn:elemcont4}
\end{eqnarray}
gives a set of linear algebraic equations to solve at the new time step
$\pbrac{n+1}\Delta t$ from the known solution $\vect{u}^{n}$ at the previous time
step $n\Delta t$.

The stability of the above scheme can be examined by expanding $\vect{u}$
(assumed to be smoothly continuous in time) in terms of the eigenvectors
$\vect{s}_{i}$ (with associated eigenvalues $\lambda_{i}$) of the matrix $\matr{A}
= \matr{M}^{-1}\matr{K}$. Writing the initial conditions $\vect{u}(0) = \dsuml{i}{}
a_{i}\vect{s}_{i}$ and steady state solution $\vect{u}_{\infty} = \dsuml{i}{}
b_{i}\vect{s}_{i}$ , the set of ordinary differential equations
\eqnref{eqn:orddiff} has solution
\begin{equation}
  \vect{u} = \dsuml{i}{} \sqbrac{b_{i} + \pbrac{a_{i}- b_{i}}e^{-\lambda_{i}t}}
  \vect{s}_{i}
\label{eqn:ordde}
\end{equation}

The time-difference scheme \eqnref{eqn:elemcont4} on the other hand, with
$\vect{u}$ now replaced by a set of discrete values $\vect{u}^{n}$ at each time
step $n\Delta t$, can be written as the recursion formula
\begin{equation}
  \sqbrac{\matr{I}+\theta \Delta t\matr{A}}\vect{u}^{n+1} = \sqbrac{\matr{I}-\pbrac{1-\theta} 
    \Delta t\matr{A}}\vect{u}^{n} +\Delta t\matr{A} \vect{u}_{\infty}
  \label{eqn:recform}
\end{equation}
with solution
\begin{equation}
  \vect{u}=\dsuml{i}{} \bbrac{b_{i} + \pbrac{a_{i}- b_{i}}\sqbrac{
      \dfrac{1 - \Delta t\pbrac{1-\theta}\lambda_{i}}{1 + \Delta t \theta 
        \lambda_{i}}}^{n}}\vect{s}_{i}
  \label{eqn:wsol}
\end{equation}
(You can verify that \eqnref{eqn:ordde} and \eqnref{eqn:wsol} are indeed the
solutions of \eqnref{eqn:orddiff} and \eqnref{eqn:elemcont3}, respectively, by
substituting and using $\matr{A}\vect{s}_{i} = \lambda_{i}\vect{s}_{i}$.)

Comparing \eqnref{eqn:ordde} and \eqnref{eqn:wsol} shows that replacing the
ordinary differential equations \bref{eqn:orddiff} by the finite difference
approximation \eqnref{eqn:elemcont3} is equivalent to replacing the exponential
$e^{-\lambda_{i}t}$ in \eqnref{eqn:ordde} by the approximation
\begin{equation}
  e^{-\lambda_{i}t} \sim \sqbrac{\dfrac{1- \Delta t \pbrac{1-\theta}\lambda_{i}}
    {1 + \Delta t \theta \lambda_{i}}}^{n}
  \label{eqn:approx}
\end{equation}
or, with $t = n\Delta t$,
\begin{equation}
  e^{-\lambda_{i}t} \sim \dfrac{1 - \Delta t\pbrac{1-\theta}\lambda_{i}}{
    1 + \Delta t\theta \lambda_{i}}=1-\dfrac{\Delta t \lambda_{i}}
  {1+\Delta t \theta \lambda_{i}}
  \label{eqn:approx2}
\end{equation}

The stability of the numerical time integration scheme can now be investigated
by examining the behaviour of this approximation to the exponential. For
stability we require
\begin{equation}
  -1 \leq 1 - \dfrac{\Delta t \lambda_{i}}{1 + \Delta t \theta 
    \lambda_{i}} \leq 1
  \label{eqn:stab}
\end{equation}
since this term appears in \eqnref{eqn:wsol} raised to the power $n$. The right
hand inequality in \eqnref{eqn:stab} is trivially satisfied, since $\Delta
t,\lambda_{i}$ and $\theta $ are all positive, and the left hand inequality
gives
\begin{equation}
  \dfrac{\Delta t \lambda_{i}}{1 + \Delta t \theta \lambda_{i}} \leq 2 \qquad 
  \text{or} \quad \Delta t \lambda_{i} \pbrac{1-2\theta} \leq 2
\label{eqn:LHineq}
\end{equation}

A consequence of \eqnref{eqn:LHineq} is that the scheme is \emph{unconditionally
  stable} if $\frac{1}{2} \leq \theta \leq 1$. For $\theta < \frac{1}{2}$ the
\emph{stability criterion} is
\begin{equation}
  \Delta t \lambda_{i}< \dfrac{2}{1-2\theta}
  \label{eqn:scrit}
\end{equation}
   
If the exponential approximation given by \eqnref{eqn:approx2} is negative for
any $\lambda_{i}$ the solution will contain components which change sign with
each time step $n$. This \emph{oscillatory noise} can be avoided by choosing
\begin{equation}
  \Delta t < \dfrac{1}{\pbrac{1-\theta}\lambda_{\text{max}}},
  \label{eqn:oscnoise}
\end{equation}
where $\lambda_{\text{max}}$ is the largest eigenvalue in the matrix $\matr{A}$,
but in practice this imposes a limit which is too severe for $\Delta t$ and a small
amount of oscillatory noise, associated with the high frequency vibration modes
of the system, is tolerated.  Alternatively the oscillatory noise can be
filtered out by averaging.

These theoretical results are explored numerically with a
Crank-Nicolson-Galerkin scheme ($\theta= \frac{1}{2}$) in \Figref{fig:annumsol},
where the one-dimensional diffusion equation
\begin{equation}
  \begin{array}{rcl}
    \delby{u}{t}= D  \deltwosqby{u}{x}   & \text{on} &0 \leq x \leq 1 \\\\
    \mbox{subject to initial conditions} && \fnof{u}{x,0} = 0 \\
    \mbox{and boundary conditions} && \fnof{u}{0,t} = 0 , \fnof{u}{1,t} = 1 
  \end{array}
  \label{1DDE}
\end{equation}
is solved for various time increments ($\Delta t$) and element lengths
($\Delta x$) for both linear and cubic Hermite elements.
\begin{figure} \centering
 \input{figs/transient_heat_condn/annumsol.pstex}
 \caption{Analytical and numerical solutions of the
   transient 1D heat equation showing the effects of element size $\Delta x$
   and time step size $\Delta t$. The top graph shows the exact and
   approximate solutions as functions of $x$ at various times. The lower
   graphs show the solution through time at the specified $x$ positions and
   with various choices of $\Delta x$ and $\Delta t$ as indicated.}
 \label{fig:annumsol}
\end{figure}

Decreasing $\Delta x$ from $0.25$ to $0.1$ with linear elements produces more
oscillation because the system has more degrees of freedom and leads to greater
oscillation. At a sufficiently small $\Delta t$ the oscillations are
negligible (bottom right, \Figref{fig:annumsol}). With this value of $\Delta
t$ (\nunit{0.01}{\s}) the numerical results agree well with the exact solution
(top, \Figref{fig:annumsol}) given by
\begin{equation}
  \fnof{u}{x,t} = x + \dfrac{2}{\pi}\dsuml{n=1}{\infty}\dfrac{(-1)^{n}}{n}
  e^{-n^{2}\pi^{2}t} \sin\pbrac{n\pi x}
  \label{eqn:exact}
\end{equation}

\section{Mass lumping}
\label{sec:mass}

A technique known as \emph{mass lumping}\index{Mass lumping} is sometimes used
in which the mass matrix $\matr{M}$ is replaced by a diagonal matrix having
diagonal terms equal to the row sums. For example, consider the mass matrix
(\eqnref{eqn:elemcont}) for a bilinear element (see \Figref{fig:2Dbbf} and
\eqnref{eqn:2,3DE}).
\begin{alignat*}{2}
    M_{11} &= \iint{}{}{\pbrac{1-\xi_{1}}^{2}
        \pbrac{1-\xi_{2}}^{2}}{\xi_{1}}{\xi_{2}} &&
        = \evalat{-\dfrac{\pbrac{1-\xi_{1}}^{3}}{3}}{0}{1}
        \evalat{-\dfrac{\pbrac{1-\xi_{2}}^{3}}{3}}{0}{1}=\dfrac{1}{3}\cdot
        \dfrac{1}{3} = \frac{1}{9} \\ 
    M_{22} &=
      \iint{}{}{\xi_{1}^{2}\pbrac{1-\xi_{2}}^{2}}
      {\xi_{1}}{\xi_{2}} &&=
      \dfrac{1}{3}\cdot\dfrac{1}{3} = \dfrac{1}{9} 
      \text{ and similarly $M_{33}$
      and $M_{44}$.} \\ 
    M_{12} &= \iint{}{}{\xi_{1}\pbrac{1-\xi_{1}}
      \pbrac{1-\xi_{1}}^{2}}{\xi_{1}}{\xi_{2}} &&= \pbrac{-\dfrac{1}{2}
      -\dfrac{1}{3}}\cdot\dfrac{1}{3} = \dfrac{1}{18} \\ 
    M_{13} &= \iint{}{}{\pbrac{1-\xi_{1}}^{2}
        \xi_{2}\pbrac{1-\xi_{2}}}{\xi_{1}}{\xi_{2}} &&= \dfrac{1}{18} \text{
        and similarly $M_{34}$ and $M_{24}$.} \\ 
    M_{14} &= \iint{}{}{\xi_{1}\pbrac{1-\xi_{1}}
        \xi_{2}\pbrac{1-\xi_{2}}}{\xi_{1}}{\xi_{2}} 
        &&= \dfrac{1}{36} \text{ and similarly $M_{23}$.}
\end{alignat*}

\begin{displaymath}
  \text{therefore } \matr{M}=\begin{bmatrix}
    \frac{1}{9} & \frac{1}{18} & \frac{1}{18} & \frac{1}{36} \\
    \frac{1}{18} & \frac{1}{9} & \frac{1}{36} & \frac{1}{18} \\
    \frac{1}{18} & \frac{1}{136} & \frac{1}{9} & \frac{1}{18} \\
    \frac{1}{36} & \frac{1}{18} & \frac{1}{18} & \frac{1}{9} \\
  \end{bmatrix}
  \stackrel{\text{mass lumping}}{\longrightarrow} \begin{bmatrix}
    \frac{1}{4} & 0 & 0 & 0 \\
    0 & \frac{1}{4} & 0 & 0 \\
    0 & 0 & \frac{1}{4} & 0 \\
    0 & 0 & 0 & \frac{1}{4}
  \end{bmatrix}
\end{displaymath}

The element mass is effectively lumped at the element vertices.  Such a scheme
has computational advantages when $\theta = 0$ in \eqnref{eqn:elemcont4}
because each component of the vector $\vect{u}^{n+1}$ is obtained directly
without the need to solve a set of coupled equations. This \emph{explicit}
time integration scheme, however, is only conditionally stable (see
\bref{eqn:scrit}) and suffers from \emph{phase lag errors} - see below. For
evenly spaced elements the finite element scheme with mass lumping is
equivalent to finite differences with central spatial differences.

In \Figref{fig:advection}, the finite element and finite differences (lumped
f.e. mass matrix) solutions of the one-dimensional advection-diffusion
equation \bref{eqn:lpe} with $V =$ \nunit{5}{\mps}, $D = 0.1$
\units{m^{2}s^{-1}}, $f = 0$ are compared for the propogation and dispersion
of an initial unit mass pulse at $x = 0$. The length of the solution domain is
sufficient to avoid reflected end effects.
\begin{figure} \centering
 \input{figs/transient_heat_condn/advection.pstex}
 \caption{Advection-diffusion of a unit mass pulse. The finite element
   solutions (at $t$=\nunit{0.01}{\s}, \nunit{0.05}{\s}, \nunit{0.2}{\s} and
   \nunit{0.4}{\s}) and finite difference solutions (at $t$=\nunit{0.4}{\s}
   only) are compared with the exact solution.  $\Delta x$= 0.1, $\Delta t$ =
   \nunit{0.001}{\s} for 0$<t<$\nunit{0.01}{\s} and $\Delta t$ = 0.01 for $t
   \geq$ \nunit{0.01}{\s}.}
 \label{fig:advection}
\end{figure}

The exact solution is a Gaussian distribution whose variance increases with
time:
\begin{equation}
  \fnof{u}{x,t} =  \dfrac{M}{\sqrt{4\pi\Delta t}} 
    e^{-\dfrac{\pbrac{x-Vt}^{2}}{4Dt}}
 \label{eqn:Gdist}
\end{equation}

The finite element solution, using the Crank-Nicolson-Galerkin technique,
shows excellent amplitude and phase characteristics when compared with the
exact solution. The finite difference, or lumped mass, solution also using
centered time differences, reproduces the amplitude of the pulse very well but
shows a slight phase lag.

\section{CMISS Examples}

\begin{enumerate}
\item   To solve for the transient heat flow in a plate run CMISS example $331$

\item  To investigate the stability of time integration schemes run CMISS
  examples $3321$ and $3322$.
\end{enumerate}

%%% Local Variables: 
%%% mode: latex
%%% TeX-master: "~/documents/notes/fembemnotes/fembemnotes"
%%% End: 



\clearemptydoublepage
\chapter{Linear Elasticity} 
\label{cha:linearelasticity}

\section{Introduction}
\label{sec:Intro}

To analyse the stress in various elastic bodies we calculate the strain energy
of the body in terms of nodal displacements and then minimize the strain
energy with respect to these parameters - a technique known as the
\index{Rayleigh-Ritz method}Rayleigh-Ritz. In fact, as we will show later, this leads to
the same algebraic equations as would be obtained by the Galerkin method (now
equivalent to virtual work) but the physical assumptions made (in neglecting
certain strain energy terms) are exposed more clearly in the Rayleigh-Ritz
method. We will first consider one-dimensional truss and beam elements, then
two-dimensional plane stress and plane strain elements, and finally
three-dimensional elasticity.

In all cases the steps are:
\begin{enumerate}
  \item Evaluate the components of strain in terms of nodal displacements,
  \item Evaluate the components of stress from strain using the elastic 
   material constants,
  \item  Evaluate the strain energy for each element by integrating the 
   products of stress and strain components over the element volume,
  \item  Evaluate the potential energy from the sum of total strain energy for
   all elements together with the work done by applied boundary forces, 
  \item  Apply the boundary conditions, \eg by fixing nodal displacements,
  \item  Minimize the potential energy with respect to the unconstrained nodal
   displacements,
  \item  Solve the resulting system of equations for the unconstrained nodal 
   displacements,
  \item  Evaluate the stresses and strains using the nodal displacements and 
   element basis functions,
  \item  Evaluate the boundary reaction forces (or moments) at the nodes where
   displacement is constrained. 
\end{enumerate}

\section{Truss Elements}

\index{Truss elements|(}
Consider the one-dimensional truss of undeformed length $L$ in
\Figref{fig:unitf} with end points $\pbrac{0,0}$ and $\pbrac{x,y}$ and making 
an angle $\theta$ with the x-axis. Under the action of forces in the $x$- and
$y$- directions the right hand end of the truss displaces by $u$ in the 
$x$-direction and $v$ in the $y$-direction, relative to the left hand end.

\begin{figure}[htbp] \centering
  \input{figs/lin_elasticity/truss.pstex}
  \caption{A truss of initial length $L$ is stretched to a new length $l$.   
    Displacements of the right hand end relative to the left hand end are $u$ 
    and $v$ in the $x$- and $y$- directions, respectively.}
  \label{fig:truss}
\end{figure}
                      
The new length is $l$ with axial strain
\begin{align*}
  e=\dfrac{l}{L}-1 &= \dfrac{\sqrt{\pbrac{X+u}^{2} +
      \pbrac{Y+v}^{2}}}{\sqrt{X^{2} + Y^{2}}} -1 \\ 
&=\dfrac{\sqrt{L^{2}+2\pbrac{Xu+Yv}+u^{2}+v^{2}}}{L} - 1 \\ 
&= \sqrt{1+2 \pbrac{\cos \theta . \dfrac{u}{L} + \sin \theta . \dfrac{v}{L}} +
    \dfrac{u^{2}+v^{2}}{L^{2}}} - 1
\end{align*}
using $\dfrac{X}{L}=\cos \theta$ and $\dfrac{Y}{L}=\sin \theta$, where
$\theta$ is defined to be positive in the anticlockwise direction. Neglecting
second order terms in the binomial expansion $\sqrt{\pbrac{1+\varepsilon}} = 1 +
\dfrac{1}{2}\varepsilon + \orderof{\varepsilon^{2}}$, the strain for small
displacements $u$ and $v$ is
\begin{equation}
  e \cong \cos \theta .\dfrac{u}{L} + \sin \theta .\dfrac{v}{L}
  \label{eqn:strain}
\end{equation}

The strain energy associated with this uniaxial stretch is 
\begin{equation}
  \text{SE} = \dfrac{1}{2} \gint{}{}{\sigma e}{V} = \dfrac{1}{2} A \gint{0}{L}
  {\sigma e}{x} = \dfrac{1}{2} \gint{0}{L}{EAe^{2}}{x} = \dfrac{1}{2}  ALEe^{2}
  \label{eqn:sen}
\end{equation}
where $\sigma=Ee$ is the stress in the truss (of cross-sectional area $A$),
linearly related to the strain $e$ via Young's modulus $E$. We now substitute
for $e$ from \eqnref{eqn:strain} into \eqnref{eqn:sen} and put $u=u_{2}-u_{1}$
and $v=v_{2}-v_{1}$, where $\pbrac{u_{1},v_{1}}$ and $\pbrac{u_{2},v_{2}}$ are
the nodal displacements of the two ends of the truss
\begin{equation}
  \text{SE}= \dfrac{1}{2} ALE \pbrac{\cos \theta . \dfrac{u_{2}-u_{1}}{L} + 
    \sin \theta . \dfrac{v_{2}-v_{1}}{L}}^{2}
  \label{eqn:tet}
\end{equation}

The potential energy is the combined strain energy from all trusses in the
structure minus the work done on the structure by external forces. The
Rayleigh-Ritz \index{Rayleigh-Ritz method} approach is to minimize this
potential energy with respect to the nodal displacements once all displacement
boundary conditions have been applied.

For example, consider the system of three trusses shown in
\Figref{fig:system}. A force of \nunit{100}{\kN} is applied in the $x$-direction 
at node $1$.  Node $2$ is a sliding joint and has zero displacement in the
y-direction only.  Node $3$ is a pivot and therefore has zero displacement in
both $x$- and $y$- directions. The problem is to find all nodal displacements and
the stress in the three trusses.

\begin{figure}[htbp] \centering
  \input{figs/lin_elasticity/system.pstex}
  \caption{A system of three trusses.}
  \label{fig:system}
\end{figure}

The strain in truss $1$ (joining nodes $1$ and $3$) is 
\begin{displaymath}
  \dfrac{u_{1}}{L}\cos 30+\dfrac{v_{1}}{L} \sin 30=
  \dfrac{\sqrt{3}}{2}\dfrac{u_{1}}{L}+\dfrac{1}{2}\dfrac{v_{1}}{L}
\end{displaymath}
The strain in truss $2$ (joining nodes $1$ and $2$) is 
\begin{displaymath}
  \dfrac{\pbrac{u_{1}-u_{2}}}{L}\cos 90+\dfrac{v_{1}}{L}\sin 90=\dfrac{v_{1}}{L}
\end{displaymath}
The strain in truss $3$ (joining nodes $2$ and $3$) is 
\begin{displaymath}
  \frac{u_{2}}{L}\cos \pbrac{-30}=\dfrac{\sqrt{3}}{2}\dfrac{u_{2}}{L}
\end{displaymath} 

Since a force of \nunit{100}{\kN} acts at node $1$ in the $x$-direction, 
the potential energy is
\begin{displaymath}  
  \text{PE} =\sum_{\text{trusses}} \dfrac{1}{2} ALEe^{2} - 100 u_{1} =
  \dfrac{1}{2}\frac{AE}{L} \sqbrac{\pbrac{\dfrac{\sqrt{3}}{2} u_{1} +
      \dfrac{1}{2} v_{1}}^{2}+\pbrac{\dfrac{\sqrt{3}}{2} u_{2}}^2 +
      \pbrac{v_{1}}^{2}}- 100u_{1}
\end{displaymath}
[Note that if the force was applied in the negative $x$-direction, the final
  term would be $+ 100 u_{1}$]

Minimizing the potential energy with respect to the three unknowns $u_{1}$,
$v_{1}$ and $u_{2}$  gives
\begin{equation}
  \delby{\text{PE}}{u_{1}} = \dfrac{AE}{L}\pbrac{\dfrac{\sqrt{3}}{2} 
    u_{1} + \dfrac{1}{2} v_{1}}\dfrac{\sqrt{3}}{2} - 100=0
  \label{eqn:poten1}
\end{equation}
\begin{equation}
  \delby{\text{PE}}{v_{1}}  =  \dfrac{AE}{L}\sqbrac{\pbrac{
    \dfrac{\sqrt{3}}{2} u_{1} + \dfrac{1}{2} v_{1}}\dfrac{1}{2} +v_{1}}=0
  \label{eqn:poten2}
\end{equation}
\begin{equation}
  \delby{\text{PE}}{u_{2}} =  \dfrac{AE}{L}\left( \dfrac{\sqrt{3}}{2} 
    u_{2}\right)\dfrac{\sqrt{3}}{2}=0
  \label{eqn:poten3}
\end{equation}
  
If we choose $A =$ \nunit{5\times\tento{-3}}{\mtwo}, $E =$ \nunit{10}{\GPa} and
 $L = $ \nunit{1}{\m} (\eg \nunit{100}{\mm} $\times$ \nunit{50}{\mm} timber truss) 
 then $\dfrac{AE}{L} =$ \nunit{5 \times \tento{-3}}{\mtwo} 
 \nunit{\times \tento{7}}{\kPa /\m} $ = $ \nunit{5 \times \tento{4}}{\kNpm}.

\Eqnref{eqn:poten3} gives 
\begin{displaymath}  
  u_{2} = 0
\end{displaymath} 
\Eqnref{eqn:poten1} gives 
\begin{displaymath}
  3u_{1}+\sqrt{3}v_{1}=4\times\tento{2}/\pbrac{5\times\tento{4}}
\end{displaymath}
\Eqnref{eqn:poten2} gives for two dimensions
\begin{displaymath}
  v_{1} = -\dfrac{\sqrt{3}}{5}u_{1}
\end{displaymath}
Solving these last two equations gives $u_{1}=\nunit{3.34}{\mm}$ and
$v_{1}=\nunit{-1.15}{\mm}$.  Thus the strain in truss $1$ is
$(\dfrac{\sqrt{3}}{2}3.34-\dfrac{1}{2}1.15) \times \tento{-3} =0.232\%$, in
truss $2$ is $-0.115\%$ and in truss $3$ is zero.

The tension in truss $1$ is $A\sigma = AEe = \nunit{5\times\tento{-3}}{\mtwo}
\nunit{\tento{7}}{\kPa} \times 0.232\times\tento{-2} = \nunit{116}{\kN}$ (tensile),
in truss $2$ is \nunit{-57.5}{\kN} (compressive) and in truss $3$ is zero. The
nodal reaction forces are shown in \Figref{fig:rf}.
    
\begin{figure}[htbp] \centering
  \input{figs/lin_elasticity/reaction.pstex}
  \caption{Reaction forces for the truss system of \Figref{fig:system}.}
  \label{fig:rf}
\end{figure}

%\begin{example}{Solving a truss system}
%  {Solving a truss system}

%  \todo{example???}

%  We solve the simple three truss system described above and shown in
%  \Figref{fig:system}. 

%  \label{xmp:Solving}
%\end{example}


%\begin{example}{Stresses in a bicycle frame modelled with truss elements}
%  {Stresses in a bicycle frame modelled with truss elements}

%  \todo{example???}

%  \begin{figure}[htbp] \centering
%    \input{figs/lin_elasticity/CMISSA.pstex}
%  \end{figure}
%  \label{xmp:stressesinabike}
%\end{example}
%\index{Truss elements|)}

\section{Beam Elements}

\index{Beam elements|(}
Simple beam theory ignores all but axial strain $e_{x}$  and stress 
$\sigma_{x}=Ee_{x}$  ($E=$ Young's modulus) along the beam (assumed here to be 
in the x-direction). The axial strain is given by $e_{x} = \dfrac{z}{R}$ ,
where $z$ is the lateral distance from the neutral axis in the plane of the 
bending and $R$ is the radius of curvature in that plane. The bending moment 
is given by $M= \gint{}{}{\sigma_{x} z}{A}$ , where $A$ is the beam crossectional 
area. Thus 
\begin{equation}
  \sigma_{x} =  Ee_{x} = E\dfrac{z}{R}
  \label{eqn:bem1}
\end{equation}
\begin{equation}
  M  = \gint{}{}{\sigma_{x} z}{A} = \frac{E}{R} \gint{}{}{z^{2}}{A} 
     = \dfrac{EI}{R}
  \label{eqn:bem2}
\end{equation}
where $I = \gint{}{}{z^{2}}{A}$ is the second moment of area of the beam
cross-section. Thus, $\dfrac{E}{R} = \dfrac{M}{I}$ and \eqnref{eqn:bem1} becomes
\begin{equation}
  \sigma_{x} = \dfrac{Mz}{I}
  \label{eqn:bem3}
\end{equation}
The slope of the beam is  $\dby{w}{x} = \theta$  and the rate of change of 
slope is the curvature 
\begin{equation}
  K=\dby{\theta}{x} = \dtwosqby{w}{x} = \dfrac{1}{R}
  \label{eqn:slope}
\end{equation}
Thus the bending moment is 
\begin{equation}
  M=EI \dtwosqby{w}{x} = EIw''
  \label{eqn:bmom}
\end{equation}
and a force balance gives the shear force
\begin{equation}
  V=-\dby{M}{x} = - \dby{ }{x}\pbrac{EIw''}
  \label{eqn:shf}
\end{equation}
and the normal force (per unit length of beam)
\begin{equation}
  p=\dby{V}{x} = -\dtwosqby{ }{x}\pbrac{EIw''}
  \label{eqn:normf}
\end{equation}
This last equation is the equilibrium equation for the beam, balancing the 
loading forces $p$ with the axial stresses associated with beam flexure
\begin{equation}
  -\dtwosqby{ }{x} \pbrac{EI \dtwosqby{w}{x}} = p
  \label{eqn:lasteq}
\end{equation}

The elastic strain energy stored in a bent beam is the sum of flexural strain
energy and shear strain energy, but this latter is ignored in the simple beam
theory considered here. Thus, the (flexural) strain energy is
\begin{align*}
  \text{SE} &= \dfrac{1}{2} \gint{x=0}{L}{\goneint{\sigma_{x}e_{x}}{A}}{x} =  
  \dfrac{1}{2} \gint{x=0}{L}{E\goneint{e_{x}^{2}}{A}}{x} \\
  &= \dfrac{1}{2} \gint{x=0}{L}{E \goneint{\pbrac{\dfrac{z}{R} 
  }^{2}}{A}}{x} =  \dfrac{1}{2} \gint{x=0}{L}{EI\pbrac{w''}^{2}}{x}
\end{align*}    
where $x$ is taken along the beam and $A$ is the cross-sectional area of the 
beam.

The external work associated with forces $p$ acting normal to the beam and
moving through a transverse displacement $w$ is $\gint{0}{L}{pw}{x}$.
The potential energy is therefore
\begin{equation}
  \text{PE}=\dfrac{1}{2} \gint{0}{L}{EI\pbrac{w''}^{2}}{x} - \gint{0}{L}{pw}{x}.
  \label{eqn:poten}
\end{equation}

The finite element approximation for the transverse displacement $w$ must be
able to represent the second derivative $w''$. A linear basis function has a
zero second derivative and therefore cannot represent the flexural strain. The
natural choice of basis function for beam deflection is in fact cubic Hermite
because the inter-element slope continuity of this basis ensures transmission
of bending moment as well as shear force across element boundaries.

The boundary conditions associated with the \nth{4} order equilibrium
\eqnref{eqn:lasteq} or the equations arising from minimum potential energy
\eqnref{eqn:poten} (which contain the square of $2^{\text{nd}}$ derivative terms) 
are more complex than the simple temperature or flux boundary conditions for the
(second order) heat equation. Three possible combinations of boundary
condition with their associated reactions are

\begin{tabular}{llll}
  && \emph{Boundary conditions} & \emph{Reactions}\\
  && \\
  (i) &  Simply supported &  zero displacement $w=0$ &  shear force $V$\\
  && zero moment $M = EIw'' = 0$ &  slope $\theta (= w')$\\
  (ii) &  Cantilever & zero displacement $w=0$ &  shear force $V$\\
  && zero slope  $\theta = w' = 0$ & moment $M$\\
  (iii)&  Free end & zero shear force $V=-\dby{ }{x}\pbrac{EIw''} = 0$ & 
  displacement $w$\\
  && zero moment $M = EI'' = 0$ & slope $\theta$
\end{tabular}

%\begin{example}{Stresses in a bicycle frame modelled with beam elements}
%  {Stresses in a bicycle frame modelled with beam elements.}

%  \todo{example???}

%  \begin{figure}[htbp] \centering
%    \input{figs/CMISSA.pstex}
%  \end{figure}

%  \label{xmp:Sbfbe}
%\end{example}
%\index{Beam elements|)}

\section{Plane Stress Elements}
\vspace{-5mm}
\index{Plane stress elements|(}
For two-dimensional problems, we define the displacement vector 
$\vect{u} = \begin{bmatrix} 
  u \\ 
  v 
\end{bmatrix}$, strain vector $\vect{e} = \begin{bmatrix}
  e_{x} \\
  e_{y} \\ 
  e_{xy}
\end{bmatrix}$ and stress vector $\vect{\sigma} = \begin{bmatrix}
  \sigma_{x} \\
  \sigma_{y} \\
  \sigma_{xy} 
\end{bmatrix}$. The stress-strain relation for two-dimensional plane stress:
\begin{gather}
  \begin{aligned}
    \sigma_{x} &= \dfrac{E}{1-\nu^{2}}\pbrac{e_{x}+\nu e_{y}} \\
    \sigma_{y} &= \dfrac{E}{1-\nu^{2}}\pbrac{e_{y}+\nu e_{x}} \\
    \sigma_{xy} &= \dfrac{E}{1+\nu}\pbrac{e_{xy}}
  \end{aligned}
  \label{eqn:planestress}
\end{gather}
can be written in matrix form
\begin{displaymath}
  \vect{\sigma}=\matr{E}\vect{e}
\end{displaymath}
where $\matr{E} = \dfrac{E}{1-\nu^{2}} \begin{bmatrix}
  1 & \nu & 0\\
  \nu & 1 & 0\\
  0 & 0 & 1-\nu
\end{bmatrix}$. 
The strain components are given in terms of displacement gradients by
\begin{equation}
  \begin{array}{rcl}
    e_{x} & = & \delby{u}{x} \\
    e_{y} & = & \delby{v}{y} \\
    e_{xy} & = & \dfrac{1}{2} \pbrac{\delby{u}{y} + \delby{v}{x}}
  \end{array}
  \label{eqn:stcom}
\end{equation}

The strain energy\index{Strain energy} is 
\begin{align*}
  \text{SE} &= \dfrac{1}{2} \goneint{\transpose{\vect{\sigma}}\vect{e}}{V} = \dfrac{1}{2}
  \goneint{\pbrac{e_{x}\sigma_{x} + e_{y}\sigma_{y} + e_{xy}\sigma_{xy}}}{V} \\ &=
  \dfrac{1}{2} \goneint{\transpose{\vect{e}}\matr{E}\vect{e}}{V} = \dfrac{1}{2}
  \goneint{\dfrac{E}{1-\nu^{2}} \sqbrac{e_{x}^{2} + e_{y}^{2} + 2\nu e_{x}e_{y} +
      \pbrac{1-\nu}e_{xy}^{2}}}{V}
\end{align*}

The potential energy\index{Potential energy} is
\begin{equation}
  \text{PE} = \text{SE} -\text{ external work }= \dfrac{1}{2}
  \goneint{\transpose{\vect{e}}\matr{E}\vect{e}}{V} -
  \goneint{\transpose{\vect{u}}\vect{l}}{A} 
  \label{eqn:pe}
\end{equation}
where $\vect{l}$ represents the external loads (forces) acting on the elastic
body.

Following the steps outlined in \Secref{sec:Intro} we approximate the
displacement field $\vect{u}$ with a finite element basis $u=\lbfnsymb{n} u_{n}$,
$v=\lbfnsymb{n} v_{n}$ and calculate the strains
\begin{gather}
  \begin{aligned}
    e_{x} &= \delby{u}{x} = \delby{\lbfnsymb{n}}{x} u_{n} \\
    e_{y} &= \delby{v}{y} = \delby{\lbfnsymb{n}}{y} v_{n} \\
    e_{xy} &= \dfrac{1}{2}\pbrac{\delby{u}{y}+ \delby{v}{x}} = 
    \dfrac{1}{2} \pbrac{\delby{\lbfnsymb{n}}{y}u_{n} + 
      \delby{\lbfnsymb{n}}{x}v_{n}}
  \end{aligned}
  \label{eqn:strains}
\end{gather}
or
\begin{equation}
  \vect{e} = \begin{bmatrix}
    e_{x} \\
    e_{y} \\
    e_{xy}
    \end{bmatrix} = \begin{bmatrix}
      \delby{\lbfnsymb{n}}{x} & 0 \\
      0 & \delby{\lbfnsymb{n}}{y} \\
      \dfrac{1}{2}\delby{\lbfnsymb{n}}{y} & \dfrac{1}{2}\delby{\lbfnsymb{n}}{x}
  \end{bmatrix} \begin{bmatrix}
    u_{n} \\
    v_{n} 
  \end{bmatrix} = \matr{B}\vect{u}
  \label{eqn:orstr}
\end{equation}
From \eqnref{eqn:pe} the potential energy is therefore
\begin{align*}
  \text{PE} &= \dfrac{1}{2}
  \goneint{\transpose{\pbrac{\matr{B}\vect{u}}}\matr{E}\pbrac{\matr{B}\vect{u}}}{V} -
  \goneint{\transpose{\vect{u}}\vect{l}}{A} \\ 
  &= \dfrac{1}{2}\transpose{\vect{u}} \left[
  \goneint{\transpose{\matr{B}}\matr{E}\matr{B}}{V} \right] \vect{u} - 
  \goneint{\transpose{\vect{u}}\vect{l}}{A} \\
  &= \dfrac{1}{2} \transpose{\vect{u}}\matr{K}\vect{u}-
  \goneint{\transpose{\vect{u}}\vect{l}}{A}
\end{align*}
where $\matr{K} = \goneint{\transpose{\matr{B}}\matr{E}\matr{B}}{V}$ is the
element stiffness matrix.

We next minimize the potential energy with respect to the nodal parameters
$u_{n}$ and $v_{n}$ giving
\begin{equation}
  \matr{K}\vect{u} = \vect{f}
  \label{eqn:pennodpam}
\end{equation}
\index{Plane stress elements|)} 
where $\vect{f} = \goneint{\vect{l}}{A}$ is a vector of nodal forces.

\subsection{Notes on calculating nodal loads}
\label{sec:noteoncalc}

If a known stress acts normal to a given surface (\eg a surface pressure), it
may be applied by calculating equivalent nodal forces. For example, consider a
uniform load $\nunit{p}{\kNpm}$ applied to the edge of the plane stress
element in \Figref{fig:uniformbound}a.

The nodal load vector $\vect{f}$ in \eqnref{eqn:pennodpam} has components
\begin{equation}
  f_{n} = \goneint{p \lbfnsymb{n}}{x} = pL \gint{0}{1}{\lbfnsymb{n}}{\xi}
  \label{eqn:nlv}
\end{equation}  
where $\xi$ is the normalized element coordinate along the side of length $L$
loaded by the constant stress \nunit{p}{\kNpm}. If the element side has a
linear basis, \eqnref{eqn:nlv} gives
\begin{align*}
  f_{1} &= pL\gint{0}{1}{\lbfnsymb{1}}{\xi} = pL\gint{0}{1}{\pbrac{1-\xi}}{\xi} =
  \dfrac{1}{2} pL \\
  f_{2} &= pL\gint{0}{1}{\lbfnsymb{2}}{\xi} = pL\gint{0}{1}{\xi}{\xi} = \dfrac{1}{2} pL
\end{align*}
as shown in \Figref{fig:uniformbound}b. If the element side has a
quadratic basis, \eqnref{eqn:nlv} gives 
\begin{align*}
  f_{1} &= pL\gint{0}{1}{\lbfnsymb{1}}{\xi} =
  pL\gint{0}{1}{2\pbrac{\dfrac{1}{2}-\xi}\pbrac{1-\xi}}{\xi} = \frac{1}{6} pL \\
  f_{2} &= pL\gint{0}{1}{\lbfnsymb{2}}{\xi} = pL \gint{0}{1}{4\xi\pbrac{1-\xi}}{\xi}  
  = \frac{2}{3} pL \\
  f_{3} &= pL\gint{0}{1}{\lbfnsymb{3}}{\xi} = pL \gint{0}{1}{2\xi\pbrac{\xi -
      \dfrac{1}{2}}}{\xi} = \frac{1}{6} pL
\end{align*}
as shown in \Figref{fig:uniformbound}c. A node common to two elements will
receive contributions from both elements, as shown in
\Figref{fig:uniformbound}d.

\begin{figure}[htbp] \centering
  \input{figs/lin_elasticity/uniformbound.pstex}
  \caption{A uniform boundary stress applied to the element side in (a) is 
   equivalent to nodal loads of $\frac{1}{2} pL$ and $\frac{1}{2} pL$ for the 
   linear basis used in (b) and to  $\frac{1}{6}pL$,  $\frac{2}{3}pL$ and  
   $\frac{1}{6}pL$ for the quadratic basis used in (c). Two adjacent quadratic
   elements both contribute to a common node in (d), where the element 
   length is now $\frac{L}{2}$.}
  \label{fig:uniformbound}
\end{figure}


\section{Three-Dimensional Elasticity}
\label{sec:3Delast}

Consider a surface $\Gamma$ enclosing a volume $\Omega$ of material of mass
density $\rho$.  Conservation of linear momentum over the domain $\Omega$
results in the governing stress equilibrium equations
\begin{equation}
  \sigma_{ij,j} + b_{i} =  0 \qquad i,j=1,2,3
  \label{eqn:equilib}
\end{equation}
where $\sigma_{ij}$ are the components of the stress tensor ($\sigma_{ij}$ is
the component of the traction or stress vector in the $\nth{i}$ direction
which is acting on the face of a rectangle whose normal is in the $\nth{j}$
direction), and $b_{i}$ is the body force/unit volume (\eg $\vect{b} =
\rho\vect{g}$). Note that the notation $\sigma_{ij,j} =
\delby{\sigma_{ij}}{x_{j}}$ has been introduced to represent the derivative.

Recall that the components of the linear (or small) strain tensor are
\begin{equation}
  e_{ij} =  \dfrac{1}{2}\pbrac{u_{i,j} + u_{j,i}} \qquad i,j=1,2,3
  \label{eqn:sst}
\end{equation}
where $\vect{u}$ is the displacement vector (\ie $\vect{u}$ is the difference
between the final and initial positions of a material point in
question). Note: we are assuming here that the displacement gradients are
small compared to unity, which is appropriate for many materials in solid
mechanics.  However, for soft materials, such as rubber or biological tissue,
then we need to use the exact finite strain tensor.

The object of solving an elasticity problem is to find the distributions of
stress and displacement in an elastic body, subject to a known set of body
forces and prescribed stresses or displacements at the boundaries.  In the
general three-dimensional case, this means finding $6$ stress components
$\sigma_{ij}$ ($=\sigma_{ji}$ which arises from the conservation of angular
momentum) and 3 displacements $u_{i}$ each as a function of position in the
body. Currently we have $15$ unknowns ($6$ stresses, $6$ strains and $3$
displacements), but only $9$ equations ($3$ equilibrium equations and $6$
strain-displacement relations).

To progress, we require an equation of state, \ie a stress-strain relation or
constitutive law.  For a linear elastic material we may propose that the
components of stress $\sigma_{ij}$ depend linearly on $e_{ij}$. \ie
\begin{displaymath}
  \sigma_{ij} = c_{ijkl}e_{kl}
\end{displaymath}  
where $c_{ijkl}$ are the $81$ components of a $\nth{4}$ order tensor, although
symmetry of the strain and stress tensors reduces the number of independent
components to $21$.

If the material is assumed to be isotropic (\ie the material response is
independent of orientation of the material element), then we end up with the
generalized Hooke's Law.
\begin{equation}
  \sigma_{ij} = \lambda e_{kk} \delta_{ij} + 2 \mu e_{ij}
  \label{eqn:genHook}
\end{equation}
or inversely
\begin{displaymath}
  e_{ij} = \dfrac{1}{2 \mu} \sigma_{ij} - \dfrac{\lambda}{2 \mu \pbrac{3 \lambda 
    + 2 \mu}} \sigma_{kk} \delta_{ij}
\end{displaymath}
where $\lambda$, $\mu$ are Lam\'{e}s constants.

Note: $\lambda$, $\mu$ are related to Young's modules $E$ and Poisson's 
ratio $\nu$ by 
\begin{displaymath}
  E  =  \dfrac{\mu \pbrac{3 \lambda + 2 \mu}}{\lambda + \mu}
\end{displaymath}
\begin{displaymath}
  \nu = \dfrac{\lambda}{2\pbrac{\lambda + \mu}}
\end{displaymath}   

Providing that the displacements are continuous functions of position, then
\eqnref{eqn:equilib}, \eqnref{eqn:sst} and \eqnref{eqn:genHook} are sufficient
to determine the $15$ unknown quantities.  This can often work with some
smaller grouping or simplification of these equations, \eg if all boundary
conditions are expressed in terms of displacements, substituting
\eqnref{eqn:sst} into \eqnref{eqn:genHook} then into \eqnref{eqn:equilib}
yields Navier's equation of motion.
\begin{displaymath}
  \mu u_{i,kk} + \pbrac{\lambda + \mu} u_{k,ki} + b_{i} = 0 \qquad i,k=1,2,3
\end{displaymath}
These $3$ equations can be solved for the unknown displacements. Then
\eqnref{eqn:sst} can be used to determine the strains and \eqnref{eqn:genHook}
to calculate the stresses.


\subsection{Weighted Residual Integral Equation}
Using weighted residuals as before we can write
\begin{equation}
  \goneint{\pbrac{\sigma_{ij,j} + b_{i}} u_{i}^{*}}{\Omega} = 0
  \label{eqn:integral}
\end{equation}
where $\vect{u}^{*} = \pbrac{u_{i}^{*}}$ is a (vector) weighting field.  The
$u_{i}^{*}$ are usually interpreted as a consistent set of virtual
displacements (hence we use the notation $u$ instead of $w$).

By the chain-rule
\begin{displaymath}
  \pbrac{\sigma_{ij}u_{i}^{*}}_{,j} = \sigma_{ij,j}u_{i}^{*} +
  \sigma_{ij}u_{i,j}^{*}
\end{displaymath}

Therefore, the first term in the integrand of \eqnref{eqn:integral} can be
re-written
\begin{align}
  \goneint{\sigma_{ij,j} u_{i}^{*}}{\Omega}
  &= \goneint{(\sigma_{ij} u_{i}^{*})_{,j}}{\Omega} - 
  \goneint{\sigma_{ij}u_{i,j}^{*}}{\Omega} \nonumber \\
%%
  &= \goneint{\diverg{\pbrac{\sigma_{ij} u_{i}^{*} }}}{\Omega}
    - \goneint{\sigma_{ij} u_{i,j}^{*}}{\Omega} \nonumber\\
%%
  &= \gint{\del \Omega}{}{\sigma_{ij} u_{i}^{*}n_{j}}{\Gamma}
  - \goneint{\sigma_{ij} u_{i,j}^{*}}{\Omega}
  \label{eqn:divtheory}
\end{align}
where the domain integral involving ``$\diverg{} = \delby{}{x_{j}}$'' has been
transformed into a surface integral using the divergence theorem
\begin{displaymath}
  \goneint{\diverg{\vect{g}}}{\Omega} = \gint{\del
  \Omega}{}{\dotprod{\vect{g}}{\vect{n}}}{\Gamma} \qquad \text{or} \qquad
  \goneint{g_{j,j}}{\Omega} = \gint{\del \Omega}{}{g_{j} n_{j}}{\Gamma}
\end{displaymath}
where $\vect{n} = n_{j}\vect{i}_{j}$ is the outward normal vector to the
surface $\Gamma$.

Thus, combining \eqnref{eqn:integral} and \eqnref{eqn:divtheory} we have
\begin{align}
  \goneint{\sigma_{ij} u_{i,j}^{*}}{\Omega} &=
  \goneint{b_{i} u_{i}^{*}}{\Omega} + \gint{\del \Omega}{}
  {\sigma_{ij} n_{j}u_{i}^{*}}{\Gamma} \nonumber \\ &= \goneint
  {b_{i} u_{i}^{*}}{\Omega} + \gint{\del \Omega}{}{t_{i} u_{i}^{*}}{\Gamma}
  \label{eqn:virtws}
\end{align}
where $t_{i}$ are the components of the internal stress vector (\vect{t}) and
are related to the components of the stress tensor ($\sigma_{ij}$) by Cauchy's
formula
\begin{equation}
\vect{t} = \sigma_{ij}n_{j}\vect{i}_{i}
\label{eqn:cauchyseqn}
\end{equation}

To arrive at this point, we have used weighted residuals to tie in with
\chapref{cha:steadystate}, however \eqnref{eqn:virtws} is more usually derived
using the principle of virtual work (below). Note that the weighted integral
\eqnref{eqn:virtws} is independent of the constitutive law of the material.


\subsection{The Principle of Virtual Work}

The governing equations for elastostatics can also be derived from a
physically appealing argument. Let $\vect{s}$ be the external traction vector
(\ie force per unit surface area).  For equilibrium, the work done by the
external surface forces $\vect{s} = s_{i}\vect{i}_{i}$, in moving through a
virtual displacement $\vect{u}^{*} = u_{i}^{*}\vect{i}_{i}$ is equal to the
work done by the stress vector $\vect{t} = t_{i}\vect{i}_{i}$ in moving
through a compatible set of virtual displacements $\vect{u}^{*}$. In
mathematical terms, the principle of virtual work can be written
\begin{equation}
  \gint{\del \Omega}{}{s_{i} u_{i}^{*}}{\Gamma} = \gint{\del \Omega}{}{t_{i} u_{i}^{*}}{\Gamma} 
  = \gint{\del \Omega}{}{\sigma_{ij}n_{j} u_{i}^{*}}{\Gamma}
  \label{eqn:virtwork1}
\end{equation}
using Cauchy's formula (\eqnref{eqn:cauchyseqn}).

The Green-Gauss theorem (\eqnref{eqn:nabla_eq}) is now used to replace the
right hand surface integral in \eqnref{eqn:virtwork1} by a volume integral,
giving
\begin{equation}
  \gint{\del \Omega}{}{s_{i} u_{i}^{*}}{\Gamma} = \goneint{\pbrac{
  \sigma_{ij,j} u_{i}^{*} + \sigma_{ij} u_{i,j}^{*} }}{\Omega}
\end{equation}

Substituting the equilibrium relation (\eqnref{eqn:equilib}) into the first
integrand on the right hand side, yields the \emph{virtual work} equation
\begin{equation}
  \goneint{\sigma_{ij} u_{i,j}^{*}}{\Omega} = \goneint{b_{i}
  u_{i}^{*}}{\Omega} + \gint{\del \Omega}{}{s_{i} u_{i}^{*}}{\Gamma}
  \label{eqn:virtwork2}
\end{equation}
where the internal work done due to the stress field is equated to the work
due to internal body forces and external surface forces.  Note that
\eqnref{eqn:virtwork2} is equivalent to \eqnref{eqn:virtws} via
\eqnref{eqn:virtwork1}. In practice, \eqnref{eqn:virtwork2} is in a more
useful form than \eqnref{eqn:virtws}, because the right hand side integrals
can be expressed in terms of the known body forces and the applied boundary
conditions (surface traction forces or stresses).


\subsection{The Finite Element Approximation}

Let $\Omega = \bigcup\thinspace\Omega_{e}$ and interpolate the virtual
displacements $u_{i}^{*}$ from their nodal values. \ie
\begin{gather}
  \begin{aligned}
    u_{i}^{*} &= \lbfnsymb{m} \dot (u_{i}^{m})^{*} \\
    \text{so }\quad u_{i,j}^{*} &= \delby{\lbfnsymb{m}}{x_{j}} \dot (u_{i}^{m})^{*} \\
    &= \lbfnsymb{m,k} \delby{\xi_{k}}{x_{j}} \dot (u_{i}^{m})^{*}    
  \end{aligned}
  \label{eqn:virtdisplapprox}
\end{gather}
where $(u_{i}^{m})^{*} = (U_{i}^{\triangle(m,e)})^{*}$, $\triangle(m,e)$ is
the global node number of local node $m$ on element $e$, and the shorthand
$\lbfnsymb{m,k}=\delby{\lbfnsymb{m}}{\xi_{k}}$ has been introduced.

Substituting this into \eqnref{eqn:virtwork2} gives
\begin{equation*}
  \sum_{e} \pbrac{
    \gint{\thickspace\Omega_{e}}{}{\sigma_{ij}\lbfnsymb{m,k}\delby{\xi_{k}}{x_{j}}}
    {\Omega} } \pbrac{ U_{i}^{\triangle(m,e)} }^{*} = \sum_{e} \pbrac{
    \gint{\Omega_{e}}{}{b_{i}\lbfnsymb{m}}{\Omega} + \gint{\del
    \Omega_{e}}{}{s_{i}\lbfnsymb{m}}{\Gamma} } \pbrac{U_{i}^{\triangle(m,e)}
    }^{*}
\end{equation*}
and since the virtual displacements are arbitrary we get
\begin{equation}
  \sum_{e}\gint{\Omega_{e}}{}{\sigma_{ij}\lbfnsymb{m,k}\delby{\xi_{k}}{x_{j}}}{\Omega}
    = \sum_{e} \pbrac{\gint{\Omega_{e}}{}{b_{i}\lbfnsymb{m}}{\Omega} 
    + \gint{\del \Omega_{e}}{}{s_{i}\lbfnsymb{m}}{\Gamma}}
  \label{eqn:intequilib}
\end{equation}

The next step is to express the stress components $\sigma_{ij}$ in terms of
the virtual displacements and their finite element approximation by
substituting \eqnref{eqn:virtdisplapprox} into \eqnref{eqn:sst} (the
strain-displacement relation) and in turn into \eqnref{eqn:genHook} (the
generalized Hooke's law).

We first introduce the finite element approximation for the displacement field
$u_{j}=\lbfnsymb{n}u_{j}^{n}$ which gives
\begin{equation}
  e_{ij} = \frac{1}{2}\pbrac{\hdelby{\lbfnsymb{n}u_{i}^{n}}{x_{j}} +
  \hdelby{\lbfnsymb{n}u_{j}^{n}}{x_{i}}} =
  \frac{1}{2}\pbrac{\delby{\lbfnsymb{n}}{\xi_{l}}\delby{\xi_{l}}{x_{j}}u_{i}^{n}
  + \delby{\lbfnsymb{n}}{\xi_{l}}\delby{\xi_{l}}{x_{i}}u_{j}^{n}}
\end{equation}
and
\begin{equation*}
  e_{kk} = u_{k,k} = \delby{\lbfnsymb{n}}{\xi_{l}}\delby{\xi_{l}}{x_{k}}u_{k}^{n}
\end{equation*}
Thus
\begin{equation*}
  \sigma_{ij} = \lambda \delta_{ij}
  \delby{\lbfnsymb{n}}{\xi_{l}}\delby{\xi_{l}}{x_{k}}u_{k}^{n} + 2\mu \pbrac{
  \frac{1}{2}\delby{\lbfnsymb{n}}{\xi_{l}}\delby{\xi_{l}}{x_{j}}u_{i}^{n} +
  \frac{1}{2}\delby{\lbfnsymb{n}}{\xi_{l}}\delby{\xi_{l}}{x_{i}}u_{j}^{n}}
\end{equation*}
which, due to symmetry of the stress tensor, simplifies to
\begin{align}
  \sigma_{ij} &= \lambda \delta_{ij}
  \delby{\lbfnsymb{n}}{\xi_{l}}\delby{\xi_{l}}{x_{k}}u_{k}^{n} + 2\mu 
  \delby{\lbfnsymb{n}}{\xi_{l}}\delby{\xi_{l}}{x_{i}}u_{j}^{n} \nonumber \\
              &= \pbrac{\lambda \delta_{i(j)}
  \lbfnsymb{n,l}\delby{\xi_{l}}{x_{j}} + 2\mu 
  \lbfnsymb{n,l}\delby{\xi_{l}}{x_{i}}}u_{j}^{n}
  \label{eqn:stresscmpts}
\end{align}
where the summation index $k$ has been replaced with $j$, but the parenthesis
in $\delta_{i(j)}$ implies that there is no sum with respect to that
particular index.

Substituting this expression into \eqnref{eqn:intequilib} and simplifying, we
get for each element
\begin{equation}
    u_{j}^{n} \gint{\Omega_{e}}{}{\pbrac{ \lambda
       \lbfnsymb{n,l}\delby{\xi_{l}}{x_{j}}
       \lbfnsymb{m,k}\delby{\xi_{k}}{x_{i}} + 2\mu
       \lbfnsymb{n,l}\delby{\xi_{l}}{x_{i}}
       \lbfnsymb{m,k}\delby{\xi_{k}}{x_{j}}}}{\Omega} = f_{im}
  \label{eqn:linelasfem}
\end{equation}
where $f_{im}$ denotes the right hand side terms in
\eqnref{eqn:intequilib}. (Note that there has been some careful manipulation
of summation indices with the substitution of \eqnref{eqn:stresscmpts} to
arrive at \eqnref{eqn:linelasfem}.)

So for each element
\begin{equation*}
  E_{imjn} u_{j}^{n} = f_{im}
\end{equation*}
where
\begin{gather}
  \begin{aligned}
    E_{imjn} &= \iiint\limits_{0}^{1} \pbrac{ \lambda
       \delby{\xi_{l}}{x_{j}}\delby{\xi_{k}}{x_{i}} + 2\mu
       \delby{\xi_{l}}{x_{i}}\delby{\xi_{k}}{x_{j}}} 
       \lbfnsymb{n,l} \lbfnsymb{m,k} J(\xione,\xitwo,\xithree) d\xione
       d\xitwo d\xithree \\ 
    f_{im} &= \iiint\limits_{0}^{1} b_{i}\lbfnsymb{m}
       J(\xione,\xitwo,\xithree) d\xione d\xitwo d\xithree +
       \iint\limits_{0}^{1} s_{i}\lbfnsymb{m} 
       J_{2D}(\xione,\xitwo) d\xione d\xitwo
  \end{aligned}
\end{gather}
where the Jacobians $J(\xione,\xitwo,\xithree)$ and $J_{2D}(\xione,\xitwo)$
have been used to transform volume and surface integrals so that they can be
can be calculated using $\xi$-coordinates. (Note: without loss of generality,
the above definition of $f_{im}$ assumes that $(\xione,\xitwo)$ are defined to
lie in the surface $\Gamma$.)

So in summary, the finite element approximation leads to element stiffness
matrix components that can be calculated from the known material parameters,
the chosen interpolation functions, and the geometry of the material (note
that the element stiffness components are independent of the unknown
displacement parameters).  Element stiffness components are then assembled
into the global stiffness matrix in the usual manner (as described
previously).  Note that this is implicitly a Galerkin formulation, since the
unknown displacement fields are interpolated using the same basis functions as
those used to weight the integral equations.

%\begin{example}{Stresses in a plate with a hole}
%  {Stresses in a plate with a hole}
  
%  \todo{example???}

%  A common problem in structural mechanics is to find the stress concentration
%  produced by a hole in an otherwise uniformly loaded structural component.
%  Consider the plate below loaded by horizontal forces of \nunit{100}{\kNpm}.
%  The plate is \nunit{1}{\m} thick and made of steel (Young's modulus
%  \nunit{100}{\GPa}, Poisson's ratio $0.3$).

%  \begin{figure}[htbp] \centering
%    \input{figs/lin_elasticity/CMISSB.pstex}
%  \end{figure}

%  \label{xmp:stressesinplate}
%\end{example}


\section{Linear Elasticity with Boundary Elements}
\label{sec:Lewbe-4.8}

\Eqnref{eqn:virtws} is the starting point for the general finite element
formulation (\Secref{sec:3Delast}). In the above derivation, we have 
essentially used the Green-Gauss theorem once to move from \eqnref{eqn:integral} 
to \eqnref{eqn:virtws} (as was done for the derivation of the FEM equation for
Laplace's equation). To continue, we firstly note that
\begin{align*}
  \sigma_{ij}e_{ij}^{*} &= \dfrac{1}{2}\sigma_{ij}u_{i,j}^{*} +
  \dfrac{1}{2}\sigma_{ij}u_{j,i}^{*}\\ 
  &= \dfrac{1}{2}\sigma_{ij}u_{i,j}^{*} + \dfrac{1}{2}\sigma_{ji}u_{j,i} ^{*} \\ 
  &= \dfrac{1}{2}\sigma_{ij}u_{i,j}^{*} + \dfrac{1}{2}\sigma_{ij}u_{i,j} ^{*} \\ 
  &= \sigma_{ij} u_{i,j}^{*}
\end{align*}
where $e_{ij}^{*}$ are the virtual strains corresponding to the virtual
displacements.

Using the constitutive law for linearly elastic materials
(\eqnref{eqn:genHook}) we have
\begin{align*}
  \goneint{\sigma_{ij}u_{i,j}^{*}}{\Omega} &=
  \goneint{\sigma_{ij} e_{ij}^{*}}{\Omega} \\ 
  &= \lambda\goneint{e_{kk} e_{ij}^{*} \delta_{ij}}{\Omega} + 
  2\mu\goneint{e_{ij} e_{ij}^{*}}{\Omega} \\ 
  &= \lambda\goneint{e_{kk} e_{kk}^{*}}{\Omega} + 
  2\mu \goneint{e_{ij} e_{ij}^{*}}{\Omega} \\ 
  &= \goneint{e_{ij} \sigma_{ij}^{*}}{\Omega}
\end{align*}
due to symmetry.

Thus from the virtual work statement, \eqnref{eqn:virtws} and the above
symmetry we have
\begin{equation}
  \goneint{b_{i}u_{i}^{*}}{\Omega} + \gint{\del \Omega}{} 
  {t_{i}u_{i}^{*}}{\Gamma} = \goneint{b_{i}^{*}u_{i}}{\Omega} + 
  \gint{\del \Omega}{}{t_{i}^{*} u_{i}}{\Gamma}
  \label{eqn:Bsrwt}
\end{equation}
This is known as Betti's second reciprical work theorem or the Maxwell-Betti
reciprocity relationship between two different elastic problems (the starred
and unstarred variables) established on the same domain.

Note that $b_{i}^{*} = -\sigma_{ij,j} ^{*}$  (\ie $\sigma_{ij,j} ^{*} +
b_{i}^{*} =  0$). Therefore \eqnref{eqn:Bsrwt} can be written as
\begin{equation}
  \goneint{\pbrac{\sigma_{ij,j} ^{*}} u_{i}}{\Omega} + 
  \goneint{b_{i} u_{i}^{*}}{\Omega} = 
  \gint{\del \Omega}{}{t_{i}^{*} u_{i}}{\Gamma} - 
  \gint{\del \Omega}{}{t_{i} u_{i}^{*}}{\Gamma}
  \label{eqn:eqnD}
\end{equation}
($\sigma_{ij} ^{*},e_{ij}^{*},t_{i}^{*}$ represents the equilibrium state
corresponding to the virtual displacements $u_{i}^{*}$).

Note: What we have essentially done is use integration of parts to get
\eqnref{eqn:virtws}, then use it again to get \eqnref{eqn:Bsrwt} above (after
noting the reciprocity between $\sigma_{ij}$ and $e_{ij}$).

Since the body forces, $b_{i}$, are known functions, the second domain
integral on the left hand side of \eqnref{eqn:eqnD} does not introduce any
unknowns into the problem (more about this later). The first domain integral
contains unknown displacements in $\Omega$ and it is this integral we wish to
remove.
 
We choose the virtual displacements such that
\begin{equation}
  \sigma_{ij,j}^{*} + e_{i} \delta = 0 
  \label{eqn:FundSol}
\end{equation}
(or equivalently $- b_{i}^{*} + e_{i} \delta = 0$), where $e_{i}$ is the \nth{i}
component of a unit vector in the \nth{i} direction and $e_{i}\delta =
e_{i}\fnof{\delta}{\vect{x} - P}$. We can interpret this as the body force
components which correspond to a positive unit point load applied at a point
$P \in \Omega$ in each of the three orthogonal directions.

Therefore

\begin{displaymath}
  \gint{\Omega}{}{\sigma_{ij,j}^{*} u_{i}}{\Omega} = -
  \gint{\Omega}{}{\fnof{\delta}{\vect{x} - P} e_{i} u_{i}}{\Omega} = - u_{i}(P)e_{i}
\end{displaymath}
\ie the volume integral is replaced with a point value (as for Laplace's 
equation).

Therefore, \eqnref{eqn:eqnD} becomes
\begin{equation}
  \fnof{u_{i}}{P}e_{i}  = \gint{\del \Omega}{}{t_{j} u_{j}^{*}}{\Gamma}  -  
  \gint{\del \Omega}{}{t_{j}^{*} u_{j}}{\Gamma} + 
  \goneint{b_{j} u_{j}^{*}}{\Omega} \qquad P \in \Omega
  \label{eqn:eqnE}
\end{equation}

If each point load is taken to be independent then $u_{j}^{*}$ and $t_{j}^{*}$
can be written as
\begin{align}
  u_{j}^{*} &= \fnof{u_{ij}^{*}}{P,x} e_{i} \\
  t_{j}^{*} &= \fnof{t_{ij}^{*}}{P,x} e_{i}
\end{align}
where $\fnof{u_{ij}^{*}}{P,x}$ and $\fnof{t_{ij}^{*}}{P,x}$ represent the
displacements and tractions in the $\nth{j}$ direction at $x$ corresponding to
a unit point force acting in the $\nth{i}$ direction ($e_{i}$) applied at $P$.
Substituting these into \eqnref{eqn:eqnE} (and equating components in each
$e_{i}$ direction) yields
\begin{multline}
  \fnof{u_{i}}{P} = \gint{\del \Omega}{}{\fnof{u_{ij}^{*}}{P,x}
    \fnof{t_{j}}{x}}{\fnof{\Gamma}{x}} - \gint{\del
    \Omega}{}{\fnof{t_{ij}^{*}}{P,x} \fnof{u_{j}}{x}}{\fnof{\Gamma}{x}} \\ 
  + \gint{\Omega}{}{\fnof{u_{ij}^{*}}{P,x}
    \fnof{b_{j}}{x}}{\fnof{\Omega}{x}}
 \label{eqn:eqnF}
\end{multline}

where $P \in \Omega$ (see later for $P \in \del\Omega$).

This is known as Somigliana's \footnote{Somigliana was an Italian
  Mathematician who published this result around 1894-1902.} identity for
displacement.

\section{Fundamental Solutions}

Recall from \eqnref{eqn:FundSol} that $\sigma_{ij}^{*}$ satisfied
\begin{equation}
  \sigma_{ij,j}^{*} + \fnof{\delta}{\vect{x} - P}e_{i}  = 0 
\end{equation} 
or equivalently
\begin{displaymath}
  b_{i}^{*} = e_{i} \fnof{\delta}{\vect{x} - P}
\end{displaymath}
Navier's equation for the displacements $u_{i}^{*}$ is
\begin{displaymath}
  G\;u_{i,kk}^{*}  +  \dfrac{G}{1-2\nu}u_{k,ki}^{*}  + b_{i}^{*} = 0 
\end{displaymath}
where $G$ = shear Modulus.

Thus $u_{i}^{*}$ satisfy
\begin{equation}
  G\;u_{i,kk}^{*}  +  \dfrac{G}{1-2\nu}u_{k,ki}^{*} + \fnof{\delta}{\vect{x} -
    P}e_{i} = 0 
  \label{eqn:satisfy}
\end{equation}

The solutions to the above equation in either two or three dimensions are
known as Kelvin~\footnote{Lord Kelvin (1824-1907) Scottish physicist who made
  great contributions to the science of thermodynamics}'s fundamental
solutions\index{Fundamental solution!Kelvin} and are given by
\begin{equation}
  \fnof{u_{ij}^{*}}{P,\vect{x}} = \dfrac{1}{16\pi\pbrac{1 - \nu}Gr}\bbrac{\pbrac{3
    -4\nu} \delta_{ij} + r_{,i}r_{,j}}
  \label{eqn:Kelvin}
\end{equation}
for three-dimensions and for two-dimensional plane strain problems,
\begin{equation}
  \fnof{u_{ij}^{*}}{P,\vect{x}} = \dfrac{-1}{8\pi\pbrac{1 -
      \nu}G}\bbrac{\pbrac{3 -4\nu} \delta_{ij} \log r - r_{,i}r_{,j}}
  \label{eqn:threeD}
\end{equation}
and 
\begin{equation}
  \fnof{t_{ij}^{*}}{P,\vect{x}} = \dfrac{-1}{4 \alpha \pi\pbrac{1 - \nu}r^{\alpha}}
  \bbrac{\pbrac{\pbrac{1-2\nu} \delta_{ij} + \beta r_{,i}r_{,j}} 
      \delby{r}{n} - \pbrac{1-2\nu}\pbrac{r_{,i}n_{j} -r_{,j}n_{i}}} 
  \label{eqn:twoD}
\end{equation}
where $\alpha = 1,2; \beta = 2,3$ for two-dimensional plane strain and
three-dimensional problems respectively.

Here $r\equiv \fnof{r}{P,\vect{x}}$, the distance between load point ($P$) and
field point ($\vect{x}$), 
$r_{i}  = \fnof{x_{i}}{\vect{x}}-\fnof{x_{i}}{P}$ and
$r_{,i} = \delby{r}{\fnof{x_{i}}{\vect{x}}}= \dfrac{r_{i}}{r}$.

In addition the strains at an point $\vect{x}$ due to a unit point load applied
at $P$ in the $\nth{i}$ direction are given by
\begin{displaymath}
  \fnof{e_{jki}^{*}}{P,\vect{x}} = \dfrac{-1}{8 \alpha \pi \pbrac{1-\nu}
    Gr^{\alpha}} \bbrac{\pbrac{1-2\nu}\pbrac{r_{,k} \delta_{ij} +r_{,j} \delta
    _{ik}} - r_{,i} \delta_{jk} + \beta r_{,i} r_{,j}r_{,k}}
\end{displaymath}
and the stresses are given by
\begin{displaymath}
  \fnof{\sigma_{ijk}^{*}}{P,\vect{x}} = \dfrac{-1}{4 \alpha \pi \pbrac{1-\nu}
    r^{\alpha}} \bbrac{\pbrac{1-2\nu}\pbrac{r_{,k} \delta_{ij} +r_{,j} \delta
      _{ki} - r_{,i} \delta_{jk}} + \beta r_{,i} r_{,j}r_{,k}}
\end{displaymath}
where $\alpha$ and $\beta$ are defined above.

The plane strain expressions are valid for plane stress if $\nu$ is replaced
by $\overline{\nu} = \dfrac{\nu}{1 + \nu}$ (This is a mathematical equivalence
of plane stress and plane strain - there are obviously physical differences.
What the mathematical equivalence allows us to do is to use one program to
solve both types of problems - all we have to do is modify the values of the
elastic constants).

Note that in three dimensions
\begin{displaymath}
  u_{ij}^{*} = \orderof{\frac{1}{r}}\hspace{0.5in} t_{ij}^{*} = 
  \orderof{\frac{1}{r^{2}}}
\end{displaymath}
 and for two dimensions
\begin{displaymath}
  u_{ij}^{*} = \orderof{\log r}\hspace{0.5in}  t_{ij}^{*} = \orderof{\frac{1}{r}}.
\end{displaymath}


Somigliana's identity (\eqnref{eqn:eqnF}) is a continuous representation of
displacements at any point $P \in \Omega$.  Consequently, one can find the
stress at any $P \in \Omega$ firstly by combining derivatives of
\bref{eqn:eqnF} to produce the strains and then substituting into Hooke's law.
Details can be found in \citeasnoun{brebbia:1984} pp 190--191, 255--258.

This yields
\begin{align*}
  \fnof{\sigma_{ij}}{P} &= \gint{\Gamma}{}{\fnof{u_{ijk}^{*}}{P,\vect{x}}
    \fnof{t_{k}}{\vect{x}}}{\fnof{\Gamma}{\vect{x}}} -
    \gint{\Gamma}{}{\fnof{t_{ijk}^{*}}{P,\vect{x}}
    \fnof{u_{k}}{\vect{x}}}{\fnof{\Gamma}{\vect{x}}} \\ 
  &+ \gint{\Omega}{}{\fnof{u_{ijk}^{*}}{P,\vect{x}}
    \fnof{b_{k}}{\vect{x}}}{\fnof{\Omega}{\vect{x}}}
\end{align*}

Note: One can find internal stress via numerical differentiation as in FE/FD
but these are not as accurate as the above expressions.

Expressions for the new tensors $ u_{ijk}^{*}$ and $t_{ijk}^{*}$ are on page
191 in \cite{brebbia:1984}.

\section{Boundary Integral Equation}
\label{sec:BIE,sec4.10}

Just as we did for Laplace's equation we need to consider the limiting case of
\eqnref{eqn:eqnF} as $P$ is moved to $\del \Omega$. (\ie we need to find the
equivalent of $\fnof{c}{P}$ (in section 3) - called here $\fnof{c_{ij}}{P}$.)
We use the same procedure as for Laplace's equation but here things are not so
easy.

If  $P \in \del \Omega$ we enlarge $\Omega$ to $\Omega^{\prime}$ as 
shown.

\begin{figure}[htbp] \centering
  \input{figs/lin_elasticity/illus.pstex}
  \caption{Illustration of enlarged domain when singular point is on the 
    boundary.}
\end{figure}
Then \eqnref{eqn:eqnF} can be written as
\begin{multline}
  \fnof{u_{i}}{P} = \gint{\Gamma_{-\varepsilon} +\Gamma_{\varepsilon}}{}
  {\fnof{u_{ij}^{*}}{P,\vect{x}}
    \fnof{t_{j}}{\vect{x}}}{\fnof{\Gamma}{\vect{x}}} -
  \gint{\Gamma_{-\varepsilon}+
    \Gamma_{\varepsilon}}{}{\fnof{t_{ij}^{*}}{P,\vect{x}}
    \fnof{u_{j}}{\vect{x}}}{\fnof{\Gamma}{\vect{x}}} \\ +
    \gint{\Omega^{\prime}}{}{\fnof{u_{ij}^*}{P,\vect{x}} \fnof{b_{j}}{\vect{x}}}
    {\fnof{\Omega}{\vect{x}}} 
  \label{eqn:Fwrit}
\end{multline} 

We need to look at each integral in turn as $\varepsilon^{\downarrow}0$ (\ie
$\varepsilon \rightarrow 0$ from above).  The only integral that presents a
problem is the second integral. This can be written as
\begin{multline}
  \gint{\Gamma_{-\varepsilon}+\Gamma_{\varepsilon}}{}{\fnof{t_{ij}^{*}}{P,\vect{x}} \fnof{u_{j}}{\vect{x}}}{\fnof{\Gamma}{\vect{x}}} =
  \gint{\Gamma_{\varepsilon}}{}{\fnof{t_{ij}^{*}}{P,\vect{x}}
    \fnof{u_{j}}{\vect{x}}}{\fnof{\Gamma}{\vect{x}}} \\ +
  \gint{\Gamma_{-\varepsilon}}{}{\fnof{t_{ij}^{*}}{P,\vect{x}}
    \fnof{u_{j}}{\vect{x}}}{\fnof{\Gamma}{\vect{x}}}
  \label{eqn:2ndint}
\end{multline} 

The first integral on the right hand side can be written as
\begin{multline}
  \gint{\Gamma_{\varepsilon}}{}{\fnof{t_{ij}^{*}}{P,\vect{x}}
    \fnof{u_{j}}{\vect{x}}}{\fnof{\Gamma}{\vect{x}}} =
  \underbrace{\gint{\Gamma_{\varepsilon}}{}{\fnof{t_{ij}^{*}}{P,\vect{x}}
      \sqbrac{\fnof{u_{j}}{\vect{x}}-\fnof{u_{j}}{P}}}{\Gamma(x)}}_{\text{$0$
      by continuity of $\fnof{u_{j}}{x}$}} \\ + \fnof{u_{j}}{P}
  \gint{\Gamma_{\varepsilon}}{}{\fnof{t_{ij}^{*}}{P,\vect{x}}}
  {\fnof{\Gamma}{\vect{x}}}
\label{eqn:1stint}
\end{multline}  
  
Let
\begin{equation}
  \fnof{c_{ij}}{P} = \delta_{ij} + \lim_{\varepsilon \downarrow 0}
   \gint{\Gamma_{\varepsilon}}{}{\fnof{t_{ij}^{*}}{P,\vect{x}}}
   {\fnof{\Gamma}{\vect{x}}}
 \label{eqn:eqnG}
\end{equation}

As $\varepsilon^{\downarrow} 0$, $\Gamma_{-\varepsilon}\rightarrow \Gamma$ and
we write the second integral of \eqnref{eqn:2ndint} as $\gint{\Gamma}{}
{\fnof{t_{ij}^{*}}{P,\vect{x}}
  \fnof{u_{j}}{\vect{x}}}{\fnof{\Gamma}{\vect{x}}}$ where we interpret this in
the Cauchy Principal Value\footnote{What is a Cauchy Principle Value? 
                \newline Consider $\fnof{f}{x} = \dfrac{1}{x}$ on
    $\Gamma_{-\varepsilon} = \sqbrac{-2,-\varepsilon) \cup (\varepsilon,2}$
   \newline Then
      \begin{alignat*}{2}
        \gint{\Gamma_{-\varepsilon}}{}{\fnof{f}{x}}{x} &=
        \gint{-2}{-\varepsilon}{\dfrac{1}{x}}{x} +
        \gint{\varepsilon}{2}{\dfrac{1}{x}}{x} 
        = \evalat{\ln \abs{x}}{{-2}}^{-\varepsilon} + 
          \evalat{\ln\abs{x}}{{\varepsilon}}^{2} && \\ 
          &= \ln \varepsilon -\ln 2 + \ln 2
        - \ln \varepsilon=0 \;\;\forall \varepsilon>0 && \\ & \Rightarrow
        \displaystyle{\lim_{\varepsilon \rightarrow 0}}
        \gint{\Gamma_{-\varepsilon}}{}{\fnof{f}{x}}{x} = 0 && 
                                \text{This is the Cauchy Principle Value of}
                        \gint{\Gamma}{}{\fnof{f}{x}}{x}
%%                                \label{eqn:CPV}
      \end{alignat*}
   But if we replace $\Gamma_{-\varepsilon}$  by 
   $\displaystyle{\lim_{\varepsilon\rightarrow 0}}\Gamma_{-\varepsilon} = 
   \sqbrac{-2,2}  = \Gamma$ then 
   \begin{align*}
        \gint{\Gamma}{}{\dfrac{1}{x}}{x} = \gint{-2}{2}
        {\dfrac{1}{x}}{x} = \pbrac{\displaystyle{\lim_{\varepsilon_{1} 
              \rightarrow 0}}\gint{-2}{\varepsilon_{1}}{\dfrac{1}{x}}{x}}  +
        \pbrac{\displaystyle{\lim_{\varepsilon_{2} 
              \rightarrow 0}}\gint{-\varepsilon_{2}}{2}{\dfrac{1}{x}}{x}}
        &\text{(by definition of improper integration)}
%        \label{eqn:bydefn}
    \end{align*}
    \newline which does NOT exist. \ie the integral does not exist in the
    proper sense, but it does in the Cauchy Principal Value sense. However, if
    an integral exists in the proper sense, then it exists in the Cauchy
    Principal Value sense and the two values are the same.}sense. 

Thus as $\varepsilon^{\downarrow}0$ we get the boundary integral equation
\begin{multline}
  \fnof{c_{ij}}{P} \fnof{u_{j}}{P} +
  \gint{\Gamma}{}{\fnof{t_{ij}^{*}}{P,\vect{x}}
    \fnof{u_{j}}{\vect{x}}}{\fnof{\Gamma}{\vect{x}}} \\
  = \gint{\Gamma}{}{\fnof{u_{ij}^{*}}{P,\vect{x}}
    \fnof{t_{j}}{\vect{x}}}{\fnof{\Gamma}{\vect{x}}} 
  + \goneint{\fnof{u_{ij}^{*}}{P,\vect{x}} \fnof{b_{j}}{\vect{x}}}{\Omega}
  \label{eqn:eqnH}
\end{multline}
(or, in brief (no body force), $c_{ij} u_{j} + \gint{\Gamma}{}{t_{ij}^{*}
u_{j}}{\Gamma} = \gint{\Gamma}{}{u_{ij}^{*} t_{j}}{\Gamma} $) where the
integral on the left hand side is interpreted in the Cauchy Principal sense.
In practical applications $c_{ij}$ and the principal value integral can be
found indirectly from using \eqnref{eqn:eqnH} to represent rigid-body
movements.

The numerical implementation of \eqnref{eqn:eqnH} is similar to the numerical 
implementation of an elliptic equation (\eg Laplace's Equation).  However, whereas
with Laplace's equation the unknowns were $u$ and $\delby{u}{n}$ (scalar 
quantities) here the unknowns are vector quantities. Thus it is more 
convenient to work with matrices instead of indicial notation.
\newline \ie use
\begin{displaymath}
  \vect{u} = \begin{bmatrix}
    u_{1} \\
    u_{2} \\
    u_{3}
  \end{bmatrix}, \qquad 
  \vect{t} = \begin{bmatrix}
    t_{1} \\ 
    t_{2} \\
    t_{3}
  \end{bmatrix}
\end{displaymath}
\begin{displaymath}
  \vect{u}^{*} = \begin{bmatrix}
    u_{11}^{*} &  u_{12}^{*} &  u_{13}^{*}\\
    u_{21}^{*} &  u_{22}^{*} &  u_{23}^{*}\\
    u_{31}^{*} &  u_{32}^{*} &  u_{33}^{*}
  \end{bmatrix}, \qquad
  \vect{t}^{*} = \begin{bmatrix}
    t_{11}^{*} &  t_{12}^{*} &  t_{13}^{*}\\
    t_{21}^{*} &  t_{22}^{*} &  t_{23}^{*}\\
    t_{31}^{*} &  t_{32}^{*} &  t_{33}^{*}
  \end{bmatrix}
\end{displaymath}
Then (in absence of a body force) we can write \eqnref{eqn:eqnH} as 
\begin{equation}
  \vect{cu} + \goneint{\vect{t}^{*}\vect{u}}{\Gamma} =
  \goneint{\vect{u}^{*}\vect{t}}{\Gamma}
  \label{eqn:eqnH*}
\end{equation}

We can discretise the boundary as before and put $P$, the singular point, at
each node (each node has $6$ unknowns - $3$ displacements and $3$ tractions - we get
$3$ equations per node). The overall matrix equation
\begin{equation}
  \matr{A}\vect{U}=\matr{B}\vect{T}
  \label{eqn:matrix}
\end{equation}
where $\vect{U} = 
\begin{bmatrix}
  \vect{u}_{1} \\ 
  \vect{u}_{2} \\ 
  \vdots \\ 
  \vect{u}_{n}
\end{bmatrix} $ and  $ \vect{t} = 
\begin{bmatrix}
  \vect{t}_{1} \\
  \vect{t}_{2} \\ 
  \vdots \\
  \vect{t}_{n}
\end{bmatrix}$ where $n$ is the number nodes.

The diagonal elements of the $\matr{A}$ matrix in \eqnref{eqn:matrix} 
(for three-dimensions, a $3$ x $3$
matrix) contains principal value components. If we have a rigid-body
displacement of a \emph{finite} body in any one direction then we get
\begin{equation*}
  \matr{A}\vect{I}_{l} = \vect{0}
\end{equation*}
($\vect{I}_{l}$ = vector defining a rigid body displacement in direction $l$)
\begin{displaymath}
    \Rightarrow a_{ii} = -\dsuml{i\neq j}{}a_{ij} \qquad \text{(no sum on $i$)} 
\end{displaymath} 
\ie the diagonal entries of $\matr{A}$ (the $c_{ij}$'s) do not need to be
determined explicitly. There is a similar result for an infinite body.

\section{Body Forces (and Domain Integrals in General)}

The body force gives rise to a domain integral although it does not give rise
to any further unknowns in the system of equations.  (This is because the body
force is known - the fundamental solution was chosen so that it removed all
unknowns appearing in domain integrals).

Thus \eqnref{eqn:eqnH} is still classed as a Boundary Integral Equation.
Integrals over the domain containing known functions (eg body force integral)
appear in many situations \eg the Poisson equation $\laplacian{u}=f$ yields a
domain integral involving $f$.

The question is how do we evaluate domain integrals such as those appearing in
the boundary integral forumalation of such equations?  Since the functions are
known a \emph{coarse} domain mesh may work.(\nb Since the integral also
contains the fundamental solution and $\Omega$ may not be a ``nice'' region it
is unlikely that it can be evaluated analytically). However, a domain mesh
nullifies one of the advantages of BEM - that of having to prepare only a
boundary mesh.

In some cases domain integrals must be used but there are techniques
developing to avoid many of them.  In some standard situations a domain
integral can be transformed to a boundary integral. \eg a body force arising
from a constant gravitational load, or a centrifugal load due to rotation
about a fixed axis or the effect of a steady state thermal load can all be
transformed to a boundary integral.

Firstly, let $G_{ij}^{*}$ (the Galerkin tension) be related to $u_{ij}^{*}$ by
\begin{align*}
    u_{ij}^{*} &= G_{ij,kk}^{*} -  \dfrac{1}{2\pbrac{1-\nu}}G_{ik,kj}^{*} \\
    \Rightarrow G_{ij} &=
    \left\{ \begin{array}{ll}
        \dfrac{1}{8 \pi G} r \delta_{ij} & \text{ (3D)}\\
        \dfrac{1}{8 \pi G}  r^{2} \log \pbrac{\dfrac{1}{r}} \delta_{ij}
        & \text{ (2D)}
      \end{array} \right.
\end{align*}

Then
\begin{displaymath}
  B_{i}  =  \goneint{u_{ij}^{*} b_{j}}{\Omega} =   
  \goneint{\pbrac{G_{ij,kk}^{*} -\dfrac{1}{2\pbrac{1-\nu}} G_{ik,kj}^{*}}
  b_{j}}{\Omega}
\end{displaymath} 

Under a constant gravitational load $\vect{g}=\pbrac{g_{j}}$
\begin{align*}
  b_{j} &= \rho g_{j}\\
  \Rightarrow B_{i} &= \rho g_{j} \goneint{\pbrac{G_{ik,j}^{*} 
  -\dfrac{1}{2\pbrac{1-\nu}} G_{ik,kj}^{*}}}{\Omega} \\
  &= \rho g_{j} \goneint{\bbrac{G_{ij,k}^{*} -\dfrac{1}{2\pbrac{1-\nu\}} 
  G_{ik,j}^{*}}}n_{k}}{\Gamma}
\end{align*}
which is a boundary integral. 

Unless the domain integrand is ``nice'' the above simple application of
Green's theorem won't work in general. There has been a considerable amount of
research on domain integrals in BEM which has produced techniques for
overcoming some domain methods. The two integrals of note are the DRM, dual
reciprocity method, developed around 1982 and the MRM, multiple reciprocity
method, developed around 1988.

\clearpage 
\section{CMISS Examples}

\begin{enumerate}
\item   To solve a truss system run CMISS example $411$
  This solves the simple three truss system shown in
  \Figref{fig:system}. 
  \label{xmp:Solving}

\item  To solve stresses in a bicycle frame modelled with truss elements 
  run CMISS example $412$.
  \begin{figure}[htbp] \centering
    \input{figs/lin_elasticity/CMISSA.pstex}
  \end{figure}
  \label{xmp:stressesinabike}
\end{enumerate}

%\begin{example}{Stresses in a bicycle frame modelled with beam elements}
%  {Stresses in a bicycle frame modelled with beam elements.}
%  \todo{example???}

%  \begin{figure}[htbp] \centering
%    \input{figs/CMISSA.pstex}
%  \end{figure}
%  \label{xmp:Sbfbe}
%\end{example}
%\index{Beam elements|)}

%\begin{example}{Stresses in a plate with a hole}
%  {Stresses in a plate with a hole}
  
%  \todo{example???}

%  A common problem in structural mechanics is to find the stress concentration
%  produced by a hole in an otherwise uniformly loaded structural component.
%  Consider the plate below loaded by horizontal forces of \nunit{100}{\kNpm}.
%  The plate is \nunit{1}{\m} thick and made of steel (Young's modulus
%  \nunit{100}{\GPa}, Poisson's ratio $0.3$).

%  \begin{figure}[htbp] \centering
%    \input{figs/lin_elasticity/CMISSB.pstex}
%  \end{figure}
%  \label{xmp:stressesinplate}
%\end{example}


%%% Local Variables: 
%%% mode: latex
%%% TeX-master: "/product/cmiss/documents/notes/fembemnotes/fembemnotes"
%%% End: 


\clearemptydoublepage
 \chapter{Modal Analysis}

\section{Introduction}

The system of ordinary differential equations which results from the
application of the Galerkin finite element (or other) discretization of the
spatial domain to linear parabolic or hyperbolic equations can either be
integrated directly - as in the last section for parabolic equations - or
analysed by \emph{mode superposition}. That is, the time-dependent solution
is expressed as the superposition of the natural (or resonant) modes of the
system. To find these modes requires the solution of an eigenvalue problem.

\section{Free Vibration Modes}

Consider an extension of \eqnref{eqn:orddiff} which includes second order
time derivatives (\eg nodal point accelerations)
\begin{equation}
  \matr{M}\fnof{\ddot{\vect{u}}}{t} + \matr{C}\fnof{\dot{\vect{u}}}{t}
         + \matr{K}\fnof{\vect{u}}{t} = \fnof{\vect{f}}{t}
  \label{eqn:npa}
\end{equation}
$\matr{M},\matr{C}$ and $\matr{K}$ are the mass, damping and stiffness
matrices, respectively, $\fnof{\vect{f}}{t}$ is the external load vector
and $\fnof{\vect{u}}{t}$ is the vector of $n$ nodal unknowns. In direct
time integration methods $\fnof{\ddot{\vect{u}}}{t}$ and
$\fnof{\dot{\vect{u}}}{t}$ are replaced by finite differences and the
resulting system of algebraic equations is solved at successive time steps.
For a small number of steps this is the most economical method of solution
but, if a solution is required over a long time period, or for a large
number of different load  vectors $\fnof{\vect{f}}{t}$, a suitable transformation
\begin{equation}
  \fnof{\vect{u}}{t}=\matr{P}\fnof{\vect{x}}{t}
  \label{eqn:suittrans}
\end{equation}
applied to \eqnref{eqn:npa} can result in the matrices of the transformed system
\begin{equation}
  \transpose{\matr{P}}\matr{M}\matr{P}\fnof{\ddot{\vect{x}}}{t} +
  \transpose{\matr{P}}\matr{C}\matr{P}\fnof{\dot{\vect{x}}}{t} +
  \transpose{\matr{P}}\matr{K}\matr{P}\fnof{\vect{x}}{t}
        = \transpose{\matr{P}}\vect{f}
  \label{eqn:tsyst}
\end{equation}
having a much smaller bandwidth than in the original system and hence being
more economical to solve. In fact, if damping is neglected, $\matr{P}$ can be
chosen to diagonalize $\matr{M}$ and $\matr{K}$ and thereby uncouple the equations
entirely.  This transformation (which is still applicable when damping is
included but does not then result in an uncoupled system unless further
simplications are made) is found by solving the free vibration problem
\begin{equation}
  \matr{M}\fnof{\ddot{\vect{u}}}{t} + \matr{K}\fnof{\vect{u}}{t} 
        = \vect{0}
  \label{eqn:vibprob}
\end{equation}
\textbf{Proof}: Consider a solution to \eqnref{eqn:vibprob} of the form
\begin{equation}
  \fnof{\vect{u}}{t}=\vect{s} \sin \omega \pbrac{t-t_{0}},
  \label{eqn:sineq}
\end{equation}
where $\omega$ and $t_{0}$ are constants and $\vect{s}$ is a vector of order $n$. 
Substituting \eqnref{eqn:sineq} into \eqnref{eqn:vibprob} gives the
\emph{generalized eigenproblem}
\begin{equation}
  \matr{K}\vect{s}=\omega^{2}\matr{M}\vect{s}
  \label{eqn:gene}
\end{equation}
having $n$ \emph{eigensolutions} $\pbrac{\omega_{1}^{2}\vect{s}_{1}},
\pbrac{\omega_{2}^{2}\vect{s}_{2}},\ldots,\pbrac{\omega_{n}^{2}\vect{s}_{n}}$.
If $\matr{K}$ is a symmetric matrix (as is the case when the original partial
differential operator is self-adjoint) the eigenvectors are orthogonal and can
be ``\emph{normalized}'' such that
\begin{equation}
  \transpose{\vect{s}_{i}}\matr{M}\vect{s}_{j}= \left\{
    \begin{array}{cc}
      1 & i=j\\
      0 & i \neq j
    \end{array} \right.
  \label{eqn:norm}
\end{equation}
(the eigenvectors are said to be $M$-\emph{orthonormalised}). Combining the
$n$ eigenvectors into a matrix $\matr{S} = \sqbrac{\vect{s}_{1}, \vect{s}_{2},\ldots,
\vect{s}_{n}}$ - the \emph{modal matrix} - rewriting \eqnref{eqn:norm} as
\begin{equation}
  \transpose{\matr{S}}\matr{M}\matr{S} = \matr{I}
  \label{eqn:mm}
\end{equation}
where $\matr{I}$ is the identity matrix, \eqref{eqn:gene} becomes
\begin{equation}
  \matr{K}\matr{S} = \matr{M}\matr{S}\matr{\Lambda}
  \label{eqn:id}
\end{equation}
where
\begin{equation}
  \matr{\Lambda} = \begin{bmatrix}
    \omega_{1}^{2} &  &  & 0 \\
    & \omega_{2}^{2} & & \\
    & & \ddots & \\
    0 & & & \omega_{n}^{2}
  \end{bmatrix}
  \label{eqn:lambda}
\end{equation} 
or
\begin{equation}
  \transpose{\matr{S}}\matr{K}\matr{S} =
  \transpose{\matr{S}}\matr{M}\matr{S}\matr{\Lambda} = \matr{I}\matr{\Lambda}
  = \matr{\Lambda} 
  \label{eqn:oreq}
\end{equation}

Thus the modal matrix - whose columns are the $M$-orthonormalised eigenvectors
of $\matr{K}$ (\ie satisfying \eqnref{eqn:gene}) - can be used as the
transformation matrix $\matr{P}$ in \eqnref{eqn:suittrans} required to reduce the
original system of equations \bref{eqn:npa} to the \emph{canonical} form
\begin{equation}
  \fnof{\ddot{\vect{x}}}{t} + 
        \transpose{\matr{S}}\matr{C}\matr{S}\fnof{\dot{\vect{x}}}{t} +
  \matr{\Lambda}\fnof{\vect{x}}{t} 
        = \transpose{\matr{S}}\fnof{\vect{f}}{t}
  \label{eqn:canon}
\end{equation}
With damping neglected equation \eqnref{eqn:canon} becomes a system of
uncoupled equations
\begin{equation}
  \fnof{\ddot{x}_{i}}{t} + \omega_{i}^{2}\fnof{x_{i}}{t} = \fnof{r_{i}}{t}
  \qquad i=1,2,\ldots, n
  \label{eqn:uncoup}
\end{equation}
where $x_{i}$ is the \nth{i} component of $\vect{x}$ and $r_{i}$ is the \nth{i}
component of the vector $\transpose{\matr{S}}\vect{f}$. The solution of this
system is given by the Duhamel integral  
\begin{equation}
  \fnof{x_{i}}{t}=\dfrac{1}{\omega_{i}}\gint{0}{t}
  {\fnof{r_{i}}{\tau}\sin\fnof{\omega_{i}}{t-\tau}}{\tau}
  +\alpha_{i}\sin\omega_{i}t+\beta_{i}\cos\omega_{i} d\tau
 \label{eqn:Duh}
\end{equation}
where the constants $\alpha_{i}$ and $\beta_{i}$ are determined from the
initial conditions
\begin{gather}
  \begin{aligned}
    \evalat{\fnof{x_{i}}{0}}{t=0} &= \transpose{{s}_{i}}\matr{M}
      \evalat{\fnof{\vect{u}}{0}}{t=0} \\
    \evalat{\fnof{\dot{x}_{i}}{0}}{t=0} &= \evalat{\fnof{x_{i}}{0}}{t=0} 
                = \transpose{{s}_{i}}\matr{M}\evalat{\fnof{\vect{u}}{0}}{t=0} 
      \transpose{{s}_{i}}\matr{M}\evalat{\fnof{\vect{u}}{0}}{t=0}
  \label{eqn:incond}
  \end{aligned}
\end{gather}

%
%
% Note :
%
%
%\remark{I think the $\dot{x}_{i}$ expression is wrong}
%
%

\section{An Analytic Example}

As an example, consider the equilibrium equations $\matr{M}\ddot{\vect{u}} +
\matr{K}\vect{u}=\vect{f}$ where
\begin{displaymath}
 \matr{M}=\begin{bmatrix}
   2 & 0 \\
   0 & 1
 \end{bmatrix}, \quad
 \matr{K}=\begin{bmatrix}
   6 & -2 \\
   -2 & 4
 \end{bmatrix}\quad\text{and}\quad
 \vect{f}=\begin{bmatrix}
   0 \\
   10
 \end{bmatrix}
\end{displaymath}
To find the solution by modal analysis we first solve the generalised
eigenproblem $\matr{K}\vect{s} = \omega^{2}\matr{M}\vect{s}$ \ie
\begin{displaymath}
  \begin{bmatrix}
   6-2\omega^{2} & -2 \\
   -2 & 4-\omega^{2}
 \end{bmatrix}\vect{s}=0
\end{displaymath}   
has a solution when $\det \sqbrac{\matr{K}-\omega^{2}\matr{M}} = 0$ or $\omega^{4} -
7\omega^{2} + 10= 0$. This \emph{characteristic polynomial} has solutions
$\omega^{2} = 2,5$ with corresponding eigenvectors $\transpose{\vect{s}}_{1} = a
\begin{bmatrix} 1 & 1 \end{bmatrix},\transpose{\vect{s}}_{1} =
b\begin{bmatrix} -1 & 2 \end{bmatrix}$. To find the magnitude of the
eigenvectors we use \eqnref{eqn:norm}, \ie
\begin{displaymath}
  a^{2}\begin{bmatrix} 1 & 1 \end{bmatrix}
  \begin{bmatrix}
    2 & 0 \\
    0 & 1
  \end{bmatrix} 
  \begin{bmatrix}
     1 \\
     1
  \end{bmatrix}
  = 1 \Rightarrow a=\dfrac{1}{\sqrt{3}} \qquad b^{2}
  \begin{bmatrix} -1 & 2 \end{bmatrix} 
  \begin{bmatrix}
    2 & 0 \\
    0 & 1
  \end{bmatrix} \begin{bmatrix}
    -1 \\
    2
  \end{bmatrix}=1 \Rightarrow b=\dfrac{1}{\sqrt{3}}
\end{displaymath}

(Notice that the orthogonality condition is satisfied: 
$ab\begin{bmatrix} 1 & 1 \end{bmatrix}
\begin{bmatrix}
  2 & 0\\
  0 & 1
\end{bmatrix} 
\begin{bmatrix}
  -1 \\
  2
\end{bmatrix}=0$).

The $M$-orthognormalised eigenvectors are now
$\transpose{\vect{s}}_{1}=\begin{bmatrix} \dfrac{1}{\sqrt{3}} &
  \dfrac{1}{\sqrt{3}}\end{bmatrix}$ and $\transpose{\vect{s}}_{2}=
\begin{bmatrix} -\dfrac{1}{\sqrt{6}} &
  \dfrac{2}{\sqrt{6}}\end{bmatrix}$, giving the modal matrix
$\matr{S}=\begin{bmatrix} 
  \dfrac{1}{\sqrt{3}} & -\dfrac{1}{\sqrt{6}} \\ 
  \dfrac{1}{\sqrt{3}} & \dfrac{2}{\sqrt{6}}
 \end{bmatrix}$
 which, when used as the transformation matrix, reduces the stiffness matrix
 to
\begin{displaymath}
  \transpose{\matr{S}}\matr{K}\matr{S}=\begin{bmatrix}
    \dfrac{1}{\sqrt{3}} & \dfrac{1}{\sqrt{3}}\\
   -\dfrac{1}{\sqrt{6}} & \dfrac{2}{\sqrt{6}}
 \end{bmatrix}\begin{bmatrix}
   6  & -2 \\
   -2 & 4
 \end{bmatrix}\begin{bmatrix}
   \dfrac{1}{\sqrt{3}} & -\dfrac{1}{\sqrt{6}} \\
   \dfrac{1}{\sqrt{3}} & \dfrac{2}{\sqrt{6}}
 \end{bmatrix} = \begin{bmatrix}
   2 & 0 \\
   0 & 5
 \end{bmatrix} = \matr{\Lambda}
\end{displaymath}
and the mass matrix to
\begin{displaymath}
  \transpose{\matr{S}}\matr{M}\matr{S}=\begin{bmatrix}
    \dfrac{1}{\sqrt{3}} & \dfrac{1}{\sqrt{3}} \\
    -\dfrac{1}{\sqrt{6}} & \dfrac{2}{\sqrt{6}}
 \end{bmatrix}\begin{bmatrix}
   2 & 0 \\
   0 & 1
 \end{bmatrix}\begin{bmatrix}
   \dfrac{1}{\sqrt{3}} & -\dfrac{1}{\sqrt{6}}\\
   \dfrac{1}{\sqrt{3}} & \dfrac{2}{\sqrt{6}}
 \end{bmatrix} = \begin{bmatrix}
   1 & 0\\
   0 & 1
 \end{bmatrix} = \matr{I}
\end{displaymath}
Thus the natural modes of the system are given by
\begin{displaymath}
 \fnof{\vect{u}_{1}}{t} = \begin{bmatrix}
   \dfrac{1}{\sqrt{3}}\\
   \dfrac{1}{\sqrt{3}}
 \end{bmatrix}
 \sin \sqrt{2}\pbrac{t-t_{0}}\quad \text{and} \quad
 \fnof{\vect{u}_{2}}{t} = \begin{bmatrix}
   -\dfrac{1}{\sqrt{6}} \\
   \dfrac{2}{\sqrt{6}}
 \end{bmatrix} \sin \sqrt{5}\pbrac{t-t'_{0}}.
\end{displaymath}      
The solution of the non-homogeneous system, subject to given initial
conditions, is found by solving the uncoupled equations
\begin{displaymath}
 \fnof{\ddot{\vect{x}}}{t} + \begin{bmatrix}
   2 & 0 \\
   0 & 5
 \end{bmatrix} \fnof{\vect{x}}{t} = \begin{bmatrix}
   \dfrac{1}{\sqrt{3}} & \dfrac{1}{\sqrt{3}} \\
   -\dfrac{1}{\sqrt{6}} & \dfrac{2}{\sqrt{6}}
 \end{bmatrix}\begin{bmatrix}
   0 \\
   10
 \end{bmatrix} = \begin{bmatrix}
   \dfrac{10}{\sqrt{3}}\\
   \dfrac{20}{\sqrt{6}}
 \end{bmatrix} 
\end{displaymath}
by means of the Duhamel integral \bref{eqn:Duh} (in this case with $\vect{r}$
constant) and then, from \eqnref{eqn:suittrans} with
$\matr{P} \equiv \matr{S} = [(\vect{s}_{1}, \vect{s}_{2}, \ldots ,\vect{s}_{n}]$
\begin{equation}
  \fnof{\vect{u}}{t} = \matr{S}\fnof{\vect{x}}{t} = \dsuml{i=1}{n}
  \vect{s}_{i}\fnof{x_{i}}{t} 
  \label{eqn:Duhint}
\end{equation}

Notice that the solution is expressed in \eqnref{eqn:Duhint} as the superposition
of the natural modes (eigenvectors) of the homogeneous equations. If the
forcing function (load vector) is close to one of these modes the
corresponding coefficient $x_{i}$ will be large and will dominate the response
- if it coincides then resonance will occur. Very often it is unnecessary to
evaluate all $n$ eigenvectors of the system; the higher frequency modes can be
ignored and the solution adequately represented by superposition of the $p$
eigenvectors associated with the $p$ lowest eigenvalues, where $p<n$.

\section{Proportional Damping}

When element damping terms are included in the original dynamic equations
\bref{eqn:npa} the transformation to a lower bandwidth system is still based
on the model matrix $\matr{S}$ but \eqnref{eqn:canon} is then not a system of
uncoupled equations. One simplification often made in order to retain the
diagonal nature of \eqnref{eqn:canon} is to approximate the overall energy
dissipation of the finite element system with \emph{proportional damping}

\begin{equation}
  \transpose{\vect{{s}_{i}}}\matr{C}\vect{s}_{j} = 2\omega_{i}\xi_{i}\delta_{ij},
 \label{eqn:propd}
\end{equation}
where $\xi_{i}$ is a modal damping parameter and $\delta_{ij}$ is the
Kronecker delta. \Eqnref{eqn:canon} now reduces to $n$ equations of the
form
\begin{equation}
  \fnof{\ddot{x}_{i}}{t} + 2 \omega_{i}\xi_{i}\fnof{\dot{x}_{i}}{t} +
  \omega_{1}^{2}\fnof{x_{1}}{t} = \fnof{r_{i}}{t}
 \label{eqn:form}
\end{equation}
with solution (the Duhamel integral)
\begin{equation}
  \fnof{x_{i}}{t}=\dfrac{1}{\overline{\omega}_{i}}\gint{0}{t}
  {\fnof{r_{i}}{\tau}.e^{\xi_{i}\omega_{i}\pbrac{t-\tau}}.
  \sin\fnof{\overline{\omega}_{i}}{t-\tau}}{t}
  +e^{-\xi_{i}\omega_{i}t}\bbrac{\alpha_{i}\sin\overline{\omega}_{i}t
    +\beta_{i}\cos\overline{\omega}_{i}t}
 \label{eqn:calc}
\end{equation}
where $\overline{\omega}_{i}=\omega_{i}\sqrt{1-\xi_{i}^{2}}$. $\alpha_{i}$ and
$\beta_{i}$ are calculated from the initial conditions \eqnref{eqn:incond}.
Once the components $\fnof{x_{i}}{t}$ have been found from \eqnref{eqn:calc} (or
alternative time integration methods applied to \eqref{eqn:form}), the
solution $\fnof{\vect{u}}{t}$ is expressed as a superposition of the mode shapes 
$\vect{s}_{i}$ by \eqnref{eqn:Duhint}.

\section{CMISS Examples}

%\begin{tabular}{ll}
%(a) & Vibration modes of a rectangular plate\\
%(b) & Vibration modes of a circular plate\\
%(c) & Vibration modes of a guitar top plate
%\end{tabular}
% LC no longer exist ? 3/3/97

\begin{enumerate}
  \item To analyse a plane stress modal analysis run CMISS example 451
  \item To analyse a clamped beam modal analysis run CMISS example 452
  \item To analyse a steel-framed building modal analysis run CMISS example 453
\end{enumerate}


%%% Local Variables: 
%%% mode: latex
%%% TeX-master: "~/documents/notes/fembemnotes/fembemnotes"
%%% End: 

\clearemptydoublepage
\chapter{Constitutive Laws}

\section{Introduction}
Continuum Mechanics deals with the movement of materials under the action of
applied forces where the materials are continuous and deformable (we will only
consider solid mechanics here).  The result of a set of forces acting on a
deformable material is a (possibly time-varying) displacement field: each
material point moves a certain amount dependent both on its position relative
to the applied forces and on the mechanical properties of the material. A
displacement ``field'' implies a continuous variation of displacement with
position. The concepts of strain, a measure of length change or displacement
gradient, and stress, the force per unit area acting on an infinitesimally
small plane surface within the material, are of fundamental importance.

We derive the equations governing the motion of deformable
materials in the following four steps:
\begin{enumerate}
\item \textbf {Kinematic Relations}, which define the components of strain in
  terms of displacement gradients and, in the case of incompressible
  materials, define the incompressibility constraint.
\item \textbf {Stress Equilibrium}, or equations of motion, derived from the laws
  of conservation of linear momentum and conservation of angular momentum.
\item \textbf {Constitutive Relations}, which express the relationship between
  stress and strain and must be established from experimental measurement,
  subject to certain theoretical restrictions.
\item \textbf {Boundary Conditions}, which specify the external loads or
  displacement constraints acting on the deforming body.
\end{enumerate}
The first two steps are concerned with relationships which hold for all
materials and will be considered in detail in \secref{sec:Kinrel} and 7.4; the
third step is concerned with relations determined experimentally for a
particular material and is treated more fully in 7.5.
\remark {fix up section references no 7.5}
\section{Kinematic Relations}
\label{sec:Kinrel} 

\index{Kinematic relations}The key to analyzing strain in a material
undergoing large displacements and deformation is to establish two types of
coordinate system and the relationship between them. The first are fixed
reference coordinates and the second (material coordinates) are embedded in
the deforming body. From the base vectors of the material coordinates we
define metric tensors - measures of the physical lengths of coordinate
increments - and thence strain tensors.
 
\subsection{Coordinate systems and metric tensors}
The coordinates $\pbrac{X_{1},X_{2},X_{3}}$ define the position of a material point
in the undeformed tissue (see Fig.2.1) \remark{fix up figure reference} with 
respect to a fixed reference
rectangular cartesian coordinate system. The same material point in the
deformed tissue has rectangular cartesian coordinates $\pbrac{x_{1},x_{2},x_{3}}$.
Another system of material coordinates $\pbrac{x_{1},x_{2},x_{3}}$ is defined in
which the coordinate lines can be thought of as being attached to material
particles and so move with the deforming body. During the deformation the
initially straight and orthogonal $X_{M}$-coordinate axes remain ``attached''
to material particles and therefore become distorted and non-orthogonal in the
deformed configuration. $\bbrac{X_{M}}$ are called material coordinates or
convected coordinates. On the other hand the $x_{i}$-coordinate axes are fixed
throughout the deformation and $\bbrac{x_{i}}$ are called spatial coordinates (see
Figure 3.1).
\remark{fix reference}
\remark{missing figure}
\begin{figure} \centering
  \caption{Coordinate systems used in the kinematic
    analysis of large deformation. $\pbrac{X_{1},X_{2},X_{3}}$ are the
    rectangular cartesian coordinates of the point $P$
    (position vector $R$) in the undeformed body. $\pbrac{x_{1},x_{2},x_{3}}$
    are the rectangular cartesian coordinates of the same
    material point $p$ (position vector $r$) in the deformed
    body after a displacement $u$. $\pbrac{x_{1},x_{2},x_{3}}$ are material
    coordinates.}
  \label{fig:coord}
\end{figure} 

The base vectors for the $x_{i}$-coordinates in the undeformed reference state
are found by differentiating the position vector $\vect{R}=X_{k}\vect{i}_{k}$
\begin{equation}
  \vect{G}_{i} = \dfrac{\delta X_{k}}{\delta x_{i}} \vect{i}_{k}
  \label{eq:pv}
\end{equation}

The covariant components $G_{ij}$ of the undeformed metric tensor
are obtained from the inner products of the base vectors
\begin{equation}
  G_{ij}= \vect{G}_{i}\vect{G}_{j}=\dfrac{\delta X_{k}}{\delta x_{i}}
  \dfrac{\delta X_{k}}{\delta x_{j}}
  \label{eq:bv}
\end{equation}
 
Similarly, derivatives of the position vector $\vect{r}$ of a material
point in the deformed body (see \figref{fig:coord}) with respect to the
same material coordinates yield base vectors
\begin{equation}
 \vect{g}_{i} = \dfrac{\delta x_{k}}{\delta x_{i}} \vect{i}_{k}
 \label{eq:bv2}
\end{equation}
and metric tensors
\begin{equation}
 g_{ij}= \vect{g}_{i} \vect{g}_{j} = \dfrac{\delta x_{k}}{\delta x_{i}}
 \dfrac{\delta x_{k}}{\delta x_{j}}
 \label{eq:mt}
\end{equation}

\subsection{Strain measures}
\index{Strain measures}An initially undeformed line segment vector $d\vect{X}$
with components $dX^{1}, dX^{2}, dX^{3}$ (expressed more succinctly as
$d\vect{X} =\bbrac{dX^{M}}$) deforms into the line segment $d\vect{x}=
\bbrac{dx^{i}}$.  The deformation field is defined by expressing $\vect{x}$ as
a function of $\vect{X}$ and the components of deformation gradient,
$\dfrac{\delta x_{i}}{\delta X_{M}}$, collectively give the deformation
gradient tensor $\matr{F} = \bbrac{\dfrac{\delta x_{i}}{\delta X_{M}}}$.
Thus, the line segment $d\vect{X}$ is carried by the deformation into
$d\vect{x} = \matr{F}d\vect{X}$, or, in component form, $dx^{i} =
F^{i}_{M}dX^{M}$, where summation is implied ($M$ = 1,2,3) over the repeated
index $M$ and
\begin{equation}
  F^{i}_{M} = \dfrac{\delta x_{i}}{\delta X_{M}}
  \label{eq:ri}
\end{equation}
 
\remark {missing figure}
\begin{figure}\centering
  \caption{Coordinate systems used in the kinematic
    analysis of large deformation. $\pbrac{X_{1},X_{2},X_{3}}$ are the
    rectangular cartesian coordinates of the point $P$
    (position vector $\vect{R}$) in the undeformed body. $\pbrac{x_{1},x_{2},x_{3}}$
    are the rectangular cartesian coordinates of the same
    material point $p$ (position vector $\vect{r}$) in the deformed
    body after a displacement $u$. $\pbrac{x_{1},x_{2},x_{3}}$ are material
    coordinates.}
  \label{fig:cs}
\end{figure} 

A very important concept is that of ``polar decomposition'' 
\index{Polar decomposition}. The deformation gradient tensor $\matr{F}$ 
can be considered
as the product $\matr{F}=\matr{R}\matr{U}$ of an orthogonal rotation tensor
$\matr{R}$ and a symmetric positive stretch tensor $\matr{U}$. Equivalently,
the line segment components $dX^{M}$ are stretched into components $dy^{L} =
U^{L}_{M }dX^{M}_{ }$ before being rotated into the components $dx^{i} =
R^{i}_{L} dy^{L}$. An equally valid interpretation of the deformation would be
to consider the rigid body rotation before the stretching but for the present
purpose it is more convenient to interprete ``stretch'' in terms of material
coordinates and then use the rotation tensor $\matr{R}$ to relate the
stretched material lines to the spatial coordinates (for further details and a
full justification of the polar decomposition theorem see Atkin and Fox (1980)
or Spencer (1980)). The main point here is that the stretch tensor $\matr{U}$
contains a complete description of the material strain, independent of rigid
body motion (see also \Figref{fig:advection}).
 
The length of the line segments $d\vect{X}$ and $d\vect{x}$ are denoted by $dS$
and $ds$, respectively, where, from Pythagoras, $dS^{2} =dX^{M} dX^{M}$ and
$ds^{2}= dx^{i}dx^{i}$

Strain is a measure of relative length change and the so-called 
``Right Cauchy-Green'' strain tensor
\begin{displaymath}
  \matr{C} = \transpose{\matr{F}}\matr{F} = \frac{\delta x_{k}}{\delta
    X_{M}}\frac{\delta x_{k}}{\delta X_{N}}
\end{displaymath}
with components $C_{MN}$ , indicates how each component of undeformed line
segment $d\vect{X}$ contributes to the squared length of the deformed line
segment $d\vect{x}$ 
\begin{displaymath} 
  ds^{2} = dx^{k} dx^{k} = \dfrac{\delta x_{k}}{\delta X_{M}} dX^{M}
  \dfrac{\delta x_{k}}{\delta X_{N}} dX^{N} = \matr{C}_{MN} dX^{M} 
  dX^{N} = \transpose{dX} \matr{C} dX
\end{displaymath}

Using polar decomposition $\matr{F} = \matr{R}\matr{U}$, gives
\begin{displaymath}
  \matr{C} = \transpose{\matr{F}}\matr{F} =
  \transpose{\pbrac{\matr{R}\matr{U}}} 
  \matr{R}\matr{U} = \transpose{\matr{U}}\transpose{\matr{R}}
  \matr{R}\matr{U} = \transpose{\matr{U}} \matr{U} = \matr{U}^{2}
\end{displaymath}
since $\matr{R}$ is orthogonal ($\transpose{\matr{R}} = \matr{R}^{-1}$ ) and
$\matr{U}$ is symmetric ($\transpose{\matr{U}} = \matr{U}$).

Thus the stretch tensor $\matr{U}$ can be obtained by taking the positive
square-root of the strain tensor (see below).  Both $\matr{U}$ and $\matr{C}$ are
expressed in terms of material coordinates.

%%% Local Variables: 
%%% mode: latex
%%% TeX-master: t
%%% End: 

\clearemptydoublepage

\chapter{Data fitting with finite elements}
\label{cha:datafitting}


\section{Introduction}

We show here how finite elements can be used for fitting one, two or
three-dimensional fields. The data could, for examples, be a series of
temperature measurememts (as in figure ***) and there is a need to find the
finite element nodal parameteres which, together with the chosen basis
functions, 'fit' the data in a least-sqquares sense. This linear least-squares
probelm is considerd in \secref{cha:linearfieldfitting} Another common
requirement is to fit nodal geometric parameters describing a finite element
surface in 3D space to a set of non-uniformly spced coordinates defining the
original surface. Geometric fitting problems are nonlinear when the $\xi_i$
-coordinates found at the orthogonal projection of the data onto the finite
element mesh are changed by the fitting process. A jucicious choice of
coordinate system can sometimes avoid the problem and keep the geometric
fitting linear, (this situation considers in **\secref{}. Full nonlinear
geometric fitting is considered in \secref)




\section{Linear field fitting}
\label{sec:linearfieldfitting}

\subsection{The linear field fitting problem}

Consider the two-dimensional bilinear element shown in
\figref{fig:linfieldfit}. The element surrounds a set of data points (shown by
the x's in \figref{fig:linfieldfit}) which consist of measured values $u_{d}$,
$d=1..D$, of some field (such as temperature) at specified locations $(x_{d},
y_{d})$. The field fitting problem is to find the values of the finite element
nodal parameters $u_{n}$, $n=1..4$, which minimise the sum of squared
differences $(u(\xi_{d})-u_{d})^{2}$ between $u_{d}$ and the finite element
field $u$ evaluated at the $(\xi_{1},\xi_{2})$ coordinates of data point $d$ 
$(u(\xi_{d}))$.

\begin{figure}[htpb] \centering
  \input{figs/datafitting/linfieldfit.pstex}
  \caption{Least-squares fitting of finite element nodal parameters $u_{n}$,
    $n=1..4$, to measured data values $u_{d}$, $d=1..D$, at point locations
    $(x_{d}, y_{d})$.}
  \label{fig:linfieldfit}
\end{figure}

\subsection{Calculation of data point projections}

The first requirement, therefore, is to find the $\xi_{i}$ coordinates of each
data point $(x_{d}, y_{d}$). For a bilinear element (with the basis functions
given by \eqref{eqn:2,3DE}) this is a straightforward inversion of the
following relations for each data point:

\begin{equation}
  \begin{array}{rcl}
    x_{d}&=&(1-\xi_{1}^{d})(1-\xi_{2}^{d})x_{1}+\xi_{1}^{d}(1-\xi_{2}^{d})x_{2}
    +(1-\xi_{1}^{d})\xi_{2}^{d}x_{3}+\xi_{1}^{d}\xi_{2}^{d}x_{4} \\
    y_{d}&=&(1-\xi_{1}^{d})(1-\xi_{2}^{d})y_{1}+\xi_{1}^{d}(1-\xi_{2}^{d})y_{2}
    +(1-\xi_{1}^{d})\xi_{2}^{d}y_{3}+\xi_{1}^{d}\xi_{2}^{d}y_{4}
  \end{array}
  \label{eqn:datapointposfield}
\end{equation}

where $(x_{n}, y_{n}), n=1..4$, are the specified node positions and $(x_{d},
y_{d}), d=1..D$, are the specified data point positions. To solve
\bref{eqn:datapointposfield} for $\xi_{1}^{d}$ and $\xi_{2}^{d}$ we rearrange
\bref{eqn:datapointposfield} as

\begin{equation}
  \begin{array}{rcl}
    a\xi_{1}^{d}+b\xi_{2}^{d}+c\xi_{1}^{d}\xi_{2}^{d}&=&d \\
    A\xi_{1}^{d}+B\xi_{2}^{d}+C\xi_{1}^{d}\xi_{2}^{d}&=&D 
  \end{array}
  \label{eqn:dataxipos1}
\end{equation}
where
\begin{displaymath}
  a=x_{2}-x_{1}, \quad b=x_{3}-x_{1}, \quad c=x_{1}-x_{2}-x_{3}+x_{4}, \quad
  \mbox{and} \quad d=x_{d}-x_{1}
\end{displaymath}
and
\begin{displaymath}
  A=y_{2}-y_{1}, \quad B=y_{3}-y_{1}, \quad C=y_{1}-y_{2}-y_{3}+y_{4}, \quad
  \mbox{and} \quad D=y_{d}-y_{1}
\end{displaymath}
then
\begin{equation}
  \xi_{2}^{d}=\dfrac{d-a\xi_{1}^{d}}{b+c\xi_{1}^{d}}=\dfrac{D-A\xi_{1}^{d}}
  {B+C\xi_{1}^{d}}
  \label{eqn:dataxipos2}
\end{equation}
or
\begin{equation}
  \alpha(\xi_{1}^{d})^{2}+\beta(\xi_{1}^{d})+\gamma=0
  \label{eqn:dataxipos3}
\end{equation}
where
\begin{displaymath}
  \alpha=Ac-aC, \quad \beta=dC-Dc+Ab-aB, \quad \mbox{and} \quad \gamma=Bd-bD
\end{displaymath}
Solving \bref{eqn:dataxipos3} gives
\begin{equation}
  \xi_{1}^{d}= \left\{ \begin{array}{ll} 
      \dfrac{-\beta\pm\sqrt{\beta^{2}-4\alpha\gamma}}{2\alpha} & 
      \mbox{when } \alpha\neq 0 \\
      -\dfrac{\gamma}{\beta} & \mbox{when } \alpha = 0
    \end{array} \right.
  \label{eqn:dataxipos4}
\end{equation}
and then $\xi_{2}^{d}$ is recovered from \bref{eqn:dataxipos2}.

\subsection{Least squares field fitting}

Once the $\xi_{i}$ coordinate positions of each data point
$\xi_{d}=(\xi_{1}^{d},\xi_{2}^{d})$ is known, and interpolation of the
(unknown) nodal values of $u_{n}$ ($n=1..4$) gives
\begin{displaymath}
  u(\xi_{d})=\Psi_{n}u_{n}
\end{displaymath}
where there is an implied sum over $n=1..4$. Now the weighted sum of squared
differences between this value and the measured value $u_{d}$ for $d=1..D$, is

\begin{equation}
%  {\cal F(\vect{u})}=\dsum_{d=1}^{D}w_{d}\left(\Psi_n(\xi_{d})u_{n}-
%    u_{d}\right)^{2}
  eqn here
  \label{eqn:datasumsqs}
\end{equation}
where $w_{d}$ is the weight for data point $d$. For measured data a good
choice for $w_{d}$ is one over the variance of the error for data point $d$.

Minimising \bref{eqn:datasumsqs} with respect to the nodal parameters $u_{m}$
gives
\begin{equation}
  \delby{{\cal F}}{u_{m}}=2\dsum_{d=1}^{D}w_{d}\left(\Psi_n(\xi_{d})u_{n}-
    u_{d}\right)\Psi_m(\xi_{d}) = 0
\end{equation}
or
\begin{equation}
  \left[\dsum_{d=1}^{D}w_{d}\Psi_m(\xi_{d})\Psi_n(\xi_{d})\right]u_{n}=
  \dsum_{d=1}^{D}w_{d}\Psi_m(\xi_{d})u_{d} \quad \quad m=1..4
  \label{eqn:linfieldfiteqn}
\end{equation}
Note that there is an implied sum over $n=1..4$ on the left hand side of
\eqref{eqn:linfieldfiteqn}. For the single 4-node element shown in
\figref{fig:linfieldfit}. 
\todo{check capital?}
\eqref{eqn:linfieldfiteqn} gives four unknowns
$u_{1}, u_{2}, u_{3}$ and $u_{4}$. If more than one element is used in the
fitting, the matrix and vector on the left and right hand sides, respectively,
of \bref{eqn:linfieldfiteqn} are
\begin{equation}
  E_{mn}=\dsum_{d=1}^{D}w_{d}\Psi_m(\xi_{d})\Psi_n(\xi_{d}) \quad \mbox{and} 
  \quad f_{m}=\dsum_{d=1}^{D}w_{d}\Psi_m(\xi_{d})u_{d}
\end{equation}
which can then be assembled into a global system of equations in exactly the
same fashion as occurs in the finite element solution of a boundary value
problem (see \secref{sec:OdSSHC-2.1}).

\subsection{Gauss point fitting}

An extension to linear field fitting is Gauss point fitting. Consider the
problem of finding the nodal stress field from a finite element analysis. The
finite element formulation for a stress analysis typically solves for
displacements, not stresses (see \chapref{cha:linearelasticity}). Stresses are
a derived quantity and are calculated from the constitutive law from the
strains which are calculated (in terms of nodal displacements) at the Gauss
points. The result of the analysis is that it is only possible to obtain a
Gauss point based description of the stresses. To obtain a nodal field based
description of the stress field we can use a fitting approach very similar to
the one described above for field fitting. The only difference in this case is
that our 'data points' are located at the Gauss points and hence the $\xi_d$
location is just the Gauss point location $\xi_{g}$. The values of the field
$u_{d}$ are just the values of stress calculated at the Gauss points and the
weights for each data point $w_{d}$ are 1.

\section{Linear geometric fitting}
\label{sec:lineargeometricfitting}

\subsection{Introduction}

Another common requirement is to fit nodal geometric parameters describing a
finite element geometry to a set of non-uniformly spaced coordinates defining
the original geometry. Geometric fitting differs from field fitting in that
when fitting with more than one geometric variable we have a non-linear
problem since the mesh data projection is no longer orthogonal.

Geometric fitting problems can be turned into a linear problem in two
ways. The first is to only fit geometries were there is one geometric
variable. The second is to keep the data projections constant throughout the
fit. Both options will now be considered.

\subsection{Fitting with one geometric variable}

A Geometric fitting problem with one variable is linear hence it is desirable
to try and arrange a geometric fitting problem so that only one variable is
being fitted. This can often be achieved by adopting a non-rectangular
cartesian coordinate system. For example consider a fitting problem in polar
coordinates as shown in \figref{fig:polargeomfit}

\todo{missing figure}

\begin{figure}[htpb] \centering
%  \input{figs/datafitting/polargeomfit.pstex}
%  \caption{Fitting the radial coordinate in a polar coordinate mesh. There are
%    four elements and four nodes. The dashed lines show the projections of the
%    data points (x) onto the starting mesh.}
  \label{fig:polargeomfit}
\end{figure}

In this case the orthogonal data point projections $\xi_{d}$ are independent
of the radius and can be easily found from the theta coordinate of the data
point $\theta_{d}$. This independence from radius means we can formulate the
geometric fitting problem with only one geometric variable, $r$ being
fitted. As with the linear field fitting case we can formulate an error
function as the weighted sum of squares of the individual errors, that is

\begin{equation}
  {\cal F(\vect{r})}=\dsum_{d=1}^{D}w_{d}\left(\Psi_{n}(\xi_{d})r_{n}-r_{d}
  \right)^{2}
\end{equation}

Minimising this error function with respect to the nodal radii $r_{n}$ we
obtain
\begin{equation}
  \delby{{\cal F}}{r_{m}}=2\dsum_{d=1}^{D}w_{d}\left(\Psi_n(\xi_{d})r_{n}-
    r_{d}\right)\Psi_m(\xi_{d}) = 0
\end{equation}
or in terms of element stiffness matrices
\begin{equation}
  E_{mn}r_{n}=f_{m}
\end{equation}
where
\begin{equation}
  E_{mn}=\dsum_{d=1}^{D}w_{d}\Psi_m(\xi_{d})\Psi_n(\xi_{d}) \quad \mbox{and} 
  \quad f_{m}=\dsum_{d=1}^{D}w_{d}\Psi_m(\xi_{d})u_{d}
\end{equation}

Another example of a change of coordinate systems is a prolate spheroidal
coordinates for fitting a heart geometry.

\subsection{Linear geometric fitting with fixed $\xi_{d}$ locations}

In some geometric fitting problems there is no appropriate change of 
coordinates and a rectangular cartesian system must be used. This means that,
in general, we require both x and y (and z) in the fit at the same time.
This results in a non-linear problem as the data point projection $\xi_{d}$
will now, in general, change as we change any geometric variable during the
fit. Hence in order to obtain a linear problem the data point projections
must remain fixed during the fit. To see how this works consider the geometric
fitting problem as shown in \figref{fig:fixedxifitbefore}.

\begin{figure}[htpb] \centering
  %\epsfig{file=epsfiles/beforefit.eps,width=15cm}
  \caption{Geometric fitting problem. The three nodal y-coordinates 
    $y_{1}, y_{2}$ and $y_{3}$ of a two linear element basis function mesh are
    fitted to five data points. The initial locations of the nodes are
    $(x_{1},y_{1})=(0,0), (x_{2},y_{2})=(1,0)$ and $(x_{3},y_{3})=(2,0)$. The
    location of the data points are
    $(x^{1},y^{1})=(0.2,1.2),(x^{2},y^{2})=(0.8,1.5),(x^{3},y^{3})=(1.3,1.5),
    (x^{4},y^{4})=(1.5,1.1)$ and $(x^{5},y^{5})=(1.7,0.8)$. The initial $\xi$
    projections are $\xi^{1}=0.2$ and $\xi^{2}=0.8$ in element 1 and
    $\xi^{3}=0.3,\xi^{4}=0.5$ and $\xi^{5}=0.7$ in element 2.}
  \label{fig:fixedxifitbefore}
\end{figure}

Now computing the error functions as a weighted sum of squares we obtain
\begin{equation}
  {\cal F(\vect{r})}=\dsum_{d=1}^{D}w_{d}\left(\Psi_{n}(\xi_{d})y_{n}-y_{d}
  \right)^{2}
\end{equation}

Minimising this error function with respect to the nodal parameters $y_{n}$ we
obtain the element matrix equation
\begin{equation}
  E_{mn}y_{n}=f_{m}
\end{equation}
where, as before,
\begin{equation}
  E_{mn}=\dsum_{d=1}^{D}w_{d}\Psi_m(\xi_{d})\Psi_n(\xi_{d}) \quad \mbox{and} 
  \quad f_{m}=\dsum_{d=1}^{D}w_{d}\Psi_m(\xi_{d})u_{d}
\end{equation}

Now for element 1
\begin{equation}
  \begin{array}{rcl}
    E_{11}&=&(1-\xi^{1})(1-\xi^{1})+(1-\xi^{2})(1-\xi^{2}) \\
    &=&(1-0.2)(1-0.2)+(1-0.8)(1-0.8)=0.68 \\
    E_{12}&=&E_{21}=(1-\xi^{1})\xi^{1}+(1-\xi^{2})\xi^{2} \\
    &=&(1-0.2)(0.2)+(1-0.8)(0.8)=0.32 \\
    E_{22}&=&\xi^{1}\xi^{1}+\xi^{2}\xi^{2} \\
    &=& (0.2)(0.2)+(0.8)(0.8)=0.68 \\
    f_{1}&=&(1-\xi^{1})y^{1}+(1-\xi^{2})y^{2} \\
    &=&(1-0.2)(1.2)+(1-0.8)(1.5)=1.26 \\
    f_{2}&=&\xi^{1}y^{1}+\xi^{2}y^{2} \\
    &=&(0.2)(1.2)+(0.8)(1.5)=1.44
  \end{array}
\end{equation}
and for element 2
\begin{equation}
  \begin{array}{rcl}
    E_{11}&=&(1-\xi^{3})(1-\xi^{3})+(1-\xi^{4})(1-\xi^{4})+(1-\xi^{5})
    (1-\xi^{5})\\
    &=&(1-0.3)(1-0.3)+(1-0.5)(1-0.5)+(1-0.7)(1-0.7)=0.83 \\
    E_{12}&=&E_{21}=(1-\xi^{3})\xi^{3}+(1-\xi^{4})\xi^{4}+(1-\xi^{5})\xi^{5}\\
    &=&(1-0.3)(0.3)+(1-0.5)(0.5)+(1-0.7)(0.7)=0.67 \\
    E_{22}&=&\xi^{3}\xi^{3}+\xi^{4}\xi^{4}+\xi^{5}\xi^{5}\\
    &=&(0.3)(0.3)+(0.5)(0.5)+(0.7)(0.7)=0.83 \\
    f_{1}&=&(1-\xi^{3})y^{3}+(1-\xi^{4})y^{4}+(1-\xi^{5})y^{5}\\
    &=&(1-0.3)(1.5)+(1-0.5)(1.1)+(1-0.7)(0.8)=1.84\\
    f_{2}&=&\xi^{3}y^{3}+\xi^{4}y^{4}+\xi^{5}y^{5}\\
    &=&(0.3)(1.5)+(0.5)(1.1)+(0.7)(0.8)=1.56
  \end{array}
\end{equation}

Assembling these element matrices and vectors into a global system of
equations we get
\begin{equation}
  \begin{bmatrix}
    0.68 & 0.32 & 0.00 \\
    0.32 & 1.51 & 0.67 \\
    0.00 & 0.67 & 0.83 
  \end{bmatrix}
  \begin{bmatrix}
    y_{1} \\
    y_{2} \\
    y_{3}
  \end{bmatrix} =
  \begin{bmatrix}
    1.26 \\
    3.28 \\
    1.56
  \end{bmatrix}
\end{equation}

Solving these gives
\begin{equation}
  \begin{bmatrix}
    y_{1} \\
    y_{2} \\
    y_{3}
  \end{bmatrix} =
  \begin{bmatrix}
    1.03210 \\
    1.74420 \\
    0.47152
  \end{bmatrix}
\end{equation}

The fitted solution is shown in \figref{fig:fixedxifitafter}.
\begin{figure}[htpb] \centering
  %\epsfig{file=epsfiles/afterfit.eps,width=15cm}
  \caption{Fitted mesh. The $\xi_{d}$ locations on the fitted mesh are
    $\xi^{1}=0.212$ and $\xi^{2}=0.752$ in element 1 and $\xi^{3}=0.233,
    \xi^{4}=0.504$ and $\xi^{5}=0.726$ in element 2.}
  \label{fig:fixedxifitafter}
\end{figure}
It should be noted that after the fit the new data point projections have
changed after the fit. Hence it general there maybe some benefit from
reapplying the fitting procedure to the new data point projection i.e. 
iterating on the fit.

\section{Geometric fitting with Hermite elements}

\subsection{Review of cubic Hermite interpolation}

\subsubsection{Cubic Hermite basis functions}

One of the most commonly used basis functions in finite elements are Lagrange
basis functions which preserve continuity of the geometric coordinates across
element boundaries by interpolating nodal coordinates which are shared by
adjacent elements i.e.  $C^{0}$ continuity.  The interpolation formula for
linear Lagrange interpolation is given in \eqref{eqn:linLagrangeinterp}.
\begin{equation}
  \fnof{\vect{x}}{\xi}=\varphi_{1}(\xi)\vect{x}_{1}+\varphi_{2}(\xi)\vect{x}_{2}
  \label{eqn:linLagrangeinterp}
\end{equation}
where $\vect{x}_{n}$ is the geometric position of local node $n$ and the two
one-dimensional linear Lagrange basis functions are given in
\eqref{eqn:linLagrangeBfuns}.
\begin{eqnarray}
  \varphi_{1}(\xi)=1-\xi & \varphi_{2}(\xi)=\xi
  \label{eqn:linLagrangeBfuns}
\end{eqnarray}

Cubic Hermite functions, on the other hand, also preserve continuity of the
derivative of these coordinates with respect to $\xi$ across element
boundaries by defining additional nodal parameters \todo{here}
%\dbyat{\vect{x}}{\xi}{n}
i.e. $C^{1}$ continuity. The interpolation formula within an element is given
by
\begin{equation}
%  \fnof{\vect{x}}{\xi}=\Psi_{1}^{0}(\xi)\vect{x}_{1}+\Psi_{1}^{1}(\xi)\dbyat{\vect{x}}
%  {\xi}{1}+\Psi_{2}^{0}(\xi)\vect{x}_{2}+\Psi_{2}^{1}(\xi)\dbyat{\vect{x}}
%  {\xi}{2}
  \todo{eqn}
  \label{eqn:cubHermxiinterp}
\end{equation}
where the four one-dimensional cubic Hermite basis functions are given in 
\eqref{eqn:cubHermBfuns}.
\begin{eqnarray}
  \Psi_{1}^{0}(\xi)=1-3\xi^{2}+2\xi^{3} & \Psi_{1}^{1}(\xi)=\xi(\xi-1)^{2} 
  \nonumber \\
  \Psi_{2}^{0}(\xi)=\xi^{2}(3-2\xi) & \Psi_{2}^{1}(\xi)=\xi^{2}(\xi-1)
  \label{eqn:cubHermBfuns}
\end{eqnarray}

One further step is required to make cubic Hermite basis functions useful in
practice.  Consider now the two cubic Hermite elements as shown in 
\figref{fig:cubHermelem}.

\begin{figure}[htbp] \centering
  \input{figs/datafitting/cubichermiteelem.pstex}
  \caption{Two cubic Hermite elements (denoted by {\bf 1} and {\bf 2}) formed
    from three nodes (shown as a $\bullet$ and denoted by 1, 2 and 3) and 
    having arc-lengths $s_{1}$ and $s_{2}$.}
  \label{fig:cubHermelem}
\end{figure}

The derivative \todo{here}
%\dbyat{\vect{x}}{\xi}{n} defined at node $n$ is dependent upon
the local element $\xi$-coordinate and is therefore, in general, different in
the two adjacent elements. Therefore we carry a physical derivative
%$\dbyat{\vect{x}}{s}{n}$ at nodes and use
\begin{equation}
%  \dbyat{\vect{x}}{\xi}{n}=\brdby{\vect{x}}{s}_{\Delta(n,e)}.\brdby{s}{\xi}_{e}
  \label{eqn:xitoarclength}
\end{equation}
%to determine \dbyat{\vect{x}}{\xi}{n}. Here \dby{\vect{x}}{s} is a physical arc-
length derivative, $\Delta(n,e)$ is the global node number of local node $n$
%in element $e$, $\brdby{s}{\xi}_{e}$ is an element 'scale factor', denoted by
$S_{e}$, which scales the arc-length derivative to the $\xi$-coordinate
derivative.  Thus $\dby{\vect{x}}{s}$ is constrained to be continuous across
element boundaries rather than $\dby{\vect{x}}{\xi}$.

There is one condition that must be placed on the $\xi$ to arc-length
transformation to ensure that we have arc-length derivatives. This condition
is that the arc-length derivative vector at a node must have unit magnitude,
that is
\begin{equation}
%  \norm{\brdby{\vect{x}}{s}_{n}}=1
  \label{eqn:cubHermnormconst}
\end{equation}
This ensures that we have continuity with respect to a physical parameter
rather than with respect to a mathematical parameter $\xi$. The set of mesh
parameters, \vect{u}, for cubic Hermite interpolation hence contains the set
of nodal values (or positions), the set of nodal arc-length derivatives and
the set of scale factors.

\subsubsection{Bicubic Hermite basis functions}

Bicubic Hermite basis functions are the two-dimensional extension of the
one-dimensional cubic Hermite basis functions. They are formed from the tensor
(or outer) product of two one-dimensional basis functions as defined in
\eqref{eqn:cubHermBfuns}. The interpolation formula for a point $(\xi_{1},
\xi_{2})$ within an element is obtained from the bicubic Hermite interpolation
formula,
\begin{eqnarray}
%\vect{x}(\xi_{1},\xi_{2}) &=& \Psi_{1}^{0}(\xi_{1})\Psi_{1}^{0}(\xi_{2})
%    \vect{x}_{1}+\Psi_{2}^{0}(\xi_{1})\Psi_{1}^{0}(\xi_{2})\vect{x}_{2} + 
%    \nonumber \\
%    & & \Psi_{1}^{0}(\xi_{1})\Psi_{2}^{0}(\xi_{2})\vect{x}_{3} +
%    \Psi_{2}^{0}(\xi_{1})\Psi_{2}^{0}(\xi_{2})\vect{x}_{4} + \nonumber \\
%    & & \Psi_{1}^{1}(\xi_{1})\Psi_{1}^{0}(\xi_{2})\delbyat{\vect{x}}
%    {\xi_{1}}{1}+\Psi_{2}^{1}(\xi_{1})\Psi_{1}^{0}(\xi_{2})\delbyat{\vect{x}}
%    {\xi_{1}}{2} + \nonumber \\
%    & & \Psi_{1}^{1}(\xi_{1})\Psi_{2}^{0}(\xi_{2})\delbyat{\vect{x}}
%    {\xi_{1}}{3}+\Psi_{2}^{1}(\xi_{1})\Psi_{2}^{0}(\xi_{2})\delbyat{\vect{x}}
%    {\xi_{1}}{4} + \nonumber \\
%    & & \Psi_{1}^{0}(\xi_{1})\Psi_{1}^{1}(\xi_{2})\delbyat{\vect{x}}
%    {\xi_{2}}{1}+\Psi_{2}^{0}(\xi_{1})\Psi_{1}^{1}(\xi_{2})\delbyat{\vect{x}}
%    {\xi_{2}}{2} + \nonumber \\
%    & & \Psi_{1}^{0}(\xi_{1})\Psi_{2}^{1}(\xi_{2})\delbyat{\vect{x}}
%    {\xi_{2}}{3}+\Psi_{2}^{0}(\xi_{1})\Psi_{2}^{1}(\xi_{2})\delbyat{\vect{x}}
%    {\xi_{2}}{4} + \nonumber \\
%    & & \Psi_{1}^{1}(\xi_{1})\Psi_{1}^{1}(\xi_{2})\deltwobyat{\vect{x}}
%    {\xi_{1}}{\xi_{2}}{1} +
%    \Psi_{2}^{1}(\xi_{1})\Psi_{1}^{1}(\xi_{2})\deltwobyat{\vect{x}}
%    {\xi_{1}}{\xi_{2}}{2} + \nonumber \\
%    & & \Psi_{1}^{1}(\xi_{1})\Psi_{2}^{1}(\xi_{2})\deltwobyat{\vect{x}}
%    {\xi_{1}}{\xi_{2}}{3} +
%    \Psi_{2}^{1}(\xi_{1})\Psi_{2}^{1}(\xi_{2})\deltwobyat{\vect{x}}
%    {\xi_{1}}{\xi_{2}}{4}
    \label{eqn:bicubHerminterp}
\end{eqnarray}

As with the one-dimensional cubic Hermite elements, the derivatives with
respect to $\xi$ in the two-dimensional interpolation formula above are
expressed as the product of a nodal arc-length derivative and a nodal scale
factor. This is, however, complicated by the fact that there are now scale
factors for each $\xi$ direction at each node. For node $n$ we
have
\begin{equation}
%  \delbyat{\vect{x}}{\xi_{i}}{n}=\brdelby{\vect{x}}{s_{i}}_{\Delta(n,e)}.
%  \left(S_{i}\right)_{e}
  \label{eqn:xitoarclength2}
\end{equation}
and for the cross-derivative
\begin{equation}
%  \deltwobyat{\vect{x}}{\xi_{1}}{\xi_{2}}{n}=\brdeltwoby{\vect{x}}{s_{1}}{s_{2}}_
%  {\Delta(n,e)}.\left(S_{1}\right)_{e}.\left(S_{2}\right)_{e}
  \label{eqn:xitoarclength3}
\end{equation}

As with the one-dimensional cubic Hermite case conditions must be placed on
this transformation in order to maintain $C^{1}$ continuity. A sufficient
condition is that the scale factor at a node in one element must be the same
as the scale factor at the same node in an adjacent element. That is, the
elemental scale factor should be nodally based so that the same scale factor
is used at a given node regardless of the current element. With this condition
satisfied any choice of scale factor will give $C^{1}$ continuity across
element boundaries. The choice of the scale factor will, however, affect the
spacing of $\xi$ with arc-length. It is often computationally desirable to
have a uniform spacing of $\xi$ with respect to arc-length (for example, not
biasing the Gaussian quadrature scheme to one end of the element). To achieve
this uniform spacing a good choice of the nodal scale factor (denoted from now
on as ${\cal S}_{n}$ for node $n$) is the average of the two arc-lengths on
either side of the node. This condition must also be applied together with
arc-length derivatives for each $\xi$ direction as defined by
\eqref{eqn:cubHermnormconst}.

If $n_{\ominus}$ is the node or line immediately before node $n$ (in the sense
of increasing $\xi$) and $n_{\oplus}$ is the node or line immediately after
node $n$, the nodal scale factor is given by
\begin{equation}
  {\cal S}_{n}=\frac{s_{n_{\ominus}}+s_{n_{\oplus}}}{2}
  \label{eqn:avearclenscale}
\end{equation}
This is termed average arc-length scaling.  For example consider node 2 in
\figref{fig:cubHermelem} then
\[{\cal S}_{2}=\frac{s_{1}+s_{2}}{2}\]
Thus, for an element $e$, the one-dimensional cubic Hermite interpolation
formula in \bref{eqn:cubHermxiinterp} becomes
\begin{equation}
%  \fnof{\vect{x}}{\xi}=\sum_{n=1,2}\left[\Psi_{n}^{0}(\xi)\vect{x}_{n}+\Psi_{n}^{1}(\xi)
%  \brdby{\vect{x}}{s}_{n}.{\cal S}_{n}\right]
  \label{eqn:cubHerminterp}
\end{equation}
To calculate the arc-length for a particular element an iterative process is
needed. The arc-length for an one-dimensional element is defined as
\begin{equation}
%  \mbox{arc-length}=\int_{0}^{1}\norm{\dby{\vect{x}(\xi)}{\xi}}d\xi=
%  \int_{0}^{1}\sqrt{\brdby{x(\xi)}{\xi}^{2}+\brdby{y(\xi)}{\xi}^{2}}d\xi
  \label{eqn:arclendef}
\end{equation}
However, since the interpolation of \fnof{\vect{x}}{\xi}, as defined in
\bref{eqn:cubHerminterp}, uses the arc-length in the calculation of the
scaling factor, an iterative root finding technique is needed to obtain the
arc-length.

With nodal based scale factors and arc-length derivatives we achieve a good
arc-length to $\xi$ spacing, continuity is maintained with respect to a
physically meaningful parameter and the numerical results when using the mesh
are more accurate.

\subsection{Least squares fitting with cubic Hermite interpolation}

\subsubsection{Problem formulation}

Consider a set of rectangular cartesian data with geometric positions
$\vect{z}_{d},d=1..D$. For each data point we can find the position on the mesh
which has the smallest distance to that data point. This point is the
orthogonal projection of the data point onto the mesh and has geometric
position $\vect{z}$. The point $\vect{z}$ is also given by the local element
co-ordinate $\vect{\xi}_{d}$ as is shown in \figref{fig:dataproj}.

To calculate $\vect{\xi}_{d}$ a non-linear iterative procedure is required.
Given a $\vect{\xi}$ position for the data point projection within an element
the geometric position of this projection is given by the standard
interpolation formula (\eqref{eqn:cubHermxiinterp} or
\eqref{eqn:bicubHerminterp}). An error function can then be set up as the
Euclidean distance between this position and the actual position of the data
point. The $\vect{\xi}$ position which minimises this function can the be found
by using the Newton-Rhapson root finding method on the derivative of this
function. This $\vect{\xi}$ position is the orthogonal projection of the data
point.

\begin{figure} \centering
  \input{figs/datafitting/datapos.pstex}
  \caption{Definition of a data point projection into an element. The data
    point at geometric location $\vect{z}_{d}$ is projected into an element at a
    geometric position $\vect{z}$ and element co-ordinate $\xi_{d}$.}
  \label{fig:dataproj}
\end{figure}

For simplicity only two-dimensional fitting (i.e. cubic Hermite elements) will
be covered in this section. Hence, given $\xi_{d}$, $\vect{z}$ can be found from
interpolation i.e.
\begin{equation}
%  \vf{z}{\xi_{d}}=\sum_{n=1,2}\left[\onevf{\Psi_{n}^{0}}{\xi_{d}}\vect{x}_{n}+
%  \onevf{\Psi_{n}^{1}}{\xi_{d}}\brdby{\vect{x}}{s}_{n}.{\cal S}_{n}\right]
  \label{eqn:datapointpos}
\end{equation}

The measure of error for each data point is defined as the Euclidean distance
between the data point and its closest projection onto the current mesh:
\begin{equation}
%  \onevf{f_{d}}{\xi_{d}}=\norm{\vf{z}{\xi_{d}}-\vect{z}_{d}}
\end{equation} 

For a given projection of the data points onto the mesh (i.e. $\xi_{d}$ is
held constant) the objective function to be minimised in the fit is then formed
as the sum-of-squares of the individual errors.
\begin{equation}
%  \vvf{{\cal F}}{u}=\sum_{d=1,D}\gamma_{d}\onevf{f_{d}^{2}}{\xi_{d}}=
%  \sum_{d=1,D}\gamma_{d}\norm{\onevf{\vect{z}}{\xi_{d}}-\vect{z}_{d}}^{2}
  \label{eqn:linfitobjfun}
\end{equation}
where $\gamma_{d}$ is a weight for each data point and \vect{u} is a vector of
mesh parameters.

The fitting problem is to find the set of mesh parameters that minimises 
this objective function. Substituting \bref{eqn:datapointpos} into
\bref{eqn:linfitobjfun} and differentiating we obtain
\begin{equation}
%  \delby{\vvf{{\cal F}}{u}}{\left(x_{j}\right)_{m}}=2\sum_{d=1,D}\gamma_{d}
%  \left(\sum_{n=1,2}\left[\onevf{\Psi_{n}^{0}}{\xi_{d}}\left(x_{j}\right)_{n}+
%  \onevf{\Psi_{n}^{1}}{\xi_{d}}\brdelby{x_{j}}{s}_{n}.{\cal S}_{n}\right]-
%  {z_{d}}_{j}\right)\onevf{\Psi_{m}^{0}}{\xi_{d}} 
  \label{eqn:lindatajacnode}
\end{equation}
\begin{equation}
%  \delby{\vvf{{\cal F}}{u}}{\brdelby{x_{j}}{s}_{m}}=2\sum_{d=1,D}\gamma_{d}
%  \left(\sum_{n=1,2}\left[\onevf{\Psi_{n}^{0}}{\xi_{d}}\left(x_{j}\right)_{n}+
%  \onevf{\Psi_{n}^{1}}{\xi_{d}}\brdelby{x_{j}}{s}_{n}.{\cal S}_{n}\right]-
%  {z_{d}}_{j}\right)\onevf{\Psi_{m}^{1}}{\xi_{d}}{\cal S}_{m} 
  \label{eqn:lindatajacderiv}
\end{equation}

A minimum can thus be found to the objective function by setting both 
\eqnrefs{eqn:lindatajacnode}{eqn:lindatajacderiv} to zero. This will result in
a linear system only if the scale factors are kept constant during the fit. 
That is the vector $\vect{u}$ will contain the nodal positions and the nodal
arc-length derivatives. With this restriction we can obtain
\begin{eqnarray*}
%  \sum_{d=1,D}\gamma_{d}\left(\sum_{n=1,2}\left[\onevf{\Psi_{n}^{0}}{\xi_{d}}
%  \left(x_{j}\right)_{n}+\onevf{\Psi_{n}^{1}}{\xi_{d}}{\cal S}_{n}
%  \brdelby{x_{j}}{s}_{n}\right]\onevf{\Psi_{m}^{0}}{\xi_{d}}\right)
%  &=&\sum_{d=1,D}\gamma_{d}\onevf{\Psi_{m}^{0}}{\xi_{d}}{z_{d}}_{j} \\ 
%  \sum_{d=1,D}\gamma_{d}\left(\sum_{n=1,2}\left[\onevf{\Psi_{n}^{0}}{\xi_{d}}
%  \left(x_{j}\right)_{n}+\onevf{\Psi_{n}^{1}}{\xi_{d}}{\cal S}_{n}
%  \brdelby{x_{j}}{s}_{n}\right]\onevf{\Psi_{m}^{1}}{\xi_{d}}{\cal S}_{m}\right)
%  &=&\sum_{d=1,D}\gamma_{d}\onevf{\Psi_{m}^{1}}{\xi_{d}}{\cal S}_{m}{z_{d}}_{j}
\end{eqnarray*}
This is a linear system of equations for the element of the form
\begin{equation}
  K_{mn}u_{n}=f_{m}
  \label{eqn:linsystem}
\end{equation}

A linear system of equations governing the entire mesh can then be found by
assembling a global stiffness matrix from all the individual element matrices.
This can then be solved to yield the nodal positions and derivatives which
minimises the error in the mesh. An example of fitting is shown in
\figref{fig:cubHermfit}.

\begin{figure} \centering
  \input{figs/datafitting/fitting.pstex}
  \caption{Geometric data fitting with cubic Hermite elements. (a) The data
     points (+) are shown projected onto the two element mesh at some 
     intermediate stage in the fitting procedure. (b) The final fitted mesh.}
  \label{fig:cubHermfit}
\end{figure}

\subsubsection{Sobelov Smoothing}

With an insufficient number of data points, fitting 'noisy' data or fitting
data that has an uneven spread, a smoothness constraint \cite{young:1989}
can be introduced by adding a second term to the objective function:
%\[\vvf{F}{u}=\sum_{d=1,D}\gamma_{d}\norm{\onevf{\vect{z}}{\vect{\xi}_{d}}-
%\vect{z}_{d}}^{2}+\int_{\Omega} \onevf{g}{\vvvf{u}{\xi}}d\vect{\xi}\]

The first term measures the error in the surface from the data and the second
term measures deformation of the surface. To measure the deformation of the
surface a $p^{th}$ order weighted Sobelov norm 
\cite{terzopoulos:1986,tikhonov:1977} is used, defined by
\begin{equation}
%  \onevf{g_{p,w}}{\vvvf{u}{\xi}}=\sum_{q=0}^{p}\sum_{i+j=q}w_{ij}
%  \norm{\frac{\del^{q}\vect{u}}{\del^{i}\xi_{1}\del^{j}\xi_{2}}}^{2}
  \label{eqn:Sobnorm}
\end{equation}
where $w_{ij}$ are the weights for the norm. The addition to the objective
function, called the Sobelov value, is defined as
\begin{equation}
%  \onevf{G}{\vvvf{u}{\xi}}=\int_{\Omega}{\onevf{g}{\vvvf{u}{\xi}}d
%    \vect{\xi}}
  \label{eqn:Sobvalue}
\end{equation}
where $\Omega$ is the mesh domain.

For the case of two-dimensional fitting (i.e. cubic Hermite elements) $j=0$
(as there is no $\xi_{2}$ direction) and for the case of three-dimensional
fitting (i.e. bicubic Hermite elements) $j=0..2$. Consider the case for $p=2$,
and for 2D: $w_{0}=0, w_{1}=\alpha, w_{2}=\beta$, and for 3D: $w_{00}=0,
w_{01}=w_{10}=\alpha, w_{20}=w_{02}= \beta, w_{11}=2\beta$. The Sobelov value
now becomes
\begin{equation}
  \begin{array}{lllll}
%    &\onevf{G}{\vvvf{u}{\xi}}&=&\displaystyle\int_{\Omega}{\left\{
%    \alpha\norm{\dby{\vect{u}}{\xi}}^{2}+\beta\norm{\dtwosqby{\vect{u}}{\xi}}^{2}
%    \right\}d\xi} & \mbox{for 2D} \nonumber \\ 
%    \mbox{or}&\onevf{G}{\vvvf{u}{\xi}}&=&\displaystyle\int_{\Omega}
%    \left\{\alpha\left(\norm{\delby{\vect{u}}{\xi_{1}}}^{2}+\norm{\delby{\vect{u}}
%    {\xi_{2}}}^{2}\right)\right.+ & \nonumber \\ 
%    &&&\left.\beta\left(\norm{\deltwosqby{\vect{u}}{\xi_{1}}}^{2}+
%    2\norm{\deltwoby{\vect{u}}{\xi_{1}}{\xi_{2}}}^{2}+
%    \norm{\deltwosqby{\vect{u}}{\xi_{2}}}^{2}\right)\right\}d\vect{\xi}
%    & \mbox{for 3D} \nonumber
  \end{array}
  \label{eqn:ptwoSobnorm}
\end{equation}

The parameter $\alpha$ controls the tension of the surface and the parameter
$\beta$ controls the degree of surface curvature
\cite{terzopoulos:1986}.

\subsection{Non-linear geometric fitting with Hermite elements}

\subsubsection{Problem formulation}

One problem that arises when using linear fitting with cubic Hermite elements
is that arc-length derivatives and average arc-length scaling are not
maintained during the fit. In linear fitting there is no information supplied
in the linear fitting model that enforces arc-length derivatives and the scale
factors are held constant during the fit. Here we consider how to fit the data
whilst maintaining arc-length derivatives and an even spacing of arc-length
with $\xi$. Because both the value of the arc-length for the element and the
relationship between the derivatives in the various spatial directions depend
upon the mesh parameters in a non-linear fashion, the only way to ensure
arc-length derivatives are maintained during fitting is to use a non-linear
fitting procedure.

Consider the following: Let \vect{u} be the vector of mesh parameters, 
$\vect{z}_{d}$ the vector of the location of the data points in space and 
%\twovf{\vect{z}}{\vect{u}}{\vect{\xi}_{d}} 
the vector of the location of the closest projection (at the points given by
%the vector $\vect{\xi}_{d}$) of data point $d$ onto the mesh. Now the error in
the $d^{th}$ data point can be expressed by an error vector
\begin{equation}
%\twovf{\vect{e}_{d}}{\vect{u}}{\vect{\xi}_{d}}=\twovf{\vect{z}}{\vect{u}}{\vect{\xi}_{d}}
%-\vect{z}_{d}
\label{eqn:errorvec}
\end{equation}
and by a residual,
\begin{equation}
%\twovf{f_{d}}{\vect{u}}{\vect{\xi}_{d}}=\norm{\twovf{\vect{e}_{d}}{\vect{u}}
%{\vect{\xi}_{d}}}^{2}
\label{eqn:dataresidvec}
\end{equation}

%An objective function, \twovf{{\cal F}}{\vect{u}}{\vect{\xi}_{d}}, is formed 
as the sum of squares of the individual residuals for the mesh \vect{u} and
projections $\vect{\xi}_{d}$
\begin{equation}
%\twovf{{\cal F}}{\vect{u}}{\vect{\xi}_{d}}= \twovft{\vect{f}}{\vect{u}}{\vect{\xi}_{d}}
%\twovf{\vect{f}}{\vect{u}}{\vect{\xi}_{d}}
\label{eqn:dataresfun}
\end{equation}
%The fitting problem then becomes, for constant $\vect{\xi}_{d}$,
\begin{eqnarray*}
%\min_{\vect{u} \in \Re} & \vvf{{\cal F}}{u} = \twovft{\vect{f}}{\vect{u}}
%{\vect{\xi}_{d}}\twovf{\vect{f}}{\vect{u}}{\vect{\xi}_{d}}
\end{eqnarray*}

In order to maintain arc-length derivatives a non-linear constraint is needed.
This constraint comes from the geometric properties of arc-length derivatives.
For a node $n$ and $\xi$-direction $i$ the magnitude of the vector of
arc-length derivatives in the various spatial directions must be 1 as detailed
in \eqref{eqn:cubHermnormconst}.  Hence the constraint is
\begin{equation}
%  c=\norm{\brdelby{\vect{x}}{s_{i}}_{n}}=1
  \label{eqn:normconst}
\end{equation}
This also implies simple bounds on the derivative variables:
%\[-1 \leq \brdelby{x_{j}}{s_{i}}_{n} \leq 1\]

Thus the fitting problem can be written in terms of a non-linearly constrained,
non-linear optimisation problem
\begin{eqnarray}
%  \min_{\vect{u} \in \Re,~\vect{\xi}_{d} \mbox{const}} & \vvf{{\cal F}}{u} = 
%  \twovft{\vect{f}}{\vect{u}}{\vect{\xi}_{d}}\twovf{\vect{f}}{\vect{u}}{\vect{\xi}_{d}} 
  \label{eqn:nloproblem} \\
%  \mbox{subject to} & \vect{a} \leq \left(\begin{array}{c}
%  \vect{u} \nonumber \\ \vvvf{c}{u} \end{array} \right) \leq \vect{b} \nonumber
\end{eqnarray}
where \vect{a} is a vector of lower bounds, \vect{b} a vector of upper bounds 
%and \vvvf{c}{u} a vector of non-linear constraints. This type of optimisation 
problem can be solved with readily available non-linear optimisation packages 
such as NPSOL \cite{gill:1986}. 

In order to ensure that there is an approximately uniform spacing of $\xi$
with arc-length two approaches can be used. The first approach is to use
an iterative technique for the scale factors and is detailed here. The
second approach is to include the scale factors in the optimisation problem
and is detailed in the next section.

The constraint on the nodal scale factors (being the average of the line
lengths either side of the node) is placed upon the problem to ensure that the
arc-length to $\xi$ spacing is approximately uniform. As we are only getting
approximately uniform arc-length to $\xi$ spacing we can relax the constraint
on nodal scale factors. If the nodal scale factors are held fixed during the
fit we will not have average arc-length scaling throughout the fitting process
but we will have a reasonable approximation. With the nodal scale factors
fixed the variables ${\cal S}_{n}$ can be removed from the vector of mesh
%parameters \vect{u}.

The approach is to hold the scale factors constant, fit the mesh with
these scale factors, and then update the scale factors (based on the new mesh)
to be average arc-length. This process can be repeated iteratively until the
desired fit has be achieved. It should also be noted that
\eqref{eqn:nloproblem} is defined only for a constant data point projection.
With this iterative approach the data point projections are also updated
at the same time as the scale factors.

The convergence of the fitting problem can be measured in two ways. The first
is the convergence of the RMS error in the data and second is the convergence
in the magnitude of the nodal scale factors.

The algorithm is therefore:
\begin{enumerate}
  \item Define initial mesh (and calculate initial nodal scale factors)
  \item Calculate the initial data point projections
  \item Repeat until converged or the maximum number of iterations is exceeded
  \begin{enumerate}
    \item Fit the mesh to the data by solving \bref{eqn:nloproblem}
    \item Update the scale factors to be average arc-lengths based on the new
      mesh
    \item Recalculate the data point projections on the new mesh
  \end{enumerate}
\end{enumerate}

\subsubsection{Alternative formulation}
\label{sec:alternativeformulation}

An alternative approach to ensuring average arc-length scale factors can be
formulated by including the scale factors in the optimisation problem as
optimisation variables.  The vector of mesh parameters, \vect{u}, is hence
extended to include the nodal scale factors. With this a new constraint
can be introduced to \eqref{eqn:nloproblem} to ensure that there is an
approximately uniform spacing of $\xi$ with arc-length. This can be achieved
if the nodal scale factor is the average on the arc-lengths on either side of
that node. If $\left({\cal S}_{i}\right)_{n}$ is the average arc-length for
node $n$ in the $\xi_{i}$ direction as given by \bref{eqn:avearclenscale} then
the constraint that the nodal scale factor equals the average arc-length is
given by
\begin{equation}
%  c=\frac{\left(s_{i}\right)_{n\ominus}+\left(s_{i}\right)_{n\oplus}}{2}-
%  \left({\cal S}_{i}\right)_{n}=0
\end{equation}
or
\begin{equation} 
%  c=\frac{1}{2}\left(\int_{0}^{1}\norm{\delby{\vvvf{x}{\xi}}{\xi_{i}}}_
%  {n_{\ominus}}d\xi_{i}+\int_{0}^{1}\norm{\delby{\vvvf{x}{\xi}}{\xi_{i}}}_
%  {n_{\oplus}}d\xi_{i}
%  \right)-\left({\cal S}_{i}\right)_{n}=0
  \label{eqn:arclenconst}
\end{equation}
Note that no summation over $\xi_{i}$ is implied.

This also generates a simple bound on the scale-factors:
%\[\left({\cal S}_{i}\right)_{n} > 0\]

This formulation of the non-linear fitting problem does have one practical
limitation. With average arc-length scaling the interpolation within an
element depends on the arc-length on the neighbouring elements, and the
arc-length of the neighbouring elements depends on their neighbouring elements
and so on.  This results in a 'global mesh' in that every part of the mesh
is dependent on every other part of the mesh. The implication of this is that
the entire mesh must be included in the fit otherwise average arc-length
scaling cannot be achieved. This is not a desirable feature for very large
problems or for problems where only a small part of the mesh is in error and
needs to be fitted. The formulation still requires iteration for the data
point projections but does have the advantage that the number of iterations 
required is reduced as the scale factors are found during the fit. This is 
at the expense of having to solve a much larger non-linear optimisation
problem with more variables and, more importantly, more non-linear constraints.

The fit can be considered converged when either the data point projections or
the RMS error in the fit does not change significantly between iterations.
The algorithm for the non-linear data fitting procedure is as follows:

\begin{enumerate}
  \item Define the initial mesh (and calculate the initial nodal scale factors)
  \item Calculate the initial data point projections
  \item Repeat until converged or the maximum number of iterations is exceeded
  \begin{enumerate}
    \item Fit the mesh to the data by solving \bref{eqn:nloproblem} (with
      the new constraints)
    \item Recalculate the data point projections on the new mesh
  \end{enumerate}
\end{enumerate}

\subsubsection{Residual and constraint Jacobians}
\label{sec:resandcontjacs}

Solution of the non-linear problem given by \bref{eqn:nloproblem} will
generally require the evaluation of objective gradient (or residual vector
Jacobian) and the constraint Jacobian with respect to the optimisation (mesh)
parameters. For simplicity in this section we will be concerned with
two-dimensions only (i.e.  one-dimensional cubic Hermite elements). The
residual and constraint Jacobians will be given for the alternative
formulation as this covers all cases. If the scale factors are found by
iteration (i.e. the problem as described in \secref{sec:probform}) the
Jacobians with respect to the nodal scale factors can be ignored, as can the
second constraint (\eqref{eqn:arclenconst}) which ensures average arc-length
scale factors.  In this case there are three basic types of variable within an
element: the nodal variables $\left(x_{j} \right)_{n}$, the derivative
%variables $\brdelby{x_{j}}{s}_{n}$ and the nodal scale factors ${\cal S}_{n}$.
The residual and constraint Jacobians are given below for each of these three
variable types.

%Consider the error vector for the data point $d$, $\vect{e}_{d}$, defined in
\bref{eqn:errorvec}, and its corresponding residual $f_{d}$, defined in
\bref{eqn:dataresidvec}. The position of the projection of data point $d$
within the element is given by \bref{eqn:datapointpos}.  Substituting
\bref{eqn:datapointpos} into \eqnrefs{eqn:errorvec}{eqn:dataresidvec} and
differentiating with respect to the various optimisation variables we can
obtain the Jacobian of the residual vector:
\begin{eqnarray}
%  \delby{f_{d}}{\left(x_{j}\right)_{n}}&=&2\Psi^{0}_{n}(
%  \xi_{d})~{e_{d}}_{j} \label{eqn:dataresjacnode} \\
%  \delby{f_{d}}{\brdelby{x_{j}}{s}_{n}}&=&2\Psi^{1}_{n}(\xi_{d})~{\cal S}_{n}~
%  {e_{d}}_{j} \label{dataresjacderiv} \\
%  \delby{f_{d}}{{\cal S}_{n}}&=&2\Psi^{1}_{n}(\xi_{d})~\brdelby{\vect{x}}
%  {s}_{n}\cdot\vect{e}_{d} \label{eqn:dataresjacline}
\end{eqnarray}

Note that if the data residual was
%\[\twovf{f_{d}}{\vect{u}}{\vect{\xi}_{d}}=\norm{\twovf{\vect{e}_{d}}{\vect{u}}
%{\vect{\xi}_{d}}}\]
\eqref{eqn:dataresjacnode} would become
%\[\delby{f_{d}}{\left(x_{j}\right)_{n}}=\frac{\Psi^{0}_{n}(\xi_{d})~{e_{d}}_{j}}
%{f_{d}}\] which is singular at the optimal solution $f_{d}=0$. To avoid
numerical problems the residual is therefore defined by
\bref{eqn:dataresidvec}.

Now consider the constraint Jacobians. Differentiating constraint
\bref{eqn:normconst} with respect to the mesh parameters gives constraint
gradients:
\begin{eqnarray}
%  \delby{c}{\left(x_{j}\right)_{n}}&=&0 \label{eqn:normjacnode} \\
%  \delby{c}{\brdelby{x_{j}}{s}_{n}}&=&\frac{\brdelby{x_{j}}{s}_{n}}{\norm{
%  \brdelby{\vect{x}}{s}_{n}}}\label{eqn:normjacderiv} \\
%  \delby{c}{{\cal S}_{n}}&=&0 \label{eqn:normjacline}
\end{eqnarray}

To calculate the gradients for constraint \bref{eqn:arclenconst} we first
compute the rate of change of \vect{x} with respect to $\xi$ within an element
by differentiating \bref{eqn:cubHerminterp}:
\begin{equation}
%  \delby{\fnof{\vect{x}}{\xi}}{\xi}=\sum_{n=1,2}\left[\delby{\onevf{\Psi_{n}^{0}}{\xi}}
%  {\xi}\vect{x}_{n}+\delby{\onevf{\Psi_{n}^{1}}{\xi}}{\xi}\brdelby{\vect{x}}
%  {s}_{n}.{\cal S}_{n}\right]
  \label{eqn:delxdelxiinterp}
\end{equation}

Substituting \bref{eqn:delxdelxiinterp} into
\bref{eqn:arclenconst} and differentiating with respect to the optimisation
variables gives the constraint gradients:
\begin{eqnarray}
%  \delby{c}{\left(x_{j}\right)_{n}}&=&\half\left({\int_{0}^{1}\frac{\left(
%  \delby{\onevf{\Psi_{2}^{0}}{\xi}}{\xi}\delby{\onevf{x_{j}}{\xi}}{\xi}
%  \right)_{n_{\ominus}}}{\norm{\delby{\fnof{\vect{x}}{\xi}}{\xi}}_{n_{\ominus}}}d\xi}+
%  \int_{0}^{1}{\frac{\left(\delby{\onevf{\Psi_{1}^{0}}{\xi}}{\xi}\delby{
%  \onevf{x_{j}}{\xi}}{\xi}\right)_{n_{\oplus}}}{\norm{\delby{\fnof{\vect{x}}{\xi}}
%  {\xi}}_{n_{\oplus}}}d\xi}\right) \label{eqn:arclenjacnode} \\
%  \delby{c}{\brdelby{x_{j}}{s}_{n}}&=&\half\left({\int_{0}^{1}\frac{\left(
%  \delby{\onevf{\Psi_{2}^{1}}{\xi}}{\xi}{\cal S}_{n}\delby{\onevf{x_{j}}
%  {\xi}}{\xi}\right)_{n_{\ominus}}}{\norm{\delby{\fnof{\vect{x}}{\xi}}{\xi}}_{n_{
%  \ominus}}}d\xi}+\int_{0}^{1}{\frac{\left(\delby{\onevf{\Psi_{1}^{1}}{\xi}}
%  {\xi}{\cal S}_{n}\delby{\onevf{x_{j}}{\xi}}{\xi}\right)_{n_{\oplus}}}
%  {\norm{\delby{\fnof{\vect{x}}{\xi}}{\xi}}_{n_{\oplus}}}d\xi}\right) 
%  \label{eqn:arclenjacderiv} \\
%  \delby{c}{{\cal S}_{n}}&=&\half\left(\int_{0}^{1}{\frac{\left(\delby{
%  \onevf{\Psi_{2}^{1}}{\xi}}{\xi}\brdelby{\vect{x}}{s}_{n}\cdot\delby{
%  \fnof{\vect{x}}{\xi}}{\xi}\right)_{n_{\ominus}}}{\norm{\delby{\fnof{\vect{x}}{\xi}}
%  {\xi}}_{n_{\ominus}}}d\xi}\right. + \nonumber \\
%  &&\hspace{3cm}\left.\int_{0}^{1}{\frac{\left(\delby{\onevf{\Psi_{1}^{1}}
%  {\xi}}{\xi}\brdelby{\vect{x}}{s}_{n}\cdot\delby{\fnof{\vect{x}}{\xi}}{\xi}\right)_{
%  n_{\oplus}}}{\norm{\delby{\fnof{\vect{x}}{\xi}}{\xi}}_{n_{\oplus}}}d\xi}\right) 
%  - 1 
   \label{eqn:arclenjacline}
\end{eqnarray}

Note that when the integrand in these formula contains $n_{\ominus}$
($n_{\oplus}$) the $\xi$ variable is assumed to be taken over the previous
(next) element.

Similar results can be found for three-dimensions (i.e. bicubic Hermite
elements).

\subsubsection{Sobelov smoothing}

In order to implement Sobelov smoothing in the form of the non-linear
optimisation problem given in \bref{eqn:nloproblem} we need to modify the
residual vector by defining an additional residual as the Sobelov value for
the mesh being fitted, that is
\begin{equation}
%  \twovf{f_{D+1}}{\vect{u}}{\vect{\xi}}=\onevf{G}{\vvvf{u}{\xi}}
  \label{eqn:addSobresidual}
\end{equation}
%where \onevf{G}{\vvvf{u}{\xi}} is defined in \bref{eqn:ptwoSobnorm}.
The objective function to minimise now becomes
%\[\vvf{F}{u}=\vvf{{\cal F}}{u}+\vvf{G^{2}}{u}=\vvf{{\cal F}}{u}+
%\vvf{{\cal G}}{u}\]
%where \vvf{{\cal F}}{u} is the data objective defined in \bref{eqn:dataresfun}
%and $\vvf{{\cal G}}{u}=\vvf{G^{2}}{u}$ is the Sobelov objective.

The gradients of this additional residual can then be found by differentiating
\bref{eqn:ptwoSobnorm} with respect to the mesh parameters. Considering the
case of two-dimensional fitting and breaking the domain up into elements we
obtain
%\[\vvf{G}{u}=\sum_{e}\int_{0}^{1}\left\{\alpha\normdelby{\fnof{\vect{x}}{\xi}}
%{\xi}^{2}+\beta\norm{\frac{\del^{2}\fnof{\vect{x}}{\xi}}{\del \xi^{2}}}^{2}
%\right\}d\xi\]
Interpolating within the element using cubic Hermite elements and 
differentiating with respect to the three different mesh parameters within
an element we obtain the additional residual gradients given in equations 
(\ref{eqn:sobjacnode}--\ref{eqn:sobjacline}).
\begin{eqnarray}
%  \delby{f_{D+1}}{\left(x_{j}\right)_{n}}&=&2\int_{0}^{1}\left\{\alpha
%  \delby{\onevf{\Psi_{n}^{0}}{\xi}}{\xi}\delby{\onevf{x_{j}}{\xi}}{\xi}+
%  \beta\frac{\del^{2}\onevf{\Psi_{n}^{0}}{\xi}}{\del\xi^{2}}\frac{\del^{2}
%  \onevf{x_{j}}{\xi}}{\del\xi^{2}}\right\}d\xi \label{eqn:sobjacnode} \\
%  \delby{f_{D+1}}{\brdelby{x_{j}}{s}_{n}}&=&2\int_{0}^{1}\left\{\alpha
%  \delby{\onevf{\Psi_{n}^{1}}{\xi}}{\xi}{\cal S}_{n}\delby{\onevf{x_{j}}
%  {\xi}}{\xi}+\beta\frac{\del^{2}\onevf{\Psi_{n}^{1}}{\xi}}{\del\xi^{2}}
%  {\cal S}_{n}\frac{\del^{2}\onevf{x_{j}}{\xi}}{\del\xi^{2}}\right\}d\xi 
%  \label{eqn:sobjacderiv} \\
%  \delby{f_{D+1}}{{\cal S}_{n}}&=&2\int_{0}^{1}\left\{\alpha\delby{
%  \onevf{\Psi_{n}^{1}}{\xi}}{\xi}\brdelby{\vect{x}}{s}_{n}\cdot\delby{\fnof{\vect{x}}
%  {\xi}}{\xi}+\beta
%  \frac{\del^{2}\onevf{\Psi_{n}^{1}}{\xi}}{\del\xi^{2}}\brdelby{\vect{x}}{s}_{n}
%  \cdot\frac{\del^{2}\fnof{\vect{x}}{\xi}}{\del\xi^{2}}\right\}d\xi \label{eqn:sobjacline}
\end{eqnarray}

Note that all the above integrals are with respect to one element. All the
contributions from each element need to be included to obtain the complete
residual gradients for the optimisation parameters.

%%% Local Variables: 
%%% mode: latex
%%% TeX-master: "/product/cmiss/documents/notes/fembemnotes/fembemnotes"
%%% compile-command: "cd ..; latex '\\nonstopmode\\documentclass[12pt,twoside,a4paper]{book}

%\includeonly{titlepage}
%\includeonly{fem_basis_fns/fem_basis_fns}
%\includeonly{heat_conduction/heat_conduction}
%\includeonly{bem/bem}
%\includeonly{transient_heat_condn/transient_heat_condn}
%\includeonly{lin_elasticity/lin_elasticity}
%\includeonly{modal_analysis/modal_analysis}
%\includeonly{con_mechanics/con_mechanics}

%\includeonly{domints_in_bem/domints_in_bem}
%\includeonly{timedep_bem/timedep_bem}
%\includeonly{datafitting/datafitting}


%%% the first three chapters
%\includeonly{titlepage,fem_basis_fns/fem_basis_fns,heat_conduction/heat_conduction,bem/bem}
%%% chapters 4-6
%\includeonly{titlepage,transient_heat_condn/transient_heat_condn,lin_elasticity/lin_elasticity,modal_analysis/modal_analysis}
%%% chapters 7-9
%\includeonly{titlepage,con_mechanics/con_mechanics,domints_in_bem/domints_in_bem,timedep_bem/timedep_bem}

\input{../latex/macros} %define new commands etc.
\input{../latex/defns} %define packages etc.
\input{../latex/abbreviations} %define packages etc.

\usepackage{../latex/mybook} %define book style
%\usepackage{../latex/lofexamples} %list of examples style

\usepackage{times} %fonts

%\usepackage{mathptm}

\renewcommand{\baselinestretch}{1.24}

% Note: single spacing used for student notes
\singlespc
\raggedbottom

% NOTE: many of the \todo and \remarks have been commented out
% so the current version can tbe printed out for teaching these
% should still be looked at NPS 22/2/99 

\normalsize
\makeindex

\title{The FEM-BEM-notes}
\author{}              % Author's name
\date{\today}          % Revision Date


\begin{document}

\include{titlepage}
\clearemptydoublepage 

\pagenumbering{roman}
\tableofcontents

%\clearemptydoublepage
%\listofexamples
\clearemptydoublepage 
\pagenumbering{arabic}

\include{fem_basis_fns/fem_basis_fns}
\clearemptydoublepage
\include{heat_conduction/heat_conduction}
\clearemptydoublepage
\include{bem/bem}
\clearemptydoublepage
\include{transient_heat_condn/transient_heat_condn}
\clearemptydoublepage
\include{lin_elasticity/lin_elasticity}
\clearemptydoublepage
\include{modal_analysis/modal_analysis}
\clearemptydoublepage
\include{con_mechanics/con_mechanics}
\clearemptydoublepage
\include{datafitting/datafitting}
\clearemptydoublepage
\include{derivative_bie/derivative_bie}
\clearemptydoublepage
%\include{chapter9/FEM9}
%\clearemptydoublepage
\include{domints_in_bem/domints_in_bem}
\clearemptydoublepage
\include{timedep_bem/timedep_bem}
\clearemptydoublepage

\include{references}
\clearemptydoublepage

\addcontentsline{toc}{chapter}{\numberline{}Index}
\printindex

\end{document}

%%% Local Variables: 
%%% mode: latex
%%% TeX-master: t
%%% End: 
'"
%%% End: 

\clearemptydoublepage
\chapter{Derivative BIE}

\section{Boundary Element Formulation}

The BEM will be used in any region of the torso in which the conductivity can
be reasonably taken to be constant (e.g. the lungs).  Thus the equation to be
solved in such a region is simply the Laplace equation.  The conventional
boundary integral equation for Laplace's equation, $\nabla\phi = 0$ in a
(closed) domain $\Omega \subset \Re^{m}\; (m=2$ or 3) can be written as
\begin{equation}
 C(P_{0})\phi(P_{0}) =
 \left(G(P,P_{0})\delby{\phi(P)}{n}-\delby{G(P,P0)}{n}\phi(P)\right)
 d\Gamma(P)
 \label{eq:closed}
\end{equation}
where $C(P_{0}) = 1$ if $P_{0}\in \Omega^{0}$, and $G(P,P_{0})$ is the
fundamental solution for the Laplace equation in either two or three
dimensions (i.e. $G(P,P_{0}) = -\frac{1}{2\pi}\log \parallel P
-P_{0}\parallel$ or $\frac{1}{4\pi}\parallel P-P_{0}\parallel$ ).

The standard procedure in a boundary element formulation is to take the point
$P_{0}$ to be one of the nodes used to approximate the dependent variable
$\phi$, and this generates one integral equation for each node.  When Lagrange
interpolation is used for $\phi$ there is only one unknown per node (allowing
for boundary conditions and assuming no free surfaces). This means that a
square system of equations is produced and a unique solution can be found.
However, when using cubic Hermite interpolation there is always more than one
unknown per node - in two-dimensions there are 2 (with Neuman boundary
conditions these are $\phi$ and $\delby{\phi}{s}$) and in three-dimensions
there are 4 (with Neuman boundary conditions these are $\phi,
\delby{\phi}{s_{1}}, \delby{\phi}{s_{2}}$ and $\frac{\del^{2}\phi}{\del
  s_{1}\del s_{2}}$).  Thus one needs to generate extra linearly independent
equations.  This is achieved by differentiating (\ref{eq:closed}) in various
directions \cite{tomlinson:1996}, yielding a hypersingular integral
formulation.

Taking the directional derivative of (\ref{eq:closed}) at $P_{0}$ in an
(initially) arbitrary direction $n_{0} (\parallel n_{0}\parallel =1)$ gives
\begin{equation}       
  \delby{\phi(P_{0})}{n_{0}} = \int_{\del\Gamma}\left(\delby{G(P,P_{0})}{n}
    \delby{\phi(P)}{n}-\frac{\del^{2}G(P,P_{0})}{\del n\del n_{0}}
    \phi(P)\right)d\Gamma(P)
 \label{eq:arbdir}
\end{equation} 
valid $\forall P_{0}\in \Omega^{0}$.  The hypersingular fundamental solution
$\frac{\del^{2}G(P,P_{0})}{\del n\del n_{0}}$ is given by either
\begin{eqnarray*}
  &\frac{1}{2\pi}\left[\frac{{\bf n.n}_{0}}{r^{2}}-\frac{2({\bf r.n}_{0})
      ({\bf r.n})}{r^{4}}\right] & \mbox{(two-dimensions)}\\ \mbox{or
    }&\frac{1}{4\pi}\left[\frac{{\bf n.n}_{0}}{r^{3}} -\frac{3({\bf r.n}_{0})
      ({\bf r.n})}{r^{5}}\right] & \mbox{(three-dimensions)}
\end{eqnarray*}
Here ${\bf r}$ is the vector of modulus $r = \parallel P-P_{0}\parallel$ from
$P_{0}$ to $P, {\bf n}$ is the unit outward normal vector and $n_{0}$ is the
unit vector in the direction of differentiation.

At this stage the integral expression (\ref{eq:arbdir}) is well defined.  With
$P_{0}$ placed on the boundary of $\Omega$ the second integrand is termed
hypersingular and it is the interpretation of this term and the procedure used
to take $P_{0}$ to the boundary that gives rise to the variety of integral
expressions currently in use in hypersingular boundary integral formulations.
To obtain a weakly-singular form of the derivative equation , one requires the
following identities
\begin{eqnarray}
 \int_{\del\Omega}\delby{G(P,P_{0})}{n}d\Gamma(P)=-1 & & 
  \forall P_{0}\in \Omega^{0}\\
 \nonumber\\
 \int_{\del\Omega}\frac{\del^{2}G(P,P_{0})}{\del n\del n_{0}}d\Gamma(P)=0  & & \forall P_{0}\in \Omega^{0}\\
 \nonumber\\
 \int_{\del\Omega}\left(n_{k}\delby{G(P,P_{0})}{n_{0}}-(x_{k}-x_{0k})
  \frac{\del^{2}G(P,P_{0})}{\del n\del n_{0}}\right)d\Gamma(P)=
  n_{0k}(P_{0}) & & 
  \forall P_{0}\in \Omega^{0}
\end{eqnarray}
The first two of these identities are obtained from (\ref{eq:closed}) and
(\ref{eq:arbdir}) respectively by putting $\phi \equiv 1$.  The third identity
is also obtained from (\ref{eq:arbdir}) by substitution of $x_{k}-x_{0k}$
(where $x_{k}, x_{0k}, n_{k}, n_{0k}$ are the $k$th cartesian components of
$P, P_{0}, n,$ and $n_{0}$ respectively).

The derivation of the weakly singular form of the derivative equation follows
the traditional derivation of the conventional boundary integral equation.
The point $P_{0}$ is placed on the boundary of the domain $\Omega$ and the
domain is enlarged about this point to include a semicircular or hemispherical
region of some small radius $\varepsilon > 0$.  The limits of the integral
expression (\ref{eq:arbdir}) over each part of the enlarged domain are then
considered in detail as$\varepsilon \rightarrow 0$, with judicious application
of the above identities \cite{tomlinson:1996,liu:1991}.  The result is
\begin{eqnarray}
 \int_{\del\Omega}\frac{\del^{2}G(P,P_{0})}{\del n\del n_{0}}
  &\left(\phi(P)-\phi(P_{0})-\delby{\phi(P_{0})}{x_{k}}\right)
  d\Gamma(P)\nonumber\\
  &=\int_{\del\Omega} n_{k}\delby{G(P,P_{0})}{n_{0}}\left(
  \delby{\phi(P)}{n}-\delby{\phi(P_{0})}{x_{k}}n_{k}(P)\right)
  d\Gamma(P)
 \label{eq:result}
\end{eqnarray}
where now $P_{0}\in \del\Omega$ (the singular point), $n$ is a unit outward
normal on $\del\Omega, n_{0}$ is an arbitrary unit vector at $P_{0}$ (the
direction in which (\ref{eq:Laplace}) was differentiated) and a sum over $k$
is implied.

\subsection{Discretisation}
The surface is discretised into ``boundary elements'' i.e.$\Omega =
\displaystyle{\bigcup_{m=1}^{M}}\Omega_{m}$.  Introducing cubic Hermite
interpolation for $\phi $, and, as above, approximating $\delby{\phi(P)}{n}$
by $N_{\alpha}\left(\delby{\phi(P)}{n}\right)_{\alpha}$ and also interpolating
the geometry as $x_{k} = M_{\alpha} x_{k}^{\alpha}$ for some set of basis
functions $\{M_{\alpha}\}$ (these will be taken to be cubic Hermite, but for
now we leave these general), Equation~(\ref{eq:result}) becomes
\begin{eqnarray}
  \sum_{m=1}^{M}\int_{\Gamma_{m}}\frac{\del^{2}G(P,P_{0})} {\del n\del n_{0}}
  &&\left(\Psi_{\alpha}^{i}\phi_{i}^{\alpha}-\phi(P_{0})
    -\delby{\phi(P_{0})}{x_{k}}(M_{\alpha}x_{k}^{\alpha} - x_{0k})
  \right)d\Gamma(P)\nonumber\\ 
  &=&\sum_{m=1}^{M}\int_{\Gamma_{m}}\delby{G(P,P_{0})}{n_{0}}\left(
    N_{\alpha}\delby{\phi^{\alpha}}{n}-\delby{\phi(P_{0})}{x_{k}}
    n_{k}(P)\right)d\Gamma(P)
 \label{eq:result2}
\end{eqnarray}
or, in terms of the local $\xi$ coordinate
\begin{eqnarray}
 \sum_{m=1}^{M}\int_{0}^{1}\frac{\del^{2}G(\xi,P_{0})}
  {\del n\del n_{0}}
  &&\left(\Psi_{\alpha}^{i}(\xi)\phi,_{i}^{\alpha}-\phi(P_{0})
  -\delby{\phi(P_{0})}{x_{k}}(M_{\alpha}(\xi)x_{k}^{\alpha} - x_{0k})
  \right)|J(\xi)d\xi\nonumber\\
 &=&\sum_{m=1}^{M}\int_{0}^{1}\delby{G(\xi,P_{0})}{n_{0}}\left(
  N_{\alpha}(\xi)\delby{\phi^{\alpha}}{n}
  -\delby{\phi(P_{0})}{x_{k}}n_{k}(\xi)
  \right)|J(\xi)d\xi
 \label{eq:result3}
\end{eqnarray}
As with the conventional boundary integral equation, $P_{0}$ is located at
each of the solution variable nodes in turn.

The global unknowns we are dealing with are nodal values of
$\phi,\delby{\phi}{s}$ and $\phi,\delby{\phi}{n}$ (and
also$\frac{\del^{2}\phi}{\del s\del n}$ if cubic Hermite interpolation is used
for the normal derivative).  By adopting a local coordinate system ($s,n$) [or
$(\xi,n)$] one has, by the chain rule,
\begin{eqnarray}
 \delby{\phi}{x_{k}}=\left(\delby{\phi}{\sigma_{l}}\delby{\sigma_{l}}
  {x_{k}}\right) & & (l=1,2)
 \label{eq:chain}
\end{eqnarray}
where ($\sigma_{1},\sigma_{2}) = (s,n)$ [or ($s_{1}, s_{2}, n$) in three
dimensions]. $\delby{\sigma_{l}}{x_{k}}$ can either be obtained exactly (as in
some of the test problems) or, more generally, by inverting the matrix
$\left[\delby{x_{k}}{\sigma_{l}}\right]$.  For a cubic Hermite geometric mesh,
$\delby{x_{k}}{s}$ are known nodal values (either entered or calculated as
part of the mesh fitting) so the matrix entries are all known.  The use of
(\ref{eq:chain}) in (\ref{eq:result3}) yields an integral expression, which,
with the application of element scale factors (if required), involves only the
required nodal unknowns as coefficients.  We concentrate now on obtaining
expressions suitable for numerical computation for these coefficients.  As in
\cite{liu:1992} two separate cases must be considered - that for which $P_{0}$
is contained in the element $\Gamma_{m}$ being considered, and that for which
$P_{0}$ is removed from the current element $\Gamma_{m}$.  For each case, we
investigate both integrals of (\ref{eq:result3}).

\subsection*{Case 1:  $P_{0}??? \Gamma_{m}$ SHOULD BE NOT AN ELEMENT} 
In this case the integrands of both integrals in (\ref{eq:result3}) are
nonsingular and both integrals exist.  The first integral can be written as
\begin{eqnarray}
 \phi,_{i}^{\alpha}\int_{0}^{1}\frac{\del^{2}G(\xi,P_{0})}
  {\del n\del n_{0}}\Psi_{\alpha}^{i}(\xi)|J(\xi)|d\xi &-& \phi(P_{0})
  \int_{0}^{1}\frac{\del^{2}G(\xi,P_{0})}
  {\del n\del n_{0}}|J(\xi)|d\xi\nonumber\\
 &-&\delby{\phi}{\sigma_{l}}(P_{0})\delby{\sigma_{l}}{x_{k}}(P_{0})
   \int_{0}^{1}\frac{\del^{2}G(\xi,P_{0})}{\del n\del n_{0}}
   (M_{\alpha}(\xi)x_{k}^{\alpha}-x_{0k})|J(\xi)|d\xi
 \label{eq:1stint}
\end{eqnarray}
Since $\phi,_{i}^{\alpha}$ is either a nodal value of $\phi$ or
$\delby{\phi}{\xi}$, multiplication of the appropriate first integrals by
element scale factors generates coefficients of nodal values of
$\delby{\phi}{s}$ .  Each integral expression is (\ref{eq:1stint}) is regular
and standard Gaussian quadrature can be used to evaluate these.

The second integral can be written as
\begin{equation}
 \delby{\phi^{\alpha}}{n}\int_{0}^{1}\frac{\del^{2}G(\xi,P_{0})}
  {\del n\del n_{0}}N_{\alpha}(\xi)|J(\xi)|d\xi - 
  \delby{\phi}{\sigma_{l}}(P_{0})\delby{\sigma_{l}}{x_{k}}(P_{0})
   \int_{0}^{1}\frac{\del G(\xi,P_{0})}{\del n\del n_{0}}
   n_{k}(\xi)|J(\xi)|d\xi
 \label{eq:2ndint}
\end{equation}
and again each integral can be evaluated by standard Gaussian quadrature.  A
more efficient grouping of the integrals in (\ref{eq:1stint}) and
(\ref{eq:2ndint}) can be obtained by grouping like coefficients, resulting in
some computational saving.

\subsection*{Case 2:  $P_{0}\in \Gamma_{m}$} 
In this case special care must be exercised to retain the weakly singular
nature of the integral expressions in (\ref{eq:result3}).  A straightforward
application of either (\ref{eq:1stint}) or (\ref{eq:2ndint}) will result in
divergent integrals.  We identify with $\beta$ the local node on element
$\Gamma_{m}$ corresponding to $P_{0}$.  Keeping the appropriate coefficients
grouped, we have from the first integral of (\ref{eq:result3})
\begin{eqnarray}
 \phi(P_{0})\int_{0}^{1}\frac{\del^{2}G(\xi,P_{0})}
  {\del n\del n_{0}}(\Psi_{\beta}^{0}(\xi)-1)|J(\xi)|d\xi & + 
  \phi ,_{i}^{\alpha\{\alpha\neq\beta\}}\int_{0}^{1}
  \frac{\del^{2}G(\xi,P_{0})}
  {\del n\del n_{0}}(\Psi_{\alpha}^{i}(\xi)-1)|J(\xi)|d\xi\nonumber\\
 +\delby{\phi}{s}(P_{0})\int_{0}^{1}\frac{\del^{2}G(\xi,P_{0})}
  {\del n\del n_{0}}\left[\delby{s}{\xi}(P_{0})\Psi_{\beta}^{1}(\xi)\right.
 & \left.- \delby{s}{x_{k}}(P_{0})(M_{\alpha}(\xi)x_{k}^{\alpha}-x_{0k})\right]
 |J(\xi)|d\xi\nonumber\\
 -\delby{\phi}{n}(P_{0})\int_{0}^{1}\frac{\del^{2}G(\xi,P_{0})}
  {\del n\del n_{0}}\delby{n}{x_{k}}(P_{0})
  \left(M_{\alpha}(\xi)x_{k}^{\alpha}-x_{0k}\right)|J(\xi)|d\xi
 \label{eq:from1stint}
\end{eqnarray}
Here we have explicitly included the element scale factors, and the second
integral involves a sum over $\alpha$ with $\alpha\neq\beta$.  With this
grouping, all integrals are at most weakly singular and amenable to numerical
integration.  The second integral of (\ref{eq:result3}) is more
straightforward, although certain groupings must still be retained.  The
appropriate expression for numerical integration is
\begin{eqnarray}
 \delby{\phi}{n}(P_{0}) &\int_{0}^{1}\frac{\del G(\xi,P_{0})}
  {\del n_{0}}\left[ N_{\beta}(\xi)-\delby{n}{x_{k}}(P_{0})n_{k}(\xi)\right]
  |J(\xi)|d\xi \nonumber\\
 &-&\delby{\phi^\alpha{\alpha\neq\beta}}{n}\int_{0}^{1}
  \frac{\del G(\xi,P_{0})}{\del n_{0}}N_{\alpha}(\xi)|J(\xi)|d\xi\nonumber\\
 &-&\delby{\phi}{s}(P_{0})\int_{0}^{1}\frac{\del G(\xi,P_{0})}
  {\del n_{0}}\delby{s}{x_{k}}(P_{0})n_{k}(\xi)|J(\xi)|d\xi
 \label{eq:approp}
\end{eqnarray}
The above equations provide numerically well-behaved expressions for the
coefficients of the discretised derivative boundary integral equation.  It
only remains to specify the direction(s) of $n_{0}$.  After extensive testing
\cite{tomlinson:1996} we concluded that the direction of differentiation
should be that of $s$ (i.e.  tangential) for two-dimensional problems (where
one extra equation is required) and for three-dimensional problems, one should
differentiate in both the $s_{1}$ and $s_{2}$ directions (and set the cross
derivative coefficients to zero).

\section{Three-dimensional element splitting}

If the singular point $P_{0}$ is contained in the current element then the
element is sub- divided in local ($\xi_{1}, \xi_{2}$) space according to the
location of $P_{0}$.  Each of these triangular elements is then transformed to
a regular four-noded element in another space [say ($s, t$) space] in such a
way that the Jacobian of this last transformation cancels the dominant
singularity in the integrand (which is $O( \frac{1}{r} )$ in the original
space, where $r$ is the distance from the singular point $P_{0}$).  The
numerical integrations are then performed over the elements in the ($s,t$)
space.  We use the following subdivisions and transformations.  In each case
the absolute value of the Jacobian of the transformation for element $A$ is
$s$ and for element $B$ is $t$.

\begin{figure}
\centering
 \input{figs/A1.pstex}
 \caption[The element splitting used when $P_{0}$ is at local node 1]{The element splitting used when $P_{0}$ is at local node 1.  The individual transformations 
   yield Jacobians that reduce the order of the singularity in the otherwise
   singular boundary element integrals.}
\label{fig:A1}
\end{figure}

\subsection{$P_{o}$ at local node 1}

Sub-divide from local node 1 to 4, giving two elements $A$ and $B$.

Transformation for element $A$:
\begin{displaymath}
\begin{array}{llllr}        
  s = \xi_{1} & t = \frac{\xi_{2}}{\xi_{1}} & \left(\mbox{inverse mapping
      is}\right. & \xi_{1} = s & \left.\xi_{2} = s t \right)
\end{array}
\end{displaymath}

Transformation for element $B$:
\begin{displaymath}
\begin{array}{llllr}        
 s = \frac{\xi_{2}}{\xi_{1}} & t = \xi_{2} & 
  \left(\mbox{inverse mapping is}\right. & \xi_{1} = st & 
  \left.\xi_{2} = t \right)
\end{array}
\end{displaymath} 
 
\subsection{$P_{o}$ at local node 2}
Sub-divide from local node 2 to 3.  Element $A$ contains local nodes 1, 2 and
3.

Transformation for element $A$:
\begin{displaymath}
\begin{array}{llllr}        
 s = 1-\xi_{1} & t = \frac{1-\xi_{1}-\xi_{2}}{1-\xi_{1}} & 
  \left(\mbox{inverse mapping is}\right. & \xi_{1} = 1-s & 
  \left.\xi_{2} = s (1-t) \right)
\end{array}
\end{displaymath}

Transformation for element $B$:
\begin{displaymath}
\begin{array}{llllr}        
 s = \frac{1-\xi_{1}}{\xi_{2}} & t = \xi_{2} & 
  \left(\mbox{inverse mapping is}\right. & \xi_{1} = 1-st & 
  \left.\xi_{2} = t \right)
\end{array}
\end{displaymath}  

\subsection{$P_{o}$ at local node 3}
Sub-divide from local node 2 to 3.  Element $A$ contains local nodes 2,3 and 4.

Transformation for element $A$:
\begin{displaymath}
\begin{array}{llllr}        
 s = \xi_{1} & t = \frac{1-\xi_{2}}{\xi_{1}} & 
  \left(\mbox{inverse mapping is}\right. & \xi_{1} = s & 
  \left.\xi_{2} = 1-st \right)
\end{array}
\end{displaymath}

Transformation for element $B$:
\begin{displaymath}
\begin{array}{llllr}        
 s = \frac{1-\xi_{1}-\xi_{2}}{1-\xi_{2}} & t = 1-\xi_{2} & 
  \left(\mbox{inverse mapping is}\right. & \xi_{1} = t(1-s) & 
  \left.\xi_{2} = 1-t \right)
\end{array}
\end{displaymath}

\subsection{$P_{o}$ at local node 4}
Sub-divide from local node 1 to 4.  Element $A$ contains local nodes 1,3 and 4.
        
Transformation for element $A$:
\begin{displaymath}
\begin{array}{llllr}        
 s = 1-\xi_{1} & t = \frac{1-\xi_{2}}{1-\xi_{1}} & 
  \left(\mbox{inverse mapping is}\right. & \xi_{1} = 1-s & 
  \left.\xi_{2} = 1-st \right)
\end{array}
\end{displaymath}

Transformation for element $B$:
\begin{displaymath}
\begin{array}{llllr}        
 s = \frac{1-\xi_{1}}{1-\xi_{2}} & t = 1-\xi_{2} & 
  \left(\mbox{inverse mapping is}\right. & \xi_{1} = 1-st & 
  \left.\xi_{2} = 1-t \right)
\end{array}
\end{displaymath}

\section{Hermite ``Simplexes''}

We describe here the special three-noded Hermite elements that have been used
to close a cubic Hermite surface in three-dimensional space.  The need for
such elements was illustrated in Figure~\ref{fig:2}.  We describe the two
special elements that were used - one in which the derivatives come to a point
at local node 3 (the top), and one in which the derivatives come together at
local node 1 (the bottom).

\begin{figure}
\centering
 \input{figs/B1.pstex}
 \caption[The two Hermite ``simplexes'']{The two Hermite ``simplexes'' and their connection to standard cubic Hermite 
elements.}
\label{fig:B1}
\end{figure}

\subsection{Apex at local node 3}
With reference to Figure~\ref{fig:B1}~a, we construct the appropriate
two-dimensional interpolation functions via a tensor product (instead of using
area functions, which are commonly used in three-noded elements).  The tensor
product formulation requires a description of the basis functions to be used
in each direction.  We use standard Hermite interlation in the $\xi_{1}$
direction, so that any standard cubic Hermite element sharing this edge
maintains consistent interpolation.  In the $\xi_{2}$ direction we require the
interpolation function to interpolate nodal values of function and derivative
at the first node, and only the value of the function at the second node (i.e.
we have dropped the arclength derivative at the second node since it is no
longer uniquely defined).  Thus we require
\begin{displaymath}
 {\bf x}(\xi_{2}) = \Psi_{1}(\xi_{2}){\bf x}_{1} + \Psi_{2}(\xi_{2}) 
  \delby{{\bf x}_{1}}{\xi_{2}}  + \Psi_{3}(\xi_{2}) {\bf x}_{3}
\end{displaymath}  
where   
\begin{eqnarray*}
 \Psi_{1} (0)  = 1,&\Psi_{1} (1)=0,&\delby{\Psi_{1}}{\xi_{2}} (0) = 0\\  \\
 \Psi_{2} (0)  = 0,&\Psi_{2} (1)=0,&\delby{\Psi_{2}}{\xi_{2}}(0) = 1\\
 \\  
 \Psi_{3} (0)  = 0,&\Psi_{3} (1)=1,&\delby{\Psi_{3}}{\xi_{2}} (0) = 0
\end{eqnarray*}  
These conditions yield
\begin{eqnarray*}
 \Psi_{1} (\xi) & = & 1 - \xi^{2}\\ 
 \Psi_{2} (\xi) & = & \xi - \xi^{2}\\ 
 \Psi_{3} (\xi) & = & \xi^{2}
\end{eqnarray*} 
and the appropriate two-dimensional basis function is obtained from a tensor
product of the standard Hermite basis function and the family given above.

\subsection{Apex at local node 1}
Again, with reference to Figure~\ref{fig:B1}~b, we construct the appropriate
two-dimensional interpolation functions via a tensor product with standard
Hermite interpolation in the $\xi_{1}$ direction.  In the $\xi_{2}$ direction
we require the interpolation function to interpolate nodal values of function
at the first node, and the value of the function and derivative at the second
node.  Thus we require
\begin{displaymath}
 {\bf x}(\xi_{2}) = \zeta_{1}(\xi_{2}){\bf x}_{1} + \zeta_{2}(\xi_{2}) 
  \delby{{\bf x}_{3}}{\xi_{2}}  + \zeta_{3}(\xi_{2}) {\bf x}_{3}
\end{displaymath}  
where   
\begin{eqnarray*}
 \zeta_{1} (0)= 1,&\zeta_{1} (1)=0,&\delby{\zeta_{1}}{\xi_{2}} (1) = 0\\  \\
 \zeta_{2} (0)= 0,&\zeta_{2} (1)=0,&\delby{\zeta_{2}}{\xi_{2}}(1) = 1\\
 \\  
 \zeta_{3} (0)= 0,&\zeta_{3} (1)=1,&\delby{\zeta_{3}}{\xi_{2}} (1) = 0
\end{eqnarray*}  
These conditions yield
\begin{eqnarray*}
 \zeta_{1} (\xi) & = & (\xi-1)^{2}\\ 
 \zeta_{2} (\xi) & = & \xi^{2} - \xi\\ 
 \zeta_{3} (\xi) & = & 2\xi-\xi^{2}
\end{eqnarray*} 
and the appropriate two-dimensional basis function are obtained as above.

%%% Local Variables: 
%%% mode: latex
%%% TeX-master: "/product/cmiss/documents/notes/fembemnotes/fembemnotes"
%%% End: 

\clearemptydoublepage
%\include{chapter9/FEM9}
%\clearemptydoublepage
\chapter{Domain Integrals in the BEM}

\section{Achieving a Boundary Integral Formulation}

The principal advantage of the BEM over other numerical methods is the
ability to reduce the problem dimension by one.  This property is
advantagous as it reduces the size of the solution system leading to
improved computational efficiency.  This reduction of dimension also eases
the burden on the engineer as it is only necessary to construct a boundary
mesh to implement the BEM.

To achieve this reduction of dimension it is necessary to formulate the
governing equation as a boundary integral equation.  To achieve a boundary
integral formulation it is necessary to find an appropriate reciprocity
relationship for the problem and to determine an appropriate fundamental
solution\index{Fundamental solution}.  If either of these requirements cannot
be satisfied then a boundary integral formulation cannot be achieved.  The
most common difficulty in applying the BEM is in determining an appropriate
fundamental solution.

A linear differential equation can be expressed in operator form as $Lu =
\gamma$ where $L$ is a linear operator, $\gamma$ is an inhomogeneous source
term and $u$ is the dependent variable.  The fundamental solution for this
equation is a solution of
\begin{equation}
  L^{*} \fnof{\omega}{\vect{x},\vect{\xi}}  + \fnof{\delta}{\vect{\xi}} = 0
\end{equation}
where * indicates the adjoint of the operator $L$ and $\delta$ is the Dirac
delta function.  No specific boundary conditions are prescribed but in some
cases regularity conditions at infinity need to be satisfied. The
fundamental solution is a Green's function which is not required to satisfy
any boundary conditions and is therefore also commonly termed the
free-space Green's function.

The mathematical theory required to determine the fundamental solution of a
constant coefficient PDE is well-developed and has been used successfully
to determine the fundamental solutions for a wide range of constant
coefficient equations \cite{walker:1980} \cite{clements:1978}
\cite{ortner:1987}.  Fundamental solutions are known and have been
published for many of the most important equations in engineering such as
Laplace's equation, the diffusion equation and the wave equation
\cite{brebbia:1984b}.  However, by no means can it be guaranteed that the
fundamental solution to a specific differential equation is known.  In
particular, PDEs with variable coefficients do not, in general, have known
fundamental solutions.  If the fundamental solution to an operator cannot
be found then domain integrals cannot be completely removed from the
integral formulation.  Domain integrals will also arise for inhomogeneous
equations.

\citeasnoun{wu:1985} argued that the BEM has several advantages over
other numerical methods which justify its use for many practical problems -
even in cases where domain integration is required.  He argued that for
problems such as flow problems a wide range of phenomena are described by
the same governing equations. What distinguishes these phenomena is the
boundary conditions of the problem.  For this reason accurate description
of the boundary conditions is vital for solution accuracy.  The BEM
generates a formulation involving both the dependent variable $u$ and the
flux $q$. This allows flux boundary conditions to be applied directly which
cannot be achieved in either the finite element or finite difference
methods.

Another advantage of the BEM over other numerical methods is that it allows an
explicit expression for the solution at an internal point.  This allows a
problem to be subdivided into a number of zones for which the BEM can be
applied individually.  This zoning approach is suited to problems with
significantly different length scales or different properties in different
areas.

Domain integration can be simply and accurately performed in the BEM.
However, the presence of domain integrals in the BEM formulation negates
one of the principal advantages of the BEM in that the problem dimension is
no longer reduced by one.  Several methods have been developed which allow
domain integrals to be expressed as equivalent boundary integrals. In this
section these methods will be discussed.

\section{Removing Domain Integrals due to Inhomogeneous Terms}

Inhomogeneous PDEs occur for a large number of physical problems.  An
inhomogeneous term may arise due to a number of factors including a source
term, a body force term, or due to initial conditions in time-dependent
problems.  An inhomogeneous linear PDE can be expressed in operator form as
$Lu = \gamma$ where $\gamma$ is a known function of position or a non-zero
constant.  If the fundamental solution is known for the operator $L$, the
resulting BEM formulation will be
\begin{equation}
  \matr{H} \vect{u} - \matr{G} \vect{q} = -\goneint{\gamma \omega}{\Omega}
\label{eq:inhomog}
\end{equation}
The domain integral in this formulation does not involve any unknowns so
domain integration can be used directly to solve this equation.  This
requires discretising the domain into internal cells in much the same way
as for the finite element method.  As the domain integral does not involve
any unknown values accurate results can generally be achieved using a
fairly coarse mesh.  This method is simple and has been shown to produce
accurate results \cite{brebbia:1984b}.  This approach, however, requires a
domain discretisation and a numerical domain integration procedure which
reduces the attraction of the BEM over domain-based numerical methods.

\subsection{The Galerkin Vector technique}

For some particular forms of the inhomogeneous function $\gamma$ the domain
integral can be transformed directly into boundary integrals. 

Consider the Poisson equation $\laplacian{u} = \gamma$.  Applying the BEM gives
an equation of the form of \eqnref{eq:inhomog}.  Using Green's second
identity
\begin{equation}
  \goneint{\pbrac{\gamma \laplacian{v} - v
  \laplacian{\gamma}}}{\Omega} = 
  \goneint{\pbrac{\gamma \delby{v}{n} - v \delby{\gamma}{n}}}{\Gamma}
\end{equation}
domain integration can be avoided for certain forms of $\gamma$.  If a
\index{Galerkin vector} $v$ can be found which satisfies $\laplacian{v} = \omega$,
where $\omega$ is the fundamental solution of Laplace's equation, then for
the specific case of $\gamma$ being harmonic ($\laplacian{\gamma} = 0$) Green's
second identity can be reduced to
\begin{equation}
  \goneint{\gamma \omega}{\Omega} = \goneint{\pbrac{\gamma \delby{v}{n} - v
      \delby{\gamma}{n}}}{\Gamma} 
\end{equation}
Therefore if a Galerkin vector can be found and $\gamma$ is harmonic the
domain integral in \eqnref{eq:inhomog} can be expressed as equivalent
boundary integrals.

\citeasnoun{fairweather:1979} determined the Galerkin vector for the
two-dimensional Poisson equation and \citeasnoun{rangogni:1982} determined
the Galerkin vector for the three-dimensional Poisson equation.
\citeasnoun{danson:1981} showed how this method can be applied
successfully for a number of physical problems involving linear isotropic
problems with body forces.  He considered the practical cases where the
body force term arose due to either a constant gravitational load, rotation
about a fixed axis or steady-state thermal loading.  In each of these cases
the domain integral can be expressed as equivalent boundary integrals.

This Galerkin vector approach provides a simple method of expressing 
domain integrals as equivalent boundary integrals.  Unfortunately, it only
applies to specific forms of the inhomogeneous term $\gamma$ (\ie $\gamma$
is required to be harmonic).

\subsection{The Monte Carlo method}

Domain discretisation could be avoided by using a Monte Carlo technique.
This technique approximates a domain integral as a sum of the integrand
at a number of random points. Specifically, in two dimensions, a domain
integral $I$ is approximated as 
\begin{equation}
  I \approx \dfrac{A}{N} \dsuml{i=1}{N} \fnof{f}{x_{i},y_{i}}
\end{equation}
where $\fnof{f}{x_{i},y_{i}}$ is the value of the integrand at random point
$\pbrac{x_{i},y_{i}}$, $N$ is the number of random points used and $A$ is the
area of the region over which the integration is performed.  This
approximation allows a domain integral to be approximated by a summation
over a set of random points so domain integration can be performed without
requiring a domain mesh.  This method has the secondary advantage of
allowing the integration to be performed over a simple geometry enclosing
the problem domain - if a random point is not in the problem domain its
contribution is ignored.

The method was proposed by \citeasnoun{gipson:1987}.  Gipson has
successfully applied this method to a number of Poisson-type problems.
Unfortunately this method often proves to be computationally expensive as a
large number of integration points are needed for accurate domain
integration.  Gipson argues however that, as this method removes the burden
of preparing a domain mesh, the extra computational expense is justified.

\subsection{Complementary Function-Particular Integral method}
\label{sec:anapim}

A more general approach can be developed using particular solutions.  Consider
the linear problem $Lu = \gamma$. $u$ can be considered as the sum of the
complementary function $u_{c}$, which is a solution of the homogeneous
equation $Lu_{c} = 0$, and a particular solution $u_{p}$ which satisfies
$Lu_{p} = \gamma$ but is not required to satisfy the boundary conditions of
the problem.  Applying BEM to the governing equation using the expansion 
 $u =u_{c} + u_{p}$ gives
\begin{equation}
  \matr{H} \vect{u} - \matr{G} \vect{q} = \matr{H} \vect{u_{p}} - \matr{G}
  \vect{q_{p}}
\label{eq:partic}
\end{equation}
If a particular solution $u_{p}$ can be found, all values on the
right-hand-side of \eqnref{eq:partic} are known - reducing the problem to
\begin{equation}
  \matr{H} \vect{u} - \matr{G} \vect{q} = \vect{d}
\end{equation}
where $\vect{d}$ is a vector of known values.  This linear system can be
solved by applying boundary conditions.  

This approach can be applied in a situation where an analytic expression
for a particular solution can be found.  Unfortunately particular solutions
are generally only known for simple operators and for simple forms of
$\gamma$.  Alternatively an approximate particular solution could be
calculated numerically.  \citeasnoun{zheng:1991} proposed a method
where a particular solution is determined by approximating the
inhomogeneous source term using a global interpolation function.  This
approach is a special case of a more general method known as the dual
reciprocity boundary element method.

\section{Domain Integrals Involving the Dependent Variable}

Consider the linear homogeneous PDE $Lu = 0$.  For many operators the
fundamental solution to the operator $L$ may be unobtainable or may be in
an unusable form.  This is especially likely if $L$ involves variable
coefficients for which case it has been shown that it is particularly
difficult to find a fundamental solution.  Instead, a BEM formulation can
be derived based on a related operator $\hat{L}$ with known fundamental
solution.  A BEM formulation for $Lu = 0$ based on the operator $\hat{L}$
will be of the form
\begin{equation}
  \matr{H} \vect{u} - \matr{G} \vect{q} 
            = -\goneint{\pbrac{\hat{L} - L} u \omega}{\Omega}
\label{eq:relate}
\end{equation}
where $\omega$ is the fundamental solution corresponding to the operator 
$\hat{L}$. This integral equation is similar to \eqnref{eq:inhomog}.  
However in this case the domain integral term involves the dependent
variable $u$.  This problem could be solved using domain integration where
the internal nodes are treated as formal problem unknowns. 

\subsection{The Perturbation Boundary Element Method}

\index{Perturbation Boundary Element Method@Perturbation BEM}
\citeasnoun{rangogni:1986} proposed solving variable coefficient PDEs by
coupling the boundary element method with a perturbation method. He
considered the two-dimensional generalised Laplace equation
\begin{equation}
  \nabla \cdot \pbrac{ \fnof{\kappa}{x,y}\nabla\fnof{V}{x,y} } = 0
\label{eq:rangge2}
\end{equation}
Using the substitution $\fnof{V}{x,y} = \kappa^{-\dfrac{1}{2}} \fnof{u}{x,y}$
\eqnref{eq:rangge2} can be recast as a heterogeneous Helmholtz equation
\begin{equation}
  \laplacian{u} + \fnof{f}{x,y}u = 0
\label{eq:varhelm2}
\end{equation}
where $f$ is a known function of position.

\citename{rangogni:1986} treated 
this equation as a perturbation about Laplace's equation. He considered the class of equations
\begin{equation}
  \laplacian{u} + \varepsilon \fnof{f}{x,y} u = 0 \qquad 
    \text{where } 0 \leq \varepsilon \leq 1
\label{eq:eqnfamily}
\end{equation}
for which he sought a solution of the form
\begin{equation}
  u = u_{0} + \varepsilon u_{1} + \varepsilon^{2} u_{2} + \ldots =
  \dsuml{j=0}{\infty} u_{j} \varepsilon^{j}
\label{eq:uexpan}
\end{equation}
Substituting \eqnref{eq:uexpan} into \eqnref{eq:eqnfamily} and grouping powers
 of $\varepsilon$ gives
\begin{equation}
  \laplacian{u_{0}} + \varepsilon \pbrac{\laplacian{u_{1}} + fu_{0}} +
  \varepsilon^{2} \pbrac{\laplacian{u_{2}} + fu_{1}} + \ldots = 0
\label{eq:groups}
\end{equation}
A solution will only exist for all values of $\varepsilon$ if the terms at
each power of $\varepsilon $ equal zero. This allows \eqnref{eq:groups} to be
treated as an infinite series of distinct problems which can be solved
using the boundary element method. $u_{0}$ can be found by solving 
 $\laplacian{u_{0}} = 0$ which Rangogni assumes will satisfy the boundary 
conditions of the original problem.  Each successive $u_{j}$ can then be
found by solving a Poisson equation with homogeneous boundary conditions as 
 $u_{j-1}$ has been previously determined.  Rangogni used a domain
 discretisation to solve these Poisson problems.

\eqnref{eq:varhelm2} is a particular member of this family of equations for
which $\varepsilon =1$.  The solution to \eqnref{eq:varhelm2} is therefore
given by $\dsuml{j=0}{\infty} u_{j}$. Rangogni reported that in practice
this series converged rapidly and in his numerical examples he achieved
accurate results using only $u_{0}$ and $u_{1}$.

\citeasnoun{rangogni:1991} extended this coupled perturbation - boundary
element method to the general second-order variable coefficient PDE
\begin{equation}
  \laplacian{u} + \fnof{f}{x,y} \delby{u}{x} + \fnof{g}{x,y} \delby{u}{y} =
  \fnof{h}{x,y}
\end{equation}
He considered the family of equations
\begin{equation}
  \laplacian{u} + \varepsilon \sqbrac{\fnof{f}{x,y} \delby{u}{x} +
    \fnof{g}{x,y} \delby{u}{y}} = \fnof{h}{x,y} 
    \quad (0 \leq \varepsilon \leq 1)
\label{eq:family2}
\end{equation}
Applying the perturbation method to this family of equations allows
\eqnref{eq:family2} to be expressed as an infinite series of distinct Poisson
equations which can be solved using the boundary element method. Again
Rangogni used an domain mesh to solve these Poisson equations.  Rangogni
found that in practice convergence was rapid and accurate results were
produced.

\citeasnoun{gipson:1987b} considered a class of hyperbolic and elliptic
problems which can be transformed into an inhomogeneous Helmholtz equation.
They used the perturbation method to recast this as an infinite sequence of
Poisson equations. They avoided domain discretisation by using a Monte
Carlo integration technique \cite{gipson:1987} to evaluate the required
domain integrals.

\citeasnoun{lafe:1987} used the perturbation method to solve
steady-state groundwater flow problems in heterogeneous aquifers.  They
showed the method produced accurate results for simply varying hydraulic
conductivities with convergence after two or three terms.
\citename{lafe:1987} investigated the convergence of the perturbation
method.  They found that for rapidly varying hydraulic conductivity
convergence is not guaranteed.  From this investigation they concluded that
accurate results can be obtained so long as the hydraulic conductivity does
not vary by more than one order of magnitude within the solution domain.
If the hydraulic conductivity variation is more significant they recommend
using the perturbation method in conjunction with a subregion technique so
that the variation of conductivity within each subregion satisfies their
requirements.  This process could become computationally expensive,
particularly if convergence is not rapid, as the solution of multiple
subproblems will be required within each subregion.

%%%
%goes brown below here
%%%
\subsection{The Multiple Reciprocity Method}

The multiple reciprocity method 
 \index{Multiple reciprocity method|(} (MRM) was
initially proposed by \citeasnoun{nowak:1987} for the solution of transient
heat conduction problems.  Since then the method has been successfully applied
to a wide range of problems.  The MRM can be viewed as a generalisation of the
Galerkin vector approach.  Instead of using one higher-order fundamental
solution, the Galerkin vector, to convert the remaining domain integrals to
equivalent boundary integrals a series of higher-order fundamental solutions
is used.

 \index{Multiple reciprocity method!Poisson equation} Consider the Poisson equation
\begin{equation}
  \laplacian{u} = b_{0}
\end{equation}
where $b_{0} = \fnof{b_{0}}{\vect{x}}$ is a known function of position.
Applying BEM to this equation, using the fundamental solution to the Laplace
operator, gives
\begin{equation}
  \fnof{c}{\vect{\xi}} \fnof{u}{\vect{\xi}} + 
  \goneint{u \delby{\omega_{0}}{n}}{\Gamma} + 
  \goneint{b_{0} \omega_{0}}{\Omega} = \goneint{\omega_{0} \delby{u}{n}}{\Gamma}
\label{eq:mrmstart}
\end{equation}
where $\omega_{0}$ is the known fundamental solution to Laplace's equation
applied at point $\vect{\xi}$.  To avoid domain discretisation the domain integral
in \eqnref{eq:mrmstart} needs to be expressed as equivalent boundary
integrals.  Using MRM this is achieved by defining a higher-order
fundamental solution $\omega_{1}$ such that
\begin{equation}
  \laplacian{\omega_{1}} = \omega_{0}
\end{equation}
Using this higher-order fundamental solution the domain integral in
\eqnref{eq:mrmstart} can be written as
\begin{equation}
  \goneint{b_{0} \omega_{0}}{\Omega} =
  \goneint{b_{0}\laplacian{\omega_{1}}}{\Omega} 
\end{equation}
or 
\begin{equation}
  \goneint{b_{0} \omega_{0}}{\Omega} = 
  \goneint{\pbrac{u \delby{\omega_{1}}{n} - \omega_{1} \delby{u}{n}}}{\Gamma} +
  \goneint{\omega_{1} \laplacian{b_{0}}}{\Omega}
\label{eq:delb}
\end{equation}
This formulation has generated a new domain integral.  $b_{0}$ is a known
function so we can introduce a new function $b_{1}$ which can be determined
analytically from the relationship
\begin{equation}
  b_{1} = \laplacian{b_{0}}
\end{equation}
giving 
\begin{equation}
  \goneint{\omega_{1} \laplacian{b_{0}}}{\Omega} = 
  \goneint{\omega_{1} b_{1}}{\Omega}
\end{equation}
This process can be repeated by introducing a new higher-order fundamental
solution $\omega_{2}$ such that
\begin{equation}
  \laplacian{\omega_{2}} = \omega_{1}
\end{equation}
and continuing until convergence is reached.

This procedure is based on the recurrence relationships
%\section{}

\begin{alignat}{2}
  b_{j+1} & = \laplacian{b_{j}} & \qquad \text{for }j &= 0,1,2,\ldots
  \label{eq:recurr} \\
  \laplacian{\omega_{j+1}} & = \omega_{j} & \qquad \text{for }j &= 0,1,2,\ldots
  \label{eq:higher}
\end{alignat}
Using these recurrence relationships gives the boundary integral formulation
\begin{equation}
  \fnof{c}{\vect{\xi}} \fnof{u}{\vect{\xi}} + 
  \goneint{\pbrac{u\delby{\omega_{0}}{n} - \omega_{0} \delby{u}{n}}}{\Gamma} +
  \dsuml{j=0}{\infty}   \goneint{\pbrac{b_{j}\delby{\omega_{j+1}}{n} 
      - \omega_{j+1} \delby{b_{j}}{n}}}{\Gamma} = 0
\end{equation}
which is an exact formulation if the infinite series converges.  Errors are
only introduced at the stage of boundary discretisation.  

Introducing interpolattion functions and discretising the boundary gives
the matrix system
\begin{equation}
  \matr{H_{0}} \vect{u} - \matr{G_{0}} \vect{q} = \dsuml{j=0}{\infty}
  \pbrac{\matr{H_{j+1}} \vect{P_{j}} - \matr{G_{j+1}} \vect{R_{j}}}
\end{equation}
where $\matr{H_{j+1}}$ and $\matr{G_{j+1}}$ are influence coefficient
matrices corresponding to the higher-order fundamental solutions and
$\vect{p_{j}}$ and $\vect{r_{j}}$ contain the nodal values of $b_{j}$ and its
normal derivative. 

The MRM can be applied based on operators other than the Laplace operator.
This approach relies on knowledge of the higher-order fundamental solutions
necessary for application of the method.  These solutions have been
determined and successfully used for the Laplace operator in both two and
three dimensions but the extension of the method to other equation types
needs further research.  \citeasnoun{itagaki:1993} have determined the higher
order fundamental solutions for the two-dimensional modified Helmholtz
equation.

The MRM can be extended to other equations by allowing the forcing function
$b_{0}$ to be a general function such that $b_{0} = \fnof{b_{0}}{\vect{x},u,t}$.
The MRM will be restricted to cases where the recurrence relationships -
\eqnrefs{eq:recurr}{eq:higher} - can be employed.  \citeasnoun{brebbia:1989}
have applied the MRM to the Helmholtz equation $\laplacian{u} + \kappa^{2} u =
0$ where $b_{0} = -\kappa^{2} u$ and the recurrence relationship defined by
\eqnref{eq:recurr} becomes simply 
\index{Multiple reciprocity method!Helmholtz equation}
\begin{equation}
  u_{j+1} = \laplacian{u_{j}} = - \kappa^{2j} u
\end{equation}
In this case the boundary integral formulation will be
\begin{equation}
  \fnof{c}{\vect{\xi}} \fnof{u}{\vect{\xi}} + \dsuml{j=0}{\infty}
  \goneint{\kappa^{2j} \pbrac{u \delby{\omega_{j}}{n} -
    \omega_{j} \delby{u}{n}}}{\Gamma} = 0
\end{equation}
\index{Multiple reciprocity method|)}

\subsection{The Dual Reciprocity Boundary Element Method}

\subsubsection{Equation Derivation}

\index{Dual reciprocity boundary element method@Dual reciprocity BEM|)}
The dual reciprocity boundary element method (DR-BEM) was developed to
avoid the need for domain integration in cases where the fundamental
solution of the governing differential equation is unknown or is
impractical to apply.  Instead the DR-BEM is applied using an appropriate
related operator with known fundamental solution.  The most common choice
is the Laplace operator \cite{partridge:1992} and in this chapter the DR-BEM
will be illustrated for this choice.

Consider a second-order PDE which can be expressed in the form
\begin{equation}
  \laplacian{u} = b
\label{eq:drmbase}
\end{equation}
The forcing function $b$ can be completely general. If $b = \fnof{b}{\vect{x}}$
then $b$ is a known function of position and the differential equation
described is simply the Poisson equation. For potential problems $b =
\fnof{b}{\vect{x},u}$ and for transient problems $b = \fnof{b}{\vect{x},u,t}$.
Applying the BEM to \eqnref{eq:drmbase} will give
\begin{equation}
  \matr{H} \vect{u} - \matr{G} \vect{q} = -\goneint{b \omega}{\Omega}
\end{equation}
where $\omega$ is the known fundamental solution to Laplace's equation.
The aim of the DR-BEM is to express the domain integral due to the forcing
function $b$ as equivalent boundary integrals.

The DR-BEM uses the idea of approximating $b$ using
interpolation functions.  A global approximation to $b$ of the form
\label{page:alpha}
\begin{equation}
  b = \dsuml{j=1}{M} \alpha_{j}f_{j}
\label{eq:globalapprox}
\end{equation}
is proposed.  $\alpha_{j}$ are unknown coefficients and $f_{j}$ are
approximating functions used in the interpolation and are generally chosen
to be functions of the source point and the field point of the fundamental
solution. The approximating functions $f_{j}$ are applied at $M$ different
collocation points - called poles - generally most, but not all, of which
are located on the boundary of the problem domain.

As discussed in the previous chapter the solution to a linear PDE $Lu =
\gamma$ can be constructed as the sum of a complimentary function $u_{c}$
(which satisfies the homogeneous equation $Lu_{c} = 0$) and a particular
solution $u_{p}$ to the equation $Lu_{p} = \gamma$.  Instead of using a
single particular solution, which may be difficult to determine, the DR-BEM
employs a series of particular solutions $\hat{u}_{j}$ which are related to
the approximating functions $f_{j}$ as shown in \eqnref{eq:harmonic}.
\begin{equation}
  \laplacian{\hat{u}_{j}} = f_{j} \quad j = 1, \ldots ,M
\label{eq:harmonic}
\end{equation}
By substituting \eqnrefs{eq:globalapprox}{eq:harmonic} into \eqnref{eq:drmbase}
the forcing function $b$ is approximated by a weighted summation of
particular solutions to the Poisson equation.
\begin{equation}
  \laplacian{u} = \dsuml{j=1}{M} \alpha_{j} \laplacian{\hat{u}_{j}}
\label{eq:rhsapprox}
\end{equation}
The DR-BEM essentially constructs an approximate particular solution to the
governing PDE as a summation of localised particular solutions.  

With the governing equation rewritten in the form of \eqnref{eq:rhsapprox} the
standard boundary element approach can be applied. \Eqnref{eq:rhsapprox} is
multiplied by a weighting function $\omega$ and integrated over the domain.
Green's theorem is applied twice and the fundamental solution of the
Laplacian is used to remove the remaining domain integrals. The name dual
reciprocity BEM is derived from the application of reciprocity
relationships to both sides of \eqnref{eq:rhsapprox}. After applying these
steps \eqnref{eq:drmmaths} is obtained, where the fundamental solution pole
is applied at point $\vect{\xi}$.
\begin{multline}
  \fnof{c}{\vect{\xi}} \fnof{u}{\vect{\xi}} + \goneint{\pbrac{u
    \delby{\omega}{n} - \omega \delby{u}{n}}}{\Gamma} \\ =
  \dsuml{j=1}{M} \alpha_{j} \pbrac{\fnof{c}{\vect{\xi}}
    \fnof{u_{j}}{\vect{\xi}} + \goneint{\pbrac{\hat{u}_{j}
      \delby{\omega}{n} - \omega \delby{hat{u}_{j}}{n}}}{\Gamma}}
\label{eq:drmmaths}
\end{multline}
In implementing a 
numerical solution of this equation similar steps are taken as for 
the standard BEM.  The boundary is
discretised into elements and interpolation functions are introduced to
approximate the dependent variable within each element. 

The form of each $\hat{u}_{j}$ is known from \eqnref{eq:harmonic} once the
approximating functions $f_{j}$ have been defined. It is not necessary to
use interpolation functions to approximate each $\hat{u}_{j}$. However by
using the same interpolation functions to approximate $u$ and $\hat{u}_{j}$
the numerical implementation will generate the same matrices $\matr{H}$ and
$\matr{G}$ on both sides of \eqnref{eq:drmmaths}. The error generated by
approximating each $\hat{u}_{j}$ in this manner has been found to be small
and can be justified by the improved computational efficiency of the method
\cite{partridge:1992}.

The application of this method results in the system
\begin{equation}
  \matr{H} \vect{u} - \matr{G} \vect{q} = \dsuml{j=1}{N+I} \alpha_{j}
  \pbrac{\matr{H} \vect{\hat{u}_{j}} - \matr{G} \vect{\hat{q}_{j}}}
\label{eq:sumin}
\end{equation}
where the $M$ poles were chosen to be the $N$ boundary nodes plus $I$ internal
points so that $M = N+I$. Although it is not generally necessary to include
poles at internal points it has been found that in general improved accuracy
is achieved by doing so \cite{nowak:1992}.  It has been shown that for many
problems \cite{partridge:1992} \cite{cruse:1993} using boundary points only in
this procedure is insufficient to define the problem.  In general using
internal points is likely to improve the solution accuracy as it increases the
number of degrees of freedom.  No theory has been developed of how many
internal collocation points should be used for optimal accuracy, or where
these points should be positioned within the problem domain. Using internal
poles in this interpolation does not require domain discretisation - it is
only necessary to specify the coordinates of the internal collocation points.
The internal points can be chosen to be locations where the solution is of
interest.

The $\vect{\hat{u}_{j}}$ and $\vect{\hat{q}_{j}}$ vectors can be treated as
columns of the matrices $\matr{\hat{U}}$ and $\matr{\hat{Q}}$ respectively.
This allows \eqnref{eq:sumin} to be rewritten as
\begin{equation}
  \matr{H} \vect{u} - \matr{G} \vect{q} = \pbrac{\matr{H} \matr{\hat{U}} -
    \matr{G} \matr{\hat{Q}}} \vect{\alpha}
\label{eq:drmalpha}
\end{equation}
where $\vect{\alpha}$ is a vector containing the nodal values of $\alpha$.
To solve this system it is necessary to evaluate $\vect{\alpha}$.
$\vect{\alpha}$ is defined by \eqnref{eq:globalapprox} which, for the nodal
values, can be expressed in matrix form as $\vect{b} = \matr{F}
\matr{\alpha}$. If the $\matr{F}$ matrix is nonsingular this expression can
be rearranged to give \eqnref{eq:expalpha} which provides an explicit
expression for $\matr{\alpha}$.
\begin{equation}
  \vect{\alpha} = \matr{F}^{-1} \vect{b}
\label{eq:expalpha}
\end{equation}
Including this explicit expression for $\vect{\alpha}$ in 
\eqnref{eq:drmalpha} gives
\begin{equation}
  \matr{H} \vect{u} - \matr{G} \vect{q} = \pbrac{\matr{H} \matr{\hat{U}} -
    \matr{G} \matr{\hat{Q}}} \matr{F}^{-1} \vect{b}
\label{eq:drmsystem}
\end{equation}
The approach taken to solve this equation will depend on the form of $b$.

\subsubsection{The Approximating Function $f$}
\label{sec:approxfnchoice}

\index{Dual reciprocity boundary element method@Dual reciprocity
  BEM!approximating function}
The accuracy of the DR-BEM hinges on the accuracy of the global
approximation to the forcing function $b$ (defined by
\eqnref{eq:globalapprox}).  Therefore the choice of the approximating
functions $f_{j}$ is a key consideration when implementing the DR-BEM. The
only requirement so far prescribed on the form of the approximating
functions $f_{j}$ is that the $\matr{F}$ matrix generated should be
nonsingular and that the related particular solutions $\hat{u}_{j}$ can be
determined and can be expressed in a practical closed form.  Some work has
been conducted into investigating what form of $f_{j}$ should be used in a
given situation to provide the highest accuracy and computational
efficiency.

Usually a form of $f_{j}$ is defined and this can be used, applying
\eqnref{eq:harmonic}, to specify $\hat{u}$ and $\hat{q}$.  The fundamental
solution of Laplace's equation is $\fnof{\omega}{\vect{x},\vect{\xi}} =
-\dfrac{1}{2\pi} \ln r$ in two-dimensional space and
$\fnof{\omega}{\vect{x},\vect{\xi}} = \dfrac{1}{4 \pi r}$ in three-dimensional
space - where $r$ is the Euclidean distance between the field point
$\vect{x}$ and the source point $\vect{\xi}$ of the fundamental
solution.  Due to the dependence of this fundamental solution only on $r$
the approximating function is generally chosen to be some radial function
\ie  $f_{j} = \fnof{f_{j}}{r}$. Several other options for $f_{j}$ have been tried
\cite{partridge:1992} but it has been found that in general the most accurate
results were generated using some radial function.  For both two and
three-dimensional problems \citeasnoun{wrobel:1986} recommended choosing
$f_{j}$ from the series
\begin{equation}
  f_{j} = 1 + r_{j} + r_{j}^{2} +\ldots+ r_{j}^{m}
\label{eq:fdefine}
\end{equation}
where $r_{j}$ is the distance between the field point (node $j$) and the
DR-BEM collocation point (node $i$).  They showed that accurate results can
be achieved using some combination of terms from this series.  Generally
the same approximating function $f_{j}$ is used at all the collocation
points so in this thesis, for simplicity, the form of approximating
functions $f_{j}$ will be referred to by a single $f$.

Choosing $f$ to be a function of only one variable simplifies the process
of determining $\hat{u}$ and $\hat{q}$.  For two-dimensional problems, if $f
= \fnof{f}{r}$ then the relationship
\begin{equation}
  \laplacian{\hat{u}} = \fnof{f}{r} 
\end{equation}
can be reduced to the ordinary differential equation
\begin{equation}
  \dtwosqby{\hat{u}}{r} + \dfrac{1}{r} \dby{\hat{u}}{r} = f 
\end{equation}
Using $f$ defined by \eqnref{eq:fdefine} the corresponding forms of $\hat{u}$
and $\hat{q}$, for two-dimensional problems, can be shown to be
\begin{align}
  \hat{u} & = \dfrac{r^{2}}{4} + \dfrac{r^{3}}{9} + \ldots +
  \dfrac{r^{m+2}}{\pbrac{m+2}^{2}} \\ \hat{q} & = \pbrac{r_{x} \delby{x}{n} +
    r_{y} \delby{y}{n}} \pbrac{\dfrac{1}{2} + \dfrac{r}{3} +\ldots+
    \dfrac{r^{m}}{m+2}}
\end{align}
where $r_{x} = x_{j} - x_{i}$ and $r_{y} = y_{j} - y_{i}$.

Any combination of terms from \eqnref{eq:fdefine} can be used for specifying
$f$. It has been found that in general including higher-order terms leads
to little improvement in accuracy \cite{partridge:1992}. The most commonly
used form is $f = 1 + r$ as this approximation will generally give accurate
results with greater computational efficiency than other choices.

\Eqnref{eq:fdefine} was recommended as a basis for the approximating function
$f$ due to the particular form of the fundamental solution of Laplace's
equation and its dependence on $r$ only. If a different operator is used as
the basis of the DR-BEM then it is likely a different form of $f$ will be
more appropriate.  The choice of $f$ in this case will be discussed in
\secref{sec:otheroperators}.

The performance of the DR-BEM hinges on the choice of the approximating
function $f$.  The theory of how to determine the best approximating
function is therefore a vital component of the DR-BEM.  Unfortunately the
approximating function has generally been chosen and used in a rather
ad-hoc manner.  Recently some more formal analysis of the use of
approximating functions has been undertaken. 

\citeasnoun{golberg:1994} argued that a formal analysis of the approximating
function $f$ can be undertaken using the theory of radial basis functions.
Radial basis functions are a generalisation of cubic splines in
multi-dimensions.  Cubic splines are known to be optimal for one-dimensional
interpolation.  Therefore, rather than being an arbitrary choice, it seems
that choosing $f$ to be a radial function is a logical extension for two
or three-dimensional problems.  \citename{golberg:1994} showed that, for the
Poisson equation, choosing $f$ to be a radial basis function ensures
convergence of the DR-BEM.

They also demonstrated that $f = 1 + r$ is a specific member of the group
of radial basis functions.  The theory of using radial basis functions for
multi-dimensional approximation is fairly advanced.  It has been shown that
$f = r$ is optimal for three-dimensional problems which justifies the use
of $f = 1 + r$ when applying the DR-BEM to three-dimensional problems - the
constant is included to ensure a non-zero diagonal for $\matr{F}$.  However
for two-dimensional problems it has been shown that optimal approximation
is attained using the thin plate spline $f = r^{2} log r$.  This
observation lead \citename{golberg:1994} to suggest that choosing $f$ to be a
thin plate spline may improve the accuracy of the DR-BEM in two dimensions.
Recently \citeasnoun{golberg:1995} has published a review of the DR-BEM
concentrating on developments since 1990 concerning the numerical
evaluation of particular solutions.

\subsubsection{Inhomogeneous Equations}

If the forcing function $b$ is a function 
of position only then the differential equation under consideration 
is simply Poisson's equation. In this case it is not necessary to 
invert the $\matr{F}$ matrix as $\vect{\alpha}$ can simply be calculated from 
$\vect{b} = \matr{F} \vect{\alpha}$ using Gaussian
elimination. \Eqnref{eq:drmsystem} can be rewritten as
\begin{equation}
  \matr{H} \vect{u} - \matr{G} \vect{q} = \vect{d}\quad\text{where
    }\vect{d} = \pbrac{\matr{H} \matr{\hat{U}} - \matr{G} \matr{\hat{Q}}}
  \vect{\alpha}
\label{eq:drmpoisson}
\end{equation}
By applying boundary 
conditions \eqnref{eq:drmpoisson} can be 
reduced to a linear system $\matr{A} \vect{x} = \vect{\tau}$ which can be
solved to give the unknown nodal values of $u$ and $q$.

\citeasnoun{zheng:1991} and \citeasnoun{coleman:1991} have proposed
a method which uses a global shape function to construct an approximate
particular solution.  As discussed by \citeasnoun{polyzos:1994} this method
is essentially equivalent to the DR-BEM.  However,
\citename{zheng:1991} and \citename{coleman:1991} suggested several
alternative ways of determining the unknown coefficients $\alpha_{j}$ for
inhomogeneous equations.  \citeasnoun{zheng:1991} used a least-squares
method where they minimised the sum of squares
\begin{equation}
  S = \dsuml{m=1}{M} \pbrac{\fnof{b}{r_{m}} - \dsuml{j=1}{N}
    \alpha_{j} \fnof{f_{j}}{r_{m}}}
\end{equation}
using singular value decomposition.  For large systems they found the
computational efficiency could be improved by employing the conjugate
gradient method.  \citeasnoun{coleman:1991} successfully solved
inhomogeneous potential and elasticity problems which are governed by
operators other than the Laplacian.

\subsubsection{Elliptic Problems}

\index{Dual reciprocity boundary element method@Dual reciprocity
  BEM!elliptic problems}
If $b$ is a function of the dependent variable then $\vect{\alpha}$ will also 
be a function of the dependent variable. Consider, for example, the 
linear second-order differential equation
\begin{equation}
  \laplacian{u} + u = 0
\label{eq:helmtypeeqn}
\end{equation}
In this case $b = -u$ so $\vect{\alpha} = \matr{F}^{-1} \vect{-u}$.
Applying the DR-BEM to \eqnref{eq:helmtypeeqn}, based on the fundamental
solution to Laplace's equation, gives
\begin{equation}
  \matr{H} \vect{u} - \matr{G}\vect{q} = - \pbrac{\matr{H} \matr{\hat{U}} -
    \matr{G} \matr{\hat{Q}}} \matr{F}^{-1} \vect{u}
\end{equation}
which can be rearranged to give 
\begin{equation}
  \pbrac{\matr{H} + \matr{S}} \vect{u} = \matr{G}
  \vect{q}\quad\text{where }\matr{S} = \pbrac{\matr{H} \matr{\hat{U}}
    - \matr{G} \matr{\hat{Q}}} \matr{F}^{-1}
\label{eq:drmpotential}
\end{equation}
Again, by applying boundary conditions \eqnref{eq:drmpotential} can be reduced 
to a linear system $\matr{A} \matr{x} = \matr{\tau}$ which can be solved to
determine the unknown nodal values.

Due to the presence 
of the fully-populated $\matr{F}^{-1}$ matrix in \eqnref{eq:drmpotential}
 it is not possible to solve the boundary problem and internal 
problem separately. Instead the solution can be treated as a coupled 
problem and the solutions at boundary and internal nodes are generated 
simultaneously.

\paragraph{Derivative Terms}
\label{sec:deriv}

\index{Dual reciprocity boundary element method@Dual reciprocity
  BEM!derivative terms}
The DR-BEM can also be applied for elliptic problems where
$b$ involves derivatives of the dependent variable
\cite{partridge:1992}. Consider, for example, the differential equation 
\begin{equation}
  \laplacian{u} + \delby{u}{x} = 0
\end{equation}
In this case applying DR-BEM, using the Laplace fundamental solution, gives
\begin{equation}
  \matr{H} \vect{u} - \matr{G} \vect{q} = - \pbrac{\matr{H} \matr{\hat{U}} -
    \matr{G} \matr{\hat{Q}}} \matr{F}^{-1} \vect{\delby{u}{x}}
\end{equation}
To solve this problem it is necessary to relate the nodal values of $u$ to
the nodal values of $\delby{u}{x}$. This is achieved by
using interpolation functions to approximate $\vect{u}$ in a similar manner
as was used to approximate $b$ in \eqnref{eq:globalapprox}.  A global
approximation function of the form
\begin{equation}
  u = \dsuml{j=1}{M} \fnof{\phi_{j}}{x,y} \beta_{j}
\end{equation}
can be used to approximate $u$ where $\phi_{j}$ are the chosen
interpolation functions and $\beta_{j}$ are the unknown coefficients.  In
system form this can be expressed as
\begin{equation}
  \vect{u} = \matr{\Phi} \vect{\beta}
  \label{eq:uapprox}
\end{equation}
Although it is not necessary, equating $\matr{\Phi}$ to $\matr{F}$ 
improves the computational efficiency of the method as only one matrix 
inversion procedure is required. Differentiating \eqnref{eq:uapprox} gives 
\begin{equation}
  \vect{\delby{u}{x}} = \matr{\delby{\Phi}{x}} \vect{\beta}
  \label{eq:uappder}
\end{equation}
Choosing $\matr{\Phi} = \matr{F}$ and inverting \eqnref{eq:uapprox} to give an
explicit expression for $\vect{\beta}$ allows \eqnref{eq:uappder} to be
rewritten as
\begin{equation}
  \vect{\delby{u}{x}} = \matr{\delby{F}{x}} \matr{F}^{-1} \vect{u}
  \label{eq:dudxexp}
\end{equation}
\Eqnref{eq:drmsystem} can now be rewritten as
\begin{equation}
  \pbrac{\matr{H} + \matr{R}} \vect{u} = \matr{G}
  \vect{q}\quad\text{where }\matr{R} = \pbrac{\matr{H} \matr{\hat{U}}
    - \matr{G} \matr{\hat{Q}}} \matr{F}^{-1} \matr{\delby{F}{x}}
  \matr{F}^{-1}
\label{eq:drmderiv}
\end{equation}
By applying boundary conditions \eqnref{eq:drmderiv} can be reduced to 
a linear system which can be solved to give the unknown nodal values.

As mentioned earlier, the approximating function $f$ is generally chosen to be
$f = 1 + r$.  This can lead to numerical problems if derivative terms are
included in the forcing function $b$.  As shown in \eqnref{eq:dudxexp}
derivative terms require derivatives of $f$ to be evaluated.  For example,
evaluating the $\matr{\delby{F}{x}}$ matrix requires calculation of
$\delby{f}{x}$.  Using the approximating function $f = 1 + r$ gives
\begin{equation}
  \delby{f}{x} = \delby{f}{r} \delby{r}{x} = \delby{f}{r} \dfrac{r_{x}}{r} 
\end{equation}
This derivative function can become singular, so - as shown by
\citeasnoun{zhang:1993} - significant numerical error may result.  This
will especially be the case in problems where collocation points are located
close together.

\citeasnoun{zhang:1993} suggested two possibilities for avoiding this
problem.  The first suggestion involved using a mapping procedure to map
the governing equation to an equation without convective terms. This method
was shown to produce accurate results but is somewhat cumbersome and can
only be applied to linear problems.  A simpler approach is to choose an
approximating function which does not lead to singularities for convective
terms.  \citename{zhang:1993} recommended use of either $f = 1 + r^{3}$
or $f = 1 + r^{2}+ r^{3}$.  These approximating functions produce accurate
results and can be simply applied for both linear and nonlinear problems.
\citename{zhang:1993} recommended the adoption of these approximating
functions for all use of the DR-BEM.

The same idea of using \eqnref{eq:uapprox} to allow nodal values of $u$ to be
associated to its derivatives can be applied to extend the DR-BEM to cases
involving higher-order derivatives or cross derivatives of the dependent
variable.  Appropriate approximating functions need to be chosen to avoid
the problem of singularities.

\paragraph{Variable Coefficients}

\index{Dual reciprocity boundary element method@Dual reciprocity
  BEM!variable coefficients}
The DR-BEM can be readily extended to equations with variable coefficients.
Consider the variable coefficient Helmholtz equation
\begin{equation}
  \laplacian{u} + \fnof{\kappa}{\vect{x}} u = 0
\end{equation}
where $\kappa$ is a function of position - $\kappa = \fnof{\kappa}{x,y}$ in
two dimensions.  If the DR-BEM is applied using the known fundamental solution
to the Laplace operator then the forcing function is $b = -\kappa u$.
Applying the DR-BEM gives
\begin{equation}
  \matr{H} \vect{u} - \matr{G} \vect{q} = \pbrac{\matr{H} \matr{\hat{U}} -
    \matr{G} \matr{\hat{Q}}} \matr{F}^{-1} \vect{b}
  \label{eq:vcb}
\end{equation}
where $\vect{b}$ is a vector of the nodal values of the forcing function
$b$.  The relationship $b = -\kappa u$ can be written in matrix form as
$\vect{b} = - \matr{K} \vect{u}$ where $\matr{K}$ is a diagonal matrix
containing the nodal values of $\fnof{\kappa}{x,y}$ \ie
\begin{equation}
  \matr{K} =  \begin{bmatrix} \fnof{\kappa}{x_{1},y_{1}} & 0 &
      \cdots & 0 \\ 0 & \fnof{\kappa}{x_{2},y_{2}} & \cdots & 0 \\ \vdots &
      \vdots & \ddots & \vdots \\ 0 & 0 & \cdots & \fnof{\kappa}{x_{M},y_{M}} 
\end{bmatrix}
\end{equation}
where $M$ is the number of collocation points used in applying the DR-BEM.

Using this matrix expression for $\vect{b}$ \eqnref{eq:vcb} can be rearranged
to give
\begin{equation}
  \pbrac{\matr{H} + \matr{S} \matr{K}} \vect{u} = \matr{G}
  \vect{q}\quad\text{where}\matr{S} = \pbrac{\matr{H} \matr{\hat{U}} -
    \matr{G} \matr{\hat{Q}}} \matr{F}^{-1}
\end{equation}
which is a boundary-only expression for the variable coefficient Helmholtz
equation.  This method is general and can easily be extended to accommodate
variable coefficient derivative terms and a sum of variable coefficient
terms. 

\paragraph{Formulating the DR-BEM for a General Elliptic Problem}
\label{sec:genform}

In this section it has been shown how the DR-BEM can be applied for
elliptic problems with varying forms of $b$.  The DR-BEM can be applied in
cases where $b$ involves a sum of terms due to the basic property
\begin{equation}
  \goneint{\pbrac{b_{1} + b_{2}}}{\Omega} = \goneint{b_{1}}{\Omega} +
  \goneint{b_{2}}{\Omega}
\end{equation}
Consider a two-dimensional equation of the form
\begin{equation}
  \laplacian{\fnof{u}{x,y}} = \fnof{k}{x,y} u + \fnof{l}{x,y} \delby{u}{x} +
  \fnof{m}{x,y} \delby{u}{y} + \fnof{n}{x,y}
\end{equation}
Applying the DR-BEM to this equation gives a matrix system of the form
\begin{equation}
  \pbrac{\matr{H} - \matr{R}} \vect{u} = \matr{G} \vect{q} + \matr{S} \vect{n}
\end{equation}
where
\begin{align}
  \matr{S} & = \pbrac{\matr{H} \matr{\hat{U}} - \matr{G} \matr{\hat{Q}}}
  \matr{F}^{-1} \label{eq:srelate} \\ \matr{R} & = \matr{S} \sqbrac{\matr{K} +
    \pbrac{\matr{L} \matr{\delby{F}{x}} + \matr{M} \matr{\delby{F}{y}}}
  \matr{F}^{-1}}
\label{eq:rrelate} 
\end{align}
$\matr{K}$, $\matr{L}$ and $\matr{M}$ are diagonal matrices where the
diagonals contain the nodal values of $k$, $l$ and $m$ respectively.
$\vect{n}$ is a vector containing the nodal values of $n$.  

\subsubsection{The DR-BEM Using Other Operators}
\label{sec:otheroperators}

The DR-BEM has been presented in this chapter based on the Laplace
operator.  However the DR-BEM can be be applied using essentially any
operator of appropriate order with known fundamental solution.  If an
appropriate operator can be found the complexity of the forcing function
$b$ can be reduced.  This should improve the accuracy of the method.  The
problem with applying the DR-BEM based on another operator is in
choosing the approximating function $f$.  A choice of $f$ which produces
accurate results is required but it is also necessary to choose an $f$ for
which a particular solution $\hat{u}$ can be determined. 

\citeasnoun{zhu:1993} has determined the particular solutions necessary for
applying the DR-BEM based on the two-dimensional Helmholtz operator.
\begin{equation}
  \laplacian{u} + \kappa^2 u = \fnof{b}{x,y,u,t}
\end{equation}
Radial functions have generally been used when applying the DR-BEM.  Along
the lines of \citeasnoun{wrobel:1986}, \citename{zhu:1993} chose an
approximating function of the form $f = r^{m}$ where $m$ is a positive
integer.  Determining the particular solution $\hat{u}$ requires solving the
ordinary differential equation
\begin{equation}
  \dtwosqby{\hat{u}}{r} + \dfrac{1}{r} \dby{\hat{u}}{r} + \kappa^2 u = r^{m}
\end{equation}
which can be achieved using a variation of coefficients method.

\citeasnoun{partridge:1992} applied the DR-BEM to the transient convection
diffusion equation
\begin{equation}
  D \laplacian{u} - v_{x} \delby{u}{x} - v_{y} \delby{u}{y} - k u = \delby{u}{t}
\end{equation}
where the material parameters $D$, $v_{x}$, $v_{y}$ and $k$ are all assumed to
be homogeneous.  They applied the DR-BEM based on the steady-state
convection-diffusion operator
\begin{equation}
  D \laplacian{u} - v_{x} \delby{u}{x} - v_{y} \delby{u}{y} - k u = 0
\label{eq:steadycd}
\end{equation}
which has a known fundamental solution.

This analysis requires the determination of a particular solution $\hat{u}$
which satisfies
\begin{equation}
  D \laplacian{\hat{u}} - v_{x} \delby{\hat{u}}{x} - v_{y} \delby{\hat{u}}{y}
  - k \hat{u} = f
\label{eq:cdpartic}
\end{equation}
Instead of defining a form of the approximating function $f$ and solving
for $\hat{u}$ \citename{partridge:1992} chose to define $\hat{u}$ and use
\eqnref{eq:cdpartic} to determine the corresponding approximating function.
Although somewhat ad-hoc this approach was found to produce accurate
results.
\index{Dual reciprocity boundary element method@Dual reciprocity BEM|)}

%%% Local Variables: 
%%% mode: latex
%%% TeX-master: t
%%% End: 

\clearemptydoublepage
\chapter{The BEM for Parabolic PDES}

\section{Time-Stepping Methods}

Several approaches have been proposed for applying the BEM to parabolic
problems.  These methods can be broadly classified into two main approaches.
Either some form of time-stepping procedure is used to advance the solution
in time, or a semi-analytic technique is used which can directly calculate
a solution at a specified time. In this section time-stepping procedures
will be considered.

Time-stepping approaches discretise the time domain in some manner and use
some form of time marching scheme to advance the solution from one discrete
time to the next.  The two most commonly used time-stepping methods are the
coupled finite difference - BEM and the direct time integration method.
These two methods will be outlined in this section for the diffusion
equation 
\begin{equation}
  \laplacian{\fnof{u}{\vect{x},t}} = \dfrac{1}{\kappa}
  \delby{\fnof{u}{\vect{x},t}}{t}
\end{equation}
where the diffusivity $\kappa$ is a material parameter which can be a
constant or a function of position.

\subsection{Coupled Finite Difference - Boundary Element Method}
\label{sec:fdbem}

\index{Coupled finite difference - boundary element method}
This approach discretises the time-domain in a finite difference form.
Consider the variation between a time $t^{m}$ and a time $t^{m+1} =
t^{m} + \Delta t$.  The most common approach \cite{brebbia:1984} is to
assume that, for sufficiently small $\Delta t$, the time derivative can be
approximated using a first-order fully implicit finite difference scheme 
\begin{equation}
  \delby{\fnof{u}{\vect{x},t^{m+1}}}{t} =
  \dfrac{\fnof{u}{\vect{x},t^{m+1}} - \fnof{u}{\vect{x},t^{m}}}{\Delta t}
\label{eq:fdapprox}
\end{equation}
which allows the diffusion equation in this time-range to be approximated as
\begin{equation}
\laplacian{\fnof{u}{\vect{x},t^{m+1}}} - \dfrac{1}{\kappa \Delta t}
\fnof{u}{\vect{x},t^{m+1}} + \dfrac{1}{\kappa \Delta t} \fnof{u}{\vect{x},t^{m}} = 0
\label{eq:diffdapprox}
\end{equation}
Using this finite difference approximation the original parabolic equation
has been reduced to an elliptic equation.  Using the weighted residuals
method an integral equation can be generated from \eqnref{eq:diffdapprox}. 
\begin{equation}
\fnof{c}{\vect{\xi}}u^{m+1} + \goneint{u^{m+1} \delby{\omega}{n}}{\Gamma} =
\goneint{q^{m+1} \omega d\Gamma + \dfrac{1}{\kappa \Delta t}
\dintl{\Omega}{} u^{m} \omega}{\Omega}
\label{eq:fdbem}
\end{equation}
where $u^{m+1} = \fnof{u}{\vect{x},t^{m+1}}$ and $u^{m} = \fnof{u}{\vect{x},t^{m}}$.
The fundamental solution $\omega$ is a solution of the modified
Helmholtz equation
\begin{equation}
\laplacian{\fnof{\omega}{\vect{x},\vect{\xi}}} - \dfrac{1}{\kappa \Delta t}
\fnof{\omega}{\vect{x},\vect{\xi}} + \fnof{\delta}{\vect{\xi}} = 0 
\end{equation}
applied at some source point $\xi$.  The fundamental solution of the
modified Helmholtz equation is known in both two and three dimensions.  If
an internal solution is required at a specific time this can be determined
explicitly from \eqnref{eq:fdbem} where the fundamental solution is applied
at internal point $\xi$ and $\fnof{c}{\vect{\xi}} = 1$.

Unfortunately \eqnref{eq:fdbem} contains a domain integral.  This integral is
generally evaluated by using a domain mesh \cite{brebbia:1984}.  The domain
integral does not include any problem unknowns so a fairly coarse domain
mesh will generally suffice. Applying the BEM to \eqnref{eq:fdbem} gives
\begin{equation}
\matr{H} \vect{u^{m+1}} - \matr{G} \vect{q^{m+1}} = \matr{B} \vect{u^{m}}
\label{eq:fdbemsys}
\end{equation}
where $\matr{B}$ is a matrix containing the influence coefficients due to the
domain integral.  Using \eqnref{eq:fdbemsys} the solution
can be advanced in time.  $\matr{u^{0}}$ is known from the initial
conditions so a solution can be calculated at $t = t_{0} + \Delta t$.  A
solution at internal nodes can then be calculated.  The time-stepping
procedure can be repeated using the internal solution at  $t = t_{0} +
\Delta t$ as pseudo-initial conditions for the next time-step.  

If a constant time-step is used the matrices $\matr{H}$, $\matr{G}$ and
$\matr{B}$ can be calculated once and stored. The boundary conditions can be
applied to form a solution system of the form $\matr{A} \vect{x^{m+1}} =
\vect{\tau}$ where $\vect{x^{m+1}}$ is the vector of unknown nodal values
at time $t^{m+1}$ and $\vect{\tau}$ is a vector constructed from known
nodal values from the previous time-step.  For a problem with
time-independent boundary conditions at each time-step it is only necessary
to update $\vect{\tau}$ and solve the system for $\vect{x^{m+1}}$.  If a
problem has time-dependent boundary conditions the solution system needs to
be reformed at each time-step.

This coupled finite difference - boundary element method (FD-BEM) was first
proposed by \citeasnoun{walker:1980} for the diffusion equation. It was
implemented and investigated by \citeasnoun{curran:1980}.  They found that
this method will only produce accurate results if \eqnref{eq:fdapprox}
accurately approximates the time derivative.  This will generally require
small time-steps to be adopted.  \citename{curran:1980} investigated the use
of a higher-order approximation to the time-derivative.  They found that
this improved the accuracy of the method.  Unfortunately it lead to a
deterioration in convergence behaviour.

\citeasnoun{tanaka:1994} proposed a generalised version of this time-stepping
scheme.  They approximated the time variation within an interval as
\begin{equation}
\fnof{u}{\vect{x},t} = \phi \fnof{u}{\vect{x},t^{m+1}} + \pbrac{1-\phi}
\fnof{u}{\vect{x},t^{m}}
\end{equation}
where $\phi$, termed the time-scheme parameter, is a constant in the range
$0 < \phi \leq 1$.  Substituting this approximation and a first-order
finite difference approximation of the time derivative into the diffusion
equation gives
\begin{equation}
  \phi \laplacian{u^{m+1}} + \pbrac{1 - \phi} \laplacian{u^{m}} = \dfrac{u^{m+1} -
    u^{m}}{\kappa \Delta t}
\label{eq:tandiffapprox}
\end{equation}
If $\phi = 1$ this approximation of the diffusion equation is
equivalent to the standard FD-BEM discussed earlier.  An integral equation
can be derived from \eqnref{eq:tandiffapprox}.  \citename{tanaka:1994}
implemented this method and found it gave accurate results for a range of
diffusion problems.  They tested the accuracy for a Crank-Nicolson scheme
($\phi = \frac{1}{2}$), a Galerkin scheme ($\phi = \frac{2}{3}$) and a
fully implicit scheme ($\phi = 1$). They found that the best results were
achieved using a Crank-Nicolson scheme. 

\subsection{Direct Time-Integration Method}

\index{Direct time-integration method}
Instead of converting the original parabolic equation to an elliptic
equation the problem can be treated directly in the time domain by directly
integrating over both time and space.  The weighted residual statement
using this approach is
\begin{equation}
  \gint{t_{0}}{t_{F}}{\goneint{\sqbrac{\laplacian{\fnof{u}{\vect{x},t}} -
        \dfrac{1}{\kappa} \delby{\fnof{u}{\vect{x},t}}{t}}
      \fnof{\omega}{\xi,\vect{x},t_{F},t}}{\Omega}}{t} = 0
\end{equation}
Integrating in time once and in space twice gives
\begin{equation}
  \fnof{c}{\xi} \fnof{u}{\xi,t_{F}} + \kappa \gint{t_{0}}{t_{F}}
  {\goneint{u \delby{\omega}{n}}{\Gamma}}{t} = \kappa
  \gint{t_{0}}{t_{F}}{\goneint{q \omega}{\Gamma}}{t} +
  \goneint{\fnof{u}{\vect{x},t_{0}} \omega}{\Omega}
\label{eq:tdfs}
\end{equation}
where the fundamental solution $\omega$ satisfies
\begin{equation}
  \kappa \laplacian{\fnof{\omega}{\xi,\vect{x},t_{F},t}} 
  + \delby{\fnof{\omega}{\xi,\vect{x},t_{F},t}}{t} 
  + \fnof{\delta}{\xi} \fnof{\delta}{t_{F}} = 0
\end{equation}
This time dependent fundamental solution is known in two and three
dimensions.  Physically this fundamental solution represents the effect at
a field point $\vect{x}$ at time $t$ of a unit point source applied at a
point $\xi$ at time $t_{F}$.  If an internal solution is required at a
specific time this can be determined from \eqnref{eq:tdfs} with $\fnof{c}{\xi} = 1$.

The variation of $u$ and $q$ with time is unknown so it is still necessary
to step in time.  However, as the time dependence is included in the
fundamental solution, accurate results can be achieved using larger
time-steps than with the FD-BEM.  Two different time-stepping schemes can
be used.  Similarly to the FD-BEM, each time-step can be treated as a new
problem so that an internal solution is constructed at the end of each
time-step to be used as pseudo-initial conditions for the next time-step.
Alternatively the time integration process can be restarted at $t_{0}$ with
increasing numbers of intermediate steps used.  These two time-stepping
approaches are discussed in detail in \citeasnoun{brebbia:1984}.

The first method requires a new domain integral to be calculated after each
time-step due to the updated pseudo-initial conditions.  The second
time-stepping procedure involves only a domain integral at $t_{0}$ so,
ideally, a domain integral only needs to be calculated once. This, however,
will still require the user to create a domain mesh. As mentioned by
\citeasnoun{brebbia:1984}, in many practical cases the domain integral can
be avoided. If the initial conditions are $u_{0} = 0$ throughout the body
the domain integral equals zero. If the initial conditions satisfy
Laplace's equation $\laplacian{u_{0}} = 0$ then a Galerkin vector can be found
and the domain integral can be expressed as equivalent boundary integrals.
This includes many practical cases such as constant initial temperature or
an initial linear temperature profile.

Unfortunately, in practice it is not always feasible to restart the
integration process at $t_{0}$. At each time-step new $\matr{H}$ and
$\matr{G}$ matrices are required so if many time-steps are required the
storage capacity of the computer is likely to be exceeded.  This requires
the procedure to be restarted at some time where an internal solution is
constructed and used as pseudo-initial conditions to repeat the process.
Therefore, in practice, both time-stepping methods are likely to require
domain integration.

\section{Laplace Transform Method}

\index{Laplace transform method}
An alternative approach which avoids time-stepping is to solve the problem
in a transform domain which removes the time dependence of the problem.
The parabolic PDE is thus converted to an elliptic problem for which the
boundary element method has been shown to generally produce accurate
results.  Once the solution to the elliptic problem is determined in the
transform space a solution in the original space can be attained using an
inverse transform procedure.  The most appropriate transform approach for
parabolic problems is the Laplace transform.

Consider the diffusion equation
\begin{equation}
  \laplacian{\fnof{u}{\vect{x},t}} = \dfrac{1}{\kappa}
  \delby{\fnof{u}{\vect{x},t}}{t}
\label{eq:lap_diff}
\end{equation}
with appropriate boundary and initial conditions.  The Laplace transform
of $\fnof{u}{\vect{x},t}$ will be symbolised as $\fnof{U}{\vect{x},\lambda}$ and is
defined as
\begin{equation}
  \fnof{U}{\vect{x},\lambda} = \gint{0}{\infty}{e^{-\lambda t}
  \fnof{u}{\vect{x},t}}{t}
\end{equation}
Applying Laplace transforms to \eqnref{eq:lap_diff} gives
\begin{equation}
  \laplacian{\fnof{U}{\vect{x},\lambda}} = \dfrac{1}{\kappa} \pbrac{\lambda
    \fnof{U}{\vect{x},\lambda} - \fnof{u_{0}}{\vect{x}}}
\label{eq:laptrans}
\end{equation}
with transformed boundary conditions. $\fnof{u_{0}}{\vect{x}}$ is the initial
conditions of $u$. \eqnref{eq:laptrans} is an elliptic PDE which can be
readily solved using the boundary element method.  Once the solution is
determined in Laplace transform space this solution can be inverted to give
a solution in the time-domain. This inversion procedure requires solutions
to be generated for several values of the transform parameter $\lambda$.

This method was first proposed by \citeasnoun{rizzo:1970} and has since
been successfully used by other practitioners \cite{moridis:1991}
\cite{zhu:1994}.  \citeasnoun{liggett:1979} compared the Laplace transform
method with the time-dependent Green's function method. They noted that the
direct method is simpler to apply.  However, due to its greater efficiency,
they recommended the Laplace transform method for solving diffusion
problems.

One limitation of the Laplace transform method is that \eqnref{eq:laptrans}
is inhomogeneous so that applying the standard BEM will generate a domain
integral involving the initial conditions.  Traditionally this domain
integral has been calculated by using a domain discretisation
\cite{brebbia:1984}.  However, recently \citeasnoun{zhu:1994} proposed
using the DR-BEM to convert this domain integral term to equivalent
boundary integrals.  They chose to apply the DR-BEM based on the known
fundamental solution to the Laplace operator.  Considering
\eqnref{eq:laptrans} this means that the DR-BEM will be used to convert the
right-hand-side to equivalent domain integrals.  Therefore the required
DR-BEM approximation is
\begin{equation}
  \dfrac{1}{\kappa} \pbrac{\lambda \fnof{U}{\vect{x},\lambda} -
    \fnof{u_{0}}{\vect{x}}} = \dsuml{j=1}{N+L} f_{j}\alpha_{j}
\label{eq:drmapprox}
\end{equation}
The DR-BEM can now be applied to \eqnref{eq:laptrans}, giving a matrix
system of the form
\begin{equation}
  \pbrac{\matr{H} - \dfrac{\lambda}{\kappa}\matr{S}} \vect{U} - \matr{G}
  \vect{q} = -\dfrac{1}{\kappa} \matr{S} \vect{u_{0}}
\end{equation}
which can be reduced to a square system by applying boundary conditions.
Once the solution is determined for this elliptic equation in the transform
space a solution at a given time can be constructed using an inversion
process.

This Laplace transform dual reciprocity method (LT-DRM) can easily be
extended to equations of the form
\begin{equation}
  \laplacian{\fnof{u}{\vect{x},t}} = \dfrac{1}{\kappa}
  \delby{\fnof{u}{\vect{x},t}}{t} + \fnof{b}{\vect{x},u}
\end{equation}
in which case a matrix expression of the form 
\begin{equation}
  \pbrac{\matr{H} -\matr{R} - \dfrac{\lambda}{\kappa}\matr{S}} \vect{U}
  - \matr{G} \vect{q} = -\dfrac{1}{\kappa} \matr{S} \vect{u_{0}}
\end{equation}
is generated.  Zhu and his colleagues have successfully extended the LT-DRM
to solve diffusion problems with nonlinear source terms.

\section{The DR-BEM For Transient Problems}
\label{sec:drmtrans}

\index{Dual reciprocity boundary element method@Dual reciprocity
  BEM!transient problems}
The DR-BEM can also be applied to parabolic problems. Consider, 
for example, the diffusion equation
\begin{equation}
\laplacian{u}  = \dfrac{1}{\kappa} \delby{u}{t}
\end{equation}
where the thermal diffusivity, $\kappa$, is a constant.  In this case the
global approximation of $b$ implies a separation of variables such that
\begin{equation}
\delby{u}{t} = \dsuml{j=1}{M} \fnof{f_{j}}{\vect{x}} \fnof{\alpha_{j}}{t}
\label{eq:sovapprox}
\end{equation}
Using \eqnref{eq:sovapprox}, \eqnref{eq:drmsystem} becomes
\begin{equation}
  \matr{H} \vect{u} - \matr{G} \vect{q} = \dfrac{1}{\kappa} \pbrac{\matr{H}
    \matr{\hat{U}} - \matr{G} \matr{\hat{Q}}} \matr{F}^{-1}
  \vect{\delby{u}{t}}
\end{equation}
or
\begin{equation}
  \matr{C} \vect{\delby{u}{t}} + \matr{H} \vect{u} = \matr{G}
  \vect{q}\quad\text{where }\matr{C} = -\dfrac{1}{\kappa}
  \pbrac{\matr{H} \vect{\hat{U}} - \matr{G} \matr{\hat{Q}}} \matr{F}^{-1}
\label{eq:drmtransient}
\end{equation}
\Eqnref{eq:drmtransient} can be solved using a standard direct
time-integration method.

\citeasnoun{partridge:1990} recommended using a first-order finite difference
approximation to the time derivative 
\begin{equation}
\delby{u}{t} = \dfrac{u^{m+1} - u^{m}}{\Delta t}
\end{equation}
and linear approximations to $u$ and $q$ within a time-step.
\begin{align}
u & = \pbrac{1 - \phi_{u}} u^{m} + \phi_{u} u^{m+1} \\
q & = \pbrac{1 - \phi_{q}} q^{m} + \phi_{q} q^{m+1}
\end{align}
where $\phi_{u}$ and $\phi_{q}$ are weighting parameters with values in the
range $\brac{(}{0,1}{]}$ and the time-step is between times $t^{m}$ and $t^{m+1} =
t^{m} + \Delta t$.  Substituting these approximations into
\eqnref{eq:drmtransient} an expression at $t^{m+1}$ can be derived in terms
of values at $t^{m}$.
\begin{equation}
  \sqbrac{\dfrac{1}{\Delta t} \matr{C} + \phi_{u} \matr{H}} \vect{u^{m+1}} -
  \phi_{q} \matr{G} \vect{q^{m+1}} = \sqbrac{\dfrac{\vect{C}}{\Delta t} -
    \pbrac{1-\phi_{u}} \matr{H}} \vect{u^{m}} + \pbrac{1 - \phi_{q}} \matr{G}
  \vect{q^{m}}
\end{equation}
The values of $u^{0}$ and $q^{0}$ are known from the initial conditions so
a time-stepping procedure can be used.  If a constant time-step is used the
matrices $\matr{C}$, $\matr{H}$ and $\matr{G}$ only need to be constructed
once.  Using this two-level time-integration scheme the most common choice
of time-scheme parameters is $\phi_{u} = 0.5, \phi_{q} = 1.0$.

\section{The MRM for Transient Problems}

\index{Multiple reciprocity method!diffusion equation}The MRM can be applied
to the diffusion equation $\laplacian{u} = \dfrac{1}{\kappa} \delby{u}{t}$
using the fundamental solution of Laplace's equation.  In this case the
forcing function becomes $b_{0} = \dfrac{1}{\kappa} \delby{u}{t}$ and the
recurrence relationship defined by \eqnref{eq:recurr} becomes
\begin{equation}
u_{j+1} = \laplacian{u_{j}} = \dfrac{\del^{j} u}{\del t^{j}}
\end{equation}
The higher-order fundamental solutions are known for Laplace's equation.
In this case the MRM formulation becomes
\begin{equation}
  \fnof{c}{\xi} \fnof{u}{\xi} + \dsuml{j=0}{\infty}
  \goneint{\delby{\omega}{n} \dfrac{\del^{j} u}{\del t^{j}}}{\Gamma} =
  \dsuml{j=0}{\infty} \goneint{\omega \dfrac{\del^{j} q}{\del t^{j}}}{\Gamma}
\end{equation}

The standard BEM numerical procedure can be applied to this boundary
integral equation.  This gives the matrix system
\begin{equation}
  \matr{H_{0}} \vect{u} + \matr{H_{1}} \vect{\dot{u}} + \matr{H_{2}}
  \vect{\ddot{u}} + \ldots = \matr{G_{0}} \vect{q} + \matr{G_{1}} \vect{\dot{q}} +
  \matr{G_{2}} \vect{\ddot{q}} + \ldots
\label{eq:infsystem}
\end{equation}
where the matrices $\matr{H_{1}},\matr{G_{1}}$ etc are the influence
coefficient matrices relating to the higher-order fundamental solutions.
This equation can be solved using a time-integration procedure.

The most common approach is to solve this system numerically by
discretising the time domain and using a time-stepping procedure.  This
requires some interpolation between the two time-levels marked by $m$ and
$m+1$.  This most common approach is to use a linear approximation to
$u$ and $q$ in this time-range
\begin{align}
u & = \pbrac{1 - \phi} u^{m} + \phi u^{m+1} \\
q & = \pbrac{1 - \phi} q^{m} + \phi q^{m+1}
\end{align}
where $\phi$ has a value in the range 0 to 1.  Differentiating these linear
approximations gives
\begin{align}
\dot{u} & = \dfrac{u^{m+1} - u^{m}}{\Delta t^{m}} \\
\dot{q} & = \dfrac{q^{m+1} - q^{m}}{\Delta t^{m}}
\end{align}
and all the other derivatives vanish.

This allows \eqnref{eq:infsystem} to be simplified to
\begin{equation}
\matr{H_{l}} \vect{u^{m+1}} - \matr{G_{l}} \vect{q^{m+1}} = - \matr{H_{r}} \vect{u^{m}} + \matr{G_{r}} \vect{q^{m}}
\label{eq:firstorder}
\end{equation}
where $\matr{H_{l}} = \dfrac{1}{\Delta t^{m}} \matr{H_{1}} + \phi
\matr{H_{0}}$, $\matr{H_{r}} = - \dfrac{1}{\Delta t^{m}} \matr{H_{1}} +
\pbrac{1-\phi} \matr{H_{0}}$, $\matr{G_{l}} = \dfrac{1}{\Delta t^{m}}
\matr{G_{1}} + \phi \matr{G_{0}}$, $\matr{G_{r}} = - \dfrac{1}{\Delta t^{m}}
\matr{G_{1}}+ \pbrac{1-\phi} \matr{G_{0}}$.  This approach is termed a first-order
approach as it removes all but the first derivatives.  A second order approach
can be formulated by using quadratic interpolation of $u$ and $q$ within the
time-range.

Using \eqnref{eq:firstorder} the solution can be advanced in time.  If a
constant time-step is used the matrices $\matr{H_{l}}, \matr{H_{r}},
\matr{G_{l}}$ and $\matr{G_{r}}$ only need to be constructed once outside the
time-stepping loop.  If the boundary conditions are not time-dependent the
boundary conditions only need to be applied once.

%%% Local Variables: 
%%% mode: latex
%%% TeX-master: t
%%% End: 

\clearemptydoublepage

\addcontentsline{toc}{chapter}{\numberline{}Bibliography}

\bibliographystyle{agsm}
\bibliography{../references/references}

%%% Local Variables: 
%%% mode: latex
%%% TeX-master: "~/documents/notes/fembemnotes/fembemnotes"
%%% End: 

\clearemptydoublepage

\addcontentsline{toc}{chapter}{\numberline{}Index}
\printindex

\end{document}

%%% Local Variables: 
%%% mode: latex
%%% TeX-master: t
%%% End: 
'"
%%% End: 
